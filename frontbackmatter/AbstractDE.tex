%*******************************************************
% Abstract in German
%*******************************************************
\begin{otherlanguage}{ngerman}
\pdfbookmark[1]{Zusammenfassung}{Zusammenfassung}
\chapter*{Zusammenfassung}
In dieser Arbeit wird ein Chatbot realisiert, welcher fachliche Nutzerfragen zu der IT-Dokumentation eines Großprojekts des IT-Beratungsunternehmens Capgemini beantworten kann. Nach einer historischen Einführung wird erläutert, welche Rolle die Wahl der eingesetzten Chat-Plattform für die Akzeptanz des Dialogroboters bei den Nutzern spielt. Anschließend folgt eine Analyse, auf welche Weise Gesprächsagenten wie Apple Siri und Google Assistant eingehende Nachrichten in Abfragen für eine Wissensdatenbank überführen und so Nutzerfragen beantworten. In Orientierung an diese Vorgehensweise wird ein Chatbot für IT-Projektwissen mit den Diensten von IBM Watson realisiert. Dieser prototypische Dialogroboter besteht aus zwei miteinander verknüpften Modulen. Eines davon wird mit der IT-Dokumentation synchronisiert und gibt zur Nutzerfrage passende Kapitel im Chat zurück. Im zweiten Modul wird manuell eine Wissensdatenbank angelegt, in welcher die Informationen der IT-Dokumentation in Dialogform gespeichert werden. Die dort verwendete Struktur ermöglicht es, Unterhaltungen zu planen und dem Nutzer zu dessen Anfrage passende Rückmeldungen zu liefern. Damit die Antworten des Chatbots jederzeit fachlich korrekt sind, soll die Wissensdatenbank nach testgetriebener Vorgehensweise aufgebaut und erweitert werden. Zu diesem Weiterentwicklungsprozess zählt auch die Analyse von Unterhaltungen zwischen dem Chatbot und den Projektmitarbeitern. Dieser als Trainingsphase bezeichnete Vorgang ermöglicht es, die bisher eingepflegten Inhalte weiter an die Erwartungen der Nutzer anzupassen. In einer anschließenden Evaluation bewerten zwei Projektmitarbeiter ihre Nutzererfahrung mit dem prototypischen Dialogroboter für IT-Projektwissen. Diese sehen im Chatbot das Potential, in einem ausgereifteren Zustand praktisch im Projekt eingesetzt zu werden. Um diesen Status zu erreichen, bedarf es zum aktuellen Zeitpunkt jedoch noch eines hohen Arbeitsaufwandes. Der Grund hierfür ist, dass sämtliche Synonyme für die im System hinterlegten Wörter manuell gepflegt werden müssen. Die Dienste von IBM Watson sind zum aktuellen Zeitpunkt noch nicht dazu in der Lage, Rechtschreibfehler und verschiedene Schreibweisen von Begriffen automatisiert zu erkennen oder zu erlernen. Außerdem müssen komplex abzubildende Informationen in manchen Anwendungsfällen mehrfach in die Wissensdatenbank eingepflegt werden. Daher kommt diese Arbeit zu dem Ergebnis, dass es zum aktuellen Zeitpunkt technisch durchaus möglich ist, einen Chabot für IT-Projektwissen auf Basis von IBM Watson zu realisieren. Die Pflege des Systems erfordert aber einen zu hohen Aufwand, um parallel zu den Anpassungen an der IT-Dokumentation im Projekt durchgeführt zu werden.
\end{otherlanguage}
