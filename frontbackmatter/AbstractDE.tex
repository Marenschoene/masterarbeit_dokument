%*******************************************************
% Abstract in German
%*******************************************************
\begin{otherlanguage}{ngerman}
\pdfbookmark[1]{Zusammenfassung}{Zusammenfassung}
\chapter*{Zusammenfassung}
Der Einsatz von Empfehlungssystemen zur Entscheidungsunterstützung hat in den vergangenen Jahren zunehmend an Bedeutung gewonnen.
Während Empfehlungssysteme traditionell für die Empfehlung von Gegenständen, Orten oder Dokumenten eingesetzt werden, hat die Empfehlung von Personen durch den Zuwachs an sozialen Netzwerken vermehrt Aufmerksamkeit erlangt.
In Systemen, in denen Personen die Inhalte von Empfehlungen bilden, kann der Erfolg einer Empfehlung von der Bedürfniserfüllung beider beteiligten Parteien, das heißt von sowohl Empfehlungsempfänger als auch empfohlener Person, beeinflusst werden.
Solche Systeme werden als wechselseitige oder auch bilaterale Empfehlungssysteme bezeichnet.

In projektgetriebenen Unternehmen können wechselseitige Empfehlungssysteme Entscheidungsträger darin unterstützen, passende Mitarbeiter zu offenen Projektpositionen zuzuordnen.
In der Forschung wurde bereits belegt, dass die beidseitige Berücksichtigung der Bedürfnisse bei der Empfehlungserstellung verglichen mit einem unilateralen Ansatz zu einer Verbesserung der Zufriedenheit der Mitarbeiter sowie der erwarteten Arbeitsleistung dieser führen kann.
Nach aktuellem Stand der Forschung blieb jedoch bislang unklar, wie die Bedürfnisse von Entscheidungsträgern und Mitarbeitern in einem bilateralen Empfehlungssystem einfliessen müssten, um die Zufriedenheit der Mitarbeiter und die erwartete Arbeitsleistung seitens der Entscheidungsträger bei der Zuordnung von Mitarbeitern zu Projektpositionen robust zu verbessern.
Diese Forschungslücke soll im Rahmen der vorliegenden Arbeit geschlossen werden.

Hierfür wurde ein multi-kriterieller Ansatz für die Berücksichtigung der Bedürfnisse von Entscheidungsträgern und Mitarbeitern bei der Zuordnung zu Projekten entwickelt.
Dessen Tauglichkeit wurde im Rahmen eines Experiments mit einem unilateralen Ansatz verglichen, welcher lediglich die Bedürfnisse der Entscheidunsträger betrachtet.

Insgesamt zeigen die Erkenntnisse, die im Rahmen dieser Arbeit gewonnen werden konnten, dass eine Berücksichtigung der beidseitigen Bedürfnisse bei der Zuordnung von Mitarbeitern zu Projektpositionen grundsätzlich vorteilhaft ist.
Jedoch konnte festgestellt werden, dass einzelne Ausreißer in den Ergebnissen vorhanden waren, bei denen ein Rückgang von Zufriedenheit bzw. Arbeitsleistung verzeichnet wurde.
Dadurch kann eine Robustheit des entwickelten Ansatzes nicht uneingeschränkt bestätigt werden.

Als Ursache für diese Beobachtung wird vermutet, dass sowohl die Zufriedenheit von Mitarbeitern als auch deren erwartete Arbeitsleistung seitens der Entscheidungsträger durch weitere Einflussfaktoren bestimmt werden.
Um den eindeutigen Einfluss der bilateralen Bedürfniserfüllung auf Zufriedenheit und Arbeitsleistung zu ermitteln, wird empfohlen, diese Kriterien zu identifizieren und in weiterführenden Forschungen zu berücksichtigen.

Bezüglich der Gewichtung der Bedürfnisse scheint es eine geringfügigere Rolle zu spielen, welches Gewicht den Bedürfnissen zukommt.
Um den Effekt der Gewichte auf die Zufriedenheit und Arbeitsleistung genauer zu untersuchen, wird empfohlen eine größere Datenmenge als Ausgangsbasis für die Ermittlung einer optimalen Gewichtung heranzuziehen, um zusätzlich zu der Robustheit des Algorithmus unter Einsatz des optimierten Gewichts beizutragen.



\end{otherlanguage}
