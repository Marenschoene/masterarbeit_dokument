%*******************************************************
% Abstract in German
%*******************************************************
\begin{otherlanguage}{ngerman}
\pdfbookmark[1]{Zusammenfassung}{Zusammenfassung}
\chapter*{Zusammenfassung}
Der Einsatz von Empfehlungssystemen zur Entscheidungsunterstützung hat in den vergangenen Jahren zunehmend an Bedeutung gewonnen.
Während Empfehlungssysteme ursprünglich für die Empfehlung von Gegenständen oder Dokumenten eingesetzt wurden, hat die Empfehlung von Personen durch den Zuwachs an sozialen Netzwerken vermehrt Aufmerksamkeit erlangt.
In Systemen, in denen Personen die Inhalte von Empfehlungen darstellen, kann der Erfolg einer Empfehlung von der Erfüllung der Präferenzen beider beteiligter Parteien, das heißt von sowohl Empfehlungsempfänger als auch empfohlener Person, beeinflusst werden.
Solche Systeme werden als wechselseitige oder auch bilaterale Empfehlungssysteme bezeichnet.

In projektgetriebenen Unternehmen können wechselseitige Empfehlungssysteme Entscheidungsträger darin unterstützen, passende Mitarbeiter offenen Projektpositionen zuzuordnen.
% In der Forschung konnte bereits nachgewiesen werden, dass eine beidseitige Berücksichtigung der Bedürfnisse bei der Empfehlungserstellung im Vergleich zu einer unilateralen Berücksichtigung zu einer Verbesserung der Zufriedenheit von Mitarbeitern sowie der erwarteten Arbeitsleistung dieser führen kann.
Nach aktuellem Stand der Forschung blieb bislang unklar, in welcher Form die Präferenzen von Entscheidungsträgern und Mitarbeitern in einem bilateralen Empfehlungssystem einfliessen müssen, um die Zufriedenheit der Mitarbeiter und deren erwartete Arbeitsleistung seitens der Entscheidungsträger bei der Zuordnung der Mitarbeiter zu Projektpositionen robust zu verbessern.
Dieser Fragestellung soll im Rahmen der vorliegenden Arbeit nachgegangen werden.

Hierfür wurde ein multi-kriterieller Ansatz für die Berücksichtigung der Präferenzen von Entscheidungsträgern und Mitarbeitern bei der Zuordnung zu Projekten konzipiert.
Innerhalb des Systems wurde ein bilateraler Algorithmus implementiert, der Empfehlungen anhand der gewichteten Summe der Präferenzerfüllung eines Mitarbeiters und der Präferenzerfüllung eines Entscheidungsträgers ermittelt.
Dessen Tauglichkeit wurde im Rahmen eines Experiments mit einem unilateralen Algorithmus verglichen, welcher lediglich die Bedürfnisse der Entscheidunsträger betrachtet.

Insgesamt belegen die Ergebnisse des Experiments, dass eine Berücksichtigung der beidseitigen Präferenzen bei der Zuordnung von Mitarbeitern zu Projektpositionen grundsätzlich vorteilhaft ist.
Jedoch konnten einzelne Ausreißer in den Ergebnissen festgestellt werden, bei denen eine geringere Zufriedenheit bzw. Arbeitsleistung verglichen mit den Ergebnissen des unilateralen Algorithmus verzeichnet wurde.
Eine Robustheit des entwickelten Ansatzes kann daher nicht uneingeschränkt bestätigt werden.

Als Ursache für diese Beobachtung wird angenommen, dass sowohl die Zufriedenheit von Mitarbeitern als auch deren erwartete Arbeitsleistung seitens der Entscheidungsträger durch weitere Einflussfaktoren bestimmt werden.
Die Ergebnisse des Experiments führten zu der Annahme, dass ein möglicher Einflussfaktor die unterschiedlich starke Bedeutung der einzelnen Fähigkeiten für einen Mitarbeiter bzw. für einen Entscheidungsträger darstellt.
Für eine robustere Gestaltung des Systems wird daher für zukünftige Arbeiten empfohlen, die Bedeutung einer präferierten Fähigkeit für einen Mitarbeiter sowie die Bedeutung einer angeforderten Fähigkeit aus Sicht der Entscheidungsträger in dem bilateralen Algorithmus zu berücksichtigen.
Zur Identifikation weiterer Einflussfaktoren wird darüber hinaus für weiterführende Forschungen das Durchführen von Experteninterviews unter Mitarbeitern und Entscheidungsträgern empfohlen.
% Die identifizierten Faktoren sollten in weiterführenden Forschungen als zusätzliche Kriterien des multi-kriteriellen Empfehlungssystems berücksichtigt werden.

Bezüglich der Gewichtung der Präferenzen wurde im Rahmen des Experiments deutlich, dass es eine geringfügigere Rolle spielt, wie die Präferenzen von Entscheidungsträgern bzw. Mitarbeitern gewichtet werden.
Um den Effekt der Gewichtung auf Zufriedenheit und Arbeitsleistung genauer zu untersuchen, wird für weiterführende Forschungen empfohlen, die Fälle genauer zu betrachten, in denen die Bedürfnisse der Entscheidungsträger und die Bedürfnisse der Mitarbeiter miteinander konkurrieren.
Darüber hinaus sollte in zukünftigen Studien die optimalen Gewichtung der Präferenzen anhand einer größeren Datenbasis ermittelt werden, um den Algorithmus gegenüber Ausreißer robuster zu gestalten.

\end{otherlanguage}
