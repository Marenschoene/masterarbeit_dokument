%*******************************************************
% Abstract in German
%*******************************************************
\begin{otherlanguage}{ngerman}
\pdfbookmark[1]{Zusammenfassung}{Zusammenfassung}
\chapter*{Zusammenfassung}
Die Besetzung offener Projektpositionen wird verschiedenen Publikationen zu Folge eine immer häufiger auftretende Tätigkeit in der Wirtschaft sein. Unterstützung können dabei Empfehlungssysteme im Bereich der Personalauswahl bieten. Diese richten sich bislang zumeist entweder an Personalverantwortliche oder an Stellensuchende. Das psychologische Konzept des \aclp{PEFit} zeigt jedoch, dass die gemeinsame Betrachtung von Personalverantwortlichen und Angestellten für beide Parteien zu besseren Ergebnissen führt. Empfehlungssysteme werden in der Literatur als bilateral bezeichnet, wenn sie diese Theorie implementieren.

Die vorliegende Arbeit überprüfte, ob bilaterale Empfehlungssysteme bei der Besetzung offener Projektpositionen die Zufriedenheit der Angestellten mit der vorgeschlagenen Stelle und gleichzeitig die von den empfohlenen Mitarbeitern erwartete Arbeitsleistung seitens der Projektmanager erhöht. Das zu diesem Zweck implementierte Empfehlungssystem sortierte die Mitarbeiter eines Unternehmens für mehrere Projektpostionen sowohl über einen uni- als auch einen bilateralen Ansatz. Die bilaterale Variante bezog die Präferenzen von Projektmanagern und Angestellten in die Empfehlungsbestimmung ein. Der unilaterale Ansatz betrachtete dagegen lediglich die Wünsche der Projektverantwortlichen.

Eine anschließende Fallstudie unter Projektmanagern und Mitarbeitern des Unternehmens zeigte, dass der bilaterale Empfehlungsansatz die Zufriedenheit der Angestellten erhöhen und die erwartete Arbeitsleistung bei den Projektmanagern steigern konnte, wenn sich die Mitarbeiter mehrheitlich mit einer Tätigkeit auf der betrachteten Projektposition zufrieden zeigen.

Als Ursache für diese Beobachtung wird die Erhebung der Präferenzen der Mitarbeiter betrachtet. Das bilaterale Empfehlungssystem nutzte booleschen Werte, um die präferierten Fähigkeiten der Angestellten stärker zu gewichten. Bei nicht gewünschten Kompetenzen wurde jedoch nicht unterschieden, ob ein Angestellter dieser Fähigkeit neutral gegenübersteht oder ob er diese nicht bei der Projektarbeit anwenden möchte. Aus diesem Grund wird für folgende Arbeiten empfohlen, den im Rahmen dieser Arbeit implementierten Empfehlungsansatz um eine dreistufige Skala mit entsprechender negativer Präferenz zu erweitern. Bei der Implementierung sollten die Mitarbeiter bei vorhandenem Wunsch weiterhin höher positioniert werden, bei negativer Präferenz sollten sie jedoch zusätzlich niedriger einsortiert werden. Unter Betrachtung dieser Veränderungen sollte die Evaluation unter Mitarbeitern und Projektmanagern erneut durchgeführt und die Forschungsfrage entsprechend untersucht werden.
\end{otherlanguage}
