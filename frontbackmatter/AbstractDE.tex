%*******************************************************
% Abstract in German
%*******************************************************
\begin{otherlanguage}{ngerman}
\pdfbookmark[1]{Zusammenfassung}{Zusammenfassung}
\chapter*{Zusammenfassung}
Die Besetzung offener Projektpositionen ist eine immer häufiger ausgeführte Tätigkeit in der Wirtschaft. Unterstützung bieten dabei Empfehlungssysteme im Bereich der Personalauswahl. Diese richten sich bislang zumeist einseitig entweder an Personalverantwortliche oder an Stellensuchende. Das Konzept des \aclp{PEFit} aus der Berufs- und Organisationspsychologie besagt jedoch, dass die gemeinsame Betrachtung der Anforderungen von Personalverantwortlichen und Angestellten gleichzeitig auf Seiten der Mitarbeiter zu einer ausgeprägteren Zufriedenheit und aus Perspektive des Unternehmens zu einer höheren Arbeitsleistung führt. Empfehlungssysteme werden als bilateral bezeichnet, wenn sie diese Theorie implementieren. Es wurde bislang jedoch noch nicht nachgewiesen, dass solche Anwendungen tatsächlich gleichzeitig Zufriedenheit und Arbeitsleistung bei der Besetzung von Projektpositionen erhöhen. Diese Forschungslücke soll durch die vorliegende Master-Thesis geschlossen werden.

Im Rahmen dieser Arbeit wurde eine Anwendung implementiert, welche die Angestellten eines Beratungsunternehmens für mehrere Stellen sowohl über einen uni- als auch einen bilateralen Ansatz sortierte. Die bilaterale Variante bezog Präferenzen von Projektmanagern und Mitarbeiter in die Empfehlungsbestimmung ein. Das unilaterale Vorgehen betrachtete dagegen lediglich Anforderungen der Projektmanager. Eine Fallstudie unter Projektverantwortlichen und Mitarbeitern des Beratungsunternehmens zeigte, dass der bilaterale Empfehlungsansatz die Zufriedenheit der Angestellten erhöhen und die erwartete Arbeitsleistung seitens der Projektmanagern steigern konnte, wenn sich die Mitarbeiter mehrheitlich mit einer Tätigkeit auf der betrachteten Projektposition zufrieden zeigen.

Als Ursache für diese Beobachtung wird die Erhebung der Präferenzen der Mitarbeiter betrachtet. Das bilaterale Empfehlungssystem nutzte boolesche Werte, um die präferierten Fähigkeiten der Angestellten stärker zu gewichten. Bei nicht gewünschten Kompetenzen wurde jedoch nicht unterschieden, ob ein Angestellter dieser Fähigkeit neutral gegenübersteht oder ob er diese nicht bei der Projektarbeit anwenden möchte. Aus diesem Grund wird für folgende Arbeiten empfohlen, den im Rahmen dieser Arbeit implementierten Empfehlungsansatz auf eine dreistufige Skala mit entsprechender negativer Antwortmöglichkeit zu erweitern. Bei der Implementierung sollten die Mitarbeiter bei vorhandener Präferenz weiterhin höher positioniert werden, bei Auswahl der negativen Option sollten sie jedoch zusätzlich niedriger einsortiert werden. Unter Betrachtung dieser Veränderungen sollte die Evaluation unter Mitarbeitern und Projektmanagern nochmals durchgeführt und die Forschungsfrage erneut untersucht werden.
\end{otherlanguage}
