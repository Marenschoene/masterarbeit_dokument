\definecolor{exxetagray}{gray}{0.75}
\definecolor{itemcolor}{RGB}{179,217,255}
\definecolor{usercolor}{RGB}{255,204,179}

\shorthandoff{"}
\chapter{Multikriterielle Optimierung}
\label{ch:erweiterungen}
% Hier Überleitung finden zu Multikriterieller Optimierung in RS (Anknüpfen über Nutzen, der sich aus mehreren Aspekten zusammensetzt (hier: Nutzen setzt sich zusammen aus Präferenzen Manager und Präferenzen Mitarbeiter)).
% Grundsätzlich basiert das Rating eines Elements auf einem Kriterium.
Im vorangegangenen Kapitel wurde erläutert, dass für reziproke Empfehlungen sowohl die Präferenzen der Nutzer als auch die Präferenzen der Elemente berücksichtigt werden müssen.
% Irgendwo noch einbringen, dass Paarung quasi den Nutzen angibt, also gutes paar = guter Nutzen
Nach \textcite[S. 36]{li:inproceedings} ist eine optimale Paarung von Nutzer und Element demnach abhängig von zwei Attributen: der Bedürfniserfüllung des Empfehlungsempfängers und der Bedürfniserfüllung der empfohlenen Person.
Eine alleinige Betrachtung der Relevanz eines Elements für einen Nutzer, wie aus traditionellen Empfehlungssystemen bekannt, reicht folglich in \ac{RRS} nicht aus (Vgl. Kapitel \ref{ch:empfehlungssysteme:empfehlungserstellung:recommendation}).
Für die Ermittlung des Nutzen einer \ac{N-E-K} in wechselseitigen Empfehlungssystemen müssen demnach mehrere Kriterien berücksichtigt werden.

\section{Einführung}
\label{ch:erweiterungen:einführung}


\section{Problemstellung}

\section{Bedeutung in Empfehlungssystemen}

\section{Lösungsansätze}

\subsection{Pareto-Optimierung}

\subsection{Aggregation}

\subsection{Bedingungen}

\shorthandon{"}