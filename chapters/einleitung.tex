\shorthandoff{"}
\chapter{Einführung}
\label{ch:intro}

\section{Motivation}
\label{sec:intro:motivation}
In Unternehmen aller Branchen ist eine Abkehr von großgewachsenen, zentralen und hierarchischen Organisationsstrukturen zu beobachten. Stattdessen setzen Betriebe zunehmend auf kleine, dezentrale und flexible Teams, welche in Projektarbeiten neue Produkte entwickeln und Dienstleistungen erbringen \cite[S. 3]{elanceEconomy:1999}. Besonders stark nimmt dieser Trend seit Mitte der 1990er Jahre zu \cite[S. 8]{whittington:1999}. Ein Hauptgrund für den Wandel ist die fortschreitende Digitalisierung. Diese ermöglicht kurze Kommunikations- und Entscheidungswege und macht damit zentrale und hierarchische Unternehmensstrukturen zunehmend überflüssig \cite[S. 5]{elanceEconomy:1999}.

Diese Entwicklungen verdeutlichen, dass die Zusammenstellung von Teams für einzelne Projekte ein immer häufiger stattfindender Prozess in der Wirtschaft sein wird. Unterstützung können dabei Empfehlungssysteme im Bereich der Personalauswahl bieten. In der Literatur existieren bereits einige Ansätze zur Implementierung solcher Anwendungen. \textcite[S. 1ff.]{malinowski:2006} zu Folge sind solche Empfehlungssysteme jedoch häufig unzureichend, da sie sich zumeist einseitig entweder an Personalsachbearbeiter oder an Stellensuchende richten. Die Wissenschaftler empfehlen stattdessen die Implementierung bilateraler Empfehlungssysteme zur Projektbesetzung. Dieses Konzept basiert auf der Theorie des \acp{PEFit} aus der Berufs- und Organisationspsychologie. Es besagt, dass Mitarbeiter und Personalsachbearbeiter je eine Angebots- und eine Nachfrageperspektive besitzen. Gleichen sich Angebot und Nachfrage beider Parteien aus, führt dies aus Sicht des Unternehmens zu einer hohen Arbeitsleistung und zugleich aus Perspektive des Mitarbeiters zu einer ausgeprägten Zufriedenheit \cite[S. 6]{su:2015}.

Bisher belegte jedoch noch keine Veröffentlichung, dass diese erwarteten Ergebnisse des \acp{PEFit} auch bei der Besetzung offener Projektpositionen durch bilaterale Empfehlungssysteme reproduziert werden können. Somit ist nicht nachgewiesen, dass der höhere Aufwand zur Erstellung bilateraler Empfehlungen gegenüber unilateralen Vorschlägen in der praktischen Anwendung gerechtfertigt ist. Diese Forschungslücke soll im Rahmen dieser Master-Thesis geschlossen werden.

\newpage
\section{Zielsetzung}
\label{sec:intro:zielsetzung}
Das Ziel der vorliegenden Master-Thesis besteht darin, die folgende Forschungsfrage zu beantworten:

\forschungsfrage

Um diese Frage zu beantworten, wird ein Experiment mit einer anschließenden Fallstudie durchgeführt. Dabei werden Mitarbeiter eines Beratungsunternehmens im IT-Bereich sowohl von einem bilateralen als auch von einem unilateralen Empfehlungsverfahren für verschiedene offene Projektpostionen vorgeschlagen. Anschließend soll evaluiert werden, ob die Projektmanager von den vorgeschlagenen Angestellten der bilateralen Anwendung eine höhere Arbeitsleistung erwarten als von den Empfehlungen des unilateralen Systems. Gleichzeitig soll überprüft werden, ob die Anordnung der Mitarbeiter zugunsten einer höheren Zufriedenheit der Angestellten erfolgt.

\section{Gang der Arbeit}
\label{sec:intro:gangDerArbeit}
Um die Forschungsfrage der vorliegenden Master-Thesis zu beantworten, werden in den Kapiteln \ref{ch:personEnvironmentFit} und \ref{ch:empfehlungssysteme} zunächst die theoretischen Grundlagen bilateraler Empfehlungssysteme erläutert. Hierzu wird in Kapitel \ref{ch:personEnvironmentFit} das psychologische Konzept des \acp{PEFit} erörtert. Das folgende Kapitel \ref{ch:empfehlungssysteme} enthält einen Überblick über Implementierungsmethoden von Empfehlungssystemen. Dabei werden neben modell- und speicherbasierten Methoden auch hybride Ansätze erläutert. Darüber hinaus werden häufig auftretende Probleme bei der Entwicklung von Empfehlungssystemen geschildert und entsprechende Lösungsmöglichkeiten aufgezeigt. In Kapitel \ref{ch:verwandteArbeiten} wird daraufhin der aktuelle Stand der Forschung wiedergegeben. In diesem Kontext werden verwandte Arbeiten diskutiert und hinsichtlich der Forschungsfrage analysiert. Anschließend wird in Kapitel \ref{ch:methodik} die Methodik der Master-Thesis erläutert. In diesem Kontext werden die implementierte Systemarchitektur und die Gestaltung der Fallstudie erörtert. Die dabei gewonnenen Ergebnisse werden in Kapitel \ref{ch:ergebnisse} vorgestellt. Kapitel \ref{ch:diskussion} beinhaltet eine Diskussion der Erkenntnisse, welche mit der Beantwortung der Forschungsfrage und der Erörterung möglicher Einschränkungen der Forschung endet. Ein abschließendes Fazit folgt in Kapitel \ref{ch:fazit}.
\shorthandon{"}