\shorthandoff{"}
\chapter{Einleitung}
\label{ch:intro}

\section{Motivation}
\label{sec:einleitungmotivation}

\begin{quotation}
    \textit{"In einer immer komplexer werdenden Welt mit überbordenden Informationsquellen sind Entscheidungshilfen von unschätzbarem Wert."} (\textcite[S. vii]{klahold:book})
\end{quotation}

Durch die explosionsartige Entwicklung von Webanwendungen in den vergangenen Jahrzehnten und der damit einhergebenden Flut an Alternativen für Nutzer solcher Anwendungen, haben entscheidungsunterstützende Systeme zunehmend an Bedeutung gewonnen.
Eines der bekanntesten und machtvollsten Werkzeuge in der Entscheidungsunterstützung stellen Empfehlungssysteme dar \cite[S. vii]{ricci:inbook}.
% Indem die Systeme aus einer Menge an Alternativen potenziell nützliche Elemente empfehlen, können Sie Anwendern helfen mit dem Informationsüberfluss umzugehen.
Diese können Nutzer darin unterstützen, aus einer Menge an Alternativen potenziell nützliche Elemente zu finden.
Die Relevanz dieser Systeme wird seit Jahren durch die Vielzahl an Publikationen, Konferenzen und Forschungen in dem Bereich belegt \cite[S. vii]{klahold:book}.

% Ist der Nutzen eines empfohlenen Elements an einen Anwender durch mehrere Kriterien bedingt, wird in der Literatur von multi-kriteriellen Empfehlungssystemen gesprochen.
% Die Empfehlungserstellung in multi-kriteriellen Systemen unterscheidet sich von der in Empfehlungssystemen, die lediglich auf einem Kriterium basieren.
% Die Wahl des richtigen Verfahrens zur Empfehlungserstellung in multi-kriteriellen Systemen spielt hierbei eine entscheidende Rolle für den Erfolg eines Empfehlungssystems.

Wechselseitige Empfehlungssysteme, in denen Personen die Inhalte von Empfehlungen bilden, haben durch die Zunahme an sozialen Netzwerken seit einigen Jahren eine große Bedeutung erlangt.
Im Vergleich zu herkömmlichen Empfehlungssystemen ist der Erfolg einer Empfehlung in wechselseitigen Systemen unter anderem von der bilateralen Bedürfniserfüllung beider beteiligten Parteien abhängig.
Das bedeutet, dass sowohl die Bedürfnisse des Empfehlungsempfängers, als auch der empfohlenen Person bei der Empfehlungserstellung berücksichtigt werden müssen.
% Eine bilaterale Bedürfniserfüllung besagt, dass der Erfolg einer Empfehlung sowohl von der Berücksichtigung der Bedürfnisse des Anwenders eines Systems, als auch der empfohlenen Person abhängt.
Konträr wird die alleinige Berücksichtigung der Bedürfnisse des Anwenders eines Systems als unilaterale Bedürfniserfüllung bezeichnet.
% Die bilaterale Bedürfniserfüllung, die besagt, dass der Erfolg einer Empfehlung sowohl von der Berücksichtigung der Bedürfnisse des Anwenders eines Systems, als auch der empfohlenen Person abhängt, stellt hierbei eine große Herausforderung dar.
% Folglich kann die Empfehlung in wechselseitigen Systemen als multi-kriterielles Problem betrachtet werden, bei dem mindestens zwei Kriterien berücksichtigt werden müssen: die Bedürfnisse des Anwenders und die Bedürfnisse der empfohlenen Person.
% Aufgrund des massiven Anstiegs an sozialen Netzwerken in den vergangenen Jahren sind Systeme für die Empfehlung von Personen 
% In traditionellen Empfehlungssystemen, in denen empfohlene Elemente Produkte wie CDs, Bücher oder Filme darstellen, 
% Empfehlungssysteme, in denen Personen die Inhalte von Empfehlungen bilden und der Erfolg einer Empfehlung von der bilateralen Bedürfniserfüllung abhängt, werden als wechselseitige Empfehlungssysteme bezeichnet.
% Die Bedürfniserfüllung wird als bilateral bezeichnet, wenn sowohl die Bedürfnisse des Anwenders eines Systems, als auch der empfohlenen Person bei der Empfehlungserstellung berücksichtigt werden.
% Hier überleiten zu bekannten domänen, dann zu people to people empfehlung

Ein Einsatzgebiet wechselseitiger Empfehlungssysteme stellt die Besetzung offener Projektpositionen durch passende Mitarbeiter in projektgetriebenen Unternehmen dar.
Aufgrund eines Wandels weg von zentralen und hierarchischen Unternehmensstrukturen hin zu dezentralen Projektteams, wird angenommen, dass die Zuordnung von Mitarbeitern zu Projektpositionen in Unternehmen zukünftig weiter an Bedeutung gewinnen wird \cite[S .iii]{link:booklet}. % hier erklären, warum davon ausgegangen wird, dass das ein spannendes thema ist
Empfehlungssysteme können Entscheidungsträger in diesem Auswahlprozess unterstützen, indem potenziell passende Mitarbeiter für Projektpositionen ermittelt und den Entscheidungsträgern vorgeschlagen werden.
% \textcite[S .iii]{link:booklet} erkennt die Bedeutung der Berücksichtigung der bilateralen Präferenuen für eine erfolgreiche Zuordnung von Mitarbeitern zu Projekten an und entwickelte ein System, dass neben den Präferenzen der Entscheidungsträger auch die Präferenzen der empfohlenen Personen bei der Zuordnung berücksichtigt.
Während vorangegangene Arbeiten zumeist lediglich die Bedürfnisse der Entscheidungsträger bei der Empfehlungserstellung berücksichtigen, betont \textcite[S. iii]{link:booklet} die Bedeutung der Berücksichtigung bilateralen Präferenzen für eine erfolgreiche Zuordnung von Mitarbeitern zu Projekten.
% HIER NOCH ZITAT!!!
Er entwickelte ein bilaterales Empfehlungssystem, dass neben den Präferenzen der Entscheidungsträger auch die Präferenzen der empfohlenen Personen, nämlich den Mitarbeitern, bei der Zuordnung berücksichtigt.
\textcite[S. iii]{link:booklet} konnte in seiner Untersuchung zeigen, dass durch die beidseitige Berücksichtigung der Präferenzen in vier von fünf Fällen die erwartete Zufriedenheit von Mitarbeitern und Entscheidungsträgern gesteigert werden konnte.
Die Zufriedenheit der Entscheidungsträger erfasste \textcite[S. iii]{link:booklet} mittels der zu erwartenden Arbeitsleistung der empfohlenen Mitarbeiter in Projekten.
% Hier noch erwähnen, was seine kritikpunkte waren, um die mitaufzunehmen?

Bislang blieb offen, wie in einem bilateralen Empfehlungssystem die Bedürfnisse von Entscheidungsträgern und Mitarbeitern einfliessen müssten, um die Zufriedenheit beider Parteien bei der Zuordnung von Mitarbeitern zu Projektpositionen robust zu verbessern.
% Bislang blieb offen, wie die bilaterale Bedürfniserfüllung gestaltet werden muss, um die Zufriedenheit von Mitarbeitern und Entscheidungsträgern bei der Zuordnung von Mitarbeitern zu Projektpositionen robust zu verbessern.
% Nach aktuellem Stand der Forschung existiert demnach kein Ansatz für die Berücksichtigung der Präferenzen von Mitarbeitern und Entscheidungsträgern, der uneingeschränkt zu einer gesteigerten Mitarbeiterzufriedenheit und Arbeitsleistung in Projekten führt.
% Diese Frage gilt es in der vorliegenden Arbeit zu beantworten.
% Diese Tätigkeit findet häufig in Unternehmen Anwendung, die viel projektpasiert Arbeiten und aus einer Fülle an zur Verfügung stehenden Mitarbeitern passende für diese Projekte bestimmen müssen.
% Empfehlungssystemen können hier abhilfe leisten, indem passende Mitarbeiter für offene Projektpositionen empfohlen werden.


% empfehlung u.a. von Personen, bspw. im Online-Recruiting oder -Dating -> auch als reciprocal recommender bezeichnet
% entscheidend für erfolg wechselseitiger empfehlungen: beide seiten müssen die präferenzen des jeweils anderen erfüllen (zitat: yang)

% Ein Einsatzbereich wechselseitiger systeme, der zunehmend an bedeutung gewinnt: zuordnung von mitarbeitern zu projektpositonen in projektgetriebenen Unternehmen (zitat link)
% empfehlungssystem als entscheidungsunterstützung
% link nannte bedeutung der berücksichtigumg der präferenzen der mitarbeiter
% wie müssen die kriterien berücksichtigt werden?
% (eine immer bedeustamer werdende tätigkeit -> zuordnung von mitarbeitern zu projektpositionen (zitat link))

% wie empfehlungssystem entwickeln, das auf mehreren kriterien für die empfehlungserstellung beruht?
% multi-kriterielle empfehlung

% hier ien bisschen was schreiben zu Anwendungen von MCRS in der Praxis, siehe S. 849, file:///C:/Users/masc6/OneDrive/Persoenliche_Unterlagen/Uni/Masterthesis/2015_Book_RecommenderSystemsHandbook.pdf
% Bezug auf Titel anführen -> nicht nur Entwicklung des Algorithmus, sondern diesen robust zu entwickeln
% Hier was schreiben zur motivation des unternehmens, kurz unternehmen vorstellen (siehe methodik)
% deduktive vorgehensweise (https://www.scribbr.de/category/methodik/)

\section{Zielsetzung}
\label{sec:einletung:zielsetzung}
% Wird das Problem der bilateralen Bedürfniserfüllung aus Sicht der Entscheidungstheorie betrachtet, kann der Sachverhalt als Entscheidungsproblem mit zwei Kriterien verstanden werden.
% überlegung, dass betrachtung als multi-kriterielles problem mögliche herangehensweisen erschließt, um system robust zu gestalten
Ziel der vorliegenden Arbeit ist es, die folgende Forschungsfrage zu beantworten:

\forschungsfrage

Zur Beantwortung der Forschungsfrage soll anknüpfend an die Arbeit von \textcite[S. iii]{link:booklet} untersucht werden, ob ein alternativer Ansatz für die Berücksichtigung der bilateralen Bedürfniserfüllung bei der Ermittlung von Empfehlungen uneingeschränkt zu einer höheren Zufriedenheit der Angestellten und deren Arbeitsleistung in Projekten führt.
Hierfür wird ein Experiment basierend auf fünf Beispielprojekten durchgeführt.
Anhand zweier Befragungen wird im Rahmen des Experiments die Zufriedenheit von Mitarbeitern und die, von den Entscheidungsträgern erwartete Arbeitsleistung dieser Mitarbeiter, in den Beispielprojekten erhoben.
Für die Auswertung des Experiments wird ein multi-kriterielles Empfehlungssystem entwickelt, welches zwei Algorithmen implementiert.
Repräsentativ für die unilaterale Bedürfniserfüllung wird ein unilateraler Algorithmus erstellt, der die fünf passensten Mitarbeiter in Abhängigkeit ihrer Fähigkeiten für die Beispielprojekte vorschlägt.
Für die bilaterale Empfehlung wird ein bilateraler Algorithmus implementiert, der die fünf passensten Mitarbeiter für eine Projektposition in Abhängigkeit der zwei Kriterien Fähigkeit und Präferenz empfiehlt.
Darauf basierend wird für jedes Projekt die Performance der Algorithmen hinsichtlich der Anzahl an zufriedenen Mitarbeitern bzw. Mitarbeitern mit zu erwartend hoher Arbeitsleistung verglichen.
In dem Experiment stellen folglich "unilateraler" bzw. "bilateraler Allgorithmus" die unabhängige Variable sowie "Zufriedenheit auf Seiten der Mitarbeiter" und "Erwartete Arbeitsleistung durch die Entscheidungsträger" die abhängigen Variablen dar.
% Hierfür wurde die bilaterale Bedürfniserfüllung aus Sicht der Entscheidungstheorie als multi-kriterielles Problem betrachtet. die wechselseitige Empfehlung  Entscheidungstheorie können wechselseitige Empfehlungssysteme als multi-kriterielles Problem betrachtet werden, bei dem mindestens zwei Kriterien berücksichtigt werden müssen: die Bedürfnisse des Anwenders und die Bedürfnisse der empfohlenen Person.
% hier erklären, dass wechselseitiges system auch multi-kriterieller fall und vermutung ist, dass kombination der kriterien e


\section{Aufbau der Arbeit}
\label{sec:einleitung:aufbau_der_arbeit}
Nach der Einleitung in Kapitel 1 erfolgt in Kapitel 2 eine Übersicht über theoretische Grundlagen von Empfehlungssystemen.
Hierfür werden Grundbegriffe eingeführt und erläutert, die für ein fundiertes Verständnis der Inhalte der darauffolgenden Kapitel erforderlich sind.
Außerdem erfolgt ein allgemeiner Überblick über den Empfehlungserstellungs-Prozess.
Eine Erläuterung der Bedeutung von Präferenzen im Kontext von Empfehlungssystemen dient der Unterscheidung zwischen Präferenzen im herkömmlichen Sinn und Präferenzen im Kontext der vorliegenden Arbeit.
Zuletzt erfolgt eine Abgrenzung wechselseitiger Empfehlungssysteme zu ordinären Systemen. %mit anschließender Erläuterung der veränderten Problemstellung in solchen Systemen.

In Kapitel 3 wird das Problem der bilateralen Bedürfniserfüllung in wechselseitigen Systemen aus Sicht der Entscheidungstheorie betrachtet und als Entscheidungsproblem mit zwei Kriterien definiert.
Darauf folgt ein Überblick über die Bedeutung multi-kriterieller Problemstellungen in Empfehlungssystemen.
Dieser dient als Grundlage für die darauffolgende Darstellung der Forschung zu Lösungsansätzen multi-kriterieller Sachverhalte in Empfehlungssystemen.

% In den verwandten Arbeiten in Kapitel 4 wird basierend auf dem aktuellen Stand der Forschung betrachtet, wie die bilaterale Bedürfniserfüllung in wechselseitigen Empfehlungssystemen bislang gelöst wurde.
In Kapitel 4 werden verwandte Arbeiten dargestellt, die exemplarisch zeigen, wie die bilaterale Bedürfniserfüllung in wechselseitigen Empfehlungssystemen bislang gelöst wurde.
% Hier noch schreiben, dass Fokus darauf liegt, wie die kriterien Bedürfnisse des Empfehlungsempfängers sowie die kriterien bedürfnisse der empfohlenen Person aggregiert werden.

Kapitel 5 dient der Beschreibung des methodischen Vorgehens zur Beantwortung der Forschungsfrage.
In dem Kontext erfolgt die Erläuterung des konzipierten multi-kriteriellen Empfehlungssystems und des zugehörigen bilateralen Algorithmus.

Die Ergebnisse der Forschung werden in Kapitel 6 vorgestellt und in Kapitel 7 diskutiert.
In dem Rahmen erfolgt ein Ausblick auf zukünftige Forschungen in diesem Bereich.

Die Arbeit schließt mit einem Fazit hinsichtlich der Beantwortung der Forschungsfrage ab.

% - theoretische Grundlagen zu Empfehlungssystemen -> Vorstellung des Ablaufs der Empfehlungserstellung, Was Präferenzen in Empfehlungssystemen bedeutet, wechselseitige empfehlungssysteme
% - überblick zu multi-kriterieller Optimierung in empfehlungssystemen: was bedeutet das in Empfehlungssystemen, Darstellen des aktuellen Stands der Forschung zur Lösung von multi-kriteriellen Problemen
% - verwandte arbeiten -> in kontext verwandter arbeiten einordnen: wie wird bilaterale Bedürfnisserfüllung bislang gelöst? Gibt es bereits arbeiten, die den multi-kriteriellen Charakter der problematik betrachten und wenn ja, wie?
% - methodik und konzeption: aufsetzen des experiments, daten erheben, daten auswerten
% - Ergebnisse präsentieren und diskutieren
% - ausblick auf zukünftige forschungen in dem Bereich


\shorthandon{"}