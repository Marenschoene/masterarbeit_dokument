\shorthandoff{"}
\chapter{Einführung}
\label{ch:intro}

\section{Motivation}
\label{sec:intro:motivation}
In der Wirtschaft ist eine allgemeine Abkehr von großgewachsenen, zentralen und hierarchischen Organisationsstrukturen zu beobachten. Stattdessen setzen Betriebe zunehmend auf kleine, dezentrale und flexible Teams, welche in Projektarbeiten neue Produkte entwickeln und Dienstleistungen erbringen. Ein Hauptgrund für den Wandel ist die fortschreitende Digitalisierung. Diese ermöglicht kurze Kommunikations- und Entscheidungswege und macht damit zentrale und hierarchische Unternehmensstrukturen zunehmend überflüssig. \cite[S. 2ff.]{elanceEconomy:1999}
% Besonders stark nimmt dieser Trend seit Mitte der 1990er Jahre zu \cite[S. 8]{whittington:1999}. / \cite[S. 2f.]{elanceEconomy:1999} /\cite[S. 5]{elanceEconomy:1999}

Aufgrund dieser Entwicklungen ist zu erwarten, dass die Besetzung offener Projektpositionen ein immer häufiger stattfindender und bedeutsamer werdender Prozess in der Wirtschaft sein wird. Bei dieser Aufgabe können Empfehlungssysteme unterstützend zum Einsatz kommen. Diese richten sich bislang zumeist einseitig entweder an Personalverantwortliche oder an Stellensuchende \cite[S. 2f.]{siting:2012}. \textcite[S. 1]{malinowski:2006} empfahlen stattdessen die Implementierung bilateraler Anwendungen. Dieses Konzept basiert auf der Theorie des \aclp{PEFit}, welches häufig in der Berufs- und Organisationspsychologie Anwendung findet \cite[S. 2]{guan:2021}\cite[S. 1ff.]{malinowski:2006}. Es besagt, dass Mitarbeiter und Personalverantwortliche jeweils eine Angebots- und eine Nachfrageperspektive besitzen. Gleichen sich Angebot und Nachfrage beider Parteien aus, führt dies aus Sicht des Unternehmens zu einer hohen Arbeitsleistung und zugleich aus Perspektive des Mitarbeiters zu einer ausgeprägten Zufriedenheit \cite[S. 6]{su:2015}.

Bisher belegte keine Publikation, dass die Theorie des \aclp{PEFit} und die damit verbundenen Ergebnisse auf Seiten von Mitarbeitern und Personalverantwortlichen auch durch bilaterale Empfehlungssysteme bei der Besetzung offener Projektpositionen erzielt werden können. Somit ist nicht nachgewiesen, dass der im Vergleich zu unilateralen Systemen höhere Aufwand zur Implementierung solcher Anwendungen gerechtfertigt ist. Diese Forschungslücke soll im Rahmen dieser Master-Thesis geschlossen werden.

\newpage
\section{Zielsetzung}
\label{sec:intro:zielsetzung}
Das Ziel der vorliegenden Master-Thesis besteht darin, die folgende Forschungsfrage zu beantworten:

\forschungsfrage

Um diese Frage zu beantworten, wird ein Experiment mit einer anschließenden Fallstudie durchgeführt. Dabei werden die Mitarbeiter eines Beratungsunternehmens sowohl von einem uni- als auch von einem bilateralen Empfehlungsverfahren für verschiedene Projektpositionen vorgeschlagen. Über eine Fallstudie wird daraufhin evaluiert, ob die verantwortlichen Projektmanager von den empfohlenen Angestellten des bilateralen Ansatzes eine höhere Arbeitsleistung erwarten als von den Vorschlägen der unilateralen Variante. Außerdem wird überprüft, ob die Positionierung der Mitarbeiter in den Vorschlägen des bilateralen Empfehlungsverfahrens im Vergleich zum unilateralen Vorgehen stärker zu deren Zufriedenheit erfolgt.

\section{Gang der Arbeit}
\label{sec:intro:gangDerArbeit}
Zu Beginn dieser Master-Thesis werden in den Kapiteln \ref{ch:personEnvironmentFit} und \ref{ch:empfehlungssysteme} zunächst die theoretischen Grundlagen bilateraler Empfehlungssysteme erläutert. Hierzu wird in Kapitel \ref{ch:personEnvironmentFit} das psychologische Konzept des \aclp{PEFit} vorgestellt. Das folgende Kapitel \ref{ch:empfehlungssysteme} enthält einen Überblick über Ansätze zur Implementierung von Empfehlungssystemen. Dabei werden neben modell- und speicherbasierten Methoden auch hybride Verfahren beleuchtet. Darüber hinaus werden häufig auftretende Probleme bei der Entwicklung von Empfehlungssystemen geschildert und entsprechende Lösungsmöglichkeiten aufgezeigt. Im folgenden Kapitel \ref{ch:verwandteArbeiten} wird der aktuelle Stand der Forschung behandelt. In diesem Kontext werden verwandte Arbeiten diskutiert und hinsichtlich der Forschungsfrage analysiert. Anschließend wird die Methodik dieser Master-Thesis in Kapitel \ref{ch:methodik} erläutert. In diesem Zusammenhang werden die implementierte Systemarchitektur und die Gestaltung der Fallstudie dargelegt. Die bei Durchführung der Studie gewonnenen Ergebnisse werden in Kapitel \ref{ch:ergebnisse} vorgestellt. Eine Diskussion der Erkenntnisse findet sich im darauffolgenden Kapitel \ref{ch:diskussion}. Dieses endet mit der Beantwortung der Forschungsfrage und daraus abgeleiteten Empfehlungen für zukünftige Arbeiten. Ein abschließendes Fazit folgt in Kapitel \ref{ch:fazit}.
\shorthandon{"}
