\shorthandoff{"}
\chapter{Einführung}
\label{ch:intro}

\section{Motivation}
\label{sec:intro:motivation}
In Unternehmen aller Branchen ist eine Abkehr von großgewachsenen, zentralen und hierarchischen Organisationsstrukturen zu beobachten. Stattdessen setzen Betriebe zunehmend auf kleine, dezentrale und flexible Teams, welche in Projektarbeiten neue Produkte entwickeln und Dienstleistungen erbringen \cite[S. 3]{elanceEconomy:1999}. Besonders stark nimmt dieser Trend seit Mitte der 1990er Jahre zu \cite[S. 8]{whittington:1999}. Ein Hauptgrund für den Wandel ist die fortschreitende Digitalisierung. Diese ermöglicht kurze Kommunikations- und Entscheidungswege und macht damit zentrale und hierarchische Unternehmensstrukturen zunehmend überflüssig \cite[S. 5]{elanceEconomy:1999}.

Aufgrund dieser Entwicklungen wird die Zusammenstellung von Mitarbeitern für neue Projekte voraussichtlich ein immer häufiger stattfindender Prozess in der Wirtschaft sein. Unterstützung können dabei Empfehlungssysteme im Bereich der Personalauswahl bieten. In der Literatur existieren bereits einige Ansätze zur Implementierung solcher Anwendungen. \textcite[S. 1ff.]{malinowski:2006} zu Folge bieten bestehende Empfehlungssysteme jedoch häufig unzureichende Lösungen, da sich diese zumeist einseitig entweder an Personalverantwortliche oder an Stellensuchende richten. Die Wissenschaftler empfehlen stattdessen die Implementierung bilateraler Empfehlungssysteme. Dieses Konzept basiert auf der Theorie des \acp{PEFit} aus der Berufs- und Organisationspsychologie \cite[S. 2]{guan:2021}\cite[S. 3.f]{malinowski:2006}. Es besagt, dass Mitarbeiter und Personalverantwortliche jeweils eine Angebots- und eine Nachfrageperspektive besitzen. Gleichen sich Angebot und Nachfrage beider Parteien aus, führt dies aus Sicht des Unternehmens zu einer hohen Arbeitsleistung und zugleich aus Perspektive des Mitarbeiters zu einer ausgeprägten Zufriedenheit \cite[S. 6]{su:2015}.

Bisher belegte jedoch noch keine Publikation, dass die Theorie des \acp{PEFit} und die damit verbundenen Ergebnisse auf Seiten von Mitarbeitern und Personalverantwortlichen auch durch Empfehlungssysteme erzielt werden können. Somit ist nicht nachgewiesen, dass der höhere Aufwand zur Implementierung bilateraler Anwendungen gegenüber unilateralen Systemen zur Besetzung offener Projektpositionen gerechtfertigt ist. Diese Forschungslücke soll im Rahmen dieser Master-Thesis geschlossen werden.

\newpage
\section{Zielsetzung}
\label{sec:intro:zielsetzung}
Das Ziel der vorliegenden Master-Thesis besteht darin, die folgende Forschungsfrage zu beantworten:

\forschungsfrage

Um diese Frage zu beantworten, wird ein Experiment mit einer anschließenden Fallstudie durchgeführt. Dabei werden Mitarbeiter eines Beratungsunternehmens im IT-Bereich sowohl von einem uni- als auch von einem bilateralen Empfehlungsverfahren für verschiedene Projektpositionen vorgeschlagen. Anschließend wird evaluiert, ob die verantwortlichen Projektmanager von den empfohlenen Angestellten der bilateralen Anwendung eine höhere Arbeitsleistung erwarten, als von den Vorschlägen des unilateralen Systems. Außerdem wird überprüft, ob die Mitarbeiter zufriedener mit ihrer Positionierung in den Vorschlägen der bilateralen Anwendung sind.

\section{Gang der Arbeit}
\label{sec:intro:gangDerArbeit}
Um die Forschungsfrage der vorliegenden Master-Thesis zu beantworten, werden in den Kapiteln \ref{ch:personEnvironmentFit} und \ref{ch:empfehlungssysteme} zunächst die theoretischen Grundlagen bilateraler Empfehlungssysteme erläutert. Hierzu wird in Kapitel \ref{ch:personEnvironmentFit} das psychologische Konzept des \acp{PEFit} erörtert. Das folgende Kapitel \ref{ch:empfehlungssysteme} enthält einen Überblick über Implementierungsansätze von Empfehlungssystemen. Dabei werden neben modell- und speicherbasierten Methoden auch hybride Verfahren erläutert. Darüber hinaus werden häufig auftretende Probleme bei der Entwicklung von Empfehlungssystemen geschildert und entsprechende Lösungsmöglichkeiten aufgezeigt. Im folgenden Kapitel \ref{ch:verwandteArbeiten} wird der aktuelle Stand der Forschung behandelt. In diesem Kontext werden verwandte Arbeiten diskutiert und hinsichtlich der Forschungsfrage analysiert. Anschließend wird in Kapitel \ref{ch:methodik} die Methodik der Master-Thesis erläutert. In diesem Zusammenhang werden die implementierte Systemarchitektur und die Gestaltung der Fallstudie dargelegt. Die bei Durchführung der Studie gewonnenen Ergebnisse werden in Kapitel \ref{ch:ergebnisse} vorgestellt. Eine Diskussion der Erkenntnisse findet sich im darauffolgenden Kapitel \ref{ch:diskussion}. Dieses endet mit der Beantwortung der Forschungsfrage und daraus abgeleiteten Empfehlungen für zukünftige Arbeiten. Ein abschließendes Fazit folgt in Kapitel \ref{ch:fazit}.
\shorthandon{"}