\shorthandoff{"}
\chapter{Einführung}
\label{ch:intro}

\section{Motivation}
\label{sec:intro:motivation}
In Unternehmen aller Branchen ist eine Abkehr von großgewachsenen, zentralen und hierarchischen Organisationsstrukturen zu beobachten. Stattdessen setzen Betriebe zunehmend auf kleine, dezentrale und flexible Teams, welche in Projektarbeiten neue Produkte entwickeln und Dienstleistungen erbringen \cite[S. 3]{elanceEconomy:1999}. Besonders stark nimmt dieser Trend seit Mitte der 1990er Jahre zu \cite[S. 8]{whittington:1999}. Ein Hauptgrund für den Wandel ist die fortschreitende Digitalisierung. Diese ermöglicht kurze Kommunikations- und Entscheidungswege und macht dadurch zentrale und hierarchische Unternehmensstrukturen zunehmend überflüssig. Gleichzeitig ermöglicht sie höhere Kreativität und Flexibilität innerhalb der Organisation \cite[S. 5]{elanceEconomy:1999}.

Die Deutsche Gesellschaft für Projektmanagement e.V. \cite[S. 16]{gpm:2015} stellte fest, dass ein durchschnittlicher Angestellter in Deutschland im Jahr 2013 mehr als ein Drittel seiner Arbeitszeit mit Projekttätigkeiten verbrachte. Die Organisation erwartet, dass die Projektarbeit in Zukunft einen noch größeren Anteil an der Gesamtarbeitszeit in Anspruch nehmen wird. Mitarbeiter sind folglich nicht mehr über mehrere Jahre hinweg mit einer Aufgabe im Unternehmen vertraut, sondern finden sich immer wieder zu neuen, kurzlebigen Projekten zusammen.

Mit Blick auf diese Entwicklung stellten \textcite[S. 2ff.]{elanceEconomy:1999} sogar in Frage, ob Unternehmen in ihrer heutigen Form zukünftig überhaupt noch existieren werden. Sie prognostizierten die Entstehung einer "e-lance economy". Darunter verstanden sie eine Gesellschaft von elektronisch verbundenen Freiberuflern, welche sie als "e-lancer" bezeichnen. Alle Marktteilnehmer sind in dieser Wirtschaftsform rechtlich selbstständig und finden sich immer wieder zu neuen, temporären Netzwerken zusammen, um gemeinsam an Projekten zu arbeiten.

Auch wenn dieses Szenario noch in ferner Zukunft liegt, zeichnet sich ab, dass die Zusammenstellung von Teams für einzelne Projekte ein immer häufiger stattfindender Prozess in der Wirtschaft sein wird \cite[S. 2]{malinowski:2008}. Gleichzeitig zeigen repräsentative Studien, dass die Personalabteilungen deutscher Unternehmen schon heute bei der Besetzung von Stellen überlastet sind \cite[S. 244]{personalbeschaffung:2017}. So fällt es insbesondere Personalsachbearbeitern bekannter Unternehmen schwer, die hohe Anzahl an Kandidaten zu überblicken. Die HR-Abteilungen weniger bekannter Organisationen verfügen dagegen häufig über zu wenig Angestellte für eine ausreichende Prüfung der Unterlagen \cite[S. 8]{hays:2013}. Diese Überlastung der Personalabteilungen wird durch die geringe Anzahl an verfügbaren Automatisierungsangeboten in Bezug auf die Zusammenführung von Kandidaten und Stellen zusätzlich verschärft \cite[S. 15]{hays:2013}.

Derzeit existieren in der Literatur bereits einige Ansätze zur Implementierung von Empfehlungssystemen zur Besetzung offener Projektpositionen \cite{malinowski:2008}. \textcite{malinowski:2008} bewerteten diese Anwendungen jedoch als unzureichend, da sich diese entweder an Personalsachbearbeiter oder an Stellensuchende richteten. Die Forscher empfehlen stattdessen die Implementierung bilateraler Empfehlungssysteme zur Projektbesetzung. Dieses Konzept basiert auf der Theorie des \acp{PEFit}. Diese entstammt der Organisationspsychologie und besagt, dass Mitarbeiter und Personalsachbearbeiter je eine Angebots- und eine Nachfrageperspektive besitzen. Nur wenn sich Angebot und Nachfrage beider Parteien ausgleichen, führt dies aus Sicht des Unternehmens zu einer hohen Leistung des Mitarbeiters und aus Perspektive des Angestellten zu einer hohen Zufriedenheit mit der Projektposition. Um sowohl für Personalsachbearbeiter als auch Mitarbeiter eine optimale Lösung zu erzielen, ist es laut \textcite{malinowski:2008} daher notwendig, auch bei der Implementierung von Empfehlungssystemen die Präferenzen von Mitarbeitern und Personalsachbearbeitern gleichermaßen zu berücksichtigen.

Bisher hat jedoch noch keine Veröffentlichung nachgewiesen, dass die Ergebnisse des \acp{PEFit} tatsächlich auch bei der Besetzung offener Projektpositionen durch bilaterale Empfehlungssysteme erwartet werden können. Somit ist nicht belegt, ob der höhere Aufwand zur Erstellung bilateraler Empfehlungen gegenüber unilateralen Vorschlägen in der praktischen Anwendung gerechtfertigt ist. Diese Forschungslücke soll im Rahmen dieser Master-Thesis geschlossen werden.

\section{Zielsetzung}
\label{sec:intro:zielsetzung}
Das Ziel der vorliegenden Master-Thesis besteht darin, die folgende Forschungsfrage zu beantworten:

%\textit{"Prognostizieren Projektmanager bei der Besetzung offener Projektpostionen durch ein bilaterales Empfehlungssystem im Vergleich zu einer unilateralen Variante eine höhere Leistung der vorgeschlagenen Mitarbeiter, während die empfohlenen Angestellten gleichzeitig eine ausgeprägtere Zufriedenheit erwarten?"}
\textit{"Steigert die Anwendung eines bilateralen Empfehlungssystems bei der Besetzung offener Projektpositionen gleichzeitig die Zufriedenheit der Angestellten und die von den vorgeschlagenen Mitarbeitern erwartete Leistung seitens der Projektmanager?"}

Um diese Frage zu beantworten, wird ein Experiment mit einer anschließenden Fallstudie durchgeführt. Dabei werden Mitarbeiter eines Beratungsunternehmens im IT-Bereich sowohl von einem bilateralen als auch von einem unilateralen Empfehlungssystem für verschiedene offene Projektpostionen empfohlen werden. Anhand der Vorschläge soll evaluiert werden, ob Projektmanager von den empfohlenen Angestellten der bilateralen Anwendung eine höhere Arbeitsleistung erwarten als von den Empfehlungen des unilateralen Systems. Gleichzeitig soll überprüft werden, ob die Mitarbeiter von der Zuordnung zu den Projekten eine höhere Zufriedenheit erwarten.

\section{Gang der Arbeit}
\label{sec:intro:gangDerArbeit}
Folgt.

\shorthandon{"}