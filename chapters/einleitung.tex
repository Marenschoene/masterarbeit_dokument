\shorthandoff{"}
\chapter{Einleitung}
\label{ch:intro}

\begin{quotation}
    \textit{"In einer immer komplexer werdenden Welt mit überbordenden Informationsquellen sind Entscheidungshilfen von unschätzbarem Wert."} (\textcite[S. vii]{klahold:book})
\end{quotation}

Durch die explosionsartige Entwicklung von Webanwendungen in den vergangenen Jahren und der damit einhergebenden Flut an Alternativen für Nutzer solcher Anwendungen, haben entscheidungsunterstützende Systeme zunehmend an Bedeutung gewonnen.
Eines der bekanntesten und machtvollsten Werkzeuge in der Entscheidungsunterstützung stellen Empfehlungssysteme dar. % \cite[S. vii]{ricci:inbook}
Dabei handelt es sich um Systeme, die Anwendern aus einer Menge an Alternativen potenziell nützliche Elemente vorschlagen.
Die Relevanz dieser Systeme wird seit Jahren durch die Vielzahl an Publikationen, Konferenzen und Forschungen in dem Bereich belegt \cite[S. vii]{klahold:book}.

Empfehlungssysteme, in denen Personen die Inhalte von Empfehlungen bilden und der Erfolg einer Empfehlung von der bilateralen Bedürfniserfüllung abhängt, werden als wechselseitige Empfehlungssysteme bezeichnet.
Die Bedürfniserfüllung wird als bilateral bezeichnet, wenn sowohl die Bedürfnisse des Anwenders eines Systems, als auch der empfohlenen Person bei der Empfehlungserstellung berücksichtigt werden.
% Hier überleiten zu bekannten domänen, dann zu people to people empfehlung

% Ein Einsatzgebiet von Empfehlungssystemen, das zunehmend an Bedeutung gewinnt, stellt die Besetzung offener Projektpositionen dar \cite[S .iii]{link:booklet}.
% Diese Tätigkeit findet häufig in Unternehmen Anwendung, die viel projektpasiert Arbeiten und aus einer Fülle an zur Verfügung stehenden Mitarbeitern passende für diese Projekte bestimmen müssen.
% Empfehlungssystemen können hier abhilfe leisten, indem passende Mitarbeiter für offene Projektpositionen empfohlen werden.


% empfehlung u.a. von Personen, bspw. im Online-Recruiting oder -Dating -> auch als reciprocal recommender bezeichnet
% entscheidend für erfolg wechselseitiger empfehlungen: beide seiten müssen die präferenzen des jeweils anderen erfüllen (zitat: yang)

% Ein Einsatzbereich wechselseitiger systeme, der zunehmend an bedeutung gewinnt: zuordnung von mitarbeitern zu projektpositonen in projektgetriebenen Unternehmen (zitat link)
% empfehlungssystem als entscheidungsunterstützung
% link nannte bedeutung der berücksichtigumg der präferenzen der mitarbeiter
% wie müssen die kriterien berücksichtigt werden?
% (eine immer bedeustamer werdende tätigkeit -> zuordnung von mitarbeitern zu projektpositionen (zitat link))

% wie empfehlungssystem entwickeln, das auf mehreren kriterien für die empfehlungserstellung beruht?
% multi-kriterielle empfehlung

\section{Motivation}
\label{sec:einleitungmotivation}

% hier ien bisschen was schreiben zu Anwendungen von MCRS in der Praxis, siehe S. 849, file:///C:/Users/masc6/OneDrive/Persoenliche_Unterlagen/Uni/Masterthesis/2015_Book_RecommenderSystemsHandbook.pdf
% Bezug auf Titel anführen -> nicht nur Entwicklung des Algorithmus, sondern diesen robust zu entwickeln
% Hier was schreiben zur motivation des unternehmens, kurz unternehmen vorstellen (siehe methodik)
% deduktive vorgehensweise (https://www.scribbr.de/category/methodik/)

\newpage
\section{Zielsetzung}
\label{sec:einletung:zielsetzung}
überlegung, dass betrachtung als multi-kriterielles problem mögliche herangehensweisen erschließt, um system robust zu gestalten

\forschungsfrage

\section{Aufbau der Arbeit}
\label{sec:einleitung:aufbau_der_arbeit}
\shorthandon{"}