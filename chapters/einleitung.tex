\shorthandoff{"}
\chapter{Einführung}
\label{ch:intro}

\section{Motivation}
\label{sec:intro:motivation}
In Unternehmen aller Branchen beobachten Wissenschaftler eine Abkehr von großgewachsenen, zentralen und hierarchischen Organisationsstrukturen. Stattdessen setzen Betriebe zunehmend auf kleine, dezentrale und flexible Teams, welche in Projektarbeiten neue Produkte entwickeln und Dienstleistungen erbringen \cite[S. 3]{elanceEconomy:1999}. Besonders stark nimmt dieser Trend seit Mitte der 1990er Jahre zu \cite[S. 8]{whittington:1999}. Ein Hauptgrund für den Wandel ist die fortschreitende Digitalisierung. Diese ermöglicht kurze Kommunikations- und Entscheidungswege und macht dadurch zentrale und hierarchische Unternehmensstrukturen zunehmend überflüssig. Gleichzeitig ermöglicht sie höhere Kreativität und Flexibilität innerhalb der Organisation \cite[S. 5]{elanceEconomy:1999}.

Die Deutsche Gesellschaft für Projektmanagement e.V. \cite[S. 16]{gpm:2015} stellte fest, dass ein durchschnittlicher Angestellter in Deutschland im Jahr 2013 mehr als ein Drittel seiner Arbeitszeit mit Projekttätigkeiten verbrachte. Die Organisation erwartet, dass die Projektarbeit in Zukunft einen noch größeren Anteil an der Gesamtarbeitszeit in Anspruch nehmen wird. Mitarbeiter sind folglich nicht mehr über mehrere Jahre hinweg mit einer Aufgabe im Unternehmen vertraut, sondern finden sich immer wieder zu neuen, kurzlebigen Projekten zusammen.

Mit Blick auf diese Entwicklung stellten \textcite[S. 2ff.]{elanceEconomy:1999} sogar in Frage, ob Unternehmen in ihrer heutigen Form zukünftig überhaupt noch existieren werden. Sie prognostizierten die Entstehung einer "e-lance economy". Darunter verstanden sie eine Gesellschaft von elektronisch verbundenen Freiberuflern, welche sie als "e-lancer" bezeichnen. Alle Marktteilnehmer sind in dieser Wirtschaftsform rechtlich selbstständig und finden sich immer wieder zu neuen, temporären Netzwerken zusammen, um gemeinsam an Projekten zu arbeiten.

Auch wenn dieses Szenario noch in ferner Zukunft liegt, zeichnet sich ab, dass die Zusammenstellung von Teams für einzelne Projekte ein immer häufiger stattfindender Prozess in der Wirtschaft sein wird \cite[S. 2]{malinowski:2008}. Gleichzeitig zeigen repräsentative Studien, dass die Personalabteilungen deutscher Unternehmen schon heute bei der Besetzung von Stellen überlastet sind \cite[S. 244]{personalbeschaffung:2017}. So fällt es insbesondere Personalsachbearbeitern bekannter Unternehmen schwer, die hohe Anzahl an Kandidaten zu überblicken. Die HR-Abteilungen weniger bekannter Organisationen verfügen dagegen häufig über zu wenig Angestellte für eine ausreichende Prüfung der Unterlagen \cite[S. 8]{hays:2013}. Diese Überlastung der Personalabteilungen wird durch fehlende Automatisierungsangebote in Bezug auf die Zusammenführung von Kandidaten und Stellen zusätzlich verschärft \cite[S. 15]{hays:2013}.

\section{Zielsetzung}
\label{sec:intro:zielsetzung}
% Empfehlungssystem trifft keine finale Entscheidung
% Erwartung (Erwarten wir bessere Performance)
Forschungsfrage: "Führt der Einsatz eines bilateralen Empfehlungssystems im Vergleich zu einem unilateralen System bei der Besetzung offener Projektpositionen zu höherer Performance auf Seiten des Unternehmens und gleichzeitig zu höherer Zufriedenheit seitens des Mitarbeiters?"

\section{Gang der Arbeit}
\label{sec:intro:gangDerArbeit}

\shorthandon{"}