\shorthandoff{"}
\chapter{Einführung}
\label{ch:intro}

\section{Motivation}
\label{sec:intro:motivation}
In Unternehmen aller Branchen beobachten Wissenschaftler eine Abkehr von großgewachsenen, zentralen und hierarchischen Organisationsstrukturen. Stattdessen setzen Betriebe zunehmend auf kleine, dezentrale und flexible Teams, welche in Projektarbeiten neue Produkte entwickeln und Dienstleistungen erbringen \cite[S. 3]{elanceEconomy:1999}. Besonders stark nimmt dieser Trend seit Mitte der 1990er Jahre zu \cite[S. 8]{whittington:1999}. Ein Hauptgrund für den Wandel ist die fortschreitende Digitalisierung. Diese ermöglicht kurze Kommunikations- und Entscheidungswege und macht dadurch zentrale und hierarchische Unternehmensstrukturen zunehmend überflüssig. Gleichzeitig ermöglicht sie höhere Kreativität und Flexibilität innerhalb der Organisation \cite[S. 5]{elanceEconomy:1999}.

Die Deutsche Gesellschaft für Projektmanagement e.V. \cite[S. 16]{gpm:2015} stellte fest, dass ein durchschnittlicher Angestellter in Deutschland im Jahr 2013 mehr als ein Drittel seiner Arbeitszeit mit Projekttätigkeiten verbrachte. Die Organisation erwartete, dass die Projektarbeit in Zukunft einen noch größeren Anteil an der Gesamtarbeitszeit in Anspruch nehmen wird. Mitarbeiter sind folglich nicht mehr über mehrere Jahre hinweg mit einer Aufgabe im Unternehmen vertraut, sondern finden sich immer wieder zu neuen, kurzlebigen Projekten zusammen.

Mit Blick auf diese Entwicklung stellten \textcite[S. 2ff.]{elanceEconomy:1999} sogar in Frage, ob Unternehmen in ihrer heutigen Form zukünftig überhaupt noch existieren werden. Sie prognostizierten die Entstehung einer "e-lance economy". Darunter verstanden sie eine Gesellschaft von elektronisch verbundenen Freiberuflern, welche sie als "e-lancer" bezeichnen. Alle Marktteilnehmer sind in dieser Wirtschaftsform rechtlich selbstständig und finden sich immer wieder zu neuen, temporären Netzwerken zusammen, um gemeinsam an Projekten zu arbeiten.

Auch wenn dieses Szenario noch in ferner Zukunft liegt, zeichnet sich ab, dass die Zusammenstellung von Teams für einzelne Projekte ein immer häufiger stattfindender Prozess in der Wirtschaft sein wird \cite[S. 2]{malinowski:2008}. Gleichzeitig zeigen repräsentative Studien, dass die Personalabteilungen deutscher Unternehmen schon heute bei der Besetzung von Stellen überlastet sind \cite[S. 244]{personalbeschaffung:2017}. So fällt es insbesondere Personalsachbearbeitern bekannter Unternehmen schwer, die hohe Anzahl an Kandidaten zu überblicken. Die HR-Abteilungen weniger bekannter Organisationen verfügen dagegen häufig über zu wenig Angestellte für eine ausreichende Prüfung der Unterlagen \cite[S. 8]{hays:2013}. Diese Überlastung der Personalabteilungen wird durch die geringe Anzahl an Automatisierungsangeboten in Bezug auf die Zusammenführung von Kandidaten und Stellen zusätzlich verschärft \cite[S. 15]{hays:2013}.

\textcite{malinowski:2008} beobachteten, dass in der Literatur zwar Ansätze zur Implementierung von Empfehlungssystemen zur Besetzung offener Projektpositionen existieren. Die Wissenschaftler bewerteten diese Anwendungen jedoch als unzureichend, da sich diese entweder an Personalsachbearbeiter oder Stellensuchende richteten. \textcite{malinowski:2008} orientierten sich dagegen an der Theorie des \acp{PEFit}. Dieses Konzept entstammt der Organisationspsychologie und geht davon aus, dass Mitarbeiter und Personalsachbearbeiter je eine Angebots- und eine Nachfrageperspektive haben. Wenn sich Angebot und Nachfrage beider Parteien ausgleichen, soll dies aus Sicht des Unternehmens zu einer hohen Leistung des Mitarbeiters und aus Perspektive des Angestellten zu einer hohen Zufriedenheit führen. Um sowohl für Personalsachbearbeiter als auch Mitarbeiter eine optimale Lösung zu erzielen, schlugen \textcite{malinowski:2008} daher die Implementierung bilateraler Empfehlungssysteme zur Projektbesetzung vor. Diese Systeme orientieren sich am \ac{PEFit} und betrachten bei der Besetzung offener Projektpositionen gleichzeitig die Präferenzen von Personalsachbearbeitern und Mitarbeitern.

In der Psychologie belegten bereits einige Publikationen, dass ein guter \ac{PEFit} bei der Stellenbesetzung zu einer hohen Leistung für das Unternehmen und gleichzeitig zu einer hohen Zufriedenheit beim Mitarbeiter führen kann. Bisher wies jedoch noch keine Veröffentlichung nach, dass dieselben Ergebnisse auch durch bilaterale Empfehlungssysteme bei der Besetzung offener Projektpositionen erzielt werden können. Somit bleibt ungewiss, ob der höhere Aufwand zur Erstellung bilateraler Empfehlungen gegenüber unilateralen Vorschlägen in der praktischen Anwendung gerechtfertigt ist. Im Rahmen dieser Master-Thesis soll diese Forschungslücke geschlossen werden.

\section{Zielsetzung}
\label{sec:intro:zielsetzung}
Das Ziel der vorliegenden Master-Thesis ist es, die Erwartungen von Projektmanagern und Mitarbeitern bezüglich der Vorschläge eines bilateralen und eines unilateralen Empfehlungssystems bei der Besetzung offener Projektpositionen zu vergleichen. Zu diesem Zweck sollen Mitarbeiter eines Unternehmens sowohl von einem bilateralen als auch von einem unilateralen Empfehlungssystem für verschiedene offene Projektpostionen empfohlen werden. Anhand der Vorschläge soll evaluiert werden, ob Projektmanager von den vorgeschlagenen Angestellten der bilateralen Anwendung eine höhere Arbeitsleistung erwarten als von den Empfehlungen des unilateralen Systems. Gleichzeitig soll überprüft werden, ob die Mitarbeiter von der Zuordnung zu den Projekten, für die sie vom bilateralen Empfehlungssystem empfohlen wurden, eine höhere Zufriedenheit erwarten.

In diesem Kontext soll die folgende Forschungsfrage beantwortet werden: "Erwarten Projektmanager von den vorgeschlagenen Mitarbeitern für offene Projektpostionen eines bilateralen Empfehlungssystems im Vergleich zu einem unilateralen System eine höhere Leistung im Projekt, während die Angestellten gleichzeitig eine höhere Zufriedenheit von den Empfehlungen erwarten?"

\section{Gang der Arbeit}
\label{sec:intro:gangDerArbeit}

\shorthandon{"}