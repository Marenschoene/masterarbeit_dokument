\shorthandoff{"}
\chapter{Notizen}
\label{ch:notizen}

\section{Person-Environment Fit}
\label{ch:notizen:personEnvironmentFit}
- \cite[S. 1]{edwards:1996}: P-E fit ist ein weit verbreitet in der Forschung zu organisationalem Verhalten / P-E fit verkörpert die Prämisse, dass Outcomes nicht separat von P un dE entstehen, sondern eher von der Beziehung zwischen beiden (Quellen)\\
- \cite[S. 3]{edwards:1996}: Values (bei SV) repräsentieren bewusste Desires, die von einer Person gehalten werden (Quellen) und umfasst somit Präferenzen, Interessen, Motive und Ziele (Quellen) / Supplies beziehen sich auf die Menge, Frequenz und Qualität der Attribute der Umgebung, die die Values der Person erfüllen können (Quelle) / Supplies können entweder objektiv oder subjektiv vermittelt werden (Quelle), nur subjektive Abweichungen von S zu V beeinflussen den Strain (Quellen) --> Deshalb ist der Kernprozess des S-V fits einen kognitiven Vergleich zwischen wahrgenommener und gewünschter Menge, Frequenz oder Qualität der Bedingungen oder Ereignisse wahrgenommen von der Person vorzunehmen \\
- \cite[S. 3]{edwards:1996}: Die meisten Theorien besagen, dass Strain zunimmt wenn S gegenüber V zu kurz kommt (Quellen), sind aber zweideutig bei einem Überschuss an S --> Harrison 1978 identifiziert 4 unabhängige Prozesse --> Erste zwei Prozesse geben an, dass ein Überschuss an S den Strain weiter reduziert. Erster wird als "Conservation" bezeichnet und tritt auf, wenn der Überschuss einbehalten werden kann, um den Fokuswert zu einem späteren Zeitpunkt zu befriedigen, z.B. angesammelte Urlaubstage oder überschüssiges Gehalt; 2. "Carryover": Überschuss von einem S hilft andere V zu erfüllen --> Tritt für V auf, die instrumentell verwandt sind, z.B. wenn ein Überschuss an Autonomie die Person dazu befähigt, gewünschte Veränderungen an der Arbeit vorzunehmen; Sowohl Conservation als auch Carryover zeigen eine monotone Beziehung zwischen S-V fit und Strain --> Strain nimmt weiter ab, wenn S V übersteigen / Die verbleibenden beiden Prozesse geben an, dass ein Überschuss an S den Strain erhöht: Erstes wird "depletion" genannt und tritt auf, wenn ein Überschuss an S die zukünftige Erfüllung von V auf der Fokusdimension verhindert --> z.B. Zu viel Unterstützung der Führungskraft zu einem bestimmten Zeitpunkt kann seine Unterstützung zu einem späteren Zeitpunkt ausschließen; 2. ist "Interference" --> \cite[S. 4]{edwards:1996}: Überschuss an S auf einer Dimension verhindert V-Erfüllung auf anderen Dimensionen --> z.B. Überschuss an Kontakt mit Mitarbeitern hindert das Verlangen nach Privatsphäre (Harrison 1978) --> Beide Prozesse zeigen eine gekrümmte Beziehung zw. Strain , sodass ein perfekter S-V fit optimal ist (und zu minimalem Strain führt) \\
- \cite[S. 5]{edwards:1996}: DA-fit bezieht sich auf das Match zwischen Anforderungen von E und Fähigkeiten (Abilities) von P / Abilities enthalten Skills, Wissen, Zeit und Energie, welche P heranziehen kann, um die Demands von E zu erfüllen / Demands bezieehn sich auf quantitative und qualitative Anforderungen an die Person un dkönnen objektiv (z.B. Fließbandgeschwindigkeit, Länge des Arbeitstages) oder sozial konstruiert (z.B. Gruppennormen, Rollenerwartungen) sein --> In beiden Fällen können nur die Demands, die P wahrnimmt Stress auslösen (Quellen); \cite[S. 5f.]{edwards:1996}: Kernprozess der dem DA fit unterliegt ist der kognitive Vergleich zw. wahrgenommenen Anforderungen und den A der Person, um diese Anforderungen zu erfüllen \\
- \cite[S. 6]{edwards:1996}: Die meisten Theorien besagen, dass Strain zunimmt wenn wahrgenommen wird, dass Demands zu hoch sind; Unterschiedliche Theorien was passiert, wenn Abilities zu hoch sind --> Ein Ansatz ist die Ansätze von Carryover, Conservation, etc. auch hier anzuwenden \\
- \cite[S. 9]{edwards:1996}: Es sind auch Sonderfälle möglich, z.B. wenn ein Überschuss an Demands den Value einer P erfüllt, sich selbst weiterzuentwickeln, dann kann der Strain trotzdem sinken \\
- \cite[S. 9]{edwards:1996}: Laut Caplan 1987 wurden verschiedene Variablen als potentielle Moderatoren des Effektes von SV fit und DA fit auf Strain --> Eine gemeinsame Variable bei SV und DA ist Wichtigkeit --> \cite[S. 10]{edwards:1996}: Es wird vermutet, dass Wichitigkeit die Effekte von SV und DA fit auf Strain intensiviert, sodass ein Misfit auf wichtigeren Dimensionen zu größerem Strain als ein Misfit auf weniger wichtigen Dimensionen führt (Quellen)\\
- \cite[S. 10]{edwards:1996}: Obwohl die Wichtigkeit die Beziehung zwischen SV und DA zu Strain moderieren, unterschiedet sich der unterliegende psychologische Prozess zwischen SV und DA fit --> Bei SV fit zeigt der moderierende Effekt der Wichtigkeit die Prämisse, dass ein Missfit ist schädlicher für die stark gehaltenen Values (Quellen); Beim DA-fit basiert die wichtigkeits-Moderation auf dem Grad zu welchem der Misfit zu wichtigen Konsequenzen führt, d.h. solche welche beinhalten substantielle Rewards oder Kosten für die Person (Quellen) --> P evaluiert, ob die Konsequenzen beehrenswert oder nicht begehrenswert sind --> Ähnlich dem Bewertungsprozess bei SV; Der DA misfit wird als wichtig angesehen, wenn er zu einem SV misfit auf anderen Dimensionen führt --> Diese Korrespondenz zw. SV und DA zeigt, dass ein DA misfit einen SV fit nicht nur verhindern kann wenn Abilities Demands übersteigen, sondern auch wenn Demands Abilities übersteigen \\
- \cite[S. 2]{edwards:1993}: Die P-E fit Theorie stellt drei hypothetische Beziehungen zwischen fit und strain vor --> Sind verkörpert in den five fit Messungen genutzt von (French) und (Caplan) / 1. Einfachste Messung genannt "fit" besteht aus der algebraischen Differenz zwischen E und P (E-P) --> Zeigt eine monotone Beziehung zum Strain --> Diese Beziehung wird erwartet, wenn z.B. Strain nicht nur abnimmt wenn S zu Motiven zunimmt sondern auch weiter abnimmt, wenn S übersteigt und kann auf andere Motive angewendet werden oder für spätere Verwendung zurückgehalten werden; Zwei Messungen, genannt "deficiency" (E-P für E <= P, 0 für E>P) und "excess" (E-P für E>=P, 0 für E<P) zur Darstellung von asymptotischen Beziehungen (Anmerkung für mich: Asymptotische Beziehung gab es auch bei \cite{edwards:1996}) zu Strain; Defiency repräsentiert eine negative Beziehung zu Strain nur wenn E ist kleiner als P --> Zunehmende S reduzieren den Strain bis zu einem Punkt an Zufriedenheit und haben nur noch wenig Effekt danach; Im Gegensatz dazu excess drückt eine positive Beziehung zum strain aus, nur wenn E größer ist als P --> Demands erhöhen Strain, wenn sie Abilities übersteigen, aber nciht wenn sie kleiner sind als Abilities / Zwei Messungen: "poor fit" (|E-P|) und die "squared difference" ((E-P)\^2) (hier genannt "fit squared") zeigen eine gekrümmte (curvilinear) Beziehung zum Strain --> Diese Beziehungen werden erwartet, wenn entweder übersteigen oder inadäquate S oder D sind schädlich, z.B. wenn zu viel Job-Komplexität zu einer Überlastung aber zu wenig zu Langeweile führt / Anm: Bilder zu allen Beziehungen vorhanden \\
- \cite[S. 2]{edwards:1993}: Es gibt viele Probleme an den 5 Messungen von Caplan und French (Quellen) --> Liegt an der Verwendung von Differenz-Berechnungen --> Es folgt Kritik dazu --> Probelme mit den fit Messungen können durch andere von Edwards beschriebenen Prozeduren überwunden werden --> Edwards führt in diesem Paper die Forschungen von French et al 1982 erneut aus und wendet seine empfohlenen Berechnungen darauf an --> \cite[S. 19]{edwards:1993}: Ergebnisse: 1. fit Messungen, die E und P in einen Wert zusammenrechnen sollten zugunsten von Polyomialgleichungen, welche E und P und geeignete Terme höherer Ordnung enthalten, aufgegeben werden (Quellen) / Edwards Studie demonstrierte, dass fit Messungen zu mehrdeutigen Ergebnissen führen, die separaten Beziehungen von E und P mit strain durcheinander bringen, einen sehr restriktiven Satz von Beschränkungen auferlegen, die selten unterstützt werden und reduzieren die inhärente dreidimensionale Beziehung zwischen E, P und strain auf zwei Dimensionen / \cite[S. 20]{edwards:1993}: Edwards (Quelle) Prozedur vermied diese Probleme und zeigte in den meisten Fällen, dass die Beziehung zwischen E, P und strain konnte nicht adäquat von den fit Messungen von French et al 1982 repräsentiert werden --> Grund: Diese Prozedur nutze Gleichungen, welche Fit-Messungen subsummieren, dies eliminierte die Notwendigkeit für diese Messungen und erlaubte darüber hinaus die Mesung von einer breiteren Rand von Oberflächen bezogen auf E und P zu strain --> Weitere Punkte (nachlesen) \\
- \cite[S. 5]{edwards:1993}: French et al 1982 haben Supplies und Preferences durch Itempaare gemessen --> Beispiel für Itempaar: "Wie viel Arbeitslast hast du?" und "Wie viel Arbeitslast würdest du gerne haben?" --> Antwortskalen reichten von "sehr wenig (1)" bis "sehr viel (5)" \\
- \cite[S. 1]{edwards:2004}: Laut P-E fit Paradigma, entstehen Einstellungen und Verhalten aus der Kongruenz zwischen den Attributen von Person und Umgebung (Quellen) --> Persönliche Charakteristiken enthalten individuelle biologische oder psychologische Needs, Werte, Ziele, Fähigkeiten oder Persönlichkeiten; Charakteristiken der Umgebung beziehen sich auf intrinsische oder extrinsische Belohnungen, physikalische oder physiologische Demands, kulturelle Werte oder Bedingungen der Umgebung wie Hitze, Verfügbarkeit von Essen, etc. / In der Organisationsumgebung haben Forschungen das P-E fit Paradigma verwendet, um Outcomes vorherzusagen, die alle Phasen des Arbeitslebens umfassen (Dazu einige Beispiele mit Quellen) \\
- \cite[S. 1]{edwards:2004}: Laut \textcite{muchinsky:1987} gibt es zwei langjährige Traditionen in der PE fit-Forschung: Complementary fit und supplementary fit --> Complementary fit tritt auf, wenn P oder E Charakteristiken anbieten, die der andere benötigt --> Zitat \cite{muchinsky:1987}: "Die Schwächen oder Needs der Umgebung weren von den Stärken des Individuums ausgeglichen und umgekehrt"; Complementary fit kann bedeuten, dass ein Mitarbeiter ein Skillset besitzt, das die Organisation benötigt oder dass eine Organisation die Belohnungen bietet, die ein Individuum will / Supplementary fit existiert, wenn eine Person und eine Organisation ähnliche oder übereinstimmende Charakteristiken besitzen --> Typischerweise bei der Untersuchung von Werte-Kongruenz zwischen Mitarbeiter und Organisation (z.B. bei Kristof 1996) verwendet \\
- \cite[S. 1]{edwards:2004}: Supplementary und Complementary fit haben sich unabhängig voneinander entwickelt \\
- \cite[S. 2]{edwards:2004}: Psychologische Needs werden mit Supplies von E vergleichen --> Diese beziehen sich auf extrinsische und intrinsische Ressourcen und Belohnungen (z.B. Geld, soziale Involviertheit, etc.) \\
- \cite[S. 2]{edwards:2004}: Need Fullfillment Literatur (Quellen) fokussiert (hier) die psychologischen Needs, welche sich eher auf die psychologischen Needs, die durch Lernen und Sozialisation erworben werden statt auf biologische Bedürfnisse (z.B. Essen) / Theorien der psychological Need Fullfillment --> Besagen, dass Menschen unzufrieden werden, wenn die Angebote der Umgebung kleiner sind als das was die Person verlangt (desires) --> Zufriedenheit nimmt zu, wenn Supplies sich zu Desires vergrößern --> Was passiert, wenn Supplies die Desires übersteigen, kommt auf die jeweiligen Needs an --> Kann unterschiedliche funktionale Formen annehmen (Quellen)\\
- \cite[S. 2]{edwards:2004}: Innerhalb der supplementary Tradition Forschung ist die Wertekongruenz am prominentesten insbesondere im Feld des Organisationalen Verhaltens (Quellen) / Individuelle Werte ist das was sie glauben was wichtig ist --> Dies steuert ihre Entscheidungen und Verhalten; Vergleichsweise bieten organisationale Werte Normen, die spezifizieren wie die Ressourcen der Organisation zugewiesen werden sollten und wie die Mitglieder einer Organisation sich verhalten sollten / Value congruence bezieht sich auf die Ähnlichkeit zwischen den Werten des Individuums und dem kulturellen Wertesystem einer Organisation (Quellen) / Menschen fühlen sich angezogener und vertrauen anderen, die ihnen ähnlich sind (Quellen); Außerdem finden es Angestellte komfortabler in einer Organisation zu arbeiten, in der Dinge, die dem Individuum wichtig sind, auch anderen Angestellten wichtig sind; Führt zu besseren persönliche Beziehungen (Quellen); Werte einer Organisation; Werte-Inkongruenz führt zu kognitiver Dissonanz und Unzufriedenheit (Quelle) \\
- \cite[S. 2]{edwards:2004}: Es gibt einen Unterschied zwischen psychologischer Need Erfüllung und Wertekongruenz: Forschung bei der Need Erfüllung beschreibt Needs als gewünschte Menge eines Attributes (z.B. wie viel Autonomie ein Mitarbeiter will); Im Gegensatz dazu konzeptualisiert die Forschung der Wertekongruenz Values als Wichtigkeit eines Attributes (z.B. Wie wichtig Autonomie dem Individuum ist) --> Es wird also nciht die Content-Dimension von Needs und Values unterschieden, sondern die konzeptuelle Dimension entlang derer Needs und Values variieren (z.B. Menge vs. Wichtigkeit) --> z.B. Job-Zufriedenheit definiert das anderes, Edwards definiert es aber so \\
- \cite[S. 3]{edwards:2004}: Menschen akzeptieren und behalten Jobs hauptsächlich basierend auf den gebotenen Rewards für ihre Zeit-Investitionen und Talent (Quellen) \\
- \cite[S. 3]{edwards:2004}: Value-Kongruenz und Need-Fullfillment sind nicht unabhängig voneinander: Die Values einer Organisation beeinflussen die Typen von Belohnungen, die die Organisation supplied (Quelle) und die Werte einer Person beeinflussen ihre Desires (Quelle) \\
- \cite[S. 3]{edwards:2004}: Bei einem bestimmten Unternehmen anzufangen, ist ein konkreter, offener Ausdruck der Werte einer Person (Quellen) --> Das was der Organisation wichtig ist, zu der die Person gehört, sendet Signale an die Gesellschaft über das Selbst der Person und gibt Implikationen für die Selbst-Definition (Quellen) --> Wenn die Werte der Person inkongruent mit den Werten der Organisation sind, wird die Person kognitive Dissonanz und negatives Job-Verhalten zeigen; Außerdem ist Kommunikation und Freundschaft mit anderen Angestellten schwieriger, wenn sie keine gemeinsamen Values halten (Quelle)\\
- \cite[S. 3]{edwards:2004}: Value Kongruenz und Need-Erfüllung sind nicht unabhängig voneinander --> Die Values der Organisation bestimmen die Arten von Rewards, die die Organisation bietet (Quelle) und die Values einer Person bestimmen seine oder ihre Desires (Quelle) \\
- \cite[S. 4]{edwards:2004}: Kristof 1996, S. 6 stellt fest: "Der optimale P-O fit ist erreicht, wenn jeder Need einer Entität ist erfüllt durch einen anderen und sie teilen dieselben grundlegenden Charakteristiken" \\
- \cite[S. 1]{edwards:2008}: Person-environment fit ist ein zentrales Konzept in der Forschung des organisationalen Verhaltens / \cite[S. 2]{edwards:2008}: P-E fit hatte für Jahrzehnte eine zentrale Person in der Forschung zu organisationalem Verhalten --> P-E fit bezieht sich auf die Kongruenz, Übereinstimmung (Match) oder Ähnlichkeit zwischen Person und Umwelt (siehe dazu: \cite{edwards:1998}, \cite{muchinsky:1987})\\
- \cite[S. 2]{edwards:2008}: Spezifische Typen des P-E fit beziehen sich auf die Bedürfnisse der Person und die Belohnungen des Umfeldes (Quellen), die Fähigkeiten einer Person und die Anforderungen der Umgebung (Quellen) und die Ähnlichkeit zwischen Person und sozialem Umfeld, welches sich auf Individuen, Gruppen, Organisationen oder Berufe beziehen kann (Quellen) \\
- \cite[S. 2]{edwards:2008}: Das Konzept eines P-E fits reicht zurück bis zu Plato (\cite{dumont:1995}), zeitgemäße P-E fit-Forschung führt oft auf \textcite{parsons:1909} zurück, der ein matching Modell entwickelt hat, um den Fit zwischen Attributen einer Person und Charakteristiken verschiedener Berufe zu beschreiben \\
- \cite[S. 2]{edwards:2008}: Den Grundstein für die P-E fit Forschung haben insbesondere Murrays (zwei Quellen) Need-Press-Modell und Lewins Feldtheorie (Zwei Quellen) gelegt --> Ausgehend von dieser Arbeit (meint wahrscheinlich Lewin) ist der P-E fit als ein Kernkonzept in Jobzufriedenheit, Job Stress, Berufswahl, Rekrutierung und Auswahl und Organisationskultur und Klima hervorgegangen (zu allem Quellen) --> Diese verschiedenen Strömungen haben hunderte Studien generiert, die in Zusammenfassungen (Quellen) und Meta-Analysen (Quellen) gereviewed wurden \\
- \cite[S. 4]{edwards:2008}: Der P-E fit bezieht sich auf eine Kongruenz, match oder Ähnlichkeit zwischen Person und Umgebung --> Diese generelle Definition wurde von \textcite{muchinsky:1987} in supplementary und complementary fit unterschieden / Laut \cite[S. 269]{muchinsky:1987} entsteht ein supplementary fit, wenn eine Person ähnlich zu anderen Individuen im Umfeld ist / Ein Complementary fit entsteht laut \cite[S. 271]{muchinsky:1987} wenn ein Individuum mit seinen Stärken Schwächen oder Bedürfnisse des Umfeldes ausgleicht und umgekehrt --> Der complementary fit wurde später nochmal unterschieden, ob die Bedürfnisse auf Seiten der Person oder des Umfeldes auftreten (Quellen) \\
- \cite[S. 4]{edwards:2008}: Der Grad, zu welchem die Bedürfnisse der Person durch intrinsische und extrinsische Belohnungen des Umfeldes belohnt werden, wird als "needs-supplies fit" bezeichnet (Quellen) / Der Grad zu welchem die Bedürfnisse des Umfeldes durch die Fähigkeiten der Person erfüllt werden, bezeichnet man als "Demands-Abilities fit" (Quellen) --> Anmerkung von mir: Es gibt also entweder einen NS-Fit einen DA-Fit oder einen supplementary fit \\
- \cite[S. 5]{edwards:2008}: Es gibt noch die UNterschiedung, b Person und Umgebung objektiv, subjektiv oder beides gleichzeitig konzeptualisiert werden --> Alle diese Unterscheidungen sind wichtig für die Bedeutung und die Auswirkungen des P-E fits (Quellen) \\
- \cite[S. 5]{edwards:2008}: Dann können noch unterschiedliche Einheiten zur Konzeption von P und E verwendet werden --> Needs können z.B. über Einheiten oder Wichtigkeiten ausgedrückt werden --> Diese Unterscheidung hat wichtige Implikationen für die Theorien des N-S fits (Quellen) \\
- \cite[S. 6]{edwards:2008}: Manche Studien identifizieren Grenzen, die Bedingungen etablieren, unter denen P-E fit-Beziehungen auftreten sollten --> Dies Bedingungen werden als Moderatoren bezeichnet, die die Form oder Stärke der P-E fit Beziehung beeifnlussen --> Beispiel: Der Effekt von D-A wird stärker, wenn das Nichterfüllen der Anforderungen wichtige Konsequenzen hat (Quellen) / Grenz-Bedingungen können sich aber auch auf Limitierungen beziehen, unter denen sich eine Theorie nicht anwenden lässt, z.B. wen die Outcomes des fits auf organisationalem Level beschränkt sind (Quellen) \\
- \cite[S. 1]{edwards:1990}: P-E fit charakterisiert Stress als eine Diskrepanz zwischen korrespondierenden Charakteristiken einer Person und der Umgebung --> Diese Diskrepanz steht im Verdacht, schädliche psychologische, physiologische und verhaltensbedingte Outcomes zu erzeugen --> Diese könnten sogar in erhöhter Krankheit und Sterblichkeit münden \\
- \cite[S. 1]{edwards:1990}: Das Framework bildet den Kern vieler aktueller Theorien des organisationalen Stresses  --> Solche Theorien wurden z.B. von \textcite{copingAndAdaption:1974}, \textcite{mechanismsOfJobStressAndStrain:1982} (Noch mehr Autoren) \\
- \cite[S. 1]{edwards:1990}: P-E fit ist ein generelles Framework mit einer langen Tradition in der Psychologie --> Wurzeln reichen zurück bis auf einflussreiche Autoren wie Lewin (1938, 1951) und Murray (1938) \\
- \cite[S. 2]{edwards:1990}: Sinn seines Papers: Viele Studien, die P-E fit bezüglich Stress untersuchen, sind mit ernsten theoretischen und methodischen Problemen geplagt, welche die Aussagekraft der Ergebnisse schmälern \\
- \cite[S. 2]{edwards:1990}: P-E fit ist laut Eulberg et al. 1988 der meist zitierte Modell im Bereich des organisationalen Stresses --> Dieses Paper (von Edwards) konnte keine einzige Studie finden, die frei von Fallstricken war \\
- \cite[S. 4]{edwards:2017}: Die P-E fit Theorie des Stresses nach \cite{mechanismsOfJobStressAndStrain:1982} unterscheidet objektive P und E Faktoren von deren subjektiven Gegenstücken und unterstreichen den fit zwischen subjektiver P und E als Schlüssel-Determinante von psychologischen, physiologischen und verhalten Strains zusammen mit Vermeidungs- und Verteidigungsmechanismen / Ähnlich dazu stellen auch \cite{cummings:1979} und Edwards 1992 Stress als eine Diskrepanz zwischen Vorstellungen und wichtigen Wünschen (Desires) fest --> Diese führt zu psychologischen und physiologischen Symptomen und Bemühungen, die wahrgenommenen Diskrepanzen zu lösen / Auch \cite{schuler:1980} sagt, dass Stress existiert, wenn eine Person mit Anforderungen (Demand) konfrontiert ist, die ihre Wünsche beschneidet\\
- \cite[S. 2]{edwards:1990}: Umfassendste Behandlung des P-E fit Ansatzes (bzgl. Stress) wurde von \textcite{mechanismsOfJobStressAndStrain:1982} durchgeführt --> Behandlung enthält zwei verschiedene Versionen des P-E fits --> Eine Version fokussiert die Korrespondenz zwischen Angeboten der Umgebung (Supplies) und persönlichen Werten, Motiven, Zielen (Values) --> S-V fit / Die andere Version fokussierte die Korrespondenz zwischen Anforderungen (Demands) der Umgebung und persönlichen Fähigkeiten und Fertigkeiten (Abilities) --> D-A fit --> \textcite{mechanismsOfJobStressAndStrain:1982} stellen fest, dass sowohl P als auch E subjektiv und objektiv beschrieben werden können --> Objektive P und E beziehen sich auf die Variablen, welche unabhängig von der Wahrnehmung des Individuums existieren --> Subjektive P und E beziehen sich dagegen auf Variablen wie sie vom Individuum wahrgenommen werden --> Zentrale These von \textcite{mechanismsOfJobStressAndStrain:1982}: Gibt es einen Misfit bei subjektiven S-V oder D-A, dann entstehen daraus negative psychologische, physiologische und verhaltens-Outcomes, die kollektiv als "strain" bezeichnet werden \\
- \cite[S. 2]{edwards:1990}: \textcite{mechanismsOfJobStressAndStrain:1982} beziehen sich explizit auf den P-E fit, aber es gibt auch viele andere Studien, die den P-E fit implizit behandeln --> S-V fit ist implizit in \textcite{schuler:1980}s Konzeptualisierung von Stress --> Diese enthält eine dynamische Bedingung, welche das Individuum potentiell davon abhält das zu sein, haben oder tun was sie oder er will (Desires) / Auch das kybernetische Framework von \textcite{cummings:1979} gibt an, dass eine Diskrepanz zwischen dem bevorzugten Status des Individuums und dem aktuellen Status in strain resultiert \\
- \cite[S. 2f.]{edwards:1990}: Der D-A fit erscheint in McGraths Stressmodell (Quelle nicht gefunden) --> Besagt, dass Stress durch Anforderungen des Umfeldes entstehen, welche die Fähigkeiten und Ressourcen einer Person übersteigen \\
- \cite[S. 3]{edwards:1990}: \textcite{karasek:1979} sagt, dass strain auftritt, wenn hohe Anforderungen mit geringen Fähigkeiten zur Beeinflussung der Aufgaben kombiniert werden (z.B. geringe Entscheidungsfreiheit) \\
- \cite[S. 3]{edwards:1990}: Transaktionales Stressmodell von Lazarus und seinen Kollegen (u.a. \textcite{lazarus:1978}) ist bemerkenswert, da es sowohl Elemente von D-A als auch S-V enthält --> Beim D-A fit gehen die Forscher davon aus, dass Stress in einer Beziehung zwischen Mensch und Umgebung entsteht, bei der die Anforderungen der Umgebung von der Person als dessen Ressourcen belastend oder übersteigend eingeschätzt werden oder dessen Wohlbefinden gefährdet / Die Werte-Komponente des S-V fit erscheint im Konzept der Commitments, welches ein Muster von Zielen, Motiven und Werten darstellt, welche von der Person gehalten werden / Laut Transaktionalem Modell werden Situationen in denen Anforderungen Ressourcen beschneiden oder übersteigen (D-A misfit) nur als stressig wahrgenommen, wenn die Anforderungen die Commitments einer Person verstärken oder erhalten (nach \textcite{harrison:1978} 1978 wird der S-V misfit gelöst oder abwendet) / Transaktionales Modell integriert D-A und S-V fit, gibt an, dass stressbezogene Transaktionen zwischen Person und Umgebung sowohl D-A als auch S-V misfit beinhalten \\
- \cite[S. 3]{edwards:1990}: Konzept des P-E fit gibt es auch in anderen Bereichen der Forschung zu organisationalem Verhalten --> z.B. bei \textcite{locke:1969} --> Job Zufriedenheit kommt von der Wahrnehmung, dass ein Job wichtige Job-Werte erfüllt \\
- \cite[S. 3]{edwards:1990}: Auch die Work Adjustment Theory von \textcite{workAdjustment:1964} zeigt, dass Zufriedenheit durch eine Korrespondenz zwischen jemandes Werten und verfügbaren Verstärkungsmustern auf der Arbeit entstehen \\
- \cite[S. 3]{edwards:1990}: Der S-V fit kommt auch in der Job Charakteristiken Theorie nach \textcite{hackmanOldham:1987} vor --> Besagt, dass Motivation und Zufriedenheit entstehen, wenn Individuen mit einem starken Bedürfnis nach persönlichem Wachstum mit bereichernden (enriched) Jobs kombiniert werden \\
- \cite[S. 3]{edwards:1990}: Der D-A fit unterliegt laut Schneider 1978 und Smith und Robertson 1989 (noch nicht nach Quellen recherchiert) dem am verbreitetsten Personalauswahl-Modell --> Job Anforderungen analysieren, benötigte Fähigkeiten definieren und Personen anstellen, die diese Fähigkeiten beherrschen (Anmerkung von mir: Genauso arbeiten heute auch die meisten Recommender Systeme) \\
- \cite[S. 3]{edwards:1990}: Dieses Paper fokussiert auf Probleme, die bei anderen Papern bzgl. Stress aufgetreten sind --> lässt sich aber auch auf andere Bereiche übertragen \\

\section{Theoretische Probleme}
\label{ch:notizen:theoretischeProbleme}
- \cite[S. 3f.]{edwards:1990}: Es gibt zwei Versionen des P-E fits: S-V fit und D-A fit --> Werden manchmal gemeinsam unter der Rubrik P-E fit zusammengefasst, unterscheiden sich aber fundamental in ihren unterliegenden Prozessen und ihren assoziierten Outcomes --> \cite[S. 4]{edwards:1990}: Liegt an den zugrunde liegenden Komponenten --> Der S-V fit empfiehlt einen Prozess bei dem das Individuum aus seiner persönlichen Wertestruktur schöpft, um die Umgebung damit zu evaluieren / Beim D-A fit sammelt das Individuum dagegen seine Fähigkeiten und Fertigkeiten, um die Anforderungen der Umgebung zu erfüllen --> Prozesse sind getrennt voneinander \\
- \cite[S. 4]{edwards:1990}: Unterschiedliche Outcomes werden in verschiedenen Arbeiten diskutiert, z.B. Wenn Angebote der Umgebung von den individuellen Werten abweichen, entsteht Unzufriedenheit bei \textcite{locke:1969} / Im Gegensatz dazu: Übersteigen die Anforderungen der Umgebung die persönlichen Fähigkeiten, werden laut \textcite{theoryOfBehaviorInOrganizations:1980} Leistungseinbußen (Performanceeinbußen) wahrscheinlich \\
- \cite[S. 4]{edwards:1990}: Manchmal kann der D-A fit indirekt das Wohlbefinden beeinflussen, wenn das Erreichen der Anforderungen der Umgebung ein inhärenter Wert des Individuums ist (und dadurch ein S-V fit entsteht) oder wenn die Auflösung einer D-A Diskrepanz ein Instrument zur Erreichung eines S-V fits in einer verwandten Dimension ist (sagt \textcite{mechanismsOfJobStressAndStrain:1982}) \\
- \cite[S. 4]{edwards:1990}: Im Unterschied dazu zeigen Beweise, dass es unwahrscheinlich ist, dass der S-V fit die Performance beeinflusst (siehe: \textcite{greene:1972} und \textcite{schwabCummings:1970}) \\
- \cite[S. 4]{edwards:1990}: Ursprünglich waren S-V und D-A komplett unterschiedliche Versionen des P-E fits (siehe \textcite{copingAndAdaption:1974}, \textcite{mechanismsOfJobStressAndStrain:1982}, \textcite{harrison:1978}) --> Unterscheidungen in unterliegenden Prozessen und den assoziierten Outcomes --> Nachfolgende Studien haben diese Unterscheidungen minimiert oder manchmal sogar ganz übersehen --> Manche haben z.B. S-V und D-A als alternative Vorhersageelemente für die selben Outcomes gesehen, z.B. strain (\textcite{jobDemandsAndWorkerHealth:1975}, \textcite{mechanismsOfJobStressAndStrain:1982}) --> Deren Begründung (siehe \textcite[S. 31]{mechanismsOfJobStressAndStrain:1982}): Effekte S-V und D-A basieren auf dem Ausmaß, zu dem Motive befriedigt werden --> Edwards sagt, dass die Trennung aber wichtig ist, weil sonst fragwürdig ist, ob man z.B. D-A überhaupt braucht --> Auch viele andere Autoren betrachten S-V und D-A als austauschbar oder verwechseln sie sogar  --> Edwards ist der Ansicht, dass die Trennung zwischen D-A und S-V bei Untersuchungen erhalten bleiben muss --> Siehe hierzu auch \textcite{mechanismsOfJobStressAndStrain:1982} \\
- \cite[S. 5]{edwards:1990}: Generell: Beziehung zw. S-V und D-A und ihre Auswirkungen auf Wohlbefinden und Performance würden stärker erleuchtet werden, wenn beide Formen in einer Untersuchungsdomäne betrachtet werden (z.B. S-V und D-A fit bzgl. den selben Jobcharakteristiken) \textcite{caplan:1987}

\section{Formen des P-E fit}
\label{ch:notizen:formen}
- \cite[S. 3]{edwards:2007}: Schlüsselunterscheidung in PE fit Literatur ist zw. Ssupplementar und complementary fit (Quellen); Supplementary fit tritt auf, wenn die Person (Zitat von \textcite[S. 269]{muchinsky:1987}): "ergänzt, verschönert oder besitzt Charakteristiken, welche ähnlich zu anderen Individuen" sind; Auch betrachtet der supplementary fit den Vergleich zwischen einer Person und seinem sozialen Umfeld / \cite[S. 4]{edwards:2007}: Complementary fit existiert, wenn eine (Zitat \cite[S. 271]{muchinsky:1987}) "Schwäche oder Need des Umfeldes durch eine Stärke der Person ausgeglichen wird und umgekehrt" --> Complementary fit bezieht sich darauf, welche Erweiterung sich P und Egegenseitig bieten, was der andere will / Der Complementary fit kann weiter unterschieden werden, ob die Anforderungen von E oder P erhoben werden (Quellen) / Anforderungen des Umfelds beziehen sich auf Demands an die Person --> Der Grad, zu dem diese Anforderungen durch Wisen, Skills, Fähigkeiten und Ressourcen durch die Person erfüllt werden, steht für den DA fit (Quellen) / Requirements der Person spiegeln dessen Needs wieder, welche biologische Notwendigkeiten zum Überleben und psychologische Desires, Motive und Ziele \cite{copingAndAdaption:1974} \\
- \cite[S. 5]{edwards:1990}: Es gibt drei grundlegende Formen des P-E fits (bzgl. Stress)
- \cite[S. 5]{edwards:1990}: Erste Form fokussiert die Differenz zwischen vergleichbaren P und E Komponenten --> Umso größer die Differenz, desto größer der strain --> Wird z.B. verwendet von \textcite{mechanismsOfJobStressAndStrain:1982} und vielen anderen \\
- \cite[S. 5]{edwards:1990}: Zweite Form fokussiert die Interaktion zwischen P und E --> Stress tritt au, wenn Eigenschaften der Umgebung mit bestimmten persönlichen Eigenschaften kombiniert werden --> so z.B. vorgestellt von \textcite{cherringtonEngland:1980}, \textcite{lyons:1971}, \textcite{obrien:1980} und anderen --> Operationalisieren den P-E fit als Produkt übereinstimmender P und E Komponenten \\
- \cite[S. 5]{edwards:1990}: Dritte Form betrachtet den Anteil/Verhältnis P, der von E erfüllt wird (manchmal wird auch die Differenz betrachtet) --> Strain nimmt zu, wenn die Proportion geringer wird --> So z.B. verwendet von \textcite{mechanismsOfJobStressAndStrain:1982} und anderen \\
- \cite[S. 5]{edwards:1990}: Viele Autoren betrachten die verschiedenen Formen als austauschbar oder als vergleichbar /  \textcite{mechanismsOfJobStressAndStrain:1982} sagen, dass proportionale Form über Diskrepanz verwendet werden sollte, wenn Ratio-Skalen verfügbar sind, sagt aber auch, dass sich die Ergebnisse nicht nennenswert unterscheiden \\
- \cite[S. 5]{edwards:1990}: Nähere Betrachtung der drei Formen zeigt, dass die drei Formen unterschiedliche theoretische Perspektiven auf die Beziehung zwischen P-E fit und strain haben / Die Diskrepanz-Form betrachtet P als einen Standard, gegen die E vergleichen wird --> Größere Abweichung wird dann mit mehr Stress assoziiert / Die Interaktive Form impliziert, dass die Stärke der Beziehung zwischen E und strain beeinflusst --> P ist also weniger ein Standard --> P modifiziert die Auswirkungen von E auf Stress / Proportionale From teilt Charakteristiken der beiden anderen Varianten --> P ist ein Standard gegen den E vergleichen wird und beefinlusst die Stärke der Beziehung zwischen E und strain / Im Gegensatz zur interaktiven Form impliziert die proportionale Form, dass der Effekt von P auf die Stärke der Beziehung zwischen E und strain fortlaufend kleiner wird, wenn P sich vergrößert \\
- \cite[S. 5]{edwards:1990}: Obwohl die Unterscheidungen eher theoretisch sind, zeigen sie auch wichtige methodische Implikationen --> Liegt daran, dass die drei Formen fundamental unterschiedliche funktionale Beziehungen zwischen P, E und strain zeigen --> Können gut als Oberflächen in einem 3D Raum betrachtet werden --> \cite[S. 7]{edwards:1990}: Obwohl die drei Formen des fits alle fundamental unterschiedlich sind, gibt es in der Literatur einige mathematische Transformationen, die diese Unterscheidungen undurchsichtig machen oder ganz entfernen --> Beispiel an einem Autor: Wollte die Diskrepanz durch die squared Differenz zwischen P und E zur Vorhersage von Stress bestimmen --> Diese Berechnung entspricht jedoch der interaktiven Form / Erkenntnis: Die drei Formen sind mathematisch und theoretisch verschieden --> Man muss sich überlegen welche Hypothese man zum Einfluss P und E auf Stress aufstellt und dann die drei Formen nicht als austauschbar betrachten --> Sonst entstehen statistische Tests, die nicht die ursprüngliche Hypothese prüfen --> Form muss mit den theoretischen Annahmen übereinstimmen \\
- \cite[S. 6]{edwards:2007}: Supplementary fit wird oft bei P-O fit genutzt

\subsection{Complementary und Supplementary fit}
\label{ch:notizen:formen:complementaryUndSupplementary}
- \cite[S. 1]{edwards:2004}: Complementary und supplementary fit repräsentieren 2 unterschiedliche Traditionen innerhalb des P-E fit Paradigmas --> sind parallele aber separate Strömungen

\section{Levels des Environments}
\label{ch:notizen:levelsDesEnvironments}
- \cite[S. 5f.]{edwards:2007}: (Quelle): PE fit wird auf unterschiedlichen Leveln unterschieden, auf denen der PE-fit ausgeführt wird / Person ist immer (ich würde sagen: meist oder überwiegend; gibt auch Ausnahmen siehe \cite[S. 6]{edwards:2007}) auf einem individuellen Level, aber E bezieht sich in der Forschung auf unterschiedlichen Level --> z.B. auf Personen in einer Umgebung; andere Individuen wie Führungskräfte; Mitarbeiter; Arbeitsgruppen; Abteilungen; Organisationen, etc. (zu allem zahlreiche Quellen)

\section{Content von P und E Dimensionen}
\label{ch:notizen:contentVonPundEDimensionen}
- \cite[S. 6]{edwards:2007}: Dimensionen können auf einem Kontinuum von Generell bis speziell reichen --> Edwards schlägt drei Punkte vor: Global-, Domäne-, Facettenlevel der P und E Dimensionen / Beim globalen Level werden den Ähnlichkeiten als Ganzes ohne Bezug auf andere Vergleichsdimensionen miteinander verglichen --> beim Supplementary Fit genutzt --> z.B. um P und E oder P und andere P zu vergleichen (Quellen) / \cite[S. 7]{edwards:2007}: Domain-Level  isoliert breite Bereiche, aber unterscheidet keine Dimensionen innerhalb des Bereichs --> z.B. Werte, Persönlichkeit (Quellen) / Beim Facetten-Level wird die Ähnlichkeit spezifischer Dimensionen inerhalb breiter Bereiche verglichen, z.B. wenn man die demographische Ähnlichkeit nach Alter, Geschlecht, Rasse und Erziehung bestimmen will (Quellen) --> Gutes Bild dazu auf S. 10 \\
- \cite[S. 7]{edwards:2007}: Inhalt der P und E Dimensionen müssen vergleichbar sein (Quellen) --> Hat 2 Features: 1. Nominale Gleichheit: P und E werden mit denselben Begriffen beschrieben (sowohl bei SV als auch DA); 2. Skalengleichheit: Heißt, dass P und E mit derselben Metrik bewertet werden \cite{copingAndAdaption:1974} --> Wird erreicht, wenn dieselbe Antwortskala für P und E und unterschiedliche item stems (?) zur Unterscheidung von P und E verwendet werden --> Dieser Ansatz wird von Porters Need Satisfaction Questionaire (Quelle) vorgestellt --> Dort steht "Wie viel ist es jetzt" und "Wie viel sollte es sein"

\section{Methodische Probleme}
\label{ch:notizen:methodischeProbleme}
- \cite[S. 7]{edwards:1990}: Es gibt zwei wichtige methodische Probleme in den P-E fit Studien bzgl. Stress: Erste bezieht sich auf die Messung er P und E Komoponenten; Zweite betrachtet die Analyse der Beziehung zw. P E und strain

\subsection{Messung der fit Komponenten}
\label{ch:notizen:methodischeProbleme:messungDerFitKomponenten}
- \cite[S. 8]{edwards:1990}: Messung der P und E Komponenten sollte vergleichbar sind --> Heißt: Sollten dieselbe theoretische Dimension haben --> \textcite{copingAndAdaption:1974} sagen, dass vergleichbare Messungen notwendig für Differenz-Berechnung sind --> Das ist laut \textcite{jobDemandsAndWorkerHealth:1975} die meistverwendete Prozedur in der P-E fit Forschung --> Differenzberechnungen werden meist verschiedenen Transformationen ausgesetzt, um den Punkt des 'Perfect Fit' zu bestimmen \textcite{jobDemandsAndWorkerHealth:1975} / Wenn diese Prozeduren genutzt werden, müssen die P und E Messungen nicht nur vergleichbar sein, sondern auch noch denselben Nullpunkt teilen --> sonst ist die Bestimmung des Punktes bedeutungslos \\
- \cite[S. 8]{edwards:1990}: Zweitens: Fragebögen, welche zur Messung der P und E Komponenten genutzt werden --> Sowohl bei SV als auch DA sind Fragebögen bzgl. E einfacher --> S-Fragebögen sollten fragen wie sehr das Attribut vorhanden ist, wohingegen D-Fragebögen sollten das Level (Höhe) der Demands erfragen, die mit einem betrachteten Attribut verbunden sind (vgl. \textcite{jobDemandsAndWorkerHealth:1975})/ Fragebögen zu P sind komplexer: Zu V gibt es zwei Möglichkeiten: Einmal zu erfragen, zu welchem Level die Attribute erfüllt sein sollen (\textcite{jobDemandsAndWorkerHealth:1975}) oder die Wichtigkeit der Attribute (\textcite{workAdjustment:1964}) --> Theoretische Diskussionen ergeben, dass der erste Ansatz eher für die Diskrepanz-Form und der zweite für die Interaktive Form geeignet ist (\textcite{copingAndAdaption:1974} und mehr Quellen) --> Es gibt auch Studien, die es anders bzgl. Diskrepanz machen (z.B. V durch Wichtigkeiten messen) oder manche Studien der interaktiven FOrm messen V in Desires --> Ergebnis: Ergebnisse dieser Studien lieern keine klare Interpretation über die grundlegenden Priznzipien des P-E fits / Für A veranschlagen Fragebögen oft ein Selbstassement der Fähigkeiten  oder indirekte Indikatoren der Fähigkeiten wie z.B. Bildung (\textcite{jobDemandsAndWorkerHealth:1975}) --> Vorteil des ersten Ansatzes: Man kann besser das Konstrukt messen, das einen tatsächlich interessiert, dafür ist der Ansatz anfällig für eine Antwortverzerrung durch soziale Erwünschtheit (social desirability response bias) --> Dieser Bias kann durch die zweite Methode vermieden werden, bietet aber eine weniger direkte Messung des Konstrukes, das einen interessiert --> Gibt keine klare Lösung, aber man sollte die Trade-Offs im Hinterkopf behalten \\
- \cite[S. 8f.]{edwards:1990}: Drittes Problem beachtet die Anzahl an fit-Dimensionen, die in die Studie eingebunden werden / Manche Studien (Quellen) verwenden nur eine Dimension / Sogar die systematischsten Untersuchungen zur Zeit von \cite{copingAndAdaption:1974} und \cite{mechanismsOfJobStressAndStrain:1982} enthalten nur acht Dimensionen des Fits / Wenn es nur wenige Dimensionen gibt, haben solche Studien zwei Nachteile --> 1. Wenn man annimmt, dass eine Inkongruenz über mehrere Dimensionen den strain stärker beeinflusst, vernachlässigen diese Studien notwendigerweise relevante Bestimmungsgrößen von Strain; 2. Diese Studien bieten nur limitierte Informationen über den P-E fit als generelles Konstrukt / Übergeordnete Ansätze involvieren umfassende Messungen von Person und Environment, um Indizes des fits zu bestimmen --> z.B. haben viele Studien zur Jobzufriedenheit den Work Values Inventory verwendet, um Indizes entlang von 15 Dimensionen zu bestimmen (Quellen genannt); Alternativ kann man auch Arbeitnehmer interviewn, um Job-relevante Aktivitäten und Konstrukt-korrespondierende Indizes des Fitszu erhalten --> Beide Prozeduren werden eine Einschätzung des fits bieten und sollten wenn möglich implementiert werden

\subsection{Analyse der Effekte des fits}
\label{ch:notizen:methodischeProbleme:analyseDerEffekteDesFits}
- \cite[S. 9]{edwards:1990}: Hier treten die größten Probleme auf / Die meisten Arbeiten bei der Diskrepanz-Form operationalisieren den Fit als algebraische Distanz (oder einer Transformation davon) zwischen vergleichbaren P und E Komponenten --> Ansatz erscheint intuitiv, die Verwendung von Differenz-Werten wurde aber stark kritisiert (einige Quellen) --> Ein Differenz-Score ist eine einfache Linearkombination seiner Komponenten --> \\
DAS NOCHMAL ANSEHEN MIT LÖSUNGSVORSCHLÄGEN

\section{Grundlagen des P-E fits (nach Edwards)}
\label{ch:notizen:grundlagen}
- \cite[S. 6f.]{edwards:2008}: Parsons Matching Modell von Karriere-Entscheidungsfindungen gilt als eine der ersten Theorien des P-E fits (Quelle) / \textcite[S. 5]{parsons:1909} sagt, dass es drei Faktoren für eine weise Berufswahl gibt: Verständnis seiner Fähigkeiten; Wissen um die Anforderungen des Jobs und eine Schlussfolgerung aus der Beziehung dieser zwei Gruppen --> Laut Edwards beschreibt Parsons hier einen D-A fit / Auch spricht \textcite{parsons:1909} einen N-S fit an, da er beschreibt, dass eine Person ihre Interessen und Ambitionen und die Mögilchkeiten der Arbeit kennen sollte --> Konsequenz (\cite[S. 3]{parsons:1909}): Wenn es kein Gleichgewicht zwischen den Fähigkeiten des Arbeiters und den Kapazitäten gitb, führt das zu Ineffizienz ... und geringerer Bezahlung, wohingegen ein Gleichgewicht mit der Natur des Mannes zu Enthusiasmus ... führt un damit in besseren Produkten und höherer Bezahlung führt / Parsons definiert Person und Environment aber nicht formal, beschreibt aber die Form der Beziehung zwischen P-E fit und den Outcomes \\
- \cite[S. 7]{edwards:2008}: Murray wird wegen seiner Forschung zum Needs-Press Modell als Gründer der P-E fit-Forschung bezeichnet / Murray betont, dass es verschiedene Needs gibt und organisiert diese in breiteren Kategorien (z.B. ob sie psychologisch oder physiologisch sind oder ob sie verborgen sind oder offen ausgedrückt werden) / Press bezieht sich auf Stimuli, die der Person nutzen oder schaden können --> Je nachdem ob der Stimuli die Erfüllung der Needs hemmt oder fördert / Murray stellte fest, dass der Druck (Press) mit denselben Begriffen beschrieben werden kann wie die Needs, z.B. wenn das Bedürfnis nach Leistung vom Druck des Versagens vereitelt wird (Quelle) / \cite[S. 7f.]{edwards:2008}: Murray definierte ein "Thema" als eine Kombination von bestimmten press mit korrespondierenden Needs --> Seiner Ansicht nach war das ein Schlüssel, um Beeinflussung, Verhalten und Wohlbefinden zu verstehen / Murray unterschied auch zwischen aktueller (alpha press) und wahrgenommener Umgebung (beta press)/ \cite[S. 8]{edwards:2008}: Murray trägt primär zur P-E Forschung bei, durch seine Typologie zur Beschreibung von Needs und der Vorstellung, dass Needs undPress können sich auf dieselbe Inhaltsdimension beziehen; Unterscheidung zwischen aktuellem und wahrgenommenem Druck und der Idee, dass Bedürfnisse durch Druck erfüllt werden können --> Diese Denkweise ist gleich der heutigen Vorstellung von Needs-Supplies Fit / Murrays Arbeit sagt wenig darüber aus, wie und warum ein Match zwischen Needs und Press Outcomes beeinflussen und betrachtet auch nicht die Form der Beziehung wie sie sich je nach Bedarf verändert oder welche Randbedingungen diese Beziehung umgeben --> Deshalb: Murrays Need-Press Modell ist ein nützlicher Startpunkt für P-E fit Theorien, die den needs-supplies fit betrachten, aber seine Arbeit fokussiert primär das Beschreiben von Bedürfnissen und weniger das Erklärung der Art oder der Auswirkungen des Needs-Press Matches \\
\cite[S. 8]{edwards:2008}: Lewin (Quellen) wird als Pionier der P-E fit Forschung betrachtet / Lewins Hauptbeitrag für die P-E fit Forschung war sein Statement, dass Verhalten eine Funktion der Person und des Umfeldes ist, ausgedrückt durch die klassische Formel B=F(P,E) --> Meint: Verhalten ist bestimmt durch die Person und das Umfeld und nicht durch die Person oder das Umfeld alleine (Zitat Lewin) --> Formel ist in der P-E fit Forschung allgegenwärtig / Lewin hat nicht gesagt, dass Verhalten aus dem fit zwischen P und E resultiert. Er hatte eher einen generellen Anspruch und behauptete, dass die Person und die Umgebung das Verhalten gemeinsam beeinflussen ohne dabei die Art des Effektes zu spezifizieren / Laut \textcite{schneider:2001} sollte die FOrmel eher interpretiert werden, dass Person und Umgebung additiv, interaktiv, proportional oder auf andere Weise miteinander verbunden sind, was nicht P-E fit bedeutet / Lewin hat keine bestimmten P oder E Konstrukte spezifiziert, welche zu einem bestimmten Verhalten kombiniert werden können --> Lewins Formel ist eine Kurzschrift, um zu sagen, dass P und E gemeinsam das Verhalten beeinflussen, aber der Effekt bezieht sich nicht auf den P-E fit. 

\section{Job Zufriedenheit}
\label{ch:notizen:jobZufriedenheit}
- \cite[S. 9]{edwards:2008}: Theorien der Job-Zufriedenheit basieren auf der Prämisse, dass Job-Zufriedenheit aus dem Vergleich zwischen was ein Job bietet und was ein Angestellter benötigt, will oder verlangt vom Job (Quellen) --> Dieser Vergleich korrespondiert mit dem N-S fit in der P-E fit Literatur \\
- \cite[S. 10]{edwards:2008}: Katzell (Quelle) entwickelte ein theoretisches Modell, um die Beziehung zwischen Diskrepanzen in spezifischen Job-Facetten, Zufriedenheit mit Job-Facetten und insgesamter Job-Zufriedenheit zu erklären --> Katzell definierte Job Zufriedenheit als einen "Affekt oder Grundton", der mit einem Job assoziiert wird, der "aus der Interaktion zwischen Arbeitnehmern und ihren Job-Ereignissen entstehen: Arbeitnehmer besitzen Werte oder Bedürfnisse und der Job ist mehr oder weniger ein Instrument Erfüllung zu bieten oder Verstärken" --> Katzell bezeichnete Job Zufriedenheit als Beeinflussung vom Vergleich von den Werten des Angestellten und dem was der Job bietet / Katzell definiert Werte als "Größenordnung eines Reizes, die ein höheres Maß an Befriedigung hervorruft als diejenige, die von anderen Größen dieser Art von Reizen hervorgerufen wird" --> Das impliziert, dass die Zufriedenheit abnimmt, wenn die Stimuli in beide Richtungen von Werten abweichen. Katzell stellt dazu fest dass, "das Ausmaß zu dem ein Stimulus eine affektive Reaktion hervorruft, die weniger als maximal angenehm ist, ist direkt proportional zur absoluten Diskrepanz zwischen der Größe des Stimulus und seinem entsprechenden Wert und umgekehrt proportional zum Wert" --> wtf --> egal, steht sowieso sehr in der Kritik \\
- \cite[S. 12]{edwards:2008}: Locke entwickelte einer der einflussreichsten Diskrepanz-Theorien zur Job-Zufriedenheit / Locke definierte Job-Zufriedenheit als "den angenehmen emotionalen Zustand, der sich aus der Einschätzung ergibt dass die eigene Arbeit das Erreichen der eigenen beruflichen Werte ermöglicht oder erleichtert" (Quelle) --> Bewertungsprozess hat nach Locke 3 Elemente: 1. Die Wahrnehmung eines bestimmten Aspektes des Jobs; 2. Impliziten oder expliziten Wertestandard; 3. bewusstes oder unbewusstes Urteil über die Beziehung (z.B. Diskrepanz) zw. der Wahrnehmung und den Werten (Quelle) / Locke definiert Werte als etwas, das eine Person subjektiv "verlangt, will oder zu erreichen anstrebt" (Quelle) und fügte hinzu, dass Werte in Bezug auf Inhalt (was will die Person?) und Intensität (oder Wichtigkeit von dem gewollten) / Locke stellte zu den Werten die Needs in Kontrast, welche er als objektive Anforderungen für Gesundheit und Überleben beschrieb --> Locke sagte, dass Werte in Bezug zu Needs stehen, sodass "die ultimative biologische Funktion der Werte des Menschen es ist, seine Handlungen und Entscheidungen so zu lenken, dass sie seine Needs befriedigen" (Quelle) --> Locke vermutet, dass die Erfüllung der Werte zu Job-Zufriedenheit führt, vorausgesetzt, dass die Werte mit den Bedürfnissen vereinbar sind / \cite[S. 12f.]{edwards:2008}: Locke unterschied auch Werte von Erwartungen --> Sind Glaubenssätze über die Zukunft --> Locke ging davon aus, dass Diskrepanz zwischen Wahrnehmung und Erwartung zu einer Überraschung führt, welche befriedigend oder unbefriedigend sein kann, abhängig davon, ob das unerwartete Ereignis gewollt (desired) ist (z.B. Lottogewinn) oder ungewünscht ist (z.B. Gefeuert Werden) / \cite[S. 13]{edwards:2008}: Locke (Quelle) formalisierte seine Perspektive durch die folgende Formel:

\begin{equation}
	S = (V_c - P)V_i
	\label{fig:formel1}
\end{equation}

S steht für die Zufriedenheit (Satisfaction),  Vc für den Wert (Value Content), P für die wahrgenommene Menge und Vi für die Wichtigkeit des Wertes (Value Importance) / Locke sagt, dass entweder die absolute oder die algebraische Differenz abhängig vom Inhalt des Wertes verwendet werden muss / Zwei Beispiele mit Geld und Temperatur (Quellen) / \cite[S. 14]{edwards:2008}: Umso wichtiger der Wert, umso steiler die Kurve \\
- \cite[S. 14]{edwards:2008}: Möchte man mehrere Job-Faketten evaluieren, kann man diese addieren, um eine insgesamte Job-Zufriedenheit zu bestimmen / Einige nachfolgende Arbeiten bauen auf Locke auf (Quellen) / \cite[S. 15]{edwards:2008}: Locke erklärte, dass die Norm für Diskrepanztheorien nicht das ist was Leute erwarten oder objektiv brauchen, sondern was sie wertschätzen (value) / Locke führte auch aus, dass Menschen Werte nutzen, um ihren Job zu bewerten wie sie ihn wahrnehmen und nicht wie er aktuell ist (kann abweichen) / Lockes Modell hat aber auch einige Schwächen: 1. Zirkulär (ggf. nochmal ansehen); 2. Formel \ref{fig:formel1} ist inkonsistent mit den Beispielen von Locke inkl. den Bildern. Insbesondere (Vc-P) indiziert, dass die Zufriedenheit abnimmt, wenn die wahrgenommene Menge relativ zur gewünschten Menge steigt --> Das ist das Gegenteil des ersten Bildes; Interpretiert man (Vc-P) als absolute Differenz, dann hätte die Funktion eine V-Form, was das Gegenteil der Funktion in Abb 2 wäre. / Auch sagt locke nicht welche der beiden Berechnungsweisen für welche Job-Facetten zu nutzen sind. Er sagt nur, dass Abb 2 für die große Mehrheit der Job-Faketten genutzt werden sollte (z.B. Task-Schwierigkeit, Reisebereitschaft, etc.) (Quelle); Auch sagt er, dass die Wendepunkte und neutralen Punkte empirisch ermittelt werden müssen (Quelle) / Es kann auch andere Formen als in den beiden Abb geben / Locke selbst (Quelle) ermittelte basierend auf einer Literaturrecherche sieben Job-Facetten, die er als relevant für die Job-Zufriedenheit erachtete --> Legt dabei aber außer beim Gehalt keine Form des Graphen fest / Auch sagt Lockes Theorie wenig über Grenzbedingungen aus

\section{Job Stress}
\label{ch:notizen:jobStress}
\subsection{McGrath}
\label{ch:notizen:jobStress:mcgrath}
- \cite[S. 16]{edwards:2008}: Konzept des P-E fit ist weit verbreitet in den Theorien des Job Stresses (Quellen); Manche Theorien fokussieren den N-S fit (Quellen), andere den D-A fit (Quellen) und andere beide gleichzeitig (\cite{mechanismsOfJobStressAndStrain:1982}) \\
- \cite[S. 16]{edwards:2008}: McGraths Modell von Stress und Performance: McGrath (Quellen) fokussiert den D-A fit. Laut McGrath (Quelle) entsteht Stress durch ein Ungleichgewicht zwischen Anforderungen des Umfeldes und den entsprechenden Fähigkeiten --> Existiert nicht objektiv, sondern wie es von der Person wahrgenommen wird; Um Stress wahrzunehmen muss er Person das Ungleichgewicht bewusst sein / Stress kann sowohl bei einer Über- als auch einer Unterbelastung entstehen / Stress tritt nur auf, wenn die Person glaubt, dass die Konsequenzen des nicht erreichens der Anforderungen wichtig sind / Stress ist ein D-A Misfit, der zunimmt umso größer die Abweichung ist / McGrath stellte die folgende Formel auf (Quelle):

\begin{equation}
	ES = C(|D-A|)
	\label{fig:formel2}
\end{equation}

- \cite[S. 16]{edwards:2008}: ES ist der erfahrene Stress (experienced stress); C sind die wahrgenommenen Konsequenzen (Consequences); D sind die wahrgenommenen Anforderungen (Demands) und A sind die wahrgenommenen Fähigkeiten (Abilities) / \cite[S. 17]{edwards:2008}: In einer Studie operationalisierte McGrath Stress als eine physikalische Erregung und fand heraus, dass diese hoch war, wenn die Differenz zwischen Anforderungen und Fähigkeiten klein war (Quelle) --> McGrath interpretierte dieses Ergebnis als ein Manifestierung von Unsicherheit basierend auf der Annahme, dass die Unsicherheit eine Aufgabe zu erfüllen zunimmt, wenn die Anforderungen und Fähigkeiten nah beieinander sind --> Annahme: Erfolg wird wahrscheinlich, wenn Fähigkeiten die Anforderungen übersteigen und Verlust wird wahrscheinlich, wenn die Anforderungen die Fähigkeiten übersteigen --> Aus dieser Schlussfolgerung passte McGrath seine Formel an:

\begin{equation}
	ES = C(K-|D-A|)
	\label{fig:formel3}
\end{equation}

- \cite[S. 17]{edwards:2008}: K ist dabei eine Konstante --> Dadurch dass |D-A| von K abgezogen wird, zeigt die Gleichung, dass Stress (bzw. Erregung) zunimmt, wenn die Lücke zwischen D und A abnimmt \\
- \cite[S. 18]{edwards:2008}: Die Definition von Stress als Erregung wird in der Stress-Literatur diskutiert (Quellen)

\subsection{French}
\label{ch:notizen:jobStress:french}
- \cite[S. 18]{edwards:2008}: French und Kahn (Quelle) lieferten eine Definition für mentale Gesundheit und Anpassung abhängig vom P-E fit --> Zitat: Die Anpassung hngt immer von den Eigenschaften einer Person in Abhängigkeit zu den Eigenschaften des objektiven Umfeldes ab; es bezieht sich auf die Güte des Fits zwischen Anforderungen der Person und den verfügbaren Angeboten der Umgebung. Entsteht ein Status der Nichtanpassung impliziert das direkt einen Verlust an Zufriedenheit, etc. in dieser spezifischen Umgebung / French und Kahn stellten auch fest, dass Person und Umgebung nur vergleichen werden können, wenn ihre Dimensionen vergleichbar sind  / French and Kahn haben auch die Beziehung zwischen der objektiven und der subjektiven Person und Umfeld diskutiert / \textcite{copingAndAdaption:1974} erweiterten die Arbeit von French und Kahn --> Auch \cite{copingAndAdaption:1974} definierten Anpassung als die Güte des fits zwischen Person und Umfeld fügten aber hinzu dass der P-E fit sich sowohl auf die subjektive als auch auf die objektive Person und Umfeld beziehen kann / \textcite{copingAndAdaption:1974} stellen auch fest, dass man zwischen zwei Paaren von Anforderungen und Angeboten unterscheiden muss, einmal fungieren die Motive der Person als Anforderungen an die Angebote des Umfeldes und einmal die Anforderungen des Umfeldes an die Angebote der Person / Das was French et al machen baut auf French und Kahn auf \\
- \cite[S. 18f.]{edwards:2008}: Supplies und Demands müssen als vergleichbare Dimensionen kontextualisiert werden und eine gemeinsame Metrik haben --> French et al nennen es Menge (Amount) \\
- \cite[S. 19]{edwards:2008}: \textcite{copingAndAdaption:1974} stellten vier Formeln zur mentalen Gesundheit auf --> wichtigste ist der fit zwischen subjektiver Person und subjektivem Umfeld --> Ist der Schlüsselvorhersager für psychologischen strain / Sagen auch, dass Starin entsteht, wenn die Demands die Supplies übersteigen / Von einem Überangebot an Supplies (Sowohl aus E als auch P Perspektive) erwarteten \cite[S. 319]{copingAndAdaption:1974} nicht, dass es einen direkten Effekt hat / \cite[S. 19f.]{edwards:2008}: Laut \textcite[S. 330f.]{copingAndAdaption:1974} ist eine Person motiviert, einen P-E misfit durch Änderung der objektiven oder subjektiven Person oder Umfeld aufzulösen, wenn der Misfit einen Mangel von einem Bedürfnis oder Wert bedeutet --> Änerungen die einen objektiven P-E fit lösen werden als "coping" bezeichnet, wohingegen Änderungen am subjektiven P-E fit als defense Bezeichnet werden \\
- \cite[S. 20]{edwards:2008}: Bild zeigt, dass die objektive Person und Umfeld ihre Subjektiven Gegenstücke beeinflussen und Strain und Krankheit eher durch den subjektiven als durch den objektiven P-E fit ausgelöst werden --> Objekte und subjektive Person und Umfeld werden von coping und defense beeinflusst und coping und defense sind Outcomes von Strain und Krankheit, welche die Person motivieren den P-E fit zu optimieren (Quelle) \\
- Zusammenfassung für mich: \textcite{copingAndAdaption:1974} haben das Modell entwickelt und festgestellt --> Damals hatten sowohl P als auch E Demands und Supplies --> Nichterfüllen der Demands führte zu Strain --> Später haben die Entwickler der Arbeit weiter an diesem Modell gearbeitet und die Konstrukte auf Seiten der Person von Demands Supplies zu Needs und Abilities umbenannt --> Seitdem heißen die beiden Formen des Fits NS und DA fit / \cite[S. 20]{edwards:2008}: Strain entsteht laut den neuen Arbeiten (Quellen) nur durch einen needs-supplies misfit - DA beeinflusst den Strain nur in Ausnahmefällen (Siehe Harrison) --> \cite[S. 21]{edwards:2008}: Weitere Arbeiten haben dieses Modell danach diskutiert (Quellen) --> Wie sich der Needs-Supply fit verhält, wird in mehren Arbeiten behandelt --> Alle sind sich einig, dass Strain zunimmt, wenn die Supplies des Environments die Needs der Person nicht erfüllen; Es gibt unterschiedliche Theorien wie es sich aber verhält, wenn die Supplies die Motive übersteigen --> Dann kann Strain zunehmen, abnehmen oder konstant bleiben --> Hängt davon ab, welche Auswirkungen die Übersteigung auf das Motiv oder andere Motive hat; Wenn das Überangebot nicht auf andere Motive angewendet werden kann oder für das Motiv nicht haltbar gemacht werden kann, dann sollte der NS-fit eine asymptotische Beziehung zum Strain haben (Kurve A); Wenn die übersteigenden Supplies zurückgehalten werden oder Motive anderer Dimensionen damit erfüllen können, sollte die monotone Beziehung von Kurve B entstehtn --> Beispiel: Mehr Gehalt kann in Zukunft für Luxusgüter oder Dienstleistungen ausgegeben werden; Es entsteht eine U-Kurve, wen der NS fit andere Dimensionen oder dieselbe Dimension hemmt (Kurve C) \\
- \cite[S. 22f.]{edwards:2008}: Ein Ähnliches Verhalten stellen verschiedene (Quellen) auch beim demand-abilities fit fest --> (Aber nur, wenn DA misfit dazu führt, dass die Nichterfüllung der Anforderung den Erhalt von gewünschten Supplies hemmt)--> Wenn Demand größer als Abilities sind, entsteht Strain; Ist es umgekehrt, kommt es darauf an, wie die Kurve aussieht \\
- \cite[S. 23]{edwards:2008}: Es gibt auch Erweiterungen des Frameworks, die Minima an anderen Punkten als dem P-E fit haben --> Caplan 1983 S. 39 sagt z.B. dass der meist emotional zufriedenstellenste Punkt so liegt, dass er eine kleine Herausforderung generiert. / Es gibt noch weitere Erweiterungen, z.B. wie Vergangenheit und Zukunft den Fit beeinflussen könnten \\
- \cite[S. 23f.]{edwards:2008}: Harrison 1985 betrachtete, wie Wichtigkeit die Auswirkungen des P-E fits beeinflusst --> Schlägt vor (Harrison 1985, S. 38), wenn man einen P-E fit über mehrere Dimensionen macht könnte man die Wichtigkeit jeder Dimension in die Formel integrieren , die die Diskrepanz jeder P-E fit Dimension mit der Wichtigkeit dieser Dimension multipliziert \\
- \cite[S. 24]{edwards:2008}: Außerdem hat Harrison die Effekte des P-E fits auf den organisationalen Strain untersucht, welcher sich auf Probleme der Funktionsweise , welche die Produktivität und das Überleben der Organisation verhindern --> Organisationaler strain entsteht, wenn die Fähigkeiten der Angestellten unzureichend sind, um die Rollen-Anforderungen zu erfüllen / Harrison 1985 S. 42 stellt fest (Zitat), dass das erfüllen der Needs und Values fundamental für das weitere Funtionieren und existieren des Individuums genauso wie das erfüllen der Rollenanforderung fundamental für das funktionieren und existieren der Organisation ist --> In Sonderfällen kann aber auch ein N-S fit zu organisationalem Strain führen / Beziehung zw. D-A Misfit und organisationalem Strain kann laut Harrison 1985 wie in Figur 4.6 betrachtet werden --> Org. strain nimmt zu, wenn die Anforderungen die Abilities übersteigen, aber bleibt konstant, nimmt ab/zu wenn die Abilities die Demands übersteigen \\
- \cite[S. 24]{edwards:2008}: (Zusammenfassung von Edwards): Theorie erklärt auch weshalb der subjektive eher als der objektive P-E fit ist er proximale Grund für Strain und weshlab die Effekte des D-A fits vom NS fit geschlichtet werden / Ursprünglich war die Theorie auf die Auswirkungen des P-E fits auf mentale Gesundheit beschränkt, später wurde sie auf weitere Outcomes wie Jobzufriedenheit, Performance, etc. erweitert

\section{Berufliche Kongruenz}
\label{ch:notizen:beruflicheKongruenz}
- \cite[S. 25]{edwards:2008}: Der P-E fit ist zentral für Theorien der beruflichen Kongruenz, viele davon betrachten das Match zwischen Needs, Interessen und Fähigkeiten der Person und Verstärkern und Anforderungen (verschiedene Quellen) \\
- \cite[S. 26]{edwards:2008}: Wichtigster Forscher: Holland (Mehrere Quellen) \\
- \cite[S. 26]{edwards:2008}: Berufliche Kongruenz steht auch bei Theorien zu Arbeitsanpassung im Vordergrund

\subsection{Dawis und Lofquist}
\label{ch:notizen:beruflicheKongruenz:dawisUndLofquist}
- \cite[S. 28f.]{edwards:2008}: \textcite{workAdjustment:1964} legten den Grundstein für die Theorie des "Work adjustment" --> Konzeptualisierten die Person mit Abilities und Needs / Environment wurde beschrieben durc Ability-Requirements (benötigte Anforderungen für eine zufriedenstellende Arbeitsperformance) und ein Verstärkungssystem (Spezifikationen der verstärkten Werte von Klassen von Stimulusbedingungen) (In Klammern Zitate) --> Diese P E Konstrukte wurden in zwei Typen von Korrespondenz gemappt: Eine bezieht sich auf die Ähnlichkeiten zwischen Fähigkeiten der Person und den Fähigkeits-Anforderungen der Umgebung; Die andere betrachtet die Ähnlichkeit zwischen den Bedürfnissen einer Person und dem Verstärkungssystem der Umgebung --> \textcite{workAdjustment:1964} sagen, dass man die Begriffe zur Beschreibung von Abilities und Needs auch zur Beschreibung benötigter Fähigkeiten und verfügbaren Verstärkern genutzt werden sollten, sodass vergleichbare Dimensionen vorliegen --> Proximal Outcome der Korrespondenz ist dann Zufriedenheit (Definiton: "Die Evaluation des Individuums der Stimulus-Bedingungen der Arbeitsumgebung mit Bezug auf deren Effektivität bzgl. der Verstärkung seines Verhaltens") und Zufriedenstellung ("Evaluation des Arbeitsverhaltens in Bezug auf die Qualität und Quanität der Arbeitsperformance und/oder Performance-Outcomes (Produkte, Services)") / Zitate von \cite{workAdjustment:1964}: Zufriedenheit ist eine Funktion der Korrespondenz zwischen Verstärkungssystem der Arbeitsumgebung und den Needs des Individuums, vorausgesetzt dass die Fähigkeiten des Individuums mit den Fähigkeitsanforderungen der Arbeitsumgebung korrespondieren ... Zufriedenstellung ist eine Funktion der Korresondenz zwischen den Fähigkeiten des Individuums und den Fähigkeitsanforderungen der Arbeitsumgebung, vorausgesetzt, dass die Needs des Individuums mit dem Verstärkungssystem der Arbeitsumgebung korrespondieren" / Erkenntnis: Zufriedenstellung mäßigt die Effekte von Needs-Verstärkungssystem auf Zufriedenheit und genauso mäßigt Zufriedenheit die Effekte von Fähigkeiten-Fähigkeitsanforderungs-Korresponenz auf Zufriedenstellung --> "Work adjustment" ist definiert als kombinierte Levels von Zufriedenheit und Zufriedenstellung einer Person --> \cite{workAdjustment:1964} stellten auch fest, dass Zufriedenstellung und Zufriedenheit die Wahrscheinlichkeit beeinflussen, mit welcher eine Person in einer Arbeitsumgebung bleibt oder sie verlässt \\
- \cite[S. 29ff.]{edwards:2008}: Die Theorie wurde von Dawis et al. 1968 und Lofquist und Dawis 1969 grundlegend geändert und aktualisiert von Dawis und Lofquist 1984 \\
- \cite[S. 31]{edwards:2008}: Die letzte Version (Dawis und Lofquist 1984) definiert Fähigkeiten als empirisch abgeleitete Faktoren, die spezifische Fähigkeiten umfassen, die "wiederkehrende Reaktionssequenzen sind, die dazu neigen, durch Wiederholung modifiziert und verfeinert zu werden" (Zitat) / Analog dazu sind Values Faktoren, die spezifische Needs umfassen, die definiert sind als "Das Bedürfnis eines Individuums nach einem Verstärker auf einem bestimmten Niveau der Stärke" --> Stärke ist die Wichtigkeit eines Bedürfnisses --> Ich schließe aus dem Kontext: Je größer die Stärke, desto Häufiger muss die Reaktion des Verstärkers sein; Werte sind Wichtigkeitsdimensionen als Referenzdimensionen für die Beschreibung von Needs --> Needs sind Präferenzen für Verstärker ausgedrückt als relative Wichtigkeit für jeden Verstärker für das Individuum / Definition Korrespondenz: "Harmonische Beziehung zwischen Individuum und Umgebung, Eignung des Individuums für die Umwelt und der Umwelt für das Individuum, Übereinstimmung oder Einvernehmen zwischen Individuum und Umgebung sowie eine wechselseitige und ergänzende Beziehung zwischen Individuum und Umwelt" --> Es besteht also eine wechselseitige Beziehung zwischen Individuum und Umgebung / Definitionen für Zufriedenheit und Zufriedenstellung \\
- \cite[S. 32]{edwards:2008}: Kritik von Edwards an Dawis and Lofquist: Deren Theorie definiert Person und Umgebungs-Konstrukte auf eigene Art und Weise / \cite[S. 33]{edwards:2008}:  Auch sind Bedeutungen von Zufriedenheit und Zufriedenstellung unklar dargestellt

\section{Recruiting und Selektion}
\label{ch:notizen:rekcruitingUndSelektion}
- \cite[S. 33]{edwards:2008}: P-E fit ist fundamental, wenn Menschen mit Jobs in Unternehmen gematcht werden (Quellen) / Forschung in diesem Bereich identifiziert meist das notwendige Wissen, Fertigkeiten und Fähigkeiten für einen Job, misst diese Attribute bei möglichen Angestellten und untersuchen den Zusammenhang zwischen diesen Messungen und späterer Arbeitsleistung (Performance) --> Diese Forschung adressiert nicht direkt den Fit zwischen Person und Job, weil die persönlichen Attribute nicht berücksichtigt werden (Schneider, 2001) --> Edwards betrachtet nur Theorien, die explizit den P-E fit behandeln

\subsection{Wanous Matching Modell}
\label{ch:notizen:rekcruitingUndSelektion:wanous}
- \cite[S. 34f.]{edwards:2008}: Modell von Wanous ist eine Adaption der Work Adjustment Theorie von Lofquist und Dawis 1984 / Betrachtet SV- und den DA-fit / Auch hier gibt es Reinforcers / Abilities sind definiert als "was Menschen jetzt tun können oder potentiell in der Zukunft tun können werden" (Genauso auch die Anforderungen); Needs sind definiert als "grundlegendes Streben oder Verlangen"; Im Gegensatz dazu werden Verstärker und Anforderungen nicht explizit definiert / Laut Modell führt DA fit zu Job-Performance --> Zitat: Mismatch führt bei Fähigkeiten und Job-Anforderungen führt zu schlechter Performance; Match zwischen Bedürfnissen und Verstärkern führt zu Jobzufriedenheit und Organisationalem Commitment --> Definition Jobzufriedenheit: "Match zwischen Needs einer Person und erhaltener Verstärkung durch die erledigte Arbeit"; Definition Organisationales Commitment: "Dem Match zwischen Menschlichen Needs und den Verstärkungen, die man durch die das nicht-job Klima er Organisation erhält" / Johannes: Aus Modell ist abzulesen, dass Job-Performance zu einer Reaktion beim Unternehmen führt (Feuern, .., bleiben) und Job Zufriedenheit und Commmitment zu einer Reaktion beim Mitarbeiter (Gehen, bleiben) \\
- \cite[S. 35]{edwards:2008}: Später erstellte Wanous (1992) ein neues Modell getrennt von seinem alten Modell (Meiner Ansicht nach ändert sich da aber nicht viel)

\subsection{Breaughs Person-Job Congruence Model}
\label{ch:notizen:rekcruitingUndSelektion:breaugh}
- \cite[S. 36]{edwards:2008}: Braugh entwickelte ein Modell für einen Rekruting-Prozess der die person-job Kongruenz als zentrale Komponente unterstützt --> P-J Kongruenz definiert er als "Diskrepanz zwischen den Attributen, die ein e Organisation von einem potentiellen Angestellten verlangt und den Charakteristiken die eine Person bietet und die Diskrepanz zwischen dem was die Person von der Organisation will und den Anreizen, die der Arbeitgeber bietet" --> \cite[S. 37]{edwards:2008}: (Zitate) D-A fit führt zu einem zufriedenstellenden Level an Job-Performance und ein guter fit zwischen den Wünschen der Person und den Attributen der Job-Angebote führt zu einem Gefühl der Wertsteigerung, was wiederum zu Arbeitszufriedenheit führen wird" / Nachteil nach Edwards: Beschreibt keine Metrik, mit der Person und Job verglichen werden können --> Aber es gibt ein paar Beispiele / In einer Fußnote schreibt Breaugh, dass der P-J fit sich auf die Wahrnehmung der Kongruenz zw. DA und SV bezieht (das muss ich nochmal nachprüfen) / Modell bleibt auch wage bei der Form der Beziehung zwischen Kongruenz udn Outcomes --> In einer Fußnote stellt Breaugh fest, dass für manche Organisationalen Attribute ein Individuum weder zu viel noch zu wenig von einem Attribut sucht (z.B. Reisen). Bei anderen Attributen gitl, umso mehr die Organisation bietet (z.B. Bezahlung) desto besser wird die Person den fit bewerten --> Bestimmte Attribute werden aber nicht genannt

\subsection{Werbel and Gillilands Facet Model of Fit}
\label{ch:notizen:rekcruitingUndSelektion:werbel}
- \cite[S. 37]{edwards:2008}: Werbel und Gilliland stelleten ein Faketten-Modell des P-E fits vor, welches den Auswahlprozess in Bezug auf Person-Job fit, Person-Workgroup fit und Person-Organization fit bezieht / P-J fit ist definiert als (Zitat) Kongruenz zwischen Job-Anforderungen und den benötigten Skills, Wissen und Fähigkeiten eines Job-Kandidaten --> \cite[S. 38]{edwards:2008}: Laut 
Modell führt der P-J fit zu Leistung, technischem Verständnis und Arbeitsinnovationen / Person-Workgroup fit bezieht sich auf das (Zitat) Match zwischen dem Neuangestellten und der gesamten Arbeitsgruppe (z.B. Mitarbeiter und Führungskräfte) --> P-W fit wird sowohl durch einen supplementary als auch einen complementary fit bestimmt: Supplementary, da Werte, Ziele, Persönlichkeit übereinstimmen müssen und complementary, da die Gruppe heterogene Skills, Leistungen und Netzwerke mitbringen muss, sodass (Zitat) Performance-Schwächen eines Individuums durch die Performance-Stärken eines anderen Individuums ausgeglichen werden können / P-O fit enthält einen supplementary fit und einen needs-supply fit; Supplementary fit ist beschrieben als Kompatibilität zw. dem Wertesystem der Person und der Organisation; NS fit bezieht sich auf das Match (Zitat) zwischen den Bedürfnissen des Bewerbers und dem organisationalen Belohnungssystem --> Ergebnis des Fits sind Organizational Citizenship Behaviors (OCBs), organisationale Zufriedenheit, organisationales Commitment und Bewahrung / Die Outcomes aller drei fits sind mit insgesamter Performance und organisationaler Effektivität verbunden / Kommentare von Edwards: Modell ist dahingehend bemerkenswert, dass es drei Typen von P-E Fits betrachtet und kollektiv NS und DA und supplementary fit anspricht

\section{Generelles}
\label{ch:notizen:generelles}
- \cite[S. 1]{edwards:2007}: Zusätzlich zum DA und NS fit gibt es auch einen Fit zwischen den Werten der Person und denen der Organisation und deren Mitgliedern (Quellen) \\
- \cite[S. 1]{edwards:2007}: Outcomes z.B. Berufswahl, Job-Zufriedenheit, Job-Performance, Wohlbefinden \\
- \cite[S. 1f.]{edwards:2007}: Die Forschung zum PE fit hat 3 grundlegende Annahmen: 1. Es wird generell angenomen, dass der PE fit zu positiven Outcomes führt (Quellen); 2. Es wird oft angenommen, dass die Auswirkungen des PE-Fits die selben über mehrere Personen- und Environments Konstruke sind (Quellen); 3. \cite[S. 2]{edwards:2007}: Effekte des PE fits sind die selben, unabhängig von den absoluten Levels von P und E oder der Richtung ihrer Differenz --> Quellen, die Ähnlichkeiten berechnen; Annahmen wurden diskutiert und in Frage gestellt, halten sie sich dennoch weitgehend in der PE Forschung \\ 
- \cite[S. 3]{edwards:2007}: PE fit wurde in verschiedenen Wegen konzeptualisiert --> Generell kann der PE fit als "die Kongruenz, Match, Ähnlichkeit oder Korrespondenz zwischen der Person und dem Umfeld" definiert werden\\
- \cite[S. 49]{edwards:2008}: Ein paar Quellen definieren P und E explizit (Quellen), die meisten Theorien geben keine expliziten Definitionen (Quellen) oder beschreiben P und E in generellen Begriffen, die einzelne Konstrukte zusammenfassen (Qeullen) / Wenige Theorien sagen, dass dass die Effekte des PE fits davon abhängen wie die Umgebung von der Person wahrgenommen wird (Quellen von Locke und McGrath) und eine Theorie betrachtet objektive und subjektive P und E als getrennte Konstrukte (Caplan, French, Harrison) --> Diese Theorien sind laut Edwards aber Ausnahmen von der Regel \\
- \cite[S. 53]{edwards:2008}: Bedeutung des Begriffes "fit": In der Literatur werden verschiedene Begriffe Synonym für Fit verwendet, z.B. Match, Ähnlichkeit, Kongruenz zw. Person und Umgebung --> Das sind klare Begriffe, welche die Nähe von Person und Umgebung zueinander bezeichnen --> Das ist laut Edwards die richtige Konzeptualisierung für fit; Viele Autoren verwenden Begriffe ohne klare Bedeutung wie Harmonie, Kompatibilität, etc.; Fit wird auch als Interaktion oder wechselseitige Beziehung bezeichnet; Um die Bedeutung des Begriffes fit zu klären, sollten laut Edwards Begriffe verwendet werden, welche sich auf die Annäherung von P und E zu beziehen und weniger Metaphern verwenden \\
- \cite[S. 53]{edwards:2008}: PE-fit als Konstrukt: P-E fit ist eine Aussage über das Niveau von P und E relativ zueinander --> Wenn P und E auf dem selben Niveau sind (egal ob auf einem niedrigen, mittleren oder hohen), dann existiert der Definition nach ein P-E fit --> Ist das Level nicht gleich, existiert ein P-E misfit \\
- \cite[S. 55]{edwards:2008}: Laut Edwards 1994 ist es das wahrscheinlich größte Problem Differenz-Werte und Profil-Ähnlichkeits-Indizes zu verwenden, um den P-E fit als eine einzige Variable auszudrücken --> Die Verwendung solcher Variablen wird oft auf theoretische Überlegungen zurückgeführt, z.B. könnte eine Theorie vorhersagen, dass der PE fit positiv mit einem Ergebnis zusammenhängt und als Reaktion wird ein Forscher Messungen von P und E zu einem Differnzwert zusammenfassung, um den P-E fit zu repräsentieren und den Wert mit der Messung eines OUtcomes korrelieren --> \cite[S. 56]{edwards:2008}: Geht nicht weiter darauf ein wurde in anderen (Quellen) schon behandelt --> Edwards macht sich eher Sorgen, dass die Berufung auf die Theorie, um die Verwendung von Differenzwerten und Profilähnlichkeitsindizes zu rechtfertigen, fehlgeleitet ist, da dies voraussetzt, dass die Theorie korrekt ist und sie von einer Überprüfung abschirmt --> Beispiel: Eine Theorie besagt, dass die absolute Differenz zwischen NS zu Zufriedenheit führt. Dann sollte nicht die Korrelation zwischen der absoluten Differenz zw. NS mit Zufriedenheit geprüft werden, sondern man sollte die Funktionsform testen, welche die absolute Differenz darstellen soll --> Diese Funktionsform sollte als zu empirisch zu testende Hypothese betrachtet werden und nicht als Annahme, die den Daten aufgezwungen wird --> ???

\section{ToDo}
\label{ch:todo}
- Manche Arbeiten unterscheiden zwar explizit zwischen subjektiven und objektiven P und E --> Ausnahme von der Regel (Edwards) --> bei SV-fit ist subjektive Wahrnehmung üblich \\
- Am Anfang Edwards Definitionen für P, E, fit und P-E fit verwenden, danach einzelne Definitionen für die Konstrukte suchen \\
- Was ist organisationales Verhalten \\
- Gliederung: Wichtigkeiten und dann sagen, dass aber auch Autoren zum Ergebnis kommen, dass DA für Individuum vollkommen egal ist --> Reinforcers --> Obwohl diese Verstärker so wichtig sind, werden sie bei der Personal-Auswahl häufig vernachlässigt (nur DA) --> Diese fehlende Beachtung von SV spiegelt sich auch in der Implementierung von Empfehlungssystemen wieder --> Switch zu RS \\
- Merke ein Konstrukt ist z.B. Demand oder Supply \\
- Es ist nicht möglich, alle Varianten und Arbeiten zu behandeln, wie geht man damit um? \\
- Interessant: Supplies werden auch als "Verstärker" für Needs betrachtet \\
- Nochmal nach den Definitionen für DA und SV bei Edwards suchen \\
- Es gibt viele Arten von Fits (PO, PJ, PW, etc.) --> Klar machen, dass hier nur PJ fokussiert wird (PW siehe Werbel; PO siehe O-Klima bzw. Schneider bei Klima) \\
- Wichtigkeit spielt vor allem beim supplementary fit eine Rolle --> Dieser wird in dieser Arbeit eher ausgeklammert, weil ich mich auf P-J fokussiere und supplementary eher verwendet wird, um zu schauen ob eine Person zu anderen Personen oder überhaupt in eine Org passt. Dennoch spielen Wichtigkeiten auch in manchen Umsetzungen des complementary fit eine Rolle. --> Wichtigkeiten \\
- Wichtigkeiten beziehen sich bei SV auf die Werte bei DA auf die Konsequenzen --> Person ist es egal, ob sie Anforderungen erfüllt --> Neues Kapitel

\section{Fazit}
\label{ch:fazit}
- Ob man die Anforderungen der Stelle erfüllt, ist dem Mitarbeiter eigentlich egal. Ihn interessiert es nur, ob dadurch seine Werte verstärkt/erfüllt werden. Es wäre also interessant, herauszufinden, wieso ein Mitarbeiter z.B. mehr Python anwenden würde. Mehr Gehalt wegen Data Science? Neugier für neue Sprache? Kumpel arbeitet in dieser Abteilung? ... \\
- Eigentlich müsste der Prozess des Motivations-Herausfindens dem kompletten Einstellungsprozess vorgelagert sein. Bzw. sogar dem Studium
\shorthandon{"}