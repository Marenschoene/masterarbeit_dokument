\shorthandoff{"}
\chapter{Person-Environment Fit}
\label{ch:personEnvironmentFit}

\section{Einführung}
\label{ch:personEnvironmentFit:einfuehrung}
Der \ac{PEFit} \cite[S. 428]{dawis:2002} ist ein Konzept in der Organisationspsychologie \cite[S. 1f.]{edwards:2008}. Er enthält drei zentrale Größen: Person, Umgebung (Environment) und Ergebnis (Outcome) \cite[S. 2f.]{livingstone:1997}. Forscher dieses Fachgebietes gehen davon aus, dass ein Ergebnis stets vom Zusammenspiel von Person und Umgebung abhängig ist und nicht durch eine der beiden Größen alleine bestimmt wird \cite[S. 1]{muchinsky:1987}.

\textcite[S. 5]{edwards:2007} stellten fest, dass die Literatur unter der Person ein menschliches Individuum versteht. Umgebung und Ergebnis interpretieren verschiedene Publikation ihren Beobachtungen zu Folge dagegen als breite Terminologien. Diese werden je nach Forschungsdomäne genauer spezifiziert. Beispiele für Ergebnisse sind Zufriedenheit \cite[S. 1]{lashani:2021}, Wechselbereitschaft \cite[S. 1]{amarneh:2021}, Kreativität \cite[S. 1]{duan:2019}, Leistung \cite[S. 7f.]{elfenbein:2007} und  Berufswahl \cite[S. 1]{cable:1996}. Als Umgebung untersuchten verschiedene Publikationen unter anderem Unternehmen \cite[S. 1]{kristof:1996}, Gruppen \cite[S. 1]{werbel:2001} und Arbeitsplätze \cite[S. 1]{lu:2014}.

Der \ac{PEFit} gibt an, zu welchem Grad sich die untersuchten Werte von Person und Umgebung auf einem Niveau befinden \cite[S. 3]{chatman:1989}. Wie in Abbildung \ref{fig:personEnvironmentFit:einfuehrung:abb1} verdeutlicht, ist der Fit selbst kein Ergebnis, sondern eine unabhängige Variable, welche zur Bestimmung eines untersuchten Resultates herangezogen wird \cite[S. 4f.]{edwards:1991}.

\begin{figure}[h]
	\centering
	\includegraphics[width=1\textwidth]{gfx/P-E Fit.png}
	\caption{Nochmal schön machen und kennzeichnen, dass die P-E-Beziehung wechselseitig ist}
	\label{fig:personEnvironmentFit:einfuehrung:abb1}
\end{figure}

Der \ac{PEFit} findet sich in verschiedenen Publikationen ebenfalls unter ähnlichen Bezeichnungen mit derselben Bedeutung wie Match \cite[S. 2]{player:2017}, Korrespondenz \cite[S. 1]{eggerth:2008} oder Kongruenz \cite[S. 1]{muchinsky:1987}.

\textcite[S. 6f.]{edwards:2007} unterschieden drei Ebenen, auf welchen ein Fit bestimmbar ist. Die Oberste bezeichnen sie als globale Ebene. Hier werden Person und Umgebung meist als Ganzes ohne weitere Untergliederungen miteinander vergleichen. Wird eine solche Einteilung in mehrere sehr breite Bereiche vorgenommen, sprechen \textcite[S. 7f.]{edwards:2007} von der Domänen-Ebene. Hier können beispielsweise Werte, Persönlichkeiten und Ziele von Person und Umgebung miteinander verglichen werden. Findet die Untersuchung ausschließlich innerhalb einer Domäne statt, bezeichnen \textcite[S. 7f.]{edwards:2007} dies als Facetten-Ebene. Als Beispiel nennen die Wissenschaftler die Bestimmung eines \acp{PEFit} hinsichtlich demographischer Ähnlichkeit anhand von Dimensionen wie Alter und Bildung.

In der Literatur wird der \ac{PEFit} nicht als Zustand, sondern als wechselseitiger Prozess betrachtet. In diesem interagieren Person und Umgebung miteinander und verändern sich dabei gegenseitig \cite[S. 21f.]{roberts:2006}. Diese Modifikationen können die Kongruenz sowohl verbessern als auch verschlechtern \cite[S. 4]{caplan:1987}. Aus diesem Grund wurden in der Literatur auch Maßnahmen erforscht, welche den \ac{PEFit} gezielt optimieren sollen \cite[S. 16]{cable:2001}.

Wie die Kongruenz von Person und Umgebung berechnet wird, ist von der konkreten Art des Fits abhängig. \textcite[S. 1]{muchinsky:1987} unterschieden dabei zwischen ergänzendem (supplementary) und komplementären (complementary) Fit.

\section{Ergänzender und komplementärer Fit}
\label{ch:personEnvironmentFit:supplementaryUndComplementary}
Ein ergänzender Fit entsteht, wenn Person und Umgebung gleiche Werte und Interessen aufweisen \cite[S. 2f.]{muchinsky:1987}. Diese Art von Kongruenz ist laut \textcite[S. 1ff.]{schneider:1987} ein entscheidender Faktor, von welchen Unternehmen sich potentielle Arbeitnehmer angezogen fühlen und welche Bewerber von Betrieben eingestellt werden. Auch \textcite[S. 7]{devendorf:2008} stellten fest, dass sich potentielle Arbeitnehmer stärker zu Unternehmen angezogen fühlen, bei deren Mitarbeitern sie eine hohe Ähnlichkeit zu sich selbst wahrnehmen. \textcite[S. 4, Z. 25f.]{popovich:1982} zu Folge kann der Beitritt einer Person zu einem Unternehmen sogar als ein "sehr konkreter, öffentlicher Ausdruck der Werte"\footnote{"a very concrete, public expression of values" - \textcite[S. 4, Z. 25f.]{popovich:1982}} eines Individuums interpretiert werden.

Verschiedene Autoren diskutierten die Ergebnisse des ergänzenden Fits in der Literatur kontrovers. \textcite[S. 6]{schneider:1987} stellte fest, dass Angestellte mit einer geringen Werte-Übereinstimmung eher dazu tendieren, ihr Unternehmen zu verlassen. So entsteht im Betrieb langfristig eine hohe Homogenität innerhalb der Belegschaft. Diese äußert sich einerseits in positiven Ergebnissen wie einer ausgeprägten Arbeitszufriedenheit, geringer Bereitschaft den Arbeitgeber zu wechseln und starker Identifikation mit dem Unternehmen \cite[S. 25ff.]{kristof:1996}\cite[S. 5]{su:2015}. Die mangelnde Diversität führt aber anderseits auch zu negativen Folgen, wie einer geringeren Bereitschaft für Veränderungen \cite[S. 10]{schneider:1987} und verminderter Kreativität und Innovation im Unternehmen \cite[S. 7]{chatman:1998}.

Wenn sich Person und Umgebung nicht ähneln, sondern gegenseitig vervollständigen, sprechen \textcite[S. 4]{muchinsky:1987} vom komplementären Fit. Dabei gleichen Person und Umgebung den Autoren zu Folge Schwächen des anderen durch eigene Stärken aus.

Die komplementäre Kongruenz wird wie in Abbildung \ref{fig:personEnvironmentFit:supplementaryUndComplementary:abb1} dargestellt, in zwei Fits untergliedert. Bei dieser Betrachtungsweise haben Person und Umgebung je eine Angebots- und eine Nachfrageperspektive. Die Nachfrage der einen Partei wird dabei durch das Angebot der anderen erfüllt \cite[S. 2ff.]{caplan:1987}\cite[S. 2f.]{edwards:1991}.

\begin{figure}[h]
	\centering
	\includegraphics[width=1\textwidth]{gfx/supplementaryComplementaryFit.png}
	\caption{Hier eine Beschreibung einfügen; Quelle: \cite[S. 4]{kristof:1996}}
	\label{fig:personEnvironmentFit:supplementaryUndComplementary:abb1}
\end{figure}

Unter der Nachfrage der Umgebung werden die in Abbildung \ref{fig:personEnvironmentFit:supplementaryUndComplementary:abb1} dargestellten Anforderungen (Demands) an die Person zusammengefasst. Hierzu zählen beispielsweise Rollen- und Leistungserwartungen. Das entsprechende Angebot der Person sind ihre Fähigkeiten (Abilities). Diese umfassen unter anderem Fertigkeiten, Wissen, Bildung und Arbeitserfahrung. Gleichen sich Nachfrage der Umgebung und Angebot der Person gegenseitig aus, entsteht der Anforderungen-Fähigkeiten Fit (Demands-Abilities Fit). Dieser resultiert in einer hohen Leistung und Effizienz von Individuum und Umgebung \cite[S. 3f.]{edwards:1991}\cite[S. 5]{edwards:1996}\cite[S. 4f.]{edwards:2007}\cite[S. 6]{su:2015}.

Die Nachfrage der Person entspricht ihren psychologischen Bedürfnissen (Needs). Dazu zählen persönliche Präferenzen, Interessen, Motive und Ziele. Die entsprechenden Angebote (Supplies) der Umgebung umfassen Ressourcen und Belohnungen wie Gehalt und Mitbestimmungsrechte, welche die Bedürfnisse des Individuums befriedigen. Sind Nachfrage der Person und Angebote der Umgebung gleich stark ausgeprägt, wird dies in der Literatur als Bedürfnisse-Angebote Fit (Needs-Supplies Fit) bezeichnet. Dieser resultiert in einem hohen Wohlbefinden des Mitarbeiters, welches sich beispielsweise in Zufriedenheit und verminderter Wechselbereitschaft äußert \cite[S. 2]{edwards:2004}\cite[S. 2f.]{edwards:1996}\cite[S. 4]{edwards:2008}\cite[S. 4f.]{edwards:2007}\cite[S. 6]{su:2015}.

Laut \textcite[S. 9ff.]{workAdjustment:1964} und \textcite{wanous:1992} führen unausgeglichene Charakteristiken auch beim komplementären Fit langfristig zu einem Wechsel des Arbeitsplatzes. Die Autoren betrachten die aus einem unzureichenden Anforderungen-Fähigkeiten Fit entstehende mangelnde Arbeitsleistung als Ursache für Kündigung oder Versetzung des Mitarbeiters seitens des Unternehmens. Die aus einem Bedürfnisse-Angebote Ungleichgewicht resultierende Unzufriedenheit bezeichnen sie als einen Wechselgrund seitens des Mitarbeiters.
%"Wenn die Bedürfnisse jeder Entität von der anderen erfüllt werden und beide ähnliche grundlegende Eigenschaften teilen"\footnote{"when each entity’s needs are fulfilled by the other and they share similar fundamental characteristics" - \textcite[S. 6]{kristof:1996}}, ist laut \textcite[S. 6]{kristof:1996} ein optimaler Fit erreicht.

\textcite[S. 1ff.]{edwards:2004} charakterisieren ergänzende und komplementäre Kongruenz als unterschiedliche, parallele Strömungen innerhalb der \ac{PEFit}-Forschung. Doch sie stellten fest, dass beide Fits nicht vollkommen unabhängig voneinander sind. Die Ursache sehen sie in den inneren Werten von Person und Umgebung. Diese sind einerseits ausschlaggebend für den ergänzenden Fit, beeinflussen aber auch stark die Bedürfnisse der Person und die Angebote der Umgebung. So würde sich ein Individuum mit ausgeprägten familiären Werten aufgrund des ergänzenden Fits stark zu Betrieben mit denselben Eigenschaften angezogen fühlen. Gleichzeitig prägt die Person aufgrund ihrer inneren Werte im komplementären Fit das Bedürfnis nach familiären Reizen wie gemeinschaftlichen Veranstaltungen aus. Da das Unternehmen dieselben Eigenschaften besitzt, wird dieses seinen Mitarbeitern die Teilnahme an derartige Ereignisse anbieten.

Dass der Abgleich der Charakteristiken von Person und Umgebung sehr bedeutsam für Zufriedenheit und Produktivität sind, erkannten Psychologen bereits vor über einhundert Jahren \cite[S. 5ff.]{parsons:1909}. Die Wurzeln des \acp{PEFit} reichen zurück bis ins Jahr 1909 \cite[S. 1]{su:2015}.

\section{Historische Entwicklung}
\label{ch:personEnvironmentFit:historisches}
Im ersten Jahrzehnt des 20. Jahrhunderts beschäftigten sich Wissenschaftler und Psychologen in zahlreichen Ländern der westlichen Welt intensiv mit dem Thema der Personalauswahl \cite[S. 1]{salgado:2001}. Ein Hauptanliegen der Forscher war es, individuelle Unterschiede zwischen den Menschen anzuerkennen und bei der Berufswahl zu berücksichtigen \cite[S. 2ff.]{stern:1900}. Deren Ansichten zu Folge würde die gesamte Gesellschaft effizienter arbeiten, wenn Menschen eine zu ihren wissenschaftlich ermittelten Fähigkeiten passende Tätigkeit aufnehmen würden \cite[S.2]{kevles:1968}\cite[S. 3]{parsons:1909}. Im Zuge dieser Entwicklungen konzipierte der Bostoner Professor Frank Parsons eine Vorgehensweise zur Berufsfindung, welche im Jahr 1909 vorgestellt wurde \cite[S. 1]{su:2015}. \textcite[S. 5ff.]{parsons:1909} erkannte schon zum damaligen Zeitpunkt, dass das Gleichgewicht von eigenen Fähigkeiten und Anforderungen des Berufsumfeldes eine wichtige Ursache für Effizienz, Produktqualität und Bezahlung waren. Aus diesem Grund empfahl er jungen Menschen vor der Berufswahl zunächst ihre eigenen Fähigkeiten, die Anforderungen verschiedener Arbeitsplätze und die Beziehung zwischen beiden Seiten zu verstehen. Erst wenn eine Person diese Punkte unter Beaufsichtigung eines Berufsberaters und durch Verwendung verschiedener wissenschaftlicher Tests erfüllt, könne sie sich für einen passenden Beruf entscheiden. Heute gilt Parsons aufgrund dieser Gedanken als "Gründungsvater der Berufsberatung"\footnote{"founding father of vocational guidance" - \textcite[S. 3, Z. 29]{porfeli:2009}} \cite[S. 3, Z. 29]{porfeli:2009} und als erster Vorläufer des \acp{PEFit} \cite[S. 2]{edwards:2008}.

Zum damaligen Zeitpunkt begegnete die Bevölkerung den neuartigen psychologischen Tests zunächst mit Skepsis \cite[S. 2]{kevles:1968}. Das änderte sich insbesondere im Jahr 1917 mit dem Eintritt der Vereinigten Staaten in den Ersten Weltkrieg. Das U.S. Militär stand vor der Herausforderung, innerhalb kürzester Zeit Millionen Männer in die verschiedenen spezialisierten Rollen des technisierten Krieges einzuordnen. Zu diesem Anlass setzten Wissenschaftler erstmals im großen Stil psychologische Tests zur Zuweisung von Personen zu passenden Militärpositionen ein \cite[S. 2ff.]{kevles:1968}. 

Nach dem Ersten Weltkrieg entstanden insbesondere in den 1930er-Jahren durch die Arbeiten der Wissenschaftler \textcite{lewin:1936} und \textcite{murray:1938} weitere bedeutende Entwicklungen für die Entstehung des \acp{PEFit} \cite[S. 1]{edwards:1990}. \textcite[S. 11f.]{lewin:1936} stellte fest, dass das Verhalten eines Menschen nicht, wie bis dahin angenommen, nur durch das Individuum selbst, sondern durch das Zusammenspiel von Person und Umgebung zu erklären ist. Aufbauend auf diesen Erkenntnissen erarbeitete \textcite[S. 38ff.]{murray:1938} sein Need-Press-Modell. Der Wissenschaftler ging davon aus, dass jeder Mensch im Laufe seines Lebens verschiedene Bedürfnisse (Needs) unterschiedlich stark ausprägt. Diese treffen je nach Umgebung auf diverse Reize. Murray stellte fest, dass manche Reize mit bestimmten Bedürfnissen kompatibel sind. Trifft ein passendes Bedürfnis-Reiz-Paar aufeinander, entsteht Druck (Press). Personen interpretieren diesen subjektiv als schädliche oder nützliche Situation und zeigen eine entsprechende Reaktion. Dieses Zusammenspiel von Bedürfnissen einer Person und Reizen der Umgebung entspricht der späteren Vorstellung des Bedürfnisse-Angebote Fits \cite[S. 8]{edwards:2008}. 

Die Erkenntnisse von Lewin und Murray gelten als wichtiger Grundstein für die Arbeiten verschiedener Forschungsgruppen rund um John R. P. French, Jr. \cite[S. 5]{caplan:1993}. Der Psychologe stellte im Jahr 1963 an der Universität in Michigan ein groß angelegtes Forschungsprogramm vor. Dieses machte es sich zum Ziel, die Auswirkungen des sozialen Umfeldes in Industrie-Unternehmen auf die Gesundheit der Mitarbeiter zu untersuchen. Zu diesem Zweck arbeiteten Experten verschiedener Fachrichtungen eng zusammen \cite[S. 1ff.]{french:1963}. Aus dieser Kollaboration entstand die erste formale Definition des \acp{PEFit}. Diese präsentierten \textcite{copingAndAdaption:1974} im Jahr 1974. Dabei unterschieden die Forscher ausgehend ihren bis dahin erzielten Erkenntnissen zwischen objektivem und subjektivem Fit \cite[S. 4f.]{caplan:1993}\cite[S. 1ff.]{french:1966}.

\section{Objektiver und subjektiver Person-Environment Fit}
\label{ch:personEnvironmentFit:subjektivObjektiv}
% Hierzu Quellen von Caplan und Harrison einfügen, sobald Bücher verfügbar
\textcite{copingAndAdaption:1974} erforschten die Auswirkungen des Zusammenspiels von Individuum und Arbeitsumgebung auf die mentale Belastung des Mitarbeiters. Wie in Abbildung \ref{fig:personEnvironmentFit:subjektivObjektiv:abb1} dargestellt, gingen die Wissenschaftler davon aus, dass von Person und Umgebung je eine objektiv messbare und eine vom Mitarbeiter subjektiv wahrgenommene Version existieren. 

\begin{figure}[h]
	\centering
	\includegraphics[width=1\textwidth]{gfx/subjektivObjektivPEFit.png}
	\caption{Hier eine Beschreibung einfügen \cite[S. 22]{edwards:2008}}
	\label{fig:personEnvironmentFit:subjektivObjektiv:abb1}
\end{figure}

Die Forscher betonten die Wichtigkeit, alle vier Subjekte anhand vergleichbarer Dimensionen zu messen. Dies betrachteten sie als wichtige Grundlage, um aussagekräftige Ähnlichkeiten berechnen zu können. In ihren Untersuchungen bestimmten sie die in Abbildung \ref{fig:personEnvironmentFit:subjektivObjektiv:abb1} dargestellten Differenzwerte des objektiven und des subjektiven \acp{PEFit} \cite{copingAndAdaption:1974}.

Individuen streben den Wissenschaftlern zu Folge an, unerfüllte Anforderungen zu verhindern. Dies gilt sowohl für unbefriedigte Bedürfnisse der Person als auch für überhöhte Ansprüche der Umgebung. Um solche Situationen zu vermeiden, existieren zwei Strategien. Ändert ein Mitarbeiter seine objektive Umgebung oder sein objektiv ermitteltes Selbst zur Verbesserung des objektiven \acp{PEFit}, sprechen \textcite{copingAndAdaption:1974} von Bewältigung (Coping). Ändert die Person ihre subjektive Wahrnehmung von Umgebung oder sich selbst zur Optimierung des subjektiven \acp{PEFit}, bezeichnen sie die Strategie als Verteidigung (Defense).

Beispielsweise könnte sich ein Bachelor-Absolvent für eine Stelle interessieren, welche als Anforderung einen Master-Abschluss voraussetzt. Diese Unterschiede im objektiven \ac{PEFit} könnte die Person bewältigen, indem sie entweder den fehlenden Abschluss erwirbt oder das Unternehmen überzeugt, die Stellenanforderungen abzuändern.

Dagegen könnte der Betrieb auch einen zukünftigen Mitarbeiter suchen, welcher über ein bestimmtes Kenntnisniveau in einer Programmiersprache verfügt. Schätzt ein potentieller Bewerber seine Fähigkeiten zunächst schlechter ein als gefordert, gerät der subjektive \ac{PEFit} ins Ungleichgewicht. Die Person kann sich hierbei verteidigen, indem sie die Wahrnehmung ihrer Kenntnisse nachträglich besser bewertet oder die Anforderungen der Stellenausschreibung abwertet.

Bei ihren Untersuchungen stellten \textcite{copingAndAdaption:1974} fest, dass ein Ungleichgewicht im subjektiven \ac{PEFit} besonders bedeutsam für die Entstehung psychischer Belastungen und daraus resultierenden Krankheiten beim Mitarbeiter ist. Andere Werte, wie der objektive \ac{PEFit}, spielen dagegen nur eine untergeordnete Rolle. Auch Publikationen anderer Forscher bestätigen die Einschätzung, dass der subjektive \ac{PEFit} aussagekräftiger für die Bestimmung von Ergebnissen ist, als die objektive Kongruenz \cite[S. 3]{carless:2005}. Dementsprechend wird die subjektive Wahrnehmung des \acp{PEFit} in der Literatur stärker fokussiert \cite[S. 8]{caplan:1987}\cite[S. 9]{caplan:1993}\cite[S. 16]{choi:2004}.

In einer auf den Erkenntnissen von \textcite{copingAndAdaption:1974} aufbauenden Arbeit kam \textcite{harrison:1978} sogar zu der Einschätzung, dass innerhalb des subjektiven \acp{PEFit} alleine die Bedürfnisse-Angebote Kongruenz Auswirkungen auf die mentale Gesundheit des Mitarbeiters hat. Ein Ungleichgewicht im Anforderungen-Fähigkeiten Fit führe dagegen nur dann zu psychischer Belastung, wenn diese der Erfüllung des Bedürfnisse-Angebote Fits schade. Als Beispiel für einen solchen Sachverhalt nennt \textcite{harrison:1978} eine leistungsabhängige Gehalts-Auszahlung. Möchte ein Mitarbeiter die Bezahlung erhalten (Bedürfnis), welche vom Arbeitgeber in Aussicht gestellt wird (Angebot), hat aber nicht ausreichende Fähigkeiten, um die dafür notwendigen Anforderungen zu erfüllen, führt dies zu Unzufriedenheit. Der Grund ist \textcite{harrison:1978} zu Folge jedoch nicht das unterschiedliche Niveau von Fähigkeiten und Anforderungen als solches, sondern die aus diesem Ungleichgewicht resultierende beeinträchtigte Bedürfniserfüllung des Mitarbeiters.

Auch \textcite[S. 1ff.]{lazarus:1978} stellen fest, dass zu hohe Anforderungen nur dann Stress bei einem Individuum auslösen, wenn dieses durch die Nichterfüllung negative Konsequenzen befürchtet. Dabei kann es sich entweder um schädliche Folgen für die Gesundheit oder die Verletzung innerer Werte und Ziele handeln.

Ungleichgewichte im \ac{PEFit} werden häufig auch als Misfit bezeichnet \cite[S. 2]{edwards:2004}, \cite[S. 4]{kristof:1996}. Mögliche Auswirkungen von P-E Misfits wurden in der Literatur in verschiedenen Arbeiten diskutiert.

\section{Auswirkungen von P-E Misfits}
\label{ch:personEnvironmentFit:auswirkungenErhoehterAngebote}
% Auch \cite[S. 21f.]{edwards:2008} äußert sich zu Kurven und nennt Beispiele
% Quellen: Mechanisms of Job Stres and Strain; harrsion:1978; harrsion:1985; caplan iwo
\textcite{mechanismsOfJobStressAndStrain:1982} stellten fest, dass ein Bedürfnisse-Angebote Misfit in unterschiedlichen Konsequenzen resultieren kann. Diese sind in Abbildung \ref{fig:personEnvironmentFit:auswirkungenErhoehterAngebote:abb1} dargestellt.

% Bild auch in Coping und Adaption?
\begin{figure}[h]
	\centering
	\includegraphics[width=0.75\textwidth]{gfx/ueberschuss_supply_motive.png}
	\caption{Hier eine Beschreibung einfügen \cite[S. 23]{edwards:2008}}
	\label{fig:personEnvironmentFit:auswirkungenErhoehterAngebote:abb1}
\end{figure}

An der durchgezogenen Linie auf der linken Hälfte von Abbildung \ref{fig:personEnvironmentFit:auswirkungenErhoehterAngebote:abb1} ist zu erkennen, dass je weniger die Bedürfnisse einer Person erfüllt werden, die mentale Belastung (Strain) des Individuums stärker zunimmt \cite{mechanismsOfJobStressAndStrain:1982}. Der Verlauf dieser Kurve kann über folgende algebraische Differenzberechnung bestimmt werden \cite[S. 2]{edwards:1993}:
\begin{equation}
	B = P - E
	\label{fig:personEnvironmentFit:auswirkungenErhoehterAngebote:formel1}
\end{equation}
In Gleichung \ref{fig:personEnvironmentFit:auswirkungenErhoehterAngebote:formel1} steht $B$ für die mentale Belastung des Mitarbeiters. $P$ stellt die von einer Person gewünschte Menge eines bestimmten Wertes dar. Die vom Mitarbeiter wahrgenommene erhaltene Menge des entsprechenden Wertes wird über Parameter $E$ ausgedrückt \cite[S. 2]{edwards:1993}.

Übersteigen die Angebote der Umgebung die Bedürfnisse der Person, mündet dies in einer der drei gepunkteten Linien A, B oder C \cite{mechanismsOfJobStressAndStrain:1982}.

Kurve A zeigt einen monotonen Verlauf der mentalen Belastung. Dieser entsteht, wenn eine Person die Übererfüllung eines Bedürfnisses entweder für einen späteren Zeitpunkt aufsparen oder in die Befriedigung verwandter Motive investieren kann \cite{mechanismsOfJobStressAndStrain:1982}. Dieser Sachverhalt ist beispielsweise erfüllt, wenn einer Person mehr Gehalt zusteht, als diese für die Zahlung ihrer Lebenskosten benötigt. Das überschüssige Geld könnte diese entweder für die Zahlung von Lebenshaltungskosten in den Folgemonaten aufsparen oder zusätzlich ihr mögliches Bedürfnis nach Luxusgütern befriedigen \cite[S. 21]{edwards:2008}. Für die Berechnung von Kurve A kann ebenfalls Gleichung \ref{fig:personEnvironmentFit:auswirkungenErhoehterAngebote:formel1} verwendet werden \cite[S. 2]{edwards:1993}.

Linie B hat den Verlauf einer quadratischen Funktion und tritt ein, wenn die Übererfüllung eines Bedürfnisses entweder die Befriedigung dieses oder eines verwandten Motivs hemmt \cite[S. 5]{caplan:1987}. \textcite{harrison:1978} nennt hierfür das Bedürfnis einer Person nach sozialem Austausch als Beispiel, welches bei Übererfüllung das Verlangen nach Privatsphäre verletzt. Der Verlauf von Kurve B kann über die in Gleichung \ref{fig:personEnvironmentFit:auswirkungenErhoehterAngebote:formel2} dargestellte absolute Differenz bestimmt werden \cite[S. 2]{edwards:1993}.
\begin{equation}
	B = |P - E|
	\label{fig:personEnvironmentFit:auswirkungenErhoehterAngebote:formel2}
\end{equation}
Alternativ bietet sich die Verwendung der quadrierten Differenz aus Gleichung \ref{fig:personEnvironmentFit:auswirkungenErhoehterAngebote:formel3} an \cite[S. 2]{edwards:1993}.
\begin{equation}
	B = (P - E)^2
	\label{fig:personEnvironmentFit:auswirkungenErhoehterAngebote:formel3}
\end{equation}
Kurve C stellt eine asymptotische Beziehung zur mentalen Belastung dar. Sie tritt ein, wenn weder die Bedingungen von Kurve A noch von Linie B zutreffen. Eine Übererfüllung dieses Bedürfnisses hat folglich weder positive noch negative Folgen für die Person \cite{mechanismsOfJobStressAndStrain:1982}. Ein Beispiel für eine solche Beziehung ist ein Überangebot an Parkplätzen beim Arbeitgeber. Da der Mitarbeiter nur ein Fahrzeug besitzt, kann dieser von zusätzlichen Angeboten keinen Gebrauch machen und diese auch nicht für einen späteren Zeitpunkt aufsparen. Die zusätzlichen Parkplätze schaden auch keinem anderen Bedürfnis des Mitarbeiters. Somit entstehen weder positive noch negative Auswirkungen auf dessen Wohlbefinden. Zur Bestimmung von Kurve C wird die Belastung gleich dem Wert null gesetzt. Dieses Vorgehen ist in Gleichung \ref{fig:personEnvironmentFit:auswirkungenErhoehterAngebote:formel4} dargestellt \cite[S. 2]{edwards:1993}.
\begin{equation}
	B = 0
	\label{fig:personEnvironmentFit:auswirkungenErhoehterAngebote:formel4}
\end{equation}
Verschiedene Autoren gehen davon aus, dass die Beziehungen aus Abbildung \ref{fig:personEnvironmentFit:auswirkungenErhoehterAngebote:abb1} auch für den Anforderungen-Fähigkeiten Fit zu erwarten sind. Dies gilt wie in Kapitel \ref{ch:personEnvironmentFit:subjektivObjektiv} beschrieben nur, wenn das Erfüllen der Anforderungen Auswirkungen auf die inneren Werte des Mitarbeiters hat \cite{mechanismsOfJobStressAndStrain:1982, harrison:1978}.

Wie an der durchgezogenen Linie auf der linken Seite von Abbildung \ref{fig:personEnvironmentFit:auswirkungenErhoehterAngebote:abb1} zu erkennen, entsteht dem zu Folge psychische Belastung, wenn die Anforderungen der Umgebung die Kenntnisse des Mitarbeiters übersteigen \cite[S. 5]{schuler:1980}. Sind dagegen die Fähigkeiten des Angestellten stärker ausgeprägt als erforderlich, resultiert eine der Kurven A bis C. Linie A tritt ein, wenn der Mitarbeiter seine gewonnenen Freiräume nutzen kann, um verwandte Bedürfnisse zu erfüllen. Schaden die zu niedrigen Anforderungen einem Bedürfnis des Mitarbeiters, entsteht Kurve B. Haben die zu hohen Fähigkeiten des Angestellten weder positive noch negative Auswirkungen auf dessen Wohlbefinden, resultiert Linie C \cite[S. 22f.]{edwards:2008}.

Unterschiedliche Publikationen stellen fest, dass der Verlauf der Kurven nicht alleine von den Auswirkungen der Misfits auf die Motive des Mitarbeiters abhängig ist. Zusätzlich muss beachtet werden, als wie wichtig der Angestellte die betroffenen Bedürfnisse bewertet \cite[S. 9f.]{edwards:1996}. 

\section{Einbeziehung der Wichtigkeiten von Bedürfnissen}
\label{ch:personEnvironmentFit:wichtigkeiten}
Bereits bei der ersten formalen Vorstellung des \acp{PEFit} empfahlen \textcite{copingAndAdaption:1974} die Einbeziehung von Wichtigkeitswerten in die Berechnung. Auf welche Weise diese Werte einbezogen werden sollten, ließen die Autoren zum damaligen Zeitpunkt jedoch unbeantwortet \cite[S. 19]{edwards:2008}.

\textcite[S. 38]{harrison:1985} schlug später vor, den \ac{PEFit} für jede untersuchte Variable einzeln zu berechnen und mit einem subjektiven Wichtigkeitswert zu multiplizieren. Diese Einschätzung teilte auch \textcite[S. 18]{locke:1969}\cite[S. 8f.]{locke:1976}. Er gilt als einer der einflussreichsten Wissenschaftler auf dem Gebiet der Arbeitszufriedenheit \cite[S. 12]{edwards:2008}. Diese kann laut \textcite[S. 8]{locke:1969} nur entstehen, wenn Mitarbeiter das Gefühl haben, für sie als wichtig erachtete Werte durch ihre Tätigkeit zu erfüllen. Aus diesem Grund empfahl er zur Berechnung der Arbeitszufriedenheit eines Angestellten zwei Kennzahlen heranzuziehen: Die Differenz aus wahrgenommener und gewünschter Menge eines Wertes und die subjektive Wichtigkeit des Motivs \cite[S. 8]{locke:1976}. \textcite[S. 16]{locke:1969} betonte, dass je nach untersuchtem Wert unterschiedliche Berechnungen wie algebraische oder absolute Differenz angewendet werden können. Ein Beispiel für eine Subtraktion mit algebraischer Differenz ist in folgender Gleichung \ref{fig:personEnvironmentFit:wichtigkeiten:formel1} dargestellt \cite[S. 9]{edwards:1990}:
\begin{equation}
	Y = b * (P - E)
	\label{fig:personEnvironmentFit:wichtigkeiten:formel1}
\end{equation}
In Gleichung \ref{fig:personEnvironmentFit:wichtigkeiten:formel1} steht $Y$ für das untersuchte Ergebnis, wie beispielsweise die Arbeitszufriedenheit. $b$ stellt den subjektiven Wichtigkeitswert dar. Dieser wird mit der Differenz aus gewünschte Menge eines Wertes einer Person $P$ und wahrgenommener Menge des Wertes der Umgebung $E$ multipliziert \cite[S. 9f.]{edwards:1990}.

Derartige Multiplikationen mit einem Differenzwert haben zur Folge, dass die in Abbildung \ref{fig:personEnvironmentFit:auswirkungenErhoehterAngebote:abb1} dargestellten Kurven mit steigender Wichtigkeit steiler werden \cite[S. 9]{locke:1976}. Abbildung \ref{fig:personEnvironmentFit:wichtigkeiten:abb1} verdeutlicht entstehende Stauchungen durch Wichtigkeitswerte.

\begin{figure}[h]
	\centering
	\includegraphics[width=0.8\textwidth]{gfx/Locke.png}
	\caption{Stauchung von Funktionsgraphen durch Wichtigkeitswerte (Aus \cite[S. 13f.]{edwards:2008}, nach \cite[S. 1305]{locke:1976})}
	\label{fig:personEnvironmentFit:wichtigkeiten:abb1}
\end{figure}

Im linken Teil von Abbildung \ref{fig:personEnvironmentFit:wichtigkeiten:abb1} ist eine monotone Funktion basierend auf einer algebraischen Differenzberechnung dargestellt. Die rechte Hälfte der Grafik zeigt den Verlauf einer Berechnung mit absoluter Differenz. In beiden Darstellungen ist zu erkennen, dass die Kurven bei geringer Wichtigkeit ausschließlich im mittleren Bereich von moderat niedrig bis moderat hoch verlaufen. Ist einem Individuum das jeweilige Bedürfnis dagegen sehr wichtig, füllt die Kurve den gesamten verfügbaren Bereich von niedrig bis hoch. Die Ausnutzung des größeren Gebiets führt zur Stauchung der Funktionsgraphen und damit zu einem stärkeren Ansteigen bzw. Absinken der Zufriedenheit des Individuums. 

\textcite[S. 51ff.]{edwards:1991}\cite[S. 9ff.]{edwards:1990}\cite[S. 2ff.]{edwards:1993}\cite[S. 2ff.]{edwards:1993b} kritisierte Berechnungen, wie in Gleichung \ref{fig:personEnvironmentFit:wichtigkeiten:formel1} dargestellt, in mehreren seiner Arbeiten. Dabei diskutierte er insbesondere die Multiplikation mit dem Differenzwert. Der Wissenschaftler ist der Auffassung, dass die aus der Differenzberechnung resultierenden zweidimensionalen Grafiken die Komplexität eines \acp{PEFit} nicht vollständig abbilden. Deshalb empfahl er die Berechnungen mittels Regressionsgleichungen durchzuführen.

\section{Anwendung von Regressionsgleichungen}
\label{ch:personEnvironmentFit:regressionsgleichungen}
In seinen Publikationen empfiehlt \textcite[S. 51ff.]{edwards:1991}\cite[S. 9ff.]{edwards:1990}\cite[S. 2ff.]{edwards:1993}\cite[S. 2ff.]{edwards:1993b} Multiplikationen separat für jeden Wert von Person und Umgebung anzuwenden. Formel \ref{fig:personEnvironmentFit:wichtigkeiten:formel1} könnte hierfür zu folgender Regressionsgleichung umgestellt werden (vgl. \cite[S. 9f.]{edwards:1990}, \cite[S. 2f.]{edwards:1993b}):
\begin{equation}
	Y = b_1 * P - b_2 * E
	\label{fig:personEnvironmentFit:wichtigkeiten:formel2}
\end{equation}
Die Koeffizienten $b_1$ und $b_2$ stehen in Gleichung \ref{fig:personEnvironmentFit:wichtigkeiten:formel2} für die separaten Wichtigkeiten von gewünschter ($P$) und wahrgenommener Menge ($E$) eines Wertes. Durch diese Art der Berechnung entstehen aus zweidimensionalen Grafiken dreidimensionale Modelle \cite[S. 2]{edwards:1993}. Solche Darstellungen sind dem Wissenschaftler zu Folge besser geeignet, Ungleichmäßigkeiten in den Oberflächen abzubilden \cite[S. 51ff.]{edwards:1991}. So stellte \textcite[S. 53ff.]{edwards:1991} beispielsweise bei der Datenanalyse einer Studie mit mehreren hundert Teilnehmern die in Abbildung \ref{fig:personEnvironmentFit:wichtigkeiten:abb2} dargestellte dreidimensionale Beziehung von \ac{PEFit} und Zufriedenheit fest.

\begin{figure}[h]
	\centering
	\includegraphics[width=0.6\textwidth]{gfx/drei_d_modell.png}
	\caption{Hier eine Beschreibung einfügen \cite[S. 57]{edwards:1991}}
	\label{fig:personEnvironmentFit:wichtigkeiten:abb2}
\end{figure}

Um die stärkere Aussagekraft der Regressionsgleichung zu untermauern, werteten \textcite[S. 18ff.]{edwards:1993b}  in ihrer Publikation einen umfangreichen Datensatz von \textcite{mechanismsOfJobStressAndStrain:1982} erneut aus. Dieser wurde ursprünglich durch Multiplikationen mit Differenzwerten aus Person und Umgebung analysiert. \textcite[S. 18ff.]{edwards:1993b} gelang es, durch Anwendung von Regressionsgleichungen im Vergleich zur erstmaligen Untersuchung einen wesentlich höheren Anteil an Varianz zu erklären \cite[S. 8]{su:2015}.
%Zu den Analysen von \textcite[S. 18ff.]{edwards:1993b} bzw. \textcite{mechanismsOfJobStressAndStrain:1982} muss betont werden, dass durch die Datenerhebung vorab Kennzahlen zu Person, Umgebung und Ergebnis festgestellt wurden. Die Forscher untersuchten anhand der Daten, über welche Fit-Berechnung am zuverlässigsten auf das Ergebnis geschlossen werden kann.

Neben Edwards stellte auch \textcite[S. 3]{schneider:2001}\cite[S. 2]{schneider:1978} Mängel an bestehenden \ac{PEFit}-Untersuchungen fest. Er betrachtete die Kongruenz von Person und Arbeitsplätzen. Bei der Literaturrecherche bemerkte er, dass bei der Fit-Bestimmung nur selten die vollständige P-E Kongruenz berechnet wird. Meist fokussieren Forscher seinen Beobachtungen zu Folge dabei häufig ausschließlich den Anforderungen-Fähigkeiten Fit. Dieses Phänomen lässt sich auch bei Publikationen zu IT-basierten Empfehlungssystemen dieser Domäne beobachten.
\shorthandon{"}