\shorthandoff{"}
\chapter{Notizen}
\label{ch:notizen}

\section{Person-Environment Fit}
\label{ch:notizen:personEnvironmentFit}
- \cite[S. 1]{edwards:2008}: Person-environment fit ist ein zentrales Konzept in der Forschung des organisationalen Verhaltens / \cite[S. 2]{edwards:2008}: P-E fit hatte für Jahrzehnte eine zentrale Person in der Forschung zu organisationalem Verhalten --> P-E fit bezieht sich auf die Kongruenz, Übereinstimmung (Match) oder Ähnlichkeit zwischen Person und Umwelt (siehe dazu: \cite{edwards:1998}, \cite{muchinsky:1987})\\
- \cite[S. 2]{edwards:2008}: Spezifische Typen des P-E fit beziehen sich auf die Bedürfnisse der Person und die Belohnungen des Umfeldes (Quellen), die Fähigkeiten einer Person und die Anforderungen der Umgebung (Quellen) und die Ähnlichkeit zwischen Person und sozialem Umfeld, welches sich auf Individuen, Gruppen, Organisationen oder Berufe beziehen kann (Quellen) \\
- \cite[S. 2]{edwards:2008}: Das Konzept eines P-E fits reicht zurück bis zu Plato (\cite{dumont:1995}), zeitgemäße P-E fit-Forschung führt oft auf \textcite{parsons:1909} zurück, der ein matching Modell entwickelt hat, um den Fit zwischen Attributen einer Person und Charakteristiken verschiedener Berufe zu beschreiben \\
- \cite[S. 1]{edwards:1990}: P-E fit charakterisiert Stress als eine Diskrepanz zwischen korrespondierenden Charakteristiken einer Person und der Umgebung --> Diese Diskrepanz steht im Verdacht, schädliche psychologische, physiologische und verhaltensbedingte Outcomes zu erzeugen --> Diese könnten sogar in erhöhter Krankheit und Sterblichkeit münden \\
- \cite[S. 1]{edwards:1990}: Das Framework bildet den Kern vieler aktueller Theorien des organisationalen Stresses  --> Solche Theorien wurden z.B. von \textcite{copingAndAdaption:1974}, \textcite{mechanismsOfJobStressAndStrain:1982} (Noch mehr Autoren) \\
- \cite[S. 1]{edwards:1990}: P-E fit ist ein generelles Framework mit einer langen Tradition in der Psychologie --> Wurzeln reichen zurück bis auf einflussreiche Autoren wie Lewin (1938, 1951) und Murray (1938) \\
- \cite[S. 2]{edwards:1990}: Sinn seines Papers: Viele Studien, die P-E fit bezüglich Stress untersuchen, sind mit ernsten theoretischen und methodischen Problemen geplagt, welche die Aussagekraft der Ergebnisse schmälern \\
- \cite[S. 2]{edwards:1990}: P-E fit ist laut Eulberg et al. 1988 der meist zitierte Modell im Bereich des organisationalen Stresses --> Dieses Paper (von Edwards) konnte keine einzige Studie finden, die frei von Fallstricken war \\
- \cite[S. 4]{edwards:2017}: Die P-E fit Theorie des Stresses nach \cite{mechanismsOfJobStressAndStrain:1982} unterscheidet objektive P und E Faktoren von deren subjektiven Gegenstücken und unterstreichen den fit zwischen subjektiver P und E als Schlüssel-Determinante von psychologischen, physiologischen und verhalten Strains zusammen mit Vermeidungs- und Verteidigungsmechanismen / Ähnlich dazu stellen auch \cite{cummings:1979} und Edwards 1992 Stress als eine Diskrepanz zwischen Vorstellungen und wichtigen Wünschen (Desires) fest --> Diese führt zu psychologischen und physiologischen Symptomen und Bemühungen, die wahrgenommenen Diskrepanzen zu lösen / Auch \cite{schuler:1980} sagt, dass Stress existiert, wenn eine Person mit Anforderungen (Demand) konfrontiert ist, die ihre Wünsche beschneidet\\
- \cite[S. 2]{edwards:1990}: Umfassendste Behandlung des P-E fit Ansatzes (bzgl. Stress) wurde von \textcite{mechanismsOfJobStressAndStrain:1982} durchgeführt --> Behandlung enthält zwei verschiedene Versionen des P-E fits --> Eine Version fokussiert die Korrespondenz zwischen Angeboten der Umgebung (Supplies) und persönlichen Werten, Motiven, Zielen (Values) --> S-V fit / Die andere Version fokussierte die Korrespondenz zwischen Anforderungen (Demands) der Umgebung und persönlichen Fähigkeiten und Fertigkeiten (Abilities) --> D-A fit --> \textcite{mechanismsOfJobStressAndStrain:1982} stellen fest, dass sowohl P als auch E subjektiv und objektiv beschrieben werden können --> Objektive P und E beziehen sich auf die Variablen, welche unabhängig von der Wahrnehmung des Individuums existieren --> Subjektive P und E beziehen sich dagegen auf Variablen wie sie vom Individuum wahrgenommen werden --> Zentrale These von \textcite{mechanismsOfJobStressAndStrain:1982}: Gibt es einen Misfit bei subjektiven S-V oder D-A, dann entstehen daraus negative psychologische, physiologische und verhaltens-Outcomes, die kollektiv als "strain" bezeichnet werden \\
- \cite[S. 2]{edwards:1990}: \textcite{mechanismsOfJobStressAndStrain:1982} beziehen sich explizit auf den P-E fit, aber es gibt auch viele andere Studien, die den P-E fit implizit behandeln --> S-V fit ist implizit in \textcite{schuler:1980}s Konzeptualisierung von Stress --> Diese enthält eine dynamische Bedingung, welche das Individuum potentiell davon abhält das zu sein, haben oder tun was sie oder er will (Desires) / Auch das kybernetische Framework von \textcite{cummings:1979} gibt an, dass eine Diskrepanz zwischen dem bevorzugten Status des Individuums und dem aktuellen Status in strain resultiert \\
- \cite[S. 2f.]{edwards:1990}: Der D-A fit erscheint in McGraths Stressmodell (Quelle nicht gefunden) --> Besagt, dass Stress durch Anforderungen des Umfeldes entstehen, welche die Fähigkeiten und Ressourcen einer Person übersteigen \\
- \cite[S. 3]{edwards:1990}: \textcite{karasek:1979} sagt, dass strain auftritt, wenn hohe Anforderungen mit geringen Fähigkeiten zur Beeinflussung der Aufgaben kombiniert werden (z.B. geringe Entscheidungsfreiheit) \\
- \cite[S. 3]{edwards:1990}: Transaktionales Stressmodell von Lazarus und seinen Kollegen (u.a. \textcite{lazarus:1978}) ist bemerkenswert, da es sowohl Elemente von D-A als auch S-V enthält --> Beim D-A fit gehen die Forscher davon aus, dass Stress in einer Beziehung zwischen Mensch und Umgebung entsteht, bei der die Anforderungen der Umgebung von der Person als dessen Ressourcen belastend oder übersteigend eingeschätzt werden oder dessen Wohlbefinden gefährdet / Die Werte-Komponente des S-V fit erscheint im Konzept der Commitments, welches ein Muster von Zielen, Motiven und Werten darstellt, welche von der Person gehalten werden / Laut Transaktionalem Modell werden Situationen in denen Anforderungen Ressourcen beschneiden oder übersteigen (D-A misfit) nur als stressig wahrgenommen, wenn die Anforderungen die Commitments einer Person verstärken oder erhalten (nach \textcite{harrison:1978} 1978 wird der S-V misfit gelöst oder abwendet) / Transaktionales Modell integriert D-A und S-V fit, gibt an, dass stressbezogene Transaktionen zwischen Person und Umgebung sowohl D-A als auch S-V misfit beinhalten \\
- \cite[S. 3]{edwards:1990}: Konzept des P-E fit gibt es auch in anderen Bereichen der Forschung zu organisationalem Verhalten --> z.B. bei \textcite{locke:1969} --> Job Zufriedenheit kommt von der Wahrnehmung, dass ein Job wichtige Job-Werte erfüllt \\
- \cite[S. 3]{edwards:1990}: Auch die Work Adjustment Theory von \textcite{workAdjustment:1964} zeigt, dass Zufriedenheit durch eine Korrespondenz zwischen jemandes Werten und verfügbaren Verstärkungsmustern auf der Arbeit entstehen \\
- \cite[S. 3]{edwards:1990}: Der S-V fit kommt auch in der Job Charakteristiken Theorie nach \textcite{hackmanOldham:1987} vor --> Besagt, dass Motivation und Zufriedenheit entstehen, wenn Individuen mit einem starken Bedürfnis nach persönlichem Wachstum mit bereichernden (enriched) Jobs kombiniert werden \\
- \cite[S. 3]{edwards:1990}: Der D-A fit unterliegt laut Schneider 1978 und Smith und Robertson 1989 (noch nicht nach Quellen recherchiert) dem am verbreitetsten Personalauswahl-Modell --> Job Anforderungen analysieren, benötigte Fähigkeiten definieren und Personen anstellen, die diese Fähigkeiten beherrschen (Anmerkung von mir: Genauso arbeiten heute auch die meisten Recommender Systeme) \\
- \cite[S. 3]{edwards:1990}: Dieses Paper fokussiert auf Probleme, die bei anderen Papern bzgl. Stress aufgetreten sind --> lässt sich aber auch auf andere Bereiche übertragen \\

\section{Theoretische Probleme}
\label{ch:notizen:theoretischeProbleme}
- \cite[S. 3f.]{edwards:1990}: Es gibt zwei Versionen des P-E fits: S-V fit und D-A fit --> Werden manchmal gemeinsam unter der Rubrik P-E fit zusammengefasst, unterscheiden sich aber fundamental in ihren unterliegenden Prozessen und ihren assoziierten Outcomes --> \cite[S. 4]{edwards:1990}: Liegt an den zugrunde liegenden Komponenten --> Der S-V fit empfiehlt einen Prozess bei dem das Individuum aus seiner persönlichen Wertestruktur schöpft, um die Umgebung damit zu evaluieren / Beim D-A fit sammelt das Individuum dagegen seine Fähigkeiten und Fertigkeiten, um die Anforderungen der Umgebung zu erfüllen --> Prozesse sind getrennt voneinander \\
- \cite[S. 4]{edwards:1990}: Unterschiedliche Outcomes werden in verschiedenen Arbeiten diskutiert, z.B. Wenn Angebote der Umgebung von den individuellen Werten abweichen, entsteht Unzufriedenheit bei \textcite{locke:1969} / Im Gegensatz dazu: Übersteigen die Anforderungen der Umgebung die persönlichen Fähigkeiten, werden laut \textcite{theoryOfBehaviorInOrganizations:1980} Leistungseinbußen (Performanceeinbußen) wahrscheinlich \\
- \cite[S. 4]{edwards:1990}: Manchmal kann der D-A fit indirekt das Wohlbefinden beeinflussen, wenn das Erreichen der Anforderungen der Umgebung ein inhärenter Wert des Individuums ist (und dadurch ein S-V fit entsteht) oder wenn die Auflösung einer D-A Diskrepanz ein Instrument zur Erreichung eines S-V fits in einer verwandten Dimension ist (sagt \textcite{mechanismsOfJobStressAndStrain:1982}) \\
- \cite[S. 4]{edwards:1990}: Im Unterschied dazu zeigen Beweise, dass es unwahrscheinlich ist, dass der S-V fit die Performance beeinflusst (siehe: \textcite{greene:1972} und \textcite{schwabCummings:1970}) \\
- \cite[S. 4]{edwards:1990}: Ursprünglich waren S-V und D-A komplett unterschiedliche Versionen des P-E fits (siehe \textcite{copingAndAdaption:1974}, \textcite{mechanismsOfJobStressAndStrain:1982}, \textcite{harrison:1978}) --> Unterscheidungen in unterliegenden Prozessen und den assoziierten Outcomes --> Nachfolgende Studien haben diese Unterscheidungen minimiert oder manchmal sogar ganz übersehen --> Manche haben z.B. S-V und D-A als alternative Vorhersageelemente für die selben Outcomes gesehen, z.B. strain (\textcite{jobDemandsAndWorkerHealth:1975}, \textcite{mechanismsOfJobStressAndStrain:1982}) --> Deren Begründung (siehe \textcite[S. 31]{mechanismsOfJobStressAndStrain:1982}): Effekte S-V und D-A basieren auf dem Ausmaß, zu dem Motive befriedigt werden --> Edwards sagt, dass die Trennung aber wichtig ist, weil sonst fragwürdig ist, ob man z.B. D-A überhaupt braucht --> Auch viele andere Autoren betrachten S-V und D-A als austauschbar oder verwechseln sie sogar  --> Edwards ist der Ansicht, dass die Trennung zwischen D-A und S-V bei Untersuchungen erhalten bleiben muss --> Siehe hierzu auch \textcite{mechanismsOfJobStressAndStrain:1982} \\
- \cite[S. 5]{edwards:1990}: Generell: Beziehung zw. S-V und D-A und ihre Auswirkungen auf Wohlbefinden und Performance würden stärker erleuchtet werden, wenn beide Formen in einer Untersuchungsdomäne betrachtet werden (z.B. S-V und D-A fit bzgl. den selben Jobcharakteristiken) \textcite{caplan:1987}

\section{Formen des P-E fit}
\label{ch:notizen:formen}
- \cite[S. 5]{edwards:1990}: Es gibt drei grundlegende Formen des P-E fits (bzgl. Stress)
- \cite[S. 5]{edwards:1990}: Erste Form fokussiert die Differenz zwischen vergleichbaren P und E Komponenten --> Umso größer die Differenz, desto größer der strain --> Wird z.B. verwendet von \textcite{mechanismsOfJobStressAndStrain:1982} und vielen anderen \\
- \cite[S. 5]{edwards:1990}: Zweite Form fokussiert die Interaktion zwischen P und E --> Stress tritt au, wenn Eigenschaften der Umgebung mit bestimmten persönlichen Eigenschaften kombiniert werden --> so z.B. vorgestellt von \textcite{cherringtonEngland:1980}, \textcite{lyons:1971}, \textcite{obrien:1980} und anderen --> Operationalisieren den P-E fit als Produkt übereinstimmender P und E Komponenten \\
- \cite[S. 5]{edwards:1990}: Dritte Form betrachtet den Anteil/Verhältnis P, der von E erfüllt wird (manchmal wird auch die Differenz betrachtet) --> Strain nimmt zu, wenn die Proportion geringer wird --> So z.B. verwendet von \textcite{mechanismsOfJobStressAndStrain:1982} und anderen \\
- \cite[S. 5]{edwards:1990}: Viele Autoren betrachten die verschiedenen Formen als austauschbar oder als vergleichbar /  \textcite{mechanismsOfJobStressAndStrain:1982} sagen, dass proportionale Form über Diskrepanz verwendet werden sollte, wenn Ratio-Skalen verfügbar sind, sagt aber auch, dass sich die Ergebnisse nicht nennenswert unterscheiden \\
- \cite[S. 5]{edwards:1990}: Nähere Betrachtung der drei Formen zeigt, dass die drei Formen unterschiedliche theoretische Perspektiven auf die Beziehung zwischen P-E fit und strain haben / Die Diskrepanz-Form betrachtet P als einen Standard, gegen die E vergleichen wird --> Größere Abweichung wird dann mit mehr Stress assoziiert / Die Interaktive Form impliziert, dass die Stärke der Beziehung zwischen E und strain beeinflusst --> P ist also weniger ein Standard --> P modifiziert die Auswirkungen von E auf Stress / Proportionale From teilt Charakteristiken der beiden anderen Varianten --> P ist ein Standard gegen den E vergleichen wird und beefinlusst die Stärke der Beziehung zwischen E und strain / Im Gegensatz zur interaktiven Form impliziert die proportionale Form, dass der Effekt von P auf die Stärke der Beziehung zwischen E und strain fortlaufend kleiner wird, wenn P sich vergrößert \\
- \cite[S. 5]{edwards:1990}: Obwohl die Unterscheidungen eher theoretisch sind, zeigen sie auch wichtige methodische Implikationen --> Liegt daran, dass die drei Formen fundamental unterschiedliche funktionale Beziehungen zwischen P, E und strain zeigen --> Können gut als Oberflächen in einem 3D Raum betrachtet werden --> \cite[S. 7]{edwards:1990}: Obwohl die drei Formen des fits alle fundamental unterschiedlich sind, gibt es in der Literatur einige mathematische Transformationen, die diese Unterscheidungen undurchsichtig machen oder ganz entfernen --> Beispiel an einem Autor: Wollte die Diskrepanz durch die squared Differenz zwischen P und E zur Vorhersage von Stress bestimmen --> Diese Berechnung entspricht jedoch der interaktiven Form / Erkenntnis: Die drei Formen sind mathematisch und theoretisch verschieden --> Man muss sich überlegen welche Hypothese man zum Einfluss P und E auf Stress aufstellt und dann die drei Formen nicht als austauschbar betrachten --> Sonst entstehen statistische Tests, die nicht die ursprüngliche Hypothese prüfen --> Form muss mit den theoretischen Annahmen übereinstimmen

\section{Methodische Probleme}
\label{ch:notizen:methodischeProbleme}
- \cite[S. 7]{edwards:1990}: Es gibt zwei wichtige methodische Probleme in den P-E fit Studien bzgl. Stress: Erste bezieht sich auf die Messung er P und E Komoponenten; Zweite betrachtet die Analyse der Beziehung zw. P E und strain

\subsection{Messung der fit Komponenten}
\label{ch:notizen:methodischeProbleme:messungDerFitKomponenten}
- \cite[S. 8]{edwards:1990}: Messung der P und E Komponenten sollte vergleichbar sind --> Heißt: Sollten dieselbe theoretische Dimension haben --> \textcite{copingAndAdaption:1974} sagen, dass vergleichbare Messungen notwendig für Differenz-Berechnung sind --> Das ist laut \textcite{jobDemandsAndWorkerHealth:1975} die meistverwendete Prozedur in der P-E fit Forschung --> Differenzberechnungen werden meist verschiedenen Transformationen ausgesetzt, um den Punkt des 'Perfect Fit' zu bestimmen \textcite{jobDemandsAndWorkerHealth:1975} / Wenn diese Prozeduren genutzt werden, müssen die P und E Messungen nicht nur vergleichbar sein, sondern auch noch denselben Nullpunkt teilen --> sonst ist die Bestimmung des Punktes bedeutungslos \\
- \cite[S. 8]{edwards:1990}: Zweitens: Fragebögen, welche zur Messung der P und E Komponenten genutzt werden --> Sowohl bei SV als auch DA sind Fragebögen bzgl. E einfacher --> S-Fragebögen sollten fragen wie sehr das Attribut vorhanden ist, wohingegen D-Fragebögen sollten das Level (Höhe) der Demands erfragen, die mit einem betrachteten Attribut verbunden sind (vgl. \textcite{jobDemandsAndWorkerHealth:1975})/ Fragebögen zu P sind komplexer: Zu V gibt es zwei Möglichkeiten: Einmal zu erfragen, zu welchem Level die Attribute erfüllt sein sollen (\textcite{jobDemandsAndWorkerHealth:1975}) oder die Wichtigkeit der Attribute (\textcite{workAdjustment:1964}) --> Theoretische Diskussionen ergeben, dass der erste Ansatz eher für die Diskrepanz-Form und der zweite für die Interaktive Form geeignet ist (\textcite{copingAndAdaption:1974} und mehr Quellen) --> Es gibt auch Studien, die es anders bzgl. Diskrepanz machen (z.B. V durch Wichtigkeiten messen) oder manche Studien der interaktiven FOrm messen V in Desires --> Ergebnis: Ergebnisse dieser Studien lieern keine klare Interpretation über die grundlegenden Priznzipien des P-E fits / Für A veranschlagen Fragebögen oft ein Selbstassement der Fähigkeiten  oder indirekte Indikatoren der Fähigkeiten wie z.B. Bildung (\textcite{jobDemandsAndWorkerHealth:1975}) --> Vorteil des ersten Ansatzes: Man kann besser das Konstrukt messen, das einen tatsächlich interessiert, dafür ist der Ansatz anfällig für eine Antwortverzerrung durch soziale Erwünschtheit (social desirability response bias) --> Dieser Bias kann durch die zweite Methode vermieden werden, bietet aber eine weniger direkte Messung des Konstrukes, das einen interessiert --> Gibt keine klare Lösung, aber man sollte die Trade-Offs im Hinterkopf behalten \\
- \cite[S. 8f.]{edwards:1990}: Drittes Problem beachtet die Anzahl an fit-Dimensionen, die in die Studie eingebunden werden / Manche Studien (Quellen) verwenden nur eine Dimension / Sogar die systematischsten Untersuchungen zur Zeit von \cite{copingAndAdaption:1974} und \cite{mechanismsOfJobStressAndStrain:1982} enthalten nur acht Dimensionen des Fits / Wenn es nur wenige Dimensionen gibt, haben solche Studien zwei Nachteile --> 1. Wenn man annimmt, dass eine Inkongruenz über mehrere Dimensionen den strain stärker beeinflusst, vernachlässigen diese Studien notwendigerweise relevante Bestimmungsgrößen von Strain; 2. Diese Studien bieten nur limitierte Informationen über den P-E fit als generelles Konstrukt / Übergeordnete Ansätze involvieren umfassende Messungen von Person und Environment, um Indizes des fits zu bestimmen --> z.B. haben viele Studien zur Jobzufriedenheit den Work Values Inventory verwendet, um Indizes entlang von 15 Dimensionen zu bestimmen (Quellen genannt); Alternativ kann man auch Arbeitnehmer interviewn, um Job-relevante Aktivitäten und Konstrukt-korrespondierende Indizes des Fitszu erhalten --> Beide Prozeduren werden eine Einschätzung des fits bieten und sollten wenn möglich implementiert werden

\subsection{Analyse der Effekte des fits}
\label{ch:notizen:methodischeProbleme:analyseDerEffekteDesFits}
- \cite[S. 9]{edwards:1990}: Hier treten die größten Probleme auf / Die meisten Arbeiten bei der Diskrepanz-Form operationalisieren den Fit als algebraische Distanz (oder einer Transformation davon) zwischen vergleichbaren P und E Komponenten --> Ansatz erscheint intuitiv, die Verwendung von Differenz-Werten wurde aber stark kritisiert (einige Quellen) --> Ein Differenz-Score ist eine einfache Linearkombination seiner Komponenten --> \\
DAS NOCHMAL ANSEHEN MIT LÖSUNGSVORSCHLÄGEN

\section{ToDo}
\label{ch:todo}
- Was ist organisationales Verhalten \\
\shorthandon{"}