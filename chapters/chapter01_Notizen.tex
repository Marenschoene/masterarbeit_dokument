\shorthandoff{"}
\chapter{Einführung (2 Seiten)}
\label{ch:intro}

\section{Problemstellung}
\label{sec:intro:motivation}
In Unternehmen aller Branchen beobachten Wissenschaftler eine Abkehr von großgewachsenen, zentralen und hierarchischen Organisationsstrukturen. Stattdessen setzen Betriebe zunehmend auf kleine, dezentrale und flexible Teams, welche in Projektarbeit neue Produkte entwickeln und Dienstleistungen erbringen \cite[S. 15]{elanceEconomy:1999}\cite[S. 8]{whittington:1999}. Besonders stark nimmt dieser Trend seit Mitte der 1990er Jahre zu \cite[S. 8]{whittington:1999}. Ein Hauptgrund für den Wandel ist die fortschreitende Digitalisierung. Diese macht zentrale, hierarchische Unternehmensstrukturen durch kurze Kommunikations- und Entscheidungswege zunehmend überflüssig. Gleichzeitig ermöglicht sie höhere Kreativität und Flexibilität in den Projektteams \cite[S. 17]{elanceEconomy:1999}.\\
Die deutsche Gesellschaft für Projektmanagement e.V. stellt fest, dass ein durchschnittlicher Angestellter in Deutschland bereits im Jahr 2013 mehr als ein Drittel seiner Arbeitszeit mit Projekttätigkeiten verbrachte. Der Verein erwartet, dass die Projektarbeit in Zukunft einen noch größeren Anteil in Anspruch nehmen wird \cite[S. 16]{gpm:2015}. Mitarbeiter befinden sich folglich nicht mehr über mehrere Jahre hinweg in einer Abteilung des Unternehmens, sondern finden sich immer wieder zu neuen, kurzlebigen Projekten zusammen. Thomas W. Malone und Robert J. Laubacher stellen dabei sogar in Frage, ob Unternehmen in Zukunft überhaupt noch existieren werden. Sie prognostizieren die Entstehung einer "E-Lance Economy". Dabei handelt es sich um eine Gesellschaft von elektronisch verbundenen Freiberuflern, welche die Wissenschaftler als "e-lancer" bezeichnen. Alle Marktteilnehmer sind  rechtlich selbstständig und finden sich immer wieder zu neuen temporären Netzwerken zusammen, um gemeinsam an Projekten zu arbeiten \cite[S. 14]{elanceEconomy:1999}. Auch wenn dieses Szenario noch in ferner Zukunft liegt, zeichnet es sich ab, dass die Zusammenstellung von Teams für einzelne Projekte ein immer häufiger stattfindender Prozess in der Wirtschaft sein wird \cite[S. 2]{malinowski:2008}. Gleichzeitig zeigen repräsentative Studien, dass die Personalabteilungen deutscher Unternehmen schon heute bei der Besetzung von Stellen überlastet sind \cite[S. 244]{personalbeschaffung:2017}. So fällt es Personalern in bekannten Unternehmen schwer, die hohe Anzahl an Bewerbern zu überblicken, wohingegen die HR-Abteilungen weniger bekannter Organisationen häufig über zu wenig Angestellte für eine ausreichende Prüfung der Unterlagen verfügen \cite[S. 8]{hays:2013}. Diese Überlastung der Personalabteilungen wird durch fehlende Automatisierungsangebote bei der Zusammenführung von Bewerbern zu Stellen zusätzlich verschärft \cite[S. 15]{hays:2013}.



- \cite{hays:2013}: Bekannte Arbeitgeber: Bewerberflug --> Ziel: Schnelles Filtern der richtigen Bewerber / KMU: Zu wenig Personal / Alle Unternehmesgroessen (S. 8): Zu wenig Kapazität in HR-Bereich und Auswahl der richigen Kandidaten / S. 15: Größte Herausforderung bei Vorauswahl: Herausfiltern geeigneter Kandidaten, Hoher Aufwand, Fehlende Automatisierung \\
- \cite{malinowski:2008}: Projekt- und Teambasierte Arbeitsstrukturen gewinnen an Bedeutung / Informationssysteme in HR sorgen für enorme Kostensenkung und gleichzeitig für Erhöhung der Produktivität  / Quellen 2, 49, 81: Viele Unternehmen haben sich schnell ändernde Org-Strukturen --> Flexible Arbeit und Projekte gewinnen an Bedeutung --> Und Teambasierte Arbeitsstrukturen / Quellen 7, 51, 73, 75, 82: Alle Org-Typen machen mehr Gebrauch von Teamarbeit / 48: Malone sagt E-Lance-Economy voraus --> Herausforderung: Teams müssen häufiger zusammengestellt werden / Unternehmen, die auf Projektbasierte Arbeit fokussiert sind (z.B. Beratung) haben schon länger Teambasierte Strukturen --> In solchen Unternehmen müssen Mitarbeiter häufig in neue Projekte gestafft werden --> Quellen 21 und 51 erwarten, dass sich dieser Trend auch auf andere Organsisationen überträgt --> Autor schließt daraus, dass Teamstaffing wichtiger wird und ein häufiger ausgeführter Prozess wird / Quelle 37 zeigt in empirischer Studie mit den Top1000 Unternehmen in Deutschland, dass der Einsatz von IT-Systemen im Personal-Marketing für mehr als 30 Prozent Kostenersparnis und im Schnitt 26 Prozent Zeitersparnis sorgen kann / 44 zeigt, dass IT-Unterstützung bei großen Unternehmen wichtig ist --> Basiert auf interner Skill-DB (diese ist sehr wichtig)\\
- \cite{elanceEconomy:1999}: S. 15: Devolution von großen, permanenten Organisation in flexible, zeitlich beschränkte Netzwerke von Individuen / S. 24: Neue Art von Economy: Electronically connected Freelancers (E-lancers) --> Verbinden sich zu temporärem und fluidem Netzwerk , um Güter zu produzieren und verkaufen --> Netzwerk löst sich auf, sobald der Job erledigt ist und jeder sucht neuen Auftrag / S. 25: Aufkommen virtueller Unternehmen; In Großunternehmen nimmt die Bedeutung von Projektteams und Intrapreneurs zu / S. 27: Trend wird durch PCs und elekronische Netzwerke ausgelöst --> Individuen können sich selbst managen --> Keine zentrale Steuerung mehr benötigt --> Vorteile der kleinren Einheiten: Flexibilität und Kreativität / Netzwerke werden effizienter \\
- \cite{lundin:2015}: S. 20: Projekte wurden in Menschheitsgeschichte schon immer für wichtige Konstruktionen wie Pyramiden verwendet / Seit 1930er Jahren wird die Projektorganisation zu einer wichtigen Form / Zunahme insbesondere in den 1960er und 1990er Jahren --> Studie von Whitington et al 1999 zeigt, dass alleine von 1992 bis 1994 der Anteil von 13 auf 42 Prozent gestiegen ist / PWC 2004 fand heraus, dass die meisten Projekte im Zusammenhang mit Innovation und IT stehen / Etwa 40\% der globalen Wirtschaft ist projektbasiert und nutzen Projektmanagement als primären Prozess, um Produkte und Dienstleistungen herzustellen (Turner 2010) / Projektbasiertes Arbeiten gibt es in allen Branchen \\
- \cite{towardsADesign:2017}: Puranam et al 2014: Projektbasierte Organisation ist eine neue Form der Organisation\\
- \cite{personalbeschaffung:2017}: Kruster schließt aus Hays-Studie, dass die meisten Unternehmen überfordert sind \\
- \cite{futurePerspectivesOnEmployeeSelection:2004}: Hauptherausforderung laut Herriot und Anderson 1997: "Bimodal Prediction" --> MA-Auswahl beruht auf Annahe, dass die Job-Rolle relativ stabil belibt --> Laut Howard 1995 ändern sich die Organisationsstrukturen aber so schnell, dass es zu einer Job-Instabilität kommt / FOJA: Future-Oriented Job Analysis
- \cite{malinowski:2006}: Quelle 23 etablierten den Begriff RS im Jahr 1997 \\

\section{Zielsetzung}
\label{sec:intro:zielsetzung}
- Entwicklung eines Recommender Systems \\
- Forschungsfrage --> - Ziel: Literatur (Stand der Technik) zu diesem Thema wiedergeben

\shorthandon{"}