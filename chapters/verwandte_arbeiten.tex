\definecolor{exxetagray}{gray}{0.75}
\definecolor{itemcolor}{RGB}{179,217,255}
\definecolor{usercolor}{RGB}{255,204,179}

\shorthandoff{"}
\chapter{Verwandte Arbeiten}
\label{ch:verwandte_arbeiten}


\section{Reziproke Systeme}

\subsection{Systeme mit identischer Gewichtung der Präferenzen}
% d.h. wie sehr zwei personen übereinstimmen, beides gleich gewichtet
% Dual perspektive graph representation hier: S. 104, file://wsl%24/Ubuntu/home/masc6/Projects/masterarbeit/literatur/recsys%202022%20modeling%20two%20way%20selection%20preference%20for%20person%20job%20fit.pdf

\subsection{Systeme mit unterschiedlicher Gewichtung der Präferenzen}
% präferenz eines nutzers höher gewichtet als präferenz der empfohlenen Elemente
% feature optionen können 3 unterschiedliche stadi annehmen. -> classification (label mit match und kein match) S. 68, file://wsl%24/Ubuntu/home/masc6/Projects/masterarbeit/literatur/DiazMetzlerAmer-Yahia%20-%20Relevance%20and%20Ranking%20in%20Online%20Dating%20Systems%20(2010)%20-%200.pdf
% Johannes Thesis

\section{Multi-kriterielle Systeme}
hier bsp anführen für arbeiten die sich mit mk systemen beschäftigen in den 3 bereichen
% Hier erwähnen, dass nicht alle systeme auch als mk-systeme bezeichnet sind, die solche Ansätze anwenden (bspw. hybride systeme)
% Hier erwähnen, dass Fokus auf aggregation mehrerer Kriterien und deren gewichtung liegt -> hiernach suchen

% Verwendung von MAUT (siehe S. 432, file://wsl%24/Ubuntu/home/masc6/Projects/masterarbeit/literatur/Analysis%20and%20Classification%20of%20Multi-Criteria.pdf)
% MAUT in utility based recommenders: file://wsl%24/Ubuntu/home/masc6/Projects/masterarbeit/literatur/Designing%20utility-based%20recommender%20systems%20for%20e-commerce.pdf

\subsection{Multi-attribut basierte Systeme}
% hybrid recommender: weighted -> S. 339, https://link.springer.com/content/pdf/10.1023/A:1021240730564.pdf?pdf=button

\subsection{Multi-objektive Systeme}
% multi-objective example: S. 237, file://wsl%24/Ubuntu/home/masc6/Projects/masterarbeit/literatur/Adaptive%20multi%20attribute%20diversity%20for%20recommender%20systems.pdf

\subsection{Multi-kriterielle Bewertungen in Empfehlungssystemen}
% Erfinder des aggregation Funktion approaches: file://wsl%24/Ubuntu/home/masc6/Projects/masterarbeit/literatur/New_Recommendation_Techniques_for_Multicriteria_Rating_Systems.pdf
% Arbeit von Jannach wie hier beschrieben mit SVM: S. 102, file://wsl%24/Ubuntu/home/masc6/Projects/masterarbeit/literatur/E-Commerce.pdf
% multi-kriteria: Aggregation function approach: file://wsl%24/Ubuntu/home/masc6/Projects/masterarbeit/literatur/Recommending%20Hotels%20based%20on%20Multidimesional%20Customer%20Ratings.pdf
% ansatz von Tang: zusammenfügen mehrere feature bewertungen, sowie durchschnittsbewertung und kommentare, file:///C:/Users/masc6/Downloads/tdladmin,+TangandMcCalla_JoDI_final.pdf
% Slope one algorithmus und adaptive genetic algorithm von Hassan, S. 327, file://wsl%24/Ubuntu/home/masc6/Projects/masterarbeit/literatur/Imrpoving%20Prediction%20accuracy%20of%20multi-criteria%20recommender.pdf
% Liu et al.: Cluster von Nutzern, welche wichtigkeit einzelner Attribute angeben, zusammenfassung siehe hier S. 4, file:///C:/Users/masc6/Downloads/79_HDIOUD.pdf
% CCSD method, siehe: file:///C:/Users/masc6/Downloads/79_HDIOUD.pdf
% Wenn viele features: liwei et al (zsfsg. siehe hier: file://wsl%24/Ubuntu/home/masc6/Projects/masterarbeit/literatur/A%20Multi-criteria%20Recommender%20System%20Incorporating%20Intensity%20of%20Preferences.pdf )

% was gibt es also nicht? -> Reziprozität als gewichtetes Kriterium (Präferenz der empfohlenen Person zählt nicht genauso viel wie Präferenz des Nutzers, kann daher als gewichtetes Kriterium einer Aggregation Function betrachtet werden, welche über historische daten erlernt werden kann)
% Neue anwendungsbereiche von multi-kriteriellen EMpfehlungen in domänen wie online dating: zitat von 
\shorthandon{"}