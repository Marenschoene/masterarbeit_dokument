\shorthandoff{"}
\chapter{Verwandte Arbeiten}
\label{ch:verwandteArbeiten}
Dieses Kapitel behandelt bilaterale Empfehlungssysteme. Diese Art von Anwendungen kombinieren die in Kapitel \ref{ch:personEnvironmentFit} betrachteten Erkenntnisse zum Konzept des \acp{PEFit} mit denen in Kapitel xyz vorgestellten Implementierungsmethoden von Empfehlungssystemen (Quelle).

\textcite{malinowski:2008}:

S. 1\\
- System zur Auswahl von Individuen für Teams
- Bestehende Systeme betrachten nur Skills - Match zwischen Person und Teammitgliedern wird nicht beachtet
- Entwickeln relationales Empfehlungssystem zur automatisierten Vorauswahl von Kandidaten, die am besten zu zukünftigen Teammitgliedern passen

S. 2\\
- Manche Industrien wie Consulting arbeiten schon länger Projektorientiert

S. 3\\
- Beziehen sich auf den person-environment (P-E) fit
- IS-Unterstützungsansatz benötigt 2 Dimensionen: 1. Person-Job und 2. Person-Team
- Fokus auf internes Team-Staffing

S. 4\\
- Edwards sagt, dass P-J fit aus zwei Klassen besteht: 1. Wünsche und 2. Demand-Abilities
- Einige Autoren sagen P-T ist gut wenn suplementär, andere sagen wenn komplementär --> Werbel und Johnson sagen, dass beide beachtet werden müssen

S. 5\\
- Bestehende HR-Systeme verfügen über eine interne Skill-Datenbank, über welche über Abfragen Kandidaten mit passenden Eigenschaften ermittelt werden können\\
- 3 Nachteile bestehender Systeme: 1. Einfaches Schlüsselwort-Matching; 2. Viele fokussieren ausschließlich auf Demand-Abilities; 3. Keine Beachtung des P-T Fits

S. 6\\
- Setzen hohes Vertrauen zwischen Teammitgliedern voraus --> Wichtiger Indikator für P-T Fit
- setzen auf Kontepte der sozial network analysis (SNA)
- Ziel: Vorhersage von Vertrauen zwischen Personen, die sich nicht kennen
- System benötigt zwei Dimensionen: 1. Unäre Attribute wie Fähigkeiten, um Fit zwischen Person und Task zu bestimmen; 2. Relationale Attribute, um den fit zwischen Individuum und Teammitgliedern zu bestimmen --> in beiden Fällen müssen auch die Needs einbezogen werden --> Daraus ergeben sich die 3 Haupt-funktionalen-Anforderungen: 1. Multilateraler Prozess, sodass Präferezen mehrerer Personen einbezogen werden müssen; 2. Muss Fit zwischen Person und anderen Teammitgliedern beachten; 3. Individuen können nicht in mehreren Teams zur selben Zeit arbeiten
- Entwickeln ein Vertrauens-Berchnungs-Modell, welchen das Vertrauen zwischen zuvor unbekannten Individuen vorhersagt; Beachten außerdem soziales Kapital der Kandidaten in Ergänzung zu deren menschlichem Kapital

S. 7\\
- Annahme: Trust kann als ein Wert ausgedrückt werden
- Vertrauen unterscheidet sich von Person zu Person, deshalb bildet jede Person ihr eigenes "Web of Trust" --> Wichtig, da Vertrauen von einem Kandidaten zu jedem anderen Teammitglied und umgekehrt bestimmt werden müssen --> Mann kann nicht einfach davon ausgehen, dass Person A Person B vertraut, nur weil es umgekehrt der Fall ist
- Vertrauen ist ein Wert zw. 0 und 1 --> 0 = Gar kein Vertrauen, 1 = Vertrauenswürdig
- $t'_{AC} = t_{A->B->C}$ heißt: A vertraut C über den Pfad über B --> Genannt "Vertrauenspfad" --> Pfad muss direkt sein
- Es muss auch möglich sein, Vertrauen zu berechnen, wenn es keine direkte Verbindung gibt --> Wird über Nutzer-Attribute bestimmt (Content)

S. 8\\
- Färber wendeten ein Probailistic Latent Aspect Model (PLSA) an, um Personen für Stellen zu empfehlen --> Gehört zu den modelbased Methoden
- Hier werden die Präferenzen der Nutzer als konvexe Kombination unterliegender latenter Aspekte betrachtet --> Die verschiedenen Aspekte, die einer Bewertung unerliegen können modelliert werden, was zu guten Empfehlungen führt

S. 9\\
- Kollaborative Trust-Vorhersage:
- Über Kollaboratives Filtern sollen Kanten (Vertrauen) vorhergesagt werden --> Voraussetzung: Explizite Vertrauensbewertungen benötigt
- Um Vertrauensbeziehungen zu modellieren wurde ein PLSA Modell in den Kontext der relationalen Empfehlung adaptiert --> Erlaubt es, latente Aspekte zu erfassen, welche einer Vertrauensbewertung von Personen unterliegen
- Modell-Parameter werden durch den Expectation Maximization Algorithmus bestimmt welcher der Standardalgorithmus für Maximum Lielihood Estimation in Latenten Variablen Modellen ist --> Hat 2 Schritte: --> Ergeben, mit welcher Wahrscheinlichkeit Person A Person B zu einem bestimmten Wert vertraut

S. 10\\
- Bestimmten ein Latent Aspect Model, um Jobs für Kandidaten basierend auf vorherigen Bewertungen zu empfehlen --> Modell wurde genutzt, um Ähnlichkeiten zwischen Kandidaten zu bestimmen basierend auf vorherigen Job-Bewertungen
- Sie erstellten Segmente von Kandidaten mit ähnlichen Präferenz-Strukturen basierend auf den Latenten Aspekten, welche sie aus vorherigen Job-Profilen ermittelten
- Ein Nutzer kann auch zu mehreren Segmenten gehören --> Unterschied zu Clustering, wo ein Nutzer immer nur in einem Cluster siein kann

S. 11
- Berechnung der Differenz von Wahrscheinlichkeiten, dass zwei Nutzer zum Selben Segment gehören, welches von latent Variable z gebildet wird --> Differenzen werden aufaddiert für alle Latent Aspekte und durch die Anzahl an Segmenten geteilt --> Gibt es kein gemeinsam Bewertetes Job-Profil, ist der Ähnlichkeitswert 0 --> $t'_{AB}=sim_{AB}$
- Angenommen es gibt die drei Nutzer A, B und C und diese haben sich nicht explizit bewertet, haben aber eine Ähnliche Präferenzstruktur, was die Bewertungen angeht --> Diese Präferenzstrukturen werden genutzt, um einen Ähnlichkeitsbasierte Trust-Werte zu bestimmen

- Oben beschriebene Szenarien werden kombiniert ausgeführt, sodass mehr als ein Trust-Pfad zwischen Individuen existieren kann --> Bestimmten Durchschnitt zur Aggregation
- Grad an Vertrauen wird als höher bewertet, wenn die Anzahl an --> Daher noch Verrechnung mti der allgemein Vertrauenswürdigsten Verbindung des Zielnutzers

S. 12\\
- Über die Pfade wurde dann der durchschnittliche Vertrauenswert von allem Gruppenmitgliedern zum neuen Mitglied und umgekehrt bestimmt --> Wie verrechnet wird nicht gesagt
- System soll nur Vorauswahl treffen --> Finale Entscheidung trifft HR

S. 13\\
- HR-Manager erhielten Liste mit relevanten Kandidaten --> Aus diesen kann HR dann auswählen
- Erstellten eine ROC-Kurve: Sensitivity (hit rate) war die Wahrscheinlihkeit, dass ein relevanter Kandidat empfohlen wurde; 1-Sensitivity (miss rate) war die Wahrscheinlichkeit, dass ein irrelevanter Kandidat empfohlen wurde
- ROC-Kurve plottete die Miss-Rate auf der x-Achse gegen die Hit-Rate auf der y-Achse --> Cut-Off-Value definiert die Suchlänge und bestimmt wie viele der Top-Kandidaten in der Liste aktuell betrachtet werden , wenn die Genauigkeit bewertet wird
- N=21 Studenten von zwei Universitäten
- 1. Phase: Studenten erhalten 100 echte Job-Profile und sollten auf einer Skala von 1 bis 5 bewerten, wie sehr die Profile ihre Präferenzen bzgl. mittel- oder langfristigen Karriereperspektiven erfüllen --> Job-Präferenzdaten wurden als Inputdaten für den Recommender genutzt
- 2. Phase: Studenten fügten Informationen über die Beziehungen zu anderen Teilnehmern hinzu --> Sollten Vertrauen ausdrücken --> 1 bis 5
- Relationales RS wurde mit einem Subset der Daten trainiert --> Trainings- und Testdaten --> Vergleich mit Originaldaten, um Qualität des Systems zu messen

S. 14\\
- Auch ROC-Kurve, um Vertrauen vorherzusagen --> dazu pro Student 10 Verbindungen zufällig entfernt und vorhergesagt

Persönliche Kritik\\
- Sie können nachweisen dass ihr System eine gute Genauigkeit aufweist, was aber fehlt: Wie werden Endergebnisse verrechnet? Ist bilateraler Ansatz tatsächlich besser? Ist Ansatz überhaupt der Richtige, also wie findet HR die Ergebnisse bzw. wie finden Studenten die Ergebnisse?

\textcite{malinowski:2005}:\\
S. 3\\
- bilaterales Prozess --> Beachtet Präferenzen von HR-Experte und Kandidaten bzw. Teammitgliedern / Beachtet Verbindungen zu anderen Personen
- Färber 2003 entwickelten ein Empfehlungssystem zur Empfehlung von Kandidaten --> Dieses Modell wird in dieser Arbeit erweitert

S. 4\\
- Anmerkung: Klingt für mich so, als hätten Färber 2003 das latente Modell entwickelt, um auf Jobs zu matchen, aber die Team-Komponente nicht beachtet

S. 5\\
- Beschreibung des Trust-Modells
- Es gibt explizites Vertrauen: Explizit erfasst zB durch Fragebogen

S. 6\\
- Zwei Möglichkeiten: Multiplikation oder PLSA-Modell (kollaboratives Filtern), um den kollaborativen Trust zu bestimmen --> Ergibt eine Matrix, welche die Wahrscheinlichkeit enthält, dass ein Kandidat einen anderen Kandidaten mit dem Wert v vertraut
- Annahme aus Literatur: Personen tendieren dazu sich mehr zu vertrauen, wenn sie selbe Werte und Verhaltensweisen teilen --> Berechnen Ähnlichkeiten zwischen Personen basierend auf ihren Job-Präferenzen --> Nutzen dafür ein adaptiertes PLSA Modell, um Job-Präferenzen basierend auf zuvor bewerteten Jobs vorherzusagen --> Erstellten Segmenten von Kandidaten mit ähnlihcen Präferenz-Strukturen basierend auf den latenten Aspkten, welche sie von den bewerteten Job-Profilen erhielten --> Ein Nutzer kann zu mehreren Segmenten gehören

S. 7\\
- Eine latente Variable bildet ein Segment und sie berechneten

S. 9\\
- 1. Schritt: PLSA Modell, um Kandidaten für Stellen zu empfehlen, die zur Stelle passen --> Gibt eine Liste mit den Top N passenden Kandidaten zurück, die Input für Schritt 2 sind --> Hier wird der Trust berechnet (Hier kommt System von oben zum Einsatz )--> Danach gibt es dann zwei Listen für PT und PJ --> Verrechnung über eine Formel --> Diese ist laut Forschern noch nicht optimal --> Priorität haben Wünsche der HR-Manager

S. 10\\
- Job-Recommender wie bei Färber

S. 11\\
- 1. Schritt: Modell wird mit vergangenen Bewertungen trainiert
- 2. Schritt: Relationaler Aspekt: Vertrauensberechnung
- Danach Finale Liste

Anmerkung von mir:
- Keine Evaluation

\textcite{keim:2005}:\\
S. 3\\
- System zur personalisierten Suche nach Individuen
- Anforderungen: 1. Kandidat muss Fähigkeiten besitzen (unäre Attribute) und 2. Zusammenarbeit mit anderen Teammitgliedern muss erfolgreich sein (relationale Attribute)
- Den Kandidaten auszuwählen ist eine bilaterale Entscheidung
- Bilateraler Prozess: Präferenzen mehrerer Personen; Es müssen relationale Attribute betrachtet werden; Ein Individuum kann nur einmal ausgewählt werden

S. 4\\
- Zwei Ansätze: CV-Recommender und ein Sozial Network Browser
- CV-Recommender: Empfiehlt CVs, die ähnlich sind zu Lebensläufen, die zuvor vom selben Rekruiter für ein Job-Profil betrachtet wurde --> Basiert auf dem latent aspect Model; Bild: Recruiter und Job-Beschreibung sind Variable X, Präferenz-Faktoren sind Variable Z, Rekruiter bewertet ein Profil mit geeignet oder ungeeignet --> Nicht Person, sondern Summe der Attribute werden bewertet --> Variable V a sind die Content-Elemente des Lebenslaufs des Kandidaten
- Social Network Browser

S. 5\\
- Vertrauen kann in einem einzelnen Wert ausgedrückt werden
- 3 Arten Vertrauen zu berechnen; 1. Multiplikation, 2. Kollaboratives Filtern, 3. Ähnlihckeitsberechnung --> 3. Basierend auf Profilen werden Ähnlichkeiten berechnet --> Ähnichkeiten zwischen Nutzerpaaren --> Basierend auf Distanzen und auf den Chrakteristiken, sagt System vor aus ob Beziehung besteht

S. 7\\
- Planten Validierung über einen Studentenworkshop

\textcite{malinowski:2006}:\\
S. 1:\\
- Match zwischen Job und Kandidat muss Präferenzen von Recruiter und Kandidat berücksichtigen

S. 3\\
- Match zwischen Task und Teammitgliedern

S. 4\\
- CV-Recommender: CVs werden empfohlen, welch ähnlich zu den Lebensläufen sind, welche uvor vom selben Rekruiter für das Job-Profil ausgewählt wurden --> Hybrid: Latent Aspect Model / Variable x: Rekruiter und Job-Beschreibung; z Präferenz-Faktoren; v=qualifiziert oder nicht; a sind Attribute des Kandidaten
- S. 4f.: Job-Recommender: Zweites Empfehlungssystem, das Jobs an Kandidaten basierend auf ihren Präferenz-Profilen empfiehlt, welche auf vorherigen Präferenz-Bewertungen basieren / Implementierung ähnlich zum CV-Recommender; x ist das Ziel-Profil; z sind die latenten Faktoren; v Ziel-Attributwert (trifft meine Präferenzen / trifft meine Präferenzen nicht); y ist Kandidat

S. 5\\
- Evaluation mit N=32 Studenten aus 2 deutschen Unis
- 1. Phase: Studenten bieten ihren Lebenslauf über ein Web-Interface ein --> Liegt strukturiert vor, konnten auch ihren CV als ganzes Dokument hochladen (Falls für HR benötigt)
- 2. Phase: Studenten erhalten 100 echte Job-Profile, welche zufällig aus einem Job-Portal im Internet heruntergeladen wurden; Studenten sollten Job Profile auf einer Skala von 1 bis 5 bewerten

S. 6\\
- Evaluation CV-Recommender: 10 Job-Profile wurden genommen -> Eine Person bewertete das Match zwischen den 32 Studentenprofilen --> Auf Basis von 10 der Kandidaten und 5 der Jobprofile wurde das Modell trainiert --> Modell sagte übrige Profile vorher ...

S. 7\\
- Ähnliche Evaluation für Job-recommender
- Ungelöste Herausforderung: Integration der beiden Systeme

\textcite{faerber:2003}:\\
- Meiner Ansicht nach nicht besonders

\textcite{keim:2007}:
S. 4\\
- Entwickelten ein mehrschichtiges Framework zum Partnermatching und Teamstaffing
- Für erfolgreiches Partner-Matching: 1. Individuen müssen mit Rollen oder Jobs zusammengebracht werden, für welche sie die richtigen Fähigkeiten und Fertigkeiten haben; 2. Individuen müssen mit anderen Individuen, z.B. ihren potentiellen Teammitgliedern oder Arbeitspartnern zusammengeführt werden

S. 5\\
- Entwickelten ein modulares Framework zur Entscheidungsunterstützung
- 3 Ebenen: 1. Speichert unäre und binäre Attribute / 2. Bietet Module, welche als Filter für diese Daten arbeiten --> Zum Filtern für Rekruiter und Kandidaten / 3. Aggregation --> Fügt die Ergebnisse der einzelnen Module zusammen

S. 6\\
- CV-Recommender: Im System von 2003 wurden nur Entscheidungen aus Sicht des Rekruiters unterstützt --> Bild mit x,z,v,a --> Sucht Kandidaten basierend auf vorherigen Entscheidungen des Rekuriters aus
- Job-Recommender: Unterstützt Jobsuchende und Teammitglieder bei der Jobsuche oder beim finden von Rollen in Projekten, die ihre Präferenzen treffen könnten --> Funktioniert ähnlich zu CV-Recommender --> Verweis auf \textcite{malinowski:2006}
- Validierung mit 32 Studenten: ROC-Kurve

S. 7\\
- Deskriptives Trust-Modul: Etablierten eine Trust-Ontology mit Elementen von Network-based or historic und ituational und swift trust --> Modul unterstützt Nutzer dabei qualifizierte Entscheidungen zu treffen, auf welchen vertrauensvolle Beziehungen aufgebaut werden können --> entscheidung bleibt aber beim Nutzer
- Nutzer können angeben, wie sehr sie anderen Personen vertrauen, wie lange die Partnerschaft besteht und andere Angaben
- Entstehendes Netzwerk kann dann durchsucht werden
- wtf

S. 8\\
- Implementierten 3 Ansätze, um den Nutzer bei der Identifikation von 

S. 9\\
- Aggregation wird nicht vorgenommen

\textcite{ding:2016}:\\
S. 1:\\
- Entwickelten ein wechselseitiges Empfehlungssystem zum Recrutment von Absolventen
- Nutzt dafür historische Daten der Universität über Absolventen und frühere Absolventen
- Nutzen CF und CB
- System betrachtet Anforderungen von Absolventen UND Arbeitgebern
- Nutzt historische Informationen der Universität über Absolventen und frühere Absolventen --> Verbessert Genauigkeit
- Anmerkung von mir: Ist zwar wechselseitig, bezieht sich aber nicht auf den P-E Fit

S. 2\\
- Es gibt 2 Arten von Nutzern: Arbeitgeber und aktuelle Absolventen (Jobsuchende)
- Jobsuchende können grundlegende Informationen und Job-Präferenzen in das Recommender System eingeben / Nach Registrierung liest RS automatisch ihre Bildungsinformationen aus dem historischen Daten-Repository
- Arbeitgeber können grundlegende Informationen über Job-Anforderungen eingeben / Nach Registierung liest RS automatisch alle Bildungsinformationen ehemaliger Absolventen, welche in den letzten drei Jahren bei diesem Arbeitgeber angefangen haben
- Danach hilft RRSGR die richtigen Bewerber an den Arbeitgeber zu empfehlen
- Weiter bei IV


\newpage
So entwickelten \textcite[S. 1ff.]{applyingDataMining:2014} ein bilaterales Empfehlungssystem auf Basis von Data Mining-Technologien. Sie verstanden dabei unter den Wünschen des Nutzers dessen Präferenzen für ein bestimmtes Gehalt oder die Bekanntheit eines potenziellen Arbeitgebers. \textcite[S. 4ff.]{malinowski:2006} interpretierten die Wünsche des Nutzers dagegen als dessen Präferenz für bestimmte Stellenprofile.\\
Allgemein ist festzustellen, dass zu bilateralen Empfehlungssystemen bislang nur sehr wenig Literatur existiert \cite[S. 2f.]{jobRecommenderSystemsASurvey:2012}.\\

% Dieses System berücksichtigt Präferenzen von Kandidaten und Arbeitgebern
% Hybrid, da Profilähnlichkeit (Contentbased) und Kollaboratives Filtern
\textcite[S. 1ff.]{lu:2013} kombinierten Methoden des inhaltsbasierten Filterns und Nutzerinteraktionen innerhalb eines hybriden graphenbasierten Empfehlungssystems. Die Wissenschaftler erstellten ein Jobportal, in welchem Stellensuchende, Arbeitgeber und Jobausschreibungen in Form von Knoten existierten. Jede dieser Entitäten verfügte über entsprechende textuelle Profilbeschreibungen. Kanten wurden im Graphen hinzugefügt, wenn eine hohe Ähnlichkeit zwischen zwei Profilen bestand. Zusätzlich legte deren System für sämtliche Interaktionen zwischen den Entitäten, wie dem Besuch eines Profils oder dem Bewerben auf eine Stelle, Kanten an. Ein auf dem PageRank basierender Algorithmus unterstützte Stellensuchende und Arbeitgeber auf Grundlage des Graphen bei der Auswahl geeigneter Ausschreibungen bzw. Kandidaten.
\shorthandon{"}
