\shorthandoff{"}
\chapter{Fazit und Ausblick}
\label{ch:fazit}
Zusammenfassend ist festzustellen, dass in der Literatur eine große Anzahl an Ansätzen existieren, Empfehlungssysteme für die Besetzung von Projektstellen mit Mitarbeitern zu implementieren. Welche Art von System am geeignetsten ist, muss im Einzelfall je nach Anforderungen der Domäne entschieden werden. So erzielen wissensbasierte Empfehlungssysteme eine sehr hohe Genauigkeit in den Resultaten, bieten aber nur durch aufwendige Erweiterungen Flexibilität in der Suche. Speicherbasierte Ansätze im Bereich des kollaborativen und inhaltsbasierten Filterns sorgen für eine höhere Flexibilität, sind jedoch anfällig für einen Cold Start und das Sparse Data-Problem. Um diese zu lösen, empfehlen verschiedene Wissenschaftler die Umsetzung hybrider oder modellbasierter Verfahren. Die Implementierung solcher Systeme ist jedoch wesentlich aufwendiger.\\
Kritisch ist zu bemerken, dass ein Großteil der vorliegenden Publikationen ausschließlich Empfehlungssysteme behandeln, welche sich entweder an Personalsachbearbeiter oder an Stellensuchende richten. Die Entwicklung bilateraler Empfehlungssysteme ist dagegen ein weitgehend unerforschtes Fachgebiet. Diese Art von Implementierungen beachten neben den im Projekt benötigten Fähigkeiten auch die Wünsche bzw. Bedürfnisse der Mitarbeiter. Um welche Art von Wünschen es sich dabei handelt, wird in der Literatur nicht einheitlich interpretiert.\\
Darüber hinaus ist zu beobachten, dass Personalsachbearbeiter in den bilateralen Systemen der vorliegenden Arbeiten meist explizit die für ein Projekt geforderten Fähigkeiten spezifizieren können. Keine der untersuchten Implementierungen lässt eine vergleichbare Spezifikation gewünschter Kompetenzen auf Seiten der Angestellten zu. So wird bei den betrachteten Systemen das mögliche Bedürfnis des Mitarbeiters, neue Fähigkeiten zu erlernen und praktisch anzuwenden, nicht berücksichtigt. Angestellte werden ausschließlich nach den Fähigkeiten ausgewählt, welche sie bereits beherrschen. Aus diesem Grund wird für eine folgende Arbeit empfohlen, ein weiteres bilaterales Empfehlungssystem zu entwickeln. In diesem sollten die Mitarbeiter neben ihren bestehenden Fähigkeiten auch Kompetenzen einpflegen können, welche sie in Zukunft erwerben bzw. vertiefen möchten. Personalsachbearbeiter könnten ein solches System zur Besetzung von Projektstellen und gleichzeitig zur strategischen Weiterbildung ihrer Angestellten verwenden. Aus Sicht der Mitarbeiter ist zu erwarten, dass eine derartige Implementierung die Zufriedenheit und die Motivation in den Projektarbeiten nachhaltig steigert. 
\shorthandon{"}