\shorthandoff{"}
\chapter{Fazit}
\label{ch:fazit}
Die vorliegende Arbeit zeigte auf, dass die Besetzung offener Projektpositionen eine immer häufiger auftretende Tätigkeit in der Wirtschaft sein wird. Bisher entwickelte Empfehlungssysteme im Bereich der Personalauswahl bieten für diese Problemstellung laut \textcite{malinowski:2008} zumeist Lösungen, welche sich entweder an Personalverantwortliche oder an Stellensuchende bzw. Mitarbeiter richten. Den Wissenschaftlern zu Folge würde ein bilaterales System, welches gemäß des Konzepts des \acfp{PEFit} die Präferenzen beider Parteien gleichermaßen berücksichtigt, aus beiden Perspektiven für noch bessere Ergebnisse sorgen. Um diese Hypothese zu überprüfen, untersuchte die vorliegende Master-Thesis die folgende Forschungsfrage: \forschungsfrage

Um diese Forschungsfrage zu überprüfen, wurde ein graphenbasiertes Empfehlungssystem entwickelt. Dieses sortierte die Mitarbeiter eines Unternehmens für fünf vordefinierte Projektpostionen über einen uni- und eine bilaterale Ansatz. Die bilaterale Variante bezog sowohl die von den Angestellten beherrschten Fähigkeiten, als auch deren präferierte Kompetenzen in die Empfehlungsbestimmung ein. Der unilaterale Ansatz betrachtete dagegen lediglich die beherrschten Fähigkeiten der Mitarbeiter.

Anschließend wurde eine Fallstudie unter Projektmanagern und Angestellten des Unternehmens durchgeführt. Hierbei wurde auf Seiten der Mitarbeiter überprüft, ob der bilaterale Empfehlungsansatz die Angestellten im Vergleich zur unilateralen Variante bei den vordefinierten Projektpositionen höher positioniert, wenn diese eine hohe Zufriedenheit mit einer betrachteten Projektposition prognostizieren bzw. niedriger positioniert, wenn diese eine geringe Zufriedenheit erwarten. Die Projektmanager erhielten die ersten fünf vorgeschlagenen Mitarbeiter jedes Empfehlungsverfahrens für jede betrachtete Projektposition. Sie bewerteten, von den Mitarbeitern welcher Liste sie eine höhere Arbeitsleistung bei einer Tätigkeit auf der betrachteten Stelle erwarten.

Bei der Auswertung der Ergebnisse wurde festgestellt, dass die Forschungsfrage bestätigt werden kann, wenn die Mitarbeiter mehrheitlich mit einer Tätigkeit auf der betrachteten Projektposition zufrieden sind. In diesem Fall sorgt der bilaterale Empfehlungsansatz im Vergleich zur unilateralen Variante sowohl für eine höhere Zufriedenheit auf Seiten der Angestellten, als auch für eine gesteigerte prognostizierte Arbeitsleistung bei den Projektmanagern. Zeigen sich die Mitarbeiter dagegen mehrheitlich unzufrieden mit einer betrachteten Projektposition, sorgt das bilaterale Vorgehen sowohl auf Seiten der Mitarbeiter für eine geringere Zufriedenheit, als auch aus Perspektive der Projektverantwortlichen für eine niedrigere erwartete Arbeitsleistung.

Als Ursache für diese Einschränkung wird die Erhebung der Mitarbeiter-Präferenzen betrachtet. Das bilaterale Empfehlungssystem nutzte die im Rahmen dieser Master-Thesis erhobenen booleschen Werte, um die präferierten Fähigkeiten der Angestellten stärker zu gewichten. Bei nicht gewünschten Kompetenzen wurde jedoch nicht unterschieden, ob ein Angestellter dieser Fähigkeit neutral gegenübersteht oder ob er diese nicht bei der Projektarbeit anwenden möchte. Aus diesem Grund wird für folgende Arbeiten empfohlen, den im Rahmen dieser Arbeit implementierten Empfehlungsansatz zu erweitern. Hierbei sollten die Präferenzen nicht über boolesche Werte, sondern über Abstufungen der Form "möchte ich anwenden", "neutral", "möchte ich nicht anwenden" erhoben werden. Bei der Implementierung sollten die Mitarbeiter bei vorhandenem Wunsch weiterhin höher positioniert werden, bei negativer Präferenz sollten sie jedoch zusätzlich niedriger einsortiert werden. Unter Betrachtung dieser Veränderungen sollte die Evaluation unter Mitarbeitern und Projektmanagern erneut durchgeführt und die Forschungsfrage entsprechend untersucht werden.
\newpage

\section{Fragen}
\label{ch:fazit:fragen}
- Die historischen Bücher sprechen immer nur von "Männern" --> kann man daraus einfach "Menschen" machen?\\ \note{Zitattechnisch: Nein. Auch inhaltlich ist das eine spannende Frage. Für Motivationslagen spielen soziale, kulturelle und gesellschaftliche Aspekte ja eine Rolle.}
- Gründungsvater ist ein englisches Zitat --> Wie zitieren?\\ \note{Üblich ist das Orginalzitat mit zusätzlicher Übersetzung}
- Wie umgehen mit den englischen Begriffen? z.B. fit, Need, Desire, etc. \note{Ggf. als Eigenbegriffe nutzen}\\
- Ergebnisse und Diskussion in zwei separate Kapitel oder in eines?

\section{Anmerkungen}
\label{ch:fazit:anmerkungen}
- "Referent" klingt komisch (Christian)\\
- Zweite Seite und "I. Thesis" ist unnötig (Christian)\\
- JSON und REST im Abkürzungsverzeichnis muss nicht sein (Christian)\\
- Wort: "Fallstudie" (Christian)\\
- Forschungsfrage: Symmetrische sollte klarer formuliert werden --> Kommt etwas hinterher --> Eher an den Anfang (Andreas)\\
- Forschungsfrage ist sehr lang (Christian)\\
- Einleitung: Warum erhöht die dezentrale Kommunikation die Kreativität? (Christian)\\
- Zu Empfehlungssystemen: Könnten wir uns nicht auf einen Teilgraphen beschränken? (Jan)\\
- Verlinkungen z.B. PE-Fit unterstreichen? (Nina)\\
- Formel \ref{frml:verwandteArbeiten:formel1} raus? (Nina)\\
- Bilddiskussion in Kapitel \ref{ch:verwandteArbeiten:aufDemPEFitBasierendeBilateraleSysteme:pjUndPtFit} in Präsens? (Nina)\\
- \ref{ch:verwandteArbeiten:aufDemPEFitBasierendeBilateraleSysteme:bilateraleVertrauensbestimmung}: Was ist an der Formel noch nicht optimal? (Nina)\\
- Zeile (Z) bei wörtlichen Zitaten aufnehmen? (Christian)\\
- Historischer Teil bei \ac{PEFit} ist unnötig, um Thesis zu verstehen - nur nice-to-know (Christian)\\
- Viel "die Wissenschaftler" (Christian)\\
- Bilder im \ac{PEFit}-Kapitel auf deutsch und dann englische Begriffe im Fließtext weglassen / Auch Long Tail-Bild auf deutsch (Christian)\\
- Ist \ref{fig:personEnvironmentFit:auswirkungenErhoehterAngebote:formel4} notwendig? (Christian)\\
- Qualität von \ref{fig:personEnvironmentFit:wichtigkeiten:abb2} könnte besser sein (Christian)\\
- "e-lancer", P-E Misfit, Pearson Korrelation, Sparsity Problem und Recommender Engine kursiv? (Christian)\\
- Aufkommen der Recommender Engines, YouTube, Filterblase --> an der Grenze, ob es relevant ist (Christian)\\
- Ist es häufig so, dass man direkte Treffer hat oder dass man indirekte Treffer hat? --> Sucht man nach verbreiteten Skills oder nach exotischem? --> Einfach nur als Beobachtung (Andreas)\\
- S. 22: Nicht nur die Skalierung ist verändert --> Verhältnisse --> Relative Werte verändern sich / Statt Skalierung: Glättung - dazu sollten noch 1-2 Sätze rein und das erklären und noch ein Argument dazu rein, warum das kein Problem ist, sondern wieso das vielleicht sogar gut ist (Andreas)\\
- Ontologien für Skills (Johannes)\\
- Rechtschreibfehler in Skills (Johannes)\\
- Abkürzungen nicht eindeutig (Imanuel)\\
- Abbildung \ref{fig:verwandteArbeiten:abb1} neu machen und statt der Buchstaben die Bedeutung rein schreiben (Christian)\\
- Grundsätzlich alle Bilder deutsch machen (Christian)\\
- Statt "Art der Forschung" --> "Forschungsansatz"? (Christian)\\
- Statt "Experiment" --> "Fallstudie"? (Christian)\\
- Pseudomitarbeiter in Empfehlungssysteme entfernen (Johannes)\\
- Fragen zu Unterforderung im Fragebogen positiv formulieren (Andreas)
- Abbildung 6.5: Mitte markieren und zeigen, dass es insgesamt 23 MA sind (Andreas)\\
- Abstrakt schreiben (Andreas)
\shorthandon{"}