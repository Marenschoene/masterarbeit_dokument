\shorthandoff{"}
\chapter{Methodik}
\label{ch:methodik}

\section{Art der Forschung}
\label{ch:methodik:art}
Die vorliegende Master-Thesis verfolgt das Ziel, die folgende Forschungsfrage zu beantworten: xyz.

Um diese Fragestellung zu untersuchen, wird eine quantitative Forschungsarbeit in Form eines Experiments durchgeführt. Hierfür werden zwei Versionen eines Empfehlungssystems zum Vorschlagen von Mitarbeitern für offene Projektpositionen in einem Unternehmen entwickelt. Eine der beiden Anwendungen verfolgt einen unilateralen, die andere einen bilateralen Ansatz. Beide Empfehlungssysteme erhalten als Eingabe dieselben offenen Projektpositionen, für welche sie die vorhandenen Mitarbeiter des Betriebs sortieren und in Form einer Liste zurückgeben sollen. Die Listen beider Systeme werden Projektmanagern des Unternehmens vorgelegt, woraufhin diese auf einer vordefinierten Skala bewerten, welche Arbeitsleistung sie von den in der vorliegenden Reihenfolge dargestellten Mitarbeitern erwarten. Die Angestellten erhalten die Beschreibungen der offenen Projektpositionen und bewerten auf einer vordefinierten Skala, wie zufrieden sie voraussichtlich mit der Tätigkeit auf den vorliegenden Projektpositionen sein werden. Hierbei wird überprüft, ob eine hohe erwartete Zufriedenheit des Angestellten mit einer hohen Positionierung in den Ergebnislisten der Empfehlungssysteme korreliert.

Durchgeführt wird das Experiment mit Projektmanagern und Mitarbeitern der EXXETA AG mit Hauptsitz in Karlsruhe. Das Unternehmen ist spezialisiert auf IT-Beratungsleistungen und arbeitet vorrangig projektbasiert. Passende Angestellte zu offenen Projektpositionen zuzuordnen ist dort eine häufig auftretende Aufgabe. Dementsprechend pflegen die Mitarbeiter ihre Fähigkeiten in einem Intranet in einer strukturieren Form, welche als Eingabe für die Empfehlungssysteme genutzt werden können.

\section{Versuchsaufbau}
\label{ch:methodik:versuchsaufbau}
Zu erhebende Daten:\\
- Skills sind im Intranet gespeichert: Bewertung auf einer Skala von 1 bis 5\\
- Zusätzlich Teamzuordnung im Intranet --> Kantengewicht: 1 --> Gedanken mit dem Manager (Kann mehr) in Ausarbeitung aufnehmen\\
- Für jeden Mitarbeiter erheben: Wichtigkeit (true/false) --> Auch für Dinge, die man noch nicht kann\\
- Für jeden Mitarbeiter erheben: Welche Auswirkung hat es, wenn im Projekt nicht ausgelastet (Kurve A bis C) --> Alle drei Antwortmöglichkeiten positiv formulieren --> z.B. Freiräume nutzen

Erstellen des Graphen:\\
- Aus Skills (kollaboratives Filtern) und Teamzuordnung (Inhaltsbasiertes Filtern) einen tripartiten Graphen erstellen\\
- Sinnvolle Kanten: Nutzer-Skill / Nutzer-Team // Besser: Nutzer-Skill und Nutzer-Manager --> Weniger Knoten\\
- Entscheidung gegen Mittelwertzentrierung --> Bewertung befolgt auf Tatsachen, klare Skala, wenig subjektiv

% Johannes: 404
% Jan: 366
% Alexander: 20
\begin{figure}[h]
	\centering	
	\begin{tikzpicture}[node distance={32mm}, thick, main/.style = {draw, circle}] 
		\node[main, fill=itemcolor] (MongoDB) {$MongoDB$}; 
		\node[main, fill=itemcolor] (Python) [below right of=MongoDB] {$Python$}; 
		\node[main, fill=itemcolor] (MySQL) [above right of=Python] {$MySQL$}; 
		\node[main, fill=itemcolor] (Java) [below right of=MySQL] {$Java$}; 
		\node[main, fill=itemcolor] (HDFS) [above right of=Java] {$HDFS$}; 
		\node[main, fill=itemcolor] (Spark) [below right of=HDFS] {$Spark$};
		
		\node[main, fill=usercolor] (Johannes) [above right of=MongoDB] {$Joh L.$}; 
		\node[main, fill=usercolor] (Jan) [above left of=HDFS] {$Jan H.$}; 
		\node[main, fill=usercolor] (Alexander) [above right of=HDFS] {$Alex G.$};
		
		%\node[main, fill=exxetagray] (JES32) [above right of=Johannes] {$JES 3.2$};
		
		\draw (Johannes) -- node[midway, right] {4} (Python);
		\draw (Johannes) -- node[midway, above] {3} (MySQL);
		\draw (Johannes) -- node[midway, above] {3} (MongoDB);
		
		\draw (Jan) -- node[midway, right] {1} (HDFS);		
		\draw (Jan) -- node[midway, right] {3} (Java);
		\draw (Jan) -- node[midway, above] {2} (MySQL);
		
		\draw (Alexander) -- node[midway, above] {5} (HDFS);
		\draw (Alexander) -- node[midway, left] {3} (Spark);
		
		%\draw (Johannes) -- node[midway, above] {1} (JES32);
		\draw (Johannes) -- node[midway, above] {1} (Jan);
		\draw (Jan) -- node[midway, above] {1} (Alexander);
		%\draw (Jan) -- node[midway, above] {1} (JES32);
	\end{tikzpicture}
	
	\caption{Darstellung der Fähigkeitsmatrix aus Tabelle \ref{tbl:empfehlungssysteme:arbeitsweise:tbl1} in der Datenstruktur eines Graphen}
	\label{fig:methodik:abb1}
\end{figure}

Eingabe der offenen Projektposition:\\
- Benötigt: Fähigkeiten und Wichtigkeit (Boolean)

Algorithmus:\\
- Berechnung der Katz-Zentralität\\
- Für jeden relevanten Mitarbeiter auf Basis von Abbildung \ref{fig:methodik:abb2} den finalen Wert bestimmen --> Hierbei je nach Wichtigkeit die Kurve stauchen --> Wenn für Projektmanger wichtig, die durchgezogene Linie doppelt so steil; Wenn für Mitarbeiter wichtig, rechte Seite doppelt so steil; Auswahl der Kurve anhand der Information des Mitarbeiters\\
- Summe für alle Fähigkeiten eines Projektes für jeden Mitarbeiter berechnen

\begin{figure}[h]
	\centering
	\includegraphics[width=0.75\textwidth]{gfx/ueberschuss_supply_motive.png}
	\caption{Auswirkungen eines Bedürfnisse-Angebote Misfits \cite[S. 23]{edwards:2008}\\(Bearbeitet von \myName)}
	\label{fig:methodik:abb2}
\end{figure}

- Algorithmus einmal durchführen mit Wichtigkeiten und einmal ohne (bilateral vs. unilateral)\\
- Ausgabe der sortierten Liste (mit allen Mitarbeitern (zB 25))\\
- Eingabe der Projektposition und Algorithmus für jede Projektposition wiederholen

\section{Geplante Evaluation}
\label{ch:methodik:evaluation}
Evaluation für Projektmanager:\\
- Erhält für jedes Projekt beide Listen und gibt auf einer Skala von 1 bis 5 an, wie hoch der die Leistung der empfohlenen Mitarbeiter in diesem Projekt einschätzen würde

Evaluation für Mitarbeiter:\\
- Jeder Mitarbeiter muss für jedes Projekt auf einer Skala von 1 bis 5 bewerten, wie zufrieden er wäre, wenn er darin arbeiten würde\\
- Ergebnisliste wird in Intervalle geteilt --> z.B. Zufriedenheit 5 bedeutet bei 25 Teilnehmern, dass der Nutzer im ersten Intervall sein muss --> Abweichung bestimmen --> Je weniger Abweichung, desto besser --> Durchschnittliche Abweichung von unilateral und bilateral vergleichen

Frage:\\
- Sollte Manager überhaupt Wichtigkeiten angeben?\\
	- Fähigkeiten sind Angebote und Wichtigkeiten Nachfrage\\
- Ist das Vergleichs-Vorgehen unilateral?\\
- Welche Daten müssen in den Anhang der Thesis?
\shorthandon{"}
