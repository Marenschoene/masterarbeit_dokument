\definecolor{exxetagray}{gray}{0.75}
\definecolor{itemcolor}{RGB}{179,217,255}
\definecolor{usercolor}{RGB}{255,204,179}

\shorthandoff{"}
\chapter{Methodik}
\label{ch:methodik}

\section{Einführung}
% Erklärung der genauen Problematik in unserem Anwendungsfall

% \subsection{Probleminterpretation}
% Nutzer = Manager, deren Anforderungen als Vektor dargestellt werden

\section{Art der Forschung}
% Art der Methodik
% quantitative Forschung in Form einer Befragung

\section{Bestimmung der Stichprobe}
% MA im JES-Team, Manager, angeben wie viele kontakiert wurden (Anzahl MA, Anzahl Manager)

\section{Ablauf der Datenerhebung}
% 2 separate Befragungen
% Befragung ist unabhängig von dem gewählten algorithmus -> dadurch reproduzierbarkeit und vergleichbarkeit möglich
% Aufteilen der Daten in Trainings- und Testdaten
% Nutzen der Trainingsdaten, um Gewichte des bilateralen Algorithmus zu lernen
% Vergleich des entwickelten bilateralen Algorithmus mit einer unilateralen Variante anhand der Testdaten

\subsection{Befragung der Mitarbeiter}
% 1. Mitarbeiter befragen nach Selbsteintschätzung der Fähigkeiten in Anlehnung an Intranet (da im Intranet nur z.T. vollständig), sowie Präferenzen (pos. und neg.)
% Zusätzlich: Bewertung der zufriedenheit eines MA bei Zuteilung zu den verschiedenen Dummy-Projekten

\subsection{Befragung der Projektmanager}
% 2. Bewertung der erwarteten Arbeitsleistung von Mitarbeitern unter Berücksichtigung ihrer Fähigkeiten und Präferenzen in den versch. Dummy-Projekten
% Hier ist ein Teil dese Outputs (Präferenzen und Skills) der 1. Befragung der Input dieser Befragung

\shorthandon{"}