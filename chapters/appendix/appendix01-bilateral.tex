\section{Beispielrechnung der bilateralen Komponente}
\label{ch:nebenrechnungen:bilateral}
Tabelle \ref{tbl:nebenrechnungen:bilateral:tbl1} zeigt das zu besetzende Beispiel-Projekt. In den Tabellen dieses Kapitels sind fehlende Kenntnisse grau, Grundkenntnisse blau und fortgeschrittene Kompetenzen rot markiert.
\begin{table}[h]
	\centering
	\begin{tabular}{c|c}
		Fähigkeit & Kompetenzniveau \\
		\hline
		Python  & \cellcolor{usercolor}Fortgeschritten\\
		MySQL   & \cellcolor{usercolor}Fortgeschritten\\
		HDFS    & \cellcolor{itemcolor}Grundkenntnisse
	\end{tabular}
	\caption{Zu besetzendes Beispiel-Projekt}
	\label{tbl:nebenrechnungen:bilateral:tbl1}
\end{table}

Tabelle \ref{tbl:nebenrechnungen:bilateral:tbl3} zeigt, welche der gesuchten Fähigkeiten die Mitarbeiter präferieren.
\begin{table}[h]
	\centering
	\begin{tabular}{c|c|c|c}
		Name & Python & MySQL & HDFS \\
		\hline
		Doe, Jane     & \cellcolor{usercolor}falsch & \cellcolor{usercolor}falsch & \cellcolor{SeaGreen}wahr\\
		Doe, John     & \cellcolor{SeaGreen}wahr   & \cellcolor{SeaGreen}wahr   & \cellcolor{SeaGreen}wahr\\
		Muster, Erika & \cellcolor{SeaGreen}wahr   & \cellcolor{usercolor}falsch & \cellcolor{usercolor}falsch\\
		Muster, Max   & \cellcolor{usercolor}falsch & \cellcolor{SeaGreen}wahr   & \cellcolor{SeaGreen}wahr
	\end{tabular}
	\caption{Präferierte Kompetenzen der Mitarbeiter}
	\label{tbl:nebenrechnungen:bilateral:tbl3}
\end{table}

Relevante Ergebnis-Spalten des Katz-Algorithmus für den Graphen aus Abbildung \ref{fig:methodik:versuchsaufbau:unilateral:abb1} anhand von Formel \ref{frml:empfehlungssysteme:cf:speicherbasiert:formel4} mit $\beta = \frac{1/\lambda}{\nenner}$. Präferierte Kompetenzen wurden mit \faktorFloat multipliziert. Für die weitere Berechnung verwendete Referenz-Werte sind hervorgehoben.

\begin{table}[h]
	\centering
	\begin{tabular}{c|c|c|c}
		& Python & MySQL & HDFS\\ 
		\hline
		Doe, Jane     & \cellcolor{usercolor}\textbf{2.77} & \cellcolor{itemcolor}2.00 & \cellcolor{exxetagray}1.22\\
		Doe, John     & \cellcolor{exxetagray}1.84 & \cellcolor{itemcolor}2.40 & \cellcolor{itemcolor}\textbf{2.32}\\
		Muster, Erika & \cellcolor{exxetagray}0.86 & \cellcolor{exxetagray}0.56 & \cellcolor{usercolor}1.57\\
		Muster, Max   & \cellcolor{itemcolor}1.93 & \cellcolor{itemcolor}\textbf{3.24} & \cellcolor{exxetagray}1.12
	\end{tabular}
	\caption{Relevante Ergebnisse des Katz-Algorithmus für den Graphen aus Abbildung \ref{fig:methodik:versuchsaufbau:unilateral:abb1}}
	\label{tbl:nebenrechnungen:bilateral:tbl2}
\end{table}

Berechnung der Abweichung für Jane Doe mit den Werten aus Tabelle \ref{tbl:nebenrechnungen:bilateral:tbl2}:
\begin{gather}
	\nonumber (2.77-2.77)^2 + (2.00-3.24)^2 + (1.22-2.32)^2\\
	\nonumber = 0 + 1.5376 + 1.21\\
	\nonumber = 2.7476\\
	\approx 2.75
	\label{frml:nebenrechnungen:bilateral:jane}
\end{gather}

Berechnung der Abweichung für John Doe mit den Werten aus Tabelle \ref{tbl:nebenrechnungen:bilateral:tbl2}:
\begin{gather}
	\nonumber (1.84-2.77)^2 + (2.40-3.24)^2 + (2.32-2.32)^2\\
	\nonumber = 0.8649 + 0.7056 + 0\\
	\nonumber = 1.5705\\
	\approx 1.57
	\label{frml:nebenrechnungen:bilateral:john}
\end{gather}

Berechnung der Abweichung für Erika Muster mit den Werten aus Tabelle \ref{tbl:nebenrechnungen:bilateral:tbl2}:
\begin{gather}
	\nonumber (0.86-2.77)^2 + (0.56-3.24)^2 + (1.57-2.32)^2\\
	\nonumber = 3.6481 + 7.1824 + 0.5625\\
	\nonumber = 11.393\\
	\approx 11.34
	\label{frml:nebenrechnungen:bilateral:erika}
\end{gather}

Berechnung der Abweichung für Max Muster mit den Werten aus Tabelle \ref{tbl:nebenrechnungen:bilateral:tbl2}:
\begin{gather}
	\nonumber (1.93-2.77)^2 + (3.24-3.24)^2 + (1.12-2.32)^2\\
	\nonumber = 0.7056 + 0 + 1.44\\
	\nonumber = 2.1456\\
	\approx 2.15
	\label{frml:nebenrechnungen:bilateral:max}
\end{gather}

Ausgabe:
\begin{table}[h]
	\centering
	\begin{tabular}{c|c|c}
		Positionierung & Mitarbeiter & Abweichung\\
		\hline
		1 & John D.  & 1.57\\
		2 & Max M.   & 2.15\\
		3 & Jane D.  & 2.75\\
		4 & Erika M. & 11.34
	\end{tabular}
	\caption{Ergebnisliste der bilateralen Empfehlungskomponente für die Daten aus Tabelle \ref{tbl:nebenrechnungen:bilateral:tbl2}}
	\label{tbl:nebenrechnungen:bilateral:ausgabe}
\end{table}