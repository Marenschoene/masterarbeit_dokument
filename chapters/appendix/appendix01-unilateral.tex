\section{Berechnung der unilateralen Empfehlungskomponente}
\label{ch:nebenrechnungen:unilateral}
% TODO: Python statt Java; MongoDB statt MySQL
Tabelle \ref{tbl:berechnungDerKatzZentralitaetPseudoMitarbeiter:tbl1} zeigt das zu besetzende Beispiel-Projekt. In den Tabellen dieses Kapitels sind fehlende Kenntnisse grau, Grundkenntnisse blau und fortgeschrittene Kompetenzen rot markiert.
\begin{table}[h]
	\centering
	\begin{tabular}{c|c}
		Fähigkeit & Kompetenzniveau \\
		\hline
		Java  & \cellcolor{usercolor}Fortgeschritten\\
		MySQL & \cellcolor{itemcolor}Grundkenntnisse\\
		HDFS  & \cellcolor{itemcolor}Grundkenntnisse
	\end{tabular}
	\caption{Zu besetzendes Beispiel-Projekt}
	\label{tbl:berechnungDerKatzZentralitaetPseudoMitarbeiter:tbl1}
\end{table}

Relevante Ergebnis-Spalten des Katz-Algorithmus für den Graphen aus Abbildung \ref{fig:methodik:versuchsaufbau:unilateral:abb1} anhand von Formel \ref{frml:empfehlungssysteme:cf:speicherbasiert:formel4} mit $\beta = \frac{1/\lambda}{\nenner}$. Für die weitere Berechnung verwendete Referenz-Werte sind hervorgehoben.

\begin{table}[h]
	\centering
	\begin{tabular}{c|c|c|c}
		& Java & MySQL & HDFS\\ 
		\hline
		Doe, Jane     & \cellcolor{exxetagray}0.907 & \cellcolor{itemcolor}\textbf{1.996} & \cellcolor{exxetagray}0.612\\
		Doe, John     & \cellcolor{itemcolor}0.887 & \cellcolor{itemcolor}1.202 & \cellcolor{itemcolor}\textbf{1.161}\\
		Muster, Erika & \cellcolor{exxetagray}0.416 & \cellcolor{exxetagray}0.565 & \cellcolor{usercolor}1.569\\
		Muster, Max   & \cellcolor{itemcolor}\textbf{1.031} & \cellcolor{itemcolor}1.623 & \cellcolor{exxetagray}0.559
	\end{tabular}
	\caption{Ergebnisse des Katz-Algorithmus für den Graphen aus Abbildung \ref{fig:methodik:versuchsaufbau:unilateral:abb1}}
	\label{tbl:berechnungDerKatzZentralitaetPseudoMitarbeiter:tbl2}
\end{table}

Berechnung der Abweichung für Jane Doe mit den Werten aus Tabelle \ref{tbl:berechnungDerKatzZentralitaetPseudoMitarbeiter:tbl2}:
\begin{gather}
	\nonumber (0.907-1.031)^2 + (1.996-1.996)^2 + (0.612-1.161)^2\\
	\nonumber = 0.015376 + 0 + 0,301401\\
	\nonumber = 0.316777\\
	\approx 0.317
	\label{frml:nebenrechnungen:unilateral:jane}
\end{gather}

Berechnung der Abweichung für John Doe mit den Werten aus Tabelle \ref{tbl:berechnungDerKatzZentralitaetPseudoMitarbeiter:tbl2}:
\begin{gather}
	\nonumber (0.887-1.031)^2 + (1.202-1.996)^2 + (1.161-1.161)^2\\
	\nonumber = 0.020736 + 0.630436 + 0\\
	\nonumber = 0.651172\\
	\approx 0.651
	\label{frml:nebenrechnungen:unilateral:john}
\end{gather}

Berechnung der Abweichung für Erika Muster mit den Werten aus Tabelle \ref{tbl:berechnungDerKatzZentralitaetPseudoMitarbeiter:tbl2}:
\begin{gather}
	\nonumber (0.416-1.031)^2 + (0.565-1.996)^2 + (1.569-1.161)^2\\
	\nonumber = 0.378225 + 2.047761 + 0.166464\\
	\nonumber = 2.59245\\
	\approx 2.592
	\label{frml:nebenrechnungen:unilateral:erika}
\end{gather}

Berechnung der Abweichung für Max Muster mit den Werten aus Tabelle \ref{tbl:berechnungDerKatzZentralitaetPseudoMitarbeiter:tbl2}:
\begin{gather}
	\nonumber (1.031-1.031)^2 + (1.623-1.996)^2 + (0.559-1.161)^2\\
	\nonumber = 0 + 0.139129 + 0.362404\\
	\nonumber = 0.501533\\
	\approx 0.502
	\label{frml:nebenrechnungen:unilateral:max}
\end{gather}

Ausgabe:
\begin{table}[h]
	\centering
	\begin{tabular}{c|c|c}
		Positionierung & Mitarbeiter & Abweichung\\
		\hline
		1 & Jane D.  & 0.317\\
		2 & Max M.   & 0.502\\
		3 & John D.  & 0.651\\
		4 & Erika M. & 2.592
	\end{tabular}
	\caption{Ergebnisliste der unilateralen Empfehlungskomponente für die Daten aus Tabelle \ref{tbl:berechnungDerKatzZentralitaetPseudoMitarbeiter:tbl2}}
	\label{tbl:nebenrechnungen:unilateral:ausgabe}
\end{table}