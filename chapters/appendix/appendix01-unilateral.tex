\section{Berechnung der unilateralen Empfehlungskomponente}
\label{ch:nebenrechnungen:unilateral}
Darstellung von Graph \ref{fig:methodik:versuchsaufbau:unilateral:abb1} in Form einer Tabelle:
\begin{table}[h]
	\centering
	\begin{tabular}{c|c|c|c|c|c|c|c|c|c|c|c}
		& \begin{sideways}Jane D.\end{sideways} & \begin{sideways}John D.\end{sideways} & \begin{sideways}Erika M.\end{sideways} & \begin{sideways}Max M.\end{sideways} & \begin{sideways}Projekt\end{sideways} & \begin{sideways}Java\end{sideways} & \begin{sideways}Python\end{sideways} & \begin{sideways}MySQL\end{sideways} & \begin{sideways}MongoDB\end{sideways} & \begin{sideways}HDFS\end{sideways} & \begin{sideways}Spark\end{sideways} \\
		\hline
		Jane D.  & 0 & \kantengewicht & \kantengewicht & \kantengewicht & 0 & 0 & 4 & 3 & 3 & 0 & 0\\
		John D.  & \kantengewicht & 0 & \kantengewicht & \kantengewicht & 0 & 3 & 0 & 2 & 0 & 1 & 0\\
		Erika M. & \kantengewicht & \kantengewicht & 0 & \kantengewicht & 0 & 0 & 0 & 0 & 0 & 5 & 3\\
		Max M.   & \kantengewicht & \kantengewicht & \kantengewicht & 0 & 0 & 2 & 3 & 1 & 0 & 0 & 0\\
		Projekt  & 0 & 0 & 0 & 0 & 0 & 0 & 4 & 0 & 3 & 0 & 3\\
		Java     & 0 & 3 & 0 & 2 & 0 & 0 & 0 & 0 & 0 & 0 & 0\\
		Python   & 4 & 0 & 0 & 3 & 4 & 0 & 0 & 0 & 0 & 0 & 0\\
		MySQL    & 3 & 2 & 0 & 1 & 0 & 0 & 0 & 0 & 0 & 0 & 0\\
		MongoDB  & 3 & 0 & 0 & 0 & 3 & 0 & 0 & 0 & 0 & 0 & 0\\
		HDFS     & 0 & 1 & 5 & 0 & 0 & 0 & 0 & 0 & 0 & 0 & 0\\
		Spark    & 0 & 0 & 3 & 0 & 3 & 0 & 0 & 0 & 0 & 0 & 0
	\end{tabular}
	\caption{Anzahl an Verbindungen im Graphen aus Abbildung \ref{fig:methodik:versuchsaufbau:unilateral:abb1}}
	\label{tbl:berechnungDerKatzZentralitaetPseudoMitarbeiter:tbl1}
\end{table}

Bestimmung der Katz-Zentralität für Graph \ref{fig:methodik:versuchsaufbau:unilateral:abb1} anhand der Formel \ref{frml:empfehlungssysteme:cf:speicherbasiert:formel4} mit $\beta = \frac{1/\lambda}{4}$. Damit ergibt sich für die Daten aus Tabelle \ref{tbl:berechnungDerKatzZentralitaetPseudoMitarbeiter:tbl1} $\beta \approx 0.011$. Nach Berechnung des Algorithmus ergeben sich die Werte aus Tabelle \ref{tbl:berechnungDerKatzZentralitaetPseudoMitarbeiter:tbl2}:

\begin{table}[h]
	\centering
	\begin{tabular}{c|c|c|c|c|c|c}
		& Java & Python & MySQL & MongoDB & HDFS & Spark\\ 
		\hline
		Doe, Jane     & 0.0052 & 0.0061 & 0.0935 & 0.0032 & 0.0951 & 0.0033\\
		Doe, John     & 0.0011 & 0.0005 & 0.0072 & 0.0017 & 0.0378 & 0.0442\\
		Muster, Erika & 0.0012 & 0.0006 & 0.0061 & 0.0002 & 0.0164 & 0.0001\\
		Muster, Max   & 0.0002 & 0.0002 & 0.0031 & 0.0012 & 0.0032 & 0.0331\\
		Projekt       & 0.0357 & 0.0163 & 0.0938 & 0.0031 & 0.0941 & 0.0005\\
	\end{tabular}
	\caption{Ergebnisse nach Bestimmung der Katz-Zentralität für Tabelle \ref{tbl:berechnungDerKatzZentralitaetPseudoMitarbeiter:tbl1}}
	\label{tbl:berechnungDerKatzZentralitaetPseudoMitarbeiter:tbl2}
\end{table}






Für die unilaterale Empfehlungskomponente relevante Daten aus Tabelle \ref{tbl:methodik:versuchsaufbau:unilateral:tbl1}:

\begin{table}[h]
	\centering
	\begin{tabular}{c|c|c|c}
		& \begin{sideways}Python\end{sideways} & \begin{sideways}MongoDB\end{sideways} & \begin{sideways}Spark\end{sideways} \\
		\hline
		Jane D.  & 0.31 & 0.22 & 0.03\\
		John D.  & 0.03 & 0.01 & 0.01\\
		Erika M. & 0.03 & 0.02 & 0.21\\
		Max M.   & 0.22 & 0.03 & 0.02\\
		Projekt  & 0.30 & 0.22 & 0.21\\
	\end{tabular}
	\caption{Für die unilaterale Empfehlungskomponente relevante Daten aus Tabelle \ref{tbl:methodik:versuchsaufbau:unilateral:tbl1}}
	\label{tbl:nebenrechnungen:unilateral:tbl1}
\end{table}

Berechnung der Abweichung für Jane Doe mit den Werten aus Tabelle \ref{tbl:nebenrechnungen:unilateral:tbl1}:
\begin{gather}
	\nonumber (0.31-0.30)^2 + (0.22-0.22)^2 + (0.03-0.21)^2\\
	\nonumber = 0.0001 + 0 + 0.0324\\
	\nonumber = 0.0325\\
	\approx 0.03
	\label{frml:nebenrechnungen:unilateral:jane}
\end{gather}

Berechnung der Abweichung für John Doe mit den Werten aus Tabelle \ref{tbl:nebenrechnungen:unilateral:tbl1}:
\begin{gather}
	\nonumber (0.03-0.30)^2 + (0.01-0.22)^2 + (0.01-0.21)^2\\
	\nonumber = 0.0729 + 0.0441 + 0.04\\
	\nonumber = 0.157\\
	\approx 0.16
	\label{frml:nebenrechnungen:unilateral:john}
\end{gather}

Berechnung der Abweichung für Erika Muster mit den Werten aus Tabelle \ref{tbl:nebenrechnungen:unilateral:tbl1}:
\begin{gather}
	\nonumber (0.03-0.30)^2 + (0.02-0.22)^2 + (0.21-0.21)^2\\
	\nonumber = 0.0729 + 0.04 + 0\\
	\nonumber = 0.1129\\
	\approx 0.11
	\label{frml:nebenrechnungen:unilateral:erika}
\end{gather}

Berechnung der Abweichung für Max Muster mit den Werten aus Tabelle \ref{tbl:nebenrechnungen:unilateral:tbl1}:
\begin{gather}
	\nonumber (0.22-0.30)^2 + (0.03-0.22)^2 + (0.02-0.21)^2\\
	\nonumber = 0.0064 + 0.0361 + 0.0361\\
	\nonumber = 0.0786\\
	\approx 0.08
	\label{frml:nebenrechnungen:unilateral:max}
\end{gather}

Ausgabe:
\begin{table}[h]
	\centering
	\begin{tabular}{c|c|c}
		Positionierung & Mitarbeiter & Abweichung\\
		\hline
		1 & Jane D.  & 0.03\\
		2 & Max M.   & 0.08\\
		3 & Erika M. & 0.11\\
		4 & John D.  & 0.16
	\end{tabular}
	\caption{Ergebnisliste der unilateralen Empfehlungskomponente für die Daten aus Tabelle \ref{tbl:nebenrechnungen:unilateral:tbl1}}
	\label{tbl:nebenrechnungen:unilateral:ausgabe}
\end{table}