\section{Beispielrechnung für die unilateralen Komponente}
\label{ch:nebenrechnungen:unilateral}
Tabelle \ref{tbl:berechnungDerKatzZentralitaetPseudoMitarbeiter:tbl1} zeigt das zu besetzende Beispiel-Projekt. In den Tabellen dieses Kapitels sind keine Kenntnisse grau, Grundkenntnisse blau und fortgeschrittene Kompetenzen rot markiert.
\begin{table}[h]
	\centering
	\begin{tabular}{c|c}
		Fähigkeit & Kompetenzniveau \\
		\hline
		Python  & \cellcolor{usercolor}Fortgeschritten\\
		MySQL   & \cellcolor{usercolor}Fortgeschritten\\
		HDFS    & \cellcolor{itemcolor}Grundkenntnisse
	\end{tabular}
	\caption{Zu besetzendes Beispiel-Projekt}
	\label{tbl:berechnungDerKatzZentralitaetPseudoMitarbeiter:tbl1}
\end{table}

Tabelle \ref{tbl:berechnungDerKatzZentralitaetPseudoMitarbeiter:tbl2} zeigt die relevanten unilateralen Ergebniswerte des Matrixservices. Für die weitere Berechnung verwendete Referenz-Werte sind hervorgehoben.

\begin{table}[h]
	\centering
	\begin{tabular}{c|c|c|c}
		& Python & MySQL & HDFS\\ 
		\hline
		Doe, Jane     & \cellcolor{usercolor}\textbf{2.92} & \cellcolor{itemcolor}\textbf{2.10} & \cellcolor{exxetagray}0.41\\
		Doe, John     & \cellcolor{exxetagray}0.79 & \cellcolor{itemcolor}0.92 & \cellcolor{itemcolor}\textbf{0.73}\\
		Muster, Erika & \cellcolor{exxetagray}0.36 & \cellcolor{exxetagray}0.42 & \cellcolor{usercolor}1.73\\
		Muster, Max   & \cellcolor{itemcolor}2.01 & \cellcolor{itemcolor}1.32 & \cellcolor{exxetagray}0.32
	\end{tabular}
	\caption{Relevante Ergebnisse des Katz-Algorithmus für den Graphen aus Abbildung \ref{fig:methodik:versuchsaufbau:unilateral:abb1}}
	\label{tbl:berechnungDerKatzZentralitaetPseudoMitarbeiter:tbl2}
\end{table}

Berechnung der Abweichung für Jane Doe mit den Werten aus Tabelle \ref{tbl:berechnungDerKatzZentralitaetPseudoMitarbeiter:tbl2}:
\begin{gather}
	\nonumber (2.92-2.92)^2 + (2.10-2.10)^2 + (0.41-0.73)^2\\
	\nonumber = 0 + 0 + 0.1024\\
	\nonumber = 0.1024\\
	\approx 0.1
	\label{frml:nebenrechnungen:unilateral:jane}
\end{gather}

Berechnung der Abweichung für John Doe mit den Werten aus Tabelle \ref{tbl:berechnungDerKatzZentralitaetPseudoMitarbeiter:tbl2}:
\begin{gather}
	\nonumber (0.79-2.92)^2 + (0.92-2.10)^2 + (0.73-0.73)^2\\
	\nonumber = 4.5369 + 1.3924 + 0\\
	\nonumber = 5,9293\\
	\approx 5.9
	\label{frml:nebenrechnungen:unilateral:john}
\end{gather}

Berechnung der Abweichung für Erika Muster mit den Werten aus Tabelle \ref{tbl:berechnungDerKatzZentralitaetPseudoMitarbeiter:tbl2}:
\begin{gather}
	\nonumber (0.36-2.92)^2 + (0.42-2.10)^2 + (1.73-0.73)^2\\
	\nonumber = 6.5536 + 2.8224 + 1\\
	\nonumber = 10.376\\
	\approx 10.4
	\label{frml:nebenrechnungen:unilateral:erika}
\end{gather}

Berechnung der Abweichung für Max Muster mit den Werten aus Tabelle \ref{tbl:berechnungDerKatzZentralitaetPseudoMitarbeiter:tbl2}:
\begin{gather}
	\nonumber (2.01-2.92)^2 + (1.32-2.10)^2 + (0.32-0.73)^2\\
	\nonumber = 0.8281 + 0.6084 + 0.1681\\
	\nonumber = 1.6046\\
	\approx 1.6
	\label{frml:nebenrechnungen:unilateral:max}
\end{gather}

Ausgabe:
\begin{table}[h]
	\centering
	\begin{tabular}{c|c|c}
		Positionierung & Mitarbeiter & Abweichung\\
		\hline
		1 & Jane D.  & 0.1\\
		2 & Max M.   & 1.6\\
		3 & John D.  & 5.9\\
		4 & Erika M. & 10.4
	\end{tabular}
	\caption{Ergebnisliste der unilateralen Empfehlungskomponente für die Daten aus Tabelle \ref{tbl:berechnungDerKatzZentralitaetPseudoMitarbeiter:tbl2}}
	\label{tbl:nebenrechnungen:unilateral:ausgabe}
\end{table}