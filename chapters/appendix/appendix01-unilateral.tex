\section{Berechnung der unilateralen Empfehlungskomponente}
\label{ch:nebenrechnungen:unilateral}
Tabelle \ref{tbl:berechnungDerKatzZentralitaetPseudoMitarbeiter:tbl1} zeigt das zu besetzende Beispiel-Projekt. In den Tabellen dieses Kapitels sind fehlende Kenntnisse grau, Grundkenntnisse blau und fortgeschrittene Kompetenzen rot markiert.
\begin{table}[h]
	\centering
	\begin{tabular}{c|c}
		Fähigkeit & Kompetenzniveau \\
		\hline
		Python  & \cellcolor{usercolor}Fortgeschritten\\
		MongoDB & \cellcolor{usercolor}Fortgeschritten\\
		HDFS    & \cellcolor{itemcolor}Grundkenntnisse
	\end{tabular}
	\caption{Zu besetzendes Beispiel-Projekt}
	\label{tbl:berechnungDerKatzZentralitaetPseudoMitarbeiter:tbl1}
\end{table}

Relevante Ergebnis-Spalten des Katz-Algorithmus für den Graphen aus Abbildung \ref{fig:methodik:versuchsaufbau:unilateral:abb1} anhand von Formel \ref{frml:empfehlungssysteme:cf:speicherbasiert:formel4} mit $\beta = \frac{1/\lambda}{\nenner}$. Für die weitere Berechnung verwendete Referenz-Werte sind hervorgehoben.

\begin{table}[h]
	\centering
	\begin{tabular}{c|c|c|c}
		& Python & MongoDB & HDFS\\ 
		\hline
		Doe, Jane     & \cellcolor{usercolor}\textbf{2.77} & \cellcolor{itemcolor}\textbf{1.09} & \cellcolor{exxetagray}0.61\\
		Doe, John     & \cellcolor{exxetagray}0.92 & \cellcolor{exxetagray}0.32 & \cellcolor{itemcolor}\textbf{1.16}\\
		Muster, Erika & \cellcolor{exxetagray}0.43 & \cellcolor{exxetagray}0.15 & \cellcolor{usercolor}1.57\\
		Muster, Max   & \cellcolor{itemcolor}1.93 & \cellcolor{exxetagray}0.59 & \cellcolor{exxetagray}0.56
	\end{tabular}
	\caption{Ergebnisse des Katz-Algorithmus für den Graphen aus Abbildung \ref{fig:methodik:versuchsaufbau:unilateral:abb1}}
	\label{tbl:berechnungDerKatzZentralitaetPseudoMitarbeiter:tbl2}
\end{table}

Berechnung der Abweichung für Jane Doe mit den Werten aus Tabelle \ref{tbl:berechnungDerKatzZentralitaetPseudoMitarbeiter:tbl2}:
\begin{gather}
	\nonumber (2.77-2.77)^2 + (1.09-1.09)^2 + (0.61-1.16)^2\\
	\nonumber = 0 + 0 + 0.3025\\
	\nonumber = 0.3025\\
	\approx 0.30
	\label{frml:nebenrechnungen:unilateral:jane}
\end{gather}

Berechnung der Abweichung für John Doe mit den Werten aus Tabelle \ref{tbl:berechnungDerKatzZentralitaetPseudoMitarbeiter:tbl2}:
\begin{gather}
	\nonumber (0.92-2.77)^2 + (0.32-1.09)^2 + (1.16-1.16)^2\\
	\nonumber = 3.4225 + 0.5929 + 0\\
	\nonumber = 4.0154\\
	\approx 4.02
	\label{frml:nebenrechnungen:unilateral:john}
\end{gather}

Berechnung der Abweichung für Erika Muster mit den Werten aus Tabelle \ref{tbl:berechnungDerKatzZentralitaetPseudoMitarbeiter:tbl2}:
\begin{gather}
	\nonumber (0.43-2.77)^2 + (0.15-1.09)^2 + (1.57-1.16)^2\\
	\nonumber = 5.4756 + 0.8836 + 0.1681\\
	\nonumber = 6.5273\\
	\approx 6.53
	\label{frml:nebenrechnungen:unilateral:erika}
\end{gather}

Berechnung der Abweichung für Max Muster mit den Werten aus Tabelle \ref{tbl:berechnungDerKatzZentralitaetPseudoMitarbeiter:tbl2}:
\begin{gather}
	\nonumber (1.93-2.77)^2 + (0.59-1.09)^2 + (0.56-1.16)^2\\
	\nonumber = 0.7056 + 0.25 + 0.36\\
	\nonumber = 1.3156\\
	\approx 1.32
	\label{frml:nebenrechnungen:unilateral:max}
\end{gather}

Ausgabe:
\begin{table}[h]
	\centering
	\begin{tabular}{c|c|c}
		Positionierung & Mitarbeiter & Abweichung\\
		\hline
		1 & Jane D.  & 0.30\\
		2 & Max M.   & 1.32\\
		3 & John D.  & 4.02\\
		4 & Erika M. & 6.53
	\end{tabular}
	\caption{Ergebnisliste der unilateralen Empfehlungskomponente für die Daten aus Tabelle \ref{tbl:berechnungDerKatzZentralitaetPseudoMitarbeiter:tbl2}}
	\label{tbl:nebenrechnungen:unilateral:ausgabe}
\end{table}