\shorthandoff{"}
%************************************************
\chapter{Vollständige Liste an Testfragen}
\label{ch:testfragen}
%************************************************
\begin{table}[h]
	\myfloatalign
	\begin{tabularx}{\textwidth}{XX} \toprule
		\tableheadline{Fragen} & \tableheadline{Erwartete Antwort} \\ \midrule
		- Welche Definition hat Use Case Textdifferenzen auswerten? \newline - Welche Definition hat der Use Case "Textdifferenzen auswerten"? \newline - Wie ist der Use Case Textdifferenzen auswerten definiert? \newline - Welche Definition hat Anwendungsfall Textdifferenzen auswerten? \newline - Welche Definition hat Textdifferenzen auswerten? \newline - Wie ist Textdifferenzen auswerten definiert? & Die Definition des Use Cases "Textdifferenzen auswerten" lautet: Der Use Case startet die Textdifferenz-Ermittlung. Diese ermittelt die folgenden Abweichungen: \newline - fehlende Texte \newline - Textänderungen an den deutschen und englischen Texten, die seit dem Datum "Geändert ab" durchgeführt wurden \\
		\bottomrule
	\end{tabularx}
	\caption[Testfragen nach der Überschrift "Definition" des Use Cases "Textdifferenzen auswerten"]{Testfragen nach der Überschrift "Definition" des Use Cases "Textdifferenzen auswerten"}
	\label{tab:testfragen:tbl1}
\end{table}
\begin{table}[h]
	\myfloatalign
	\begin{tabularx}{\textwidth}{XX}
		\toprule
		\tableheadline{Fragen} & \tableheadline{Erwartete Antwort} \\ \midrule
		- Welchen Anstoß hat Use Case Textdifferenzen auswerten? \newline - Welchen Anstoß hat der Use Case "Textdifferenzen auswerten"? \newline - Wie wird der Use Case Textdifferenzen auswerten angestoßen? \newline - Wie wird der Use Case Textdifferenzen auswerten gestartet? \newline - Welchen Anstoß hat Anwendungsfall Textdifferenzen auswerten? \newline - Welchen Anstoß hat Textdifferenzen auswerten? & Der Use Case "Textdifferenzen auswerten" wird durch den Batch Texte angestoßen. \\
		\bottomrule
	\end{tabularx}
	\caption[Testfragen nach der {\"U}berschrift "Anstoß" des Use Cases "Textdifferenzen auswerten"]{Testfragen nach der {\"U}berschrift "Anstoß" des Use Cases "Textdifferenzen auswerten"}
	\label{tab:testfragen:tbl2}
\end{table}
\begin{table}[h]
	\myfloatalign
	\begin{tabularx}{\textwidth}{XX}
		\toprule
		\tableheadline{Fragen} & \tableheadline{Erwartete Antwort} \\
		\midrule
		- Welche Berechtigung hat Use Case Textdifferenzen auswerten? \newline - Welche Berechtigung hat der Use Case "Textdifferenzen auswerten"? \newline - Welche Berechtigung ist für den Use Case Textdifferenzen auswerten nötig? \newline - Welche Berechtigungsstufe hat Use Case Textdifferenzen auswerten? \newline - Wer hat Zugriff auf den Use Case Textdifferenzen auswerten? \newline - Wofür muss der Benutzer bei Textdifferenzen auswerten berechtigt sein? & Der Benutzer muss beim Use Case "Textdifferenzen ermitteln" für die Textpflege und für das gewählte Segment und die gewählte Produktgruppe berechtigt sein. \\
		\bottomrule
	\end{tabularx}
\caption[Testfragen nach der {\"U}berschrift "Berechtigung" des Use Cases "Textdifferenzen auswerten"]{Testfragen nach der {\"U}berschrift "Berechtigung" des Use Cases "Textdifferenzen auswerten"}
\label{tab:testfragen:tbl3}
\end{table}
\begin{table}[h]
	\myfloatalign
	\begin{tabularx}{\textwidth}{XX}
		\toprule
		\tableheadline{Fragen} & \tableheadline{Erwartete Antwort} \\
		\midrule
		- Welche Vorbedingung hat Use Case Textdifferenzen auswerten? \newline - Welche Vorbedingungen hat der Use Case "Textdifferenzen auswerten"? \newline - Welche Vorbedingung gelten für Textdifferenzen auswerten? \newline - Welche Vorbedingung gelten für den Anwendungsfall Textdifferenzen auswerten? & Für den Use Case "Textdifferenzen ermitteln" muss keine Vorbedingung erfüllt sein. \\
		\bottomrule
	\end{tabularx}
	\caption[Testfragen nach der {\"U}berschrift "Vorbedingung" des Use Cases "Textdifferenzen auswerten"]{Testfragen nach der {\"U}berschrift "Vorbedingung" des Use Cases "Textdifferenzen auswerten"}
	\label{tab:testfragen:tbl4}
\end{table}
\begin{table}[h]
	\myfloatalign
	\begin{tabularx}{\textwidth}{XX}
		\toprule
		\tableheadline{Fragen} & \tableheadline{Erwartete Antwort} \\
		\midrule
		- Welche Plausibilität hat Use Case Textdifferenzen auswerten? \newline - Welche Plausibilität hat der Use Case "Textdifferenzen auswerten"? \newline - Welche Plausibilität gilt für Textdifferenzen auswerten? \newline - Wie plausibel ist der Anwendungsfall Textdifferenzen auswerten? & Für den Use Case "Textdifferenzen auswerten" muss keine Plausibilität erfüllt sein. \\
		\bottomrule
	\end{tabularx}
	\caption[Testfragen nach der {\"U}berschrift "Plausibilität" des Use Cases "Textdifferenzen auswerten"]{Testfragen nach der {\"U}berschrift "Plausibilität" des Use Cases "Textdifferenzen auswerten"}
	\label{tab:testfragen:tbl5}
\end{table}
\begin{table}[h]
	\myfloatalign
	\begin{tabularx}{\textwidth}{XX}
		\toprule
		\tableheadline{Fragen} & \tableheadline{Erwartete Antwort} \\
		\midrule
		- Welchen Effekt hat Use Case Textdifferenzen auswerten? \newline - Zu welchem Effekt führt Textdifferenzen auswerten? \newline - Zu welchem Effekt kommt der Use Case Textdifferenzen auswerten? & Die Abweichungen werden im Use Case "Textdiffernzen auswerten" in einer CSV-Datei protokolliert und per Email an konfigurierte Email-Adressen konfiguriert. \\
		\bottomrule
	\end{tabularx}
	\caption[Testfragen nach der {\"U}berschrift "Effekt" des Use Cases "Textdifferenzen auswerten"]{Testfragen nach der {\"U}berschrift "Effekt" des Use Cases "Textdifferenzen auswerten"}
	\label{tab:testfragen:tbl6}
\end{table}
\begin{table}[h]
	\myfloatalign
	\begin{tabularx}{\textwidth}{XX}
		\toprule
		\tableheadline{Fragen} & \tableheadline{Erwartete Antwort} \\
		\midrule
		- Welche fachlichen Meldungen hat Use Case Textdifferenzen auswerten? \newline - Welche fachliche Meldung gehört zum Use Case "Textdifferenzen auswerten"? \newline - Zeige die fachliche Meldung des Use Cases Textdifferenzen auswerten! \newline - Welche Meldung wird bei Textdifferenzen auswerten ausgegeben? & Beim Use Case "Textdifferenzen auswerten" werden keine fachlichen Meldungen ausgegeben. \\
		\bottomrule
	\end{tabularx}
	\caption[Testfragen nach der {\"U}berschrift "Fachliche Meldung" des Use Cases "Textdifferenzen auswerten"]{Testfragen nach der {\"U}berschrift "Fachliche Meldung" des Use Cases "Textdifferenzen auswerten"}
	\label{tab:testfragen:tbl7}
\end{table}
\begin{table}[h]
	\myfloatalign
	\begin{tabularx}{\textwidth}{XX}
		\toprule
		\tableheadline{Fragen} & \tableheadline{Erwartete Antwort} \\
		\midrule
		- Wie ist die fachliche Schnittstelle des Use Cases Textdifferenzen auswerten aufgebaut? \newline - Zeige mir die fachliche Schnittstelle von Textdifferenzen bewerten! \newline - Zeige die API von Textdifferenzen bewerten! \newline - Welche Parameter hat die fachliche Schnittstelle von Textdifferenzen auswerten? & Für die fachliche Schnittstelle des Use Cases "Textdifferenzen auswerten" existieren die folgenden Parameter: \newline Parameter: Segment, Beschreibung: Nummer (Segment), IN/OUT: IN \newline Parameter: Geändert ab Datum, Beschreibung: Gültiges Datum, IN/OUT: IN \newline Parameter: Produktgruppe, Beschreibung: Produktgruppe, IN/OUT: IN \newline Parameter: Fahrzeug, Beschreibung: Fahrzeug, IN/OUT: IN \newline Parameter: Email, Beschreibung: Email-Adressen, IN/OUT: IN \\
		\bottomrule
	\end{tabularx}
	\caption[Testfragen nach der {\"U}berschrift "Fachliche Schnittstelle" bzw. den Parametern des Use Cases "Textdifferenzen auswerten"]{Testfragen nach der {\"U}berschrift "Fachliche Schnittstelle" bzw. den Parametern des Use Cases "Textdifferenzen auswerten"}
	\label{tab:testfragen:tbl8}
\end{table}
\begin{table}[h]
	\myfloatalign
	\begin{tabularx}{\textwidth}{XX}
		\toprule
		\tableheadline{Fragen} & \tableheadline{Erwartete Antwort} \\
		\midrule
		- Zeige mir die Eingangsparameter an der fachlichen Schnittstelle von Textdifferenzen bewerten! \newline - Welche Eingangsparameter hat die fachliche Schnittstelle von Textdifferenzen auswerten? \newline - Welche Eingangsparameter hat die API von Textdifferenzen auswerten? & Für die fachliche Schnittstelle des Use Cases "Textdifferenzen auswerten" existieren die folgenden Eingangsparameter: \newline Parameter: Segment, Beschreibung: Nummer (Segment), IN/OUT: IN \newline Parameter: "'Geändert ab' Datum", Beschreibung: Gültiges Datum, IN/OUT: IN \newline Parameter: Produktgruppe, Beschreibung: Produktgruppe, IN/OUT: IN \newline Parameter: Fahrzeug, Beschreibung: Fahrzeug, IN/OUT: IN \newline Parameter: Email, Beschreibung: Email-Adressen, IN/OUT: IN \\
		\bottomrule
	\end{tabularx}
	\caption[Testfragen nach den Eingangsparametern des Use Cases "Textdifferenzen auswerten"]{Testfragen nach den Eingangsparametern des Use Cases "Textdifferenzen auswerten"}
	\label{tab:testfragen:tbl9}
\end{table}
\begin{table}[h]
	\myfloatalign
	\begin{tabularx}{\textwidth}{XX}
		\toprule
		\tableheadline{Fragen} & \tableheadline{Erwartete Antwort} \\
		\midrule
		- Zeige mir die Ausgangsparameter an der fachlichen Schnittstelle von Textdifferenzen bewerten! \newline - Welche Ausgangsparameter hat die fachliche Schnittstelle von Textdifferenzen auswerten? \newline - Welche Ausgangsparameter hat die API von Textdifferenzen auswerten? & Für die fachliche Schnittstelle des Use Cases "Textdifferenzen auswerten" existieren keine Ausgangsparameter. \\
		\bottomrule
	\end{tabularx}
	\caption[Testfragen nach den Ausgangsparametern des Use Cases "Textdifferenzen auswerten"]{Testfragen nach den Ausgangsparametern des Use Cases "Textdifferenzen auswerten"}
	\label{tab:testfragen:tbl10}
\end{table}
\begin{table}[h]
	\myfloatalign
	\begin{tabularx}{\textwidth}{XX} \toprule
		\tableheadline{Fragen} & \tableheadline{Erwartete Antwort} \\
		\midrule
		- Welche Use Cases haben den Parameter "'Geändert ab' Datum"? & Der Use Case "Textdifferenzen auswerten" hat den Parameter "'Geändert ab' Datum". \\
		\midrule
		- Welche Use Cases haben den Eingangsparameter "'Geändert ab' Datum"? \newline - Bei welchen Anwendungsfällen ist "'Geändert ab' Datum" ein Eingangsparameter? & Der Use Case "Textdifferenzen auswerten" hat den Eingangsparameter "'Geändert ab' Datum". \\
		\midrule
		- Welche Use Cases haben den Ausgangsparameter "'Geändert ab' Datum"? \newline - Bei welchen Anwendungsfällen ist Geändert ab Datum ein Ausgangsparameter? & Es liegt kein Use Case vor, bei welchem "'Geändert ab' Datum" ein Ausgangsparameter ist. \\
		\bottomrule
	\end{tabularx}
	\caption[Testfragen, bei welchen Use Cases "'Geändert ab' Datum" als Parameter, Eingangsparameter oder  Ausgangsparametern vorkommt]{Testfragen, bei welchen Use Cases "'Geändert ab' Datum" als Parameter, Eingangsparameter oder  Ausgangsparametern vorkommt}
	\label{tab:testfragen:tbl12}
\end{table}
\begin{table}[h]
	\myfloatalign
	\begin{tabularx}{\textwidth}{XX}
	\toprule
	\tableheadline{Fragen} & \tableheadline{Erwartete Antwort} \\
	\midrule
	- Welche Use Cases haben den Parameter Produktgruppe? & Der Use Case "Textdifferenzen auswerten" hat den Parameter "Produktgruppe". \\
	\midrule
	- Welche Use Cases haben den Eingangsparameter Produktgruppe? \newline - Bei welchen Anwendungsfällen ist Produktgruppe ein Eingangsparameter? & Der Use Case "Textdifferenzen auswerten" hat den Eingangsparameter "Produktgruppe". \\
	\midrule
	- Welche Use Cases haben den Ausgangsparameter Produktgruppe? \newline - Bei welchen Anwendungsfällen ist Produktgruppe ein Ausgangsparameter? & Es liegt kein Use Case vor, bei welchem "Produktgruppe" ein Ausgangsparameter ist. \\
	\bottomrule
	\end{tabularx}
	\caption[Testfragen, bei welchen Use Cases "Produktgruppe" als Parameter, Eingangsparameter oder  Ausgangsparametern vorkommt]{Testfragen, bei welchen Use Cases "Produktgruppe" als Parameter, Eingangsparameter oder  Ausgangsparametern vorkommt}
	\label{tab:testfragen:tbl13}
\end{table}
\begin{table}[h]
	\myfloatalign
	\begin{tabularx}{\textwidth}{XX}
		\toprule
		\tableheadline{Fragen} & \tableheadline{Erwartete Antwort} \\
		\midrule
		- Welche Use Cases haben den Parameter Fahrzeug? & Der Use Case "Textdifferenzen auswerten" hat den Parameter "Fahrzeug". \\
		\midrule
		- Welche Use Cases haben den Eingangsparameter Fahrzeug? \newline - Bei welchen Anwendungsfällen ist Fahrzeug ein Eingangsparameter? & Der Use Case "Textdifferenzen auswerten" hat den Eingangsparameter "Fahrzeug". \\
		\midrule
		- Welche Use Cases haben den Ausgangsparameter Fahrzeug? \newline - Bei welchen Anwendungsfällen ist Fahrzeug ein Ausgangsparameter? & Es liegt kein Use Case vor, bei welchem "Fahrzeug" ein Ausgangsparameter ist. \\
		\bottomrule
	\end{tabularx}
	\caption[Testfragen, bei welchen Use Cases "Fahrzeug" als Parameter, Eingangsparameter oder  Ausgangsparametern vorkommt]{Testfragen, bei welchen Use Cases "Fahrzeug" als Parameter, Eingangsparameter oder  Ausgangsparametern vorkommt}
	\label{tab:testfragen:tbl14}
\end{table}
\begin{table}[h]
	\myfloatalign
	\begin{tabularx}{\textwidth}{XX}
		\toprule
		\tableheadline{Fragen} & \tableheadline{Erwartete Antwort} \\
		\midrule
		- Welche Use Cases haben den Parameter Email? & Der Use Case "Textdifferenzen auswerten" hat den Parameter "Email". \\
		\midrule
		- Welche Use Cases haben den Eingangsparameter E-Mail? \newline - Bei welchen Anwendungsfällen ist Email ein Eingangsparameter? & Der Use Case "Textdifferenzen auswerten" hat den Eingangsparameter "Email". \\
		\midrule
		- Welche Use Cases haben den Ausgangsparameter Email? \newline - Bei welchen Anwendungsfällen ist E-Mail ein Ausgangsparameter? & Es liegt kein Use Case vor, bei welchem "Email" ein Ausgangsparameter ist. \\
		\bottomrule
	\end{tabularx}
	\caption[Testfragen, bei welchen Use Cases "Email" als Parameter, Eingangsparameter oder  Ausgangsparametern vorkommt]{Testfragen, bei welchen Use Cases "Email" als Parameter, Eingangsparameter oder  Ausgangsparametern vorkommt}
	\label{tab:testfragen:tbl15}
\end{table}
\begin{table}[h]
	\myfloatalign
	\begin{tabularx}{\textwidth}{XX} \toprule
		\tableheadline{Fragen} & \tableheadline{Erwartete Antwort} \\
		\midrule
		- Welche Use Cases haben den Parameter Segment? & Der Use Case "Textdifferenzen auswerten" hat den Parameter "Segment". \\
		\midrule
		- Welche Use Cases haben den Eingangsparameter Segment? \newline - Bei welchen Anwendungsfällen ist Segment ein Eingangsparameter? & Der Use Case "Textdifferenzen auswerten" hat den Eingangsparameter "Segment". \\
		\midrule
		- Welche Use Cases haben den Ausgangsparameter Segment? \newline - Bei welchen Anwendungsfällen ist Segment ein Ausgangsparameter? & Es liegt kein Use Case vor, bei welchem "Segment" ein Ausgangsparameter ist. \\
		\bottomrule
	\end{tabularx}
	\caption[Testfragen, bei welchen Use Cases "Segment" als Parameter, Eingangsparameter oder  Ausgangsparametern vorkommt]{Testfragen, bei welchen Use Cases "Segment" als Parameter, Eingangsparameter oder  Ausgangsparametern vorkommt}
	\label{tab:testfragen:tbl16}
\end{table}
\begin{table}[h]
	\myfloatalign
	\begin{tabularx}{\textwidth}{XX} \toprule
		\tableheadline{Fragen} & \tableheadline{Erwartete Antwort} \\
		\midrule
		- Welche Use Cases haben den Parameter "'Geändert ab' Datum"? \newline - Welche Use Cases haben den Parameter Geändert ab Datum? \newline - Welche Use Cases haben den Parameter Geändert ab? & Der Use Case "Textdifferenzen auswerten" hat den Parameter "'Geändert ab' Datum". \\
		\midrule
		- Welche Use Cases haben den Eingangsparameter "'Geändert ab' Datum"? \newline - Bei welchen Anwendungsfällen ist Geändert ab Datum ein Eingangsparameter? & Der Use Case "Textdifferenzen auswerten" hat den Eingangsparameter "'Geändert ab' Datum". \\
		\midrule
		- Welche Use Cases haben den Ausgangsparameter Geändert ab Datum? \newline - Bei welchen Anwendungsfällen ist Geändert ab Datum ein Ausgangsparameter? & Es liegt kein Use Case vor, bei welchem "'Geändert ab' Datum" ein Ausgangsparameter ist. \\
		\bottomrule
	\end{tabularx}
	\caption[Testfragen, bei welchen Use Cases "'Geändert ab' Datum" als Parameter, Eingangsparameter oder  Ausgangsparametern vorkommt]{Testfragen, bei welchen Use Cases "'Geändert ab' Datum" als Parameter, Eingangsparameter oder  Ausgangsparametern vorkommt}
	\label{tab:testfragen:tbl17}
\end{table}
\begin{table}[h]
	\myfloatalign
	\begin{tabularx}{\textwidth}{XX} \toprule
		\tableheadline{Fragen} & \tableheadline{Erwartete Antwort} \\
		\midrule
		- Welche Use Cases haben den Parameter Produktgruppe? & Der Use Case "Textdifferenzen auswerten" hat den Parameter "Produktgruppe". \\
		\midrule
		- Welche Use Cases haben den Eingangsparameter Produktgruppe? \newline - Bei welchen Anwendungsfällen ist Produktgruppe ein Eingangsparameter? & Der Use Case "Textdifferenzen auswerten" hat den Eingangsparameter "Produktgruppe". \\
		\midrule
		- Welche Use Cases haben den Ausgangsparameter Produktgruppe? \newline - Bei welchen Anwendungsfällen ist Produktgruppe ein Ausgangsparameter? & Es liegt kein Use Case vor, bei welchem "Produktgruppe" ein Ausgangsparameter ist. \\
		\bottomrule
	\end{tabularx}
	\caption[Testfragen, bei welchen Use Cases "Produktgruppe" als Parameter, Eingangsparameter oder  Ausgangsparametern vorkommt]{Testfragen, bei welchen Use Cases "Produktgruppe" als Parameter, Eingangsparameter oder  Ausgangsparametern vorkommt}
	\label{tab:testfragen:tbl18}
\end{table}
\begin{table}[h]
	\myfloatalign
	\begin{tabularx}{\textwidth}{XX} \toprule
		\tableheadline{Fragen} & \tableheadline{Erwartete Antwort} \\
		\midrule
		- Welche Use Cases haben den Parameter Fahrzeug? & Der Use Case "Textdifferenzen auswerten" hat den Parameter "Fahrzeug". \\
		\midrule
		- Welche Use Cases haben den Eingangsparameter Fahrzeug? \newline - Bei welchen Anwendungsfällen ist Fahrzeug ein Eingangsparameter? & Der Use Case "Textdifferenzen auswerten" hat den Eingangsparameter "Fahrzeug". \\
		\midrule
		- Welche Use Cases haben den Ausgangsparameter Fahrzeug? \newline - Bei welchen Anwendungsfällen ist Fahrzeug ein Ausgangsparameter? & Es liegt kein Use Case vor, bei welchem "Fahrzeug" ein Ausgangsparameter ist. \\
		\bottomrule
	\end{tabularx}
	\caption[Testfragen, bei welchen Use Cases "Fahrzeug" als Parameter, Eingangsparameter oder  Ausgangsparametern vorkommt]{Testfragen, bei welchen Use Cases "Fahrzeug" als Parameter, Eingangsparameter oder  Ausgangsparametern vorkommt}
	\label{tab:testfragen:tbl19}
\end{table}
\shorthandon{"}