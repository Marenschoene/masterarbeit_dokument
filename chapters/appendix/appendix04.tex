%************************************************
\chapter{Fragebögen der Evaluation}\label{ch:evaluation}
%************************************************
\newcommand{\frageEins}{Wie zufrieden waren Sie mit der Kommunikation mit dem Chatbot?}
\newcommand{\frageZwei}{Vertrauen Sie den Antworten des Chatbots?}
\newcommand{\frageDrei}{Was hat Ihnen gut gefallen?}
\newcommand{\frageVier}{Was hat Ihnen weniger gut gefallen?}
\newcommand{\frageFuenf}{Welche Eigenschaften des Chatbots fehlen Ihnen noch?}
\newcommand{\frageSechs}{Welche Fragen/Fragentypen sollte der Chatbot zusätzlich beantworten können?}
\newcommand{\frageSieben}{Könnten Sie sich vorstellen, den Chatbot anstelle der Dokumentation im Alltag zu benutzen?}
\newcommand{\frageAcht}{Würde Ihnen der Chatbot helfen, effizienter zu arbeiten?}
\newcommand{\frageNeun}{Worauf sollte bei der Weiterentwicklung des Chatbots geachtet werden?}
\newcommand{\frageZehn}{Könnten Sie sich vorstellen parallel zur Bearbeitung der XML-Dateien auch den Chatbot zu erweitern und zu testen?}
\newcommand{\frageElf}{Könnten Sie sich vorstellen, dass der Chatbot langfristig die Suche in der IT-Dokumentation ersetzt?}

\section{Fragebogen von Eva Hägele}
\label{ch:evaluation:eva}
\speaker{\frageEins} \\
Ich denke, der Chatbot hat potenzial. Ist aber aktuell noch etwas unausgereift, da er oft keinen Treffer bekommt, wenn man die Fragenstruktur nicht kennt.\\\\
\speaker{\frageZwei} \\
Ja, meistens wurde als Antwort einfach der Use Case verlinkt, der ja korrekt sein sollte :) \\\\
\speaker{\frageDrei} \\
Wenn er keine explizite Antwort findet, dann wird der passende Use Case verlinkt. Das hilft vorallem, wenn man sich noch nicht so gut mit der Struktur der Systemdokumentation auskennt. \\\\
\speaker{\frageVier} \\
Wenn die Frage zu unspezifisch ist und mehrere Use Cases treffen, könnte nochmal rückgefragt werden um die Trefferliste zu reduzieren. \\
Die Suchbegriffe müssen relativ exakt zu den hinterlegten Texten passen, damit es einen Treffer gibt. \\\\
\speaker{\frageFuenf} \\
Ich fände es gut, wenn er durch Nachfragen die Treffermenge reduzieren könnte bzw. dadurch unterstützt schneller an die gesuchte Info zu kommen. \\\\
\speaker{\frageSechs} \\
Vermutlich sind beliebte Anwendungsfälle auch die Suche nach bestimmten Funktionalitäten, wie z.B. Wo kann ich Preise (oder anderweitige Stammdaten) anlegen/pflegen? \\
$[$Oftmals sucht man bei uns auch nach Abkürzungen, bspw. : "Wo für steht der Begriff D-Land?", "Was bedeutet AMDS?" Da du aber nur in deinem Test Use Cases als Basis genutzt hast, wäre das vermutlich relativ leicht den aktuellen Chatbot zu erweitern in dem du noch die entsprechenden Kaptiel zu Begriffsdefinitionen aus der aktuellen Doku mit aufnimmst.$]$ \\\\
\speaker{\frageSieben} \\
Ja, vorallem für Anwender, welcher sich noch wenig mit der Struktur der Systemdokumentation auskennen, ist der Chatbot ein guter Einstieg um relevante Themen zu finden. \\\\
\speaker{\frageAcht} \\
Ja, wenn der Chatbot noch entsprechend erweitert wird, dass er auf mehr Fragen antworten kann, führt er bestimmt zu einer Effizienzsteigerung. \\\\
\speaker{\frageNeun} \\
Es müssten noch gezielter ermittelt werden, welche Art von Fragen häufig gestellt werden. Wahrscheinlich müssten man mal eine längere Testphase durhführen um diese zu ermitteln. \\\\
\speaker{\frageZehn} \\
Ja, man könnte bei der Erweiterung/Pfelge der Dokumentation auch parallel noch den Chatbot erweitern und testen. Das stelle ich mir jetzt nicht als zu großen Mehraufwand vor. \\\\
\speaker{\frageElf} \\
Ja, mit ein bisschen mehr künstlicher Intelligenz kann er die Suche bestimmt in vielen Fällen ablösen, da durch seine Unterstützung die Suche zielgerichteter wird.  Wenn man aber nur einen Überblick über alle für den Suchbegriff relevaten Themen möchte, wäre die Suche vermutlich aktuell effizienter. Der Chatbot würde sich aber bestimmt auch für diesen Use Case anpassen lassen. 
\newpage

\section{Fragebogen von Robin Gebauer}
\label{ch:evaluation:robin}
\speaker{\frageEins} \\
Ohne das Wissen wie Fragen strukturiert sind, ist es schwierig Antworten zu bekommen. \\\\
\speaker{\frageZwei} \\
Ja, so weit ich der Sysdoku vertraue ;-) \\\\
\speaker{\frageDrei} \\
Easy to use (sofern man das Fragen-pattern kennt) \\
Direkte Antworten sofern etwas gefunden wird, bzw die Alternative Notfalllösung mit dem Link zum Use Case ist gut gelungen. \\\\
\speaker{\frageVier} \\
Darstellung der Daten ist nicht immer optimal in einem Chatfenster. Liegt aber auch an der Datenstruktur (Tabellen etc.) \\\\
\speaker{\frageFuenf} \\
Suche mit Synonymen \\\\
\speaker{\frageSechs} \\
Fragen aus User Sicht / Zur Fehleranalyse / Funktionssuche \\\\
\speaker{\frageSieben} \\
Ja \\\\
\speaker{\frageAcht} \\
Ja, erleichtert die Suche in der Sysdoku \\\\
\speaker{\frageNeun} \\
Prüfen auf welchem Chatsystem man arbeiten möchte. Vor Allem bei der Darstellung komplexer Strukturen kann ein Skype Chatfenster nicht ausreichend sein. \\\\
\speaker{\frageZehn} \\
Sollte so weit wie möglich automatisch geschehen, notweniger Zusatzaufwand wird normal nicht vom Projekt getragen. \\\\
\speaker{\frageElf} \\
Eine Konventionelle Suche gibt mir bei einer Stichwortsuche eine größere strukturiertere Übersicht als es einem Chatfenster dargestellt werden kann. Daher wird er die Suche vermutlich nicht ganz ersetzen.