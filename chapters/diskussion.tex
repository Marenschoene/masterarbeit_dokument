\shorthandoff{"}
\chapter{Diskussion}
\label{ch:diskussion}

\section{Zusammenfassung der Forschungsergebnisse}
\label{ch:diskussion:zusammenfassung}
Im Rahmen der vorliegenden Master-Thesis wurde eine Fallstudie durchgeführt, über welche die Vorschläge eines unilateralen und eines bilateralen Empfehlungsansatzes hinsichtlich der Besetzung offener Projektpositionen miteinander verglichen wurden. Dabei konnte festgestellt werden, dass das bilaterale Vorschlagsverfahren bei vier der fünf evaluierten Projektpositionen eine höhere Zufriedenheit bei den Angestellten erzielen konnte. Bei einer Projektstelle sorgten dagegen der unilaterale Empfehlungsansatz für eine etwas höhere Zufriedenheit.

Vergleichbar fielen auch die Ergebnisse auf Seiten der Projektmanager aus. Hier prognostizierten die Verantwortlichen für vier der fünf Projektpositionen eine höhere Arbeitsleistung von den vorgeschlagenen Mitarbeitern des bilateralen Empfehlungssystems. Einzig für die Stelle, bei welchen auch das bilaterale Vorschlagsverfahren eine geringere Zufriedenheit bei den Mitarbeitern erzielte, prognostizierten die Projektmanager für beide Empfehlungsansätze gleiche Arbeitsleistungen.

Zusätzlich wurde evaluiert, wie Mitarbeiter und Projektmanager mit möglicher Unterforderung bei der Projektarbeit umgehen. Diese Information ist gemäß der Theorie des \acp{PEFit} zur korrekten Berechnung der Kongruenz von Mitarbeiter und Projektposition notwendig. Im Rahmen der Befragung konnte hierbei festgestellt werden, dass sowohl Projektmanager als auch Mitarbeiter mehrheitlich eine Unterforderung vermeiden möchten.

\newpage
\section{Interpretation der Forschungsergebnisse}
\label{ch:diskussion:interpretation}
Bei Betrachtung der Ergebnisse ist festzustellen, 

\section{Beschränkungen der Forschung}
\label{ch:diskussion:beschraenkungen}

\section{Empfehlung für weiterführende Forschung}
\label{ch:diskussion:empfehlungen}
- Wesentlich mehr Bewertungen bei Präferenzen --> Lag an Liste? --> Liste: Kein Coldstart
- Kein grundsätzlicher Cold Start zu erwarten
- Kein Gemischtwarenladen, sondern spezialisieren --> Spiegelt sich in UX-Projekt wieder
- Die Leute präferieren 50 Prozent der gesuchten Fähigkeiten in den Beispielprojektpositionen --> Ist andere Hälfte nicht präferiert oder egal?
- Im hinteren Teil würde man immer wieder die gleichen kriegen --> Neue kommen nur schwer rein, dabei wollen es ja sehr viele machen
- vllt noch schreiben, dass wenn die mit eingespielt werden, wahrscheinlich dann hinten landen und es deswegen wichtig ist, die zu pflegen
- Auf Kurven eingehen

Aus Ergebnissen:
- Wie viel von gesuchten Fähigkeiten aus Projekten aus Long Tail?
- Wenn ich einen Mitarbeiter für eine bestimmte Technologie suche, wird er diese in 50 Prozent aller Fälle nicht präferieren

- Wie viele Cloud, KI? --> Wie viele nur 1 Bewertung (Ausreißer)
- Kein Coldstart bei Präferenzen
- Es sind immer mind. 19 andere Personen auf einem Skill verbunden

Projekt A: unilateral/bilateral\\
Projekt B: bilateral/unilateral\\
Projekt C: bilateral/unilateral\\
Projekt D: unilateral/bilateral\\
Projekt E: bilateral/unilateral


\shorthandon{"}
