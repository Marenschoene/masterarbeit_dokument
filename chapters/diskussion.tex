\definecolor{exxetagray}{gray}{0.75}
\definecolor{itemcolor}{RGB}{179,217,255}
\definecolor{usercolor}{RGB}{255,204,179}

\shorthandoff{"}
\chapter{Diskussion}
\label{ch:diskussion}

% Fragen:
% \begin{itemize}
%     \item Was sind Randfälle? Bspw. 0.9 anforderungs-fähigkeiten-fit und 0.1 bedürfnisse-angebot-fit. Wenn der PJ fit nicht ganz erfüllt ist, geht es dann trotzdem vor weiterhin die MA-Präferenzen zu berücksichtigen, unter der Prämisse, dass dann ggf. skills eingespart werdne müssen, oder gilt das nur, wenn sowieso alle die skills erfüllen? Was sind realistische Randbedingungen? -> habe ich dann lieber MA, die unzufrieden sind, oder Projekte, die nicht gestafft werden können?
% \end{itemize}

% Was offen bleibt: Endlichkeit von Elementen -> Wie MA zuordnen, dass nicht immer ein MA, der alles kann für alles Projekte vorgeschlagen wird? (Berücksichtigung der Verfügbarkeit als Slider)
% Berücksichtigung von Teamzugehörigkeit

% Nutzerbasierte (in unserem Fall elementbasiert, da MA die Elemente sind) Gewichtung -> Gewichtung der Präferenzen im Verhältnis zu Fähigkeiten von MA zu MA unterschiedlich wichtig

\shorthandon{"}