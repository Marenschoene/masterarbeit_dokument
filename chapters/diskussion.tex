\definecolor{exxetagray}{gray}{0.75}
\definecolor{itemcolor}{RGB}{179,217,255}
\definecolor{usercolor}{RGB}{255,204,179}

\shorthandoff{"}
\chapter{Diskussion}
\label{ch:diskussion}

Kurz eingehen auf die ergebnisse der präferenz und fähigkeitsbefragung

% Abgleich Präferenzen angeforderte Projektpositionen:
% Ein detaillierter Abgleich der Mitarbeitenden, die angaben mit Projekt 5 zufrieden zu sein, mit den Mitarbeitenden, die eine hohe Übereinstimmung aufwiesen (> 80 Prozent) zeigte, dass lediglich 5 dieser 13 Mitarbeitenden auch angaben, mit dem Projekt zufrieden zu sein.
% Weiter ist zu erkennen, dass Projekt 4 zwar bei der Befragung der Zufriedenheit die höchste Anzahl an zufriedenen Mitarbeitenden aufwies ($\approx$ 30 Befragte), in der Übereinstimmung der Präferenzen jedoch im Vergleich am zweitwenigsten Mitarbeitende eine Übereinstimmung über 40 Prozent aufwiesen.
% Darüber hinaus liegt die Anzahl an Mitarbeitenden mit einer Übereinstimmung der Präferenzen unter 20 Prozent\footnote{inkl.} bei Projekt 4 mit 23 Befragten mit Abstand am höchsten.
% Hier wieder bei diskussion die  schwankung erklären -> annahme, dass das mit der ANzahl der angeforderten Projektpositionen zusammenhängt!

Argumentation:
\begin{itemize}
    \item Zufriedenheit zu integrieren ist auf jeden Fall gut -> daher ergebnisse besser
    \item Wie dann gewichtet wird scheint weniger eine Rolle zu spielen -> sehr schwankende Ergebnisse
    \item Die geringe Bedeutung der Gewichtung kann auch darauf zurückzuführen sein, dass viele Mitarbeitende auch das präferieren, was sie können (daher ist die Diskrepanz zwischen hohem und kleinen gewicht nicht so groß) -> überprüfen!!
    \item Haupterkenntnis ist also, dass es wichtig ist die Zufriedenheit zu berücksichtigen, Gewicht an sich aber kaum eine Rolle gespiel that, hauptsache die Präferenzen sind dabei
    \item Die Präferenzen ist dann das was den unterschied macht zwischen "Mitarbeitender passt" und "Mitarbeitender passt gut"
    \item Viele unbekannte Einflussfaktoren auf die Zufriedenheit: Gewichtung der einzelnen Präferenzen für einen Mitarbeitenden, andere Projekte die zur Auswahl stehen
    \item Unbekannte Einflussfaktoren auf die Arbeitsleistung: welche Fähigkeiten einem Mitarbeitendem fehlen spielt eine entscheidende rolle -> kann er diese schnell erlernen oder nicht?
    \item Next steps: Experteninterviews um zu identifizieren, was das kernproblem des bestehenden algorithmus ist und dann kriterium mit aufnehmen anhand einer der vorgestellten maßnahmen
    \item perspektivisch auch die Präferenzen integrieren, aber geringere Bedeutung als andere Kriterien (Verfügbarkeit, Vollständigkeit und Aktualität der Daten, Teamzugehörigkeit)
    \item Wichtige überlegung: problem der auswahl mitarbeitender für projekte ist entscheidungsproblem -> Frage, wieviel man dem system übergibt und was man den Managern noch an Entscheidung überlässt (bspw. Zufriedenheit als slider)
\end{itemize}

% Für die diskussion bzgl. der ergebnisse bei den ergebnissen auf kommentare achten!
% Hier auf die unterschiede zwischen histogramm und zufriedenheit bzw. erwartete arbeitsleistung angaben hinweisen!

%Ein Vergleich der Veränderungen bei beiden Kennzahlen zeigt zudem, dass die Zufriedenheit im Verhältnis zu der erwarteten Arbeitsleistung in mehr Fällen gestiegen ist.
% Zugleich führte der bilaterale Algorithmus in zwei Fällen zu einer Verschlechterung der zu erwarteten Arbeitsleistung und in einem Fall zu einer Verschlechterung der Zufriedenheit.

% Bei den Gewichten: wenn sich sowohl alpha als auch mitarbeitende ändern dann verändert sich zu viel input -> ob ein mitarbeitender dann in liste ist oder nicht ist zu ausschlaggebend
% wenn alle mitarbeitende gleich bleiben und nur alpha sich verändert kommen trotzdem komische werte raus :D
% dass bei 4 der 5 verschiedenen gewichte eine steigerung erzielt werden konnte deutet eher darauf hin, dass die gewichtung hier nicht ausschlaggebend war. dies spricht auch dafür, dass bei demselben gewicht die zufriedenheit bei manchen gewichten sowohl gestiegen als auch gesunken ist 
% -> hier auch ergebnisse über alle mitarbeitenden hinweg anführen für die edge cases 0,1 und 0,9 und zeigen, wie es sich da verhält
Daraus geht hervor, dass keines der Gewichte eindeutig zu einer Verbesserung oder einer Verschlechterung im Vergleich zum unilateralen Algorithmus geführt hat.

% Fragen:
% \begin{itemize}
%     \item Was sind Randfälle? Bspw. 0.9 anforderungs-fähigkeiten-fit und 0.1 bedürfnisse-angebot-fit. Wenn der PJ fit nicht ganz erfüllt ist, geht es dann trotzdem vor weiterhin die MA-Präferenzen zu berücksichtigen, unter der Prämisse, dass dann ggf. skills eingespart werdne müssen, oder gilt das nur, wenn sowieso alle die skills erfüllen? Was sind realistische Randbedingungen? -> habe ich dann lieber MA, die unzufrieden sind, oder Projekte, die nicht gestafft werden können?
% \end{itemize}

% Was offen bleibt: Endlichkeit von Elementen -> Wie MA zuordnen, dass nicht immer ein MA, der alles kann für alles Projekte vorgeschlagen wird? (Berücksichtigung der Verfügbarkeit als Slider)
% Berücksichtigung von Teamzugehörigkeit

% Nutzerbasierte (in unserem Fall elementbasiert, da MA die Elemente sind) Gewichtung -> Gewichtung der Präferenzen im Verhältnis zu Fähigkeiten von MA zu MA unterschiedlich wichtig

\shorthandon{"}