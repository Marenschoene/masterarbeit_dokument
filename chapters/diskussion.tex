\shorthandoff{"}
\chapter{Diskussion}
\label{ch:diskussion}

\section{Zusammenfassung der Forschungsergebnisse}
\label{ch:diskussion:zusammenfassung}
Im Rahmen der vorliegenden Master-Thesis wurde eine Fallstudie durchgeführt. Über diese wurden die Vorschläge eines unilateralen und eines bilateralen Empfehlungsansatzes bei der Besetzung offener Projektpositionen miteinander verglichen. Dabei konnte festgestellt werden, dass das bilaterale Vorschlagsverfahren bei vier der fünf evaluierten Projektpositionen eine höhere Zufriedenheit bei den Angestellten erzielen konnte. Bei einer Projektstelle sorgten dagegen der unilaterale Empfehlungsansatz für eine höhere Zufriedenheit.

Vergleichbar fielen auch die Ergebnisse auf Seiten der Projektmanager aus. Hier prognostizierten die Verantwortlichen für vier der fünf Projektpositionen eine höhere Arbeitsleistung von den vorgeschlagenen Mitarbeitern des bilateralen Empfehlungssystems. Einzig für die Stelle, bei welchen auch das bilaterale Vorschlagsverfahren eine geringere Zufriedenheit bei den Mitarbeitern erzielte, prognostizierten die Projektmanager für beide Empfehlungsansätze gleiche Arbeitsleistungen.

Außerdem wurde festgestellt, dass 17 Prozent der Mitarbeiter keine Kompetenzbewertung im Intranet der EXXETA AG vorgenommen hatten. Bei grafischer Darstellung von Präferenzen und beherrschten Fähigkeiten der Mitarbeiter konnte außerdem das lange (Ratten-)Schwanz beobachtet werden. Bezüglich der in den vordefinierten Projektpositionen benötigten Kompetenzen konnte festgestellt werden, dass die Anteile an Mitarbeitern, welche eine gesuchte Fähigkeit beherrschen und gleichzeitig präferieren und Angestellten, welche nicht über eine benötigte Kompetenz verfügen und diese dennoch präferieren, im Durchschnitt gleich groß sind. Darüber hinaus konnte beobachtet werden, dass knapp über 40 Prozent aller Mitarbeiter, welche eine gesuchte Fähigkeit beherrschen, deren Anwendung nicht als Präferenz angaben.

Abschließend wurde evaluiert, wie Mitarbeiter und Projektmanager mit möglicher Unterforderung bei der Projektarbeit umgehen. Diese Information ist gemäß der Theorie des \acp{PEFit} zur korrekten Berechnung der Kongruenz von Mitarbeiter und Projektposition notwendig. Im Rahmen der Befragung konnte hierbei festgestellt werden, dass sowohl Projektmanager als auch Mitarbeiter mehrheitlich eine Unterforderung vermeiden möchten.

\section{Interpretation der Forschungsergebnisse}
\label{ch:diskussion:interpretation}
Bei Implementierung der beiden Empfehlungsansätze wurde aufgrund der Erkenntnisse aus Kapitel \ref{ch:empfehlungssysteme} erwartet, dass der lange (Ratten-)Schwanz und der Kaltstart die Vorschlagserstellung beeinträchtigen könnten. Daher lag beiden Empfehlungsmethoden ein hybrider und graphenbasierter Ansatz zugrunde, welcher über die Einbeziehung von Fähigkeitsbewertungen und Teamzuordnungen beide Probleme löste. Dieses Vorgehen ist mit Blick auf die Ergebnisse in Abbildung \ref{fig:ergebnisse:analyse:abb1} als sinnvoll zu bewerten, da sowohl bei beherrschten als auch Präferierten Fähigkeiten ein langer (Ratten-)Schwanz erkennbar ist. Zusätzlich ist Kapitel \ref{ch:ergebnisse:analyse:intranetUndUmfrage} zu entnehmen, dass vier Mitarbeiter im Intranet keine einzige Fähigkeit bewertet hatten und diese ohne Einbeziehung der Teamzuordnungen folglich von einem Kaltstart betroffen wären.

Zu den drei Kurven
\newpage
- Annahme Kurve B war zutreffend\\
- Sehr viele Fähigkeiten, die die MA nicht beherrschen, aber präferieren --> Informelles Lernen?\\
- Von allem was sie können präferieren sie nur ca. die Hälfte --> ohne bilaterales System ist die Wahrscheinlichkeit 50 Prozent, dass man eine Fähigkeit nicht präferiert --> Bilateral hilft, aus "Filterblase" zu kommen / Bezug eher auf Abb. 6.3\\
- Ergebnis Forschungsfrage: Außer bei Projektposition C ist zu erkennen, dass sowohl Zufriedenheit als auch erwartete Leistung zunimmt / Bei C ist zu erkennen, dass die Mitarbeiter unzufriedener mi Projekt sind (im Vgl zu anderen) --> Man sollte Präferenzen nicht nur als boolschen Wert betrachten, sondern auch "Unwichtigkeit" mit einbeziehen --> Nicht mögen und egal werden aktuell gleich behandelt --> Mitarbeiter werden dann runter sortiert, wenn sie etwas nicht mögen --> Future Work

\section{Beschränkungen der Forschung}
\label{ch:diskussion:beschraenkungen}
- Sehr homogene Gruppe --> Alles Java-Entwickler (auch hier heterogen) --> Müsste überprüft werden, ob diese Ergebnisse auch für heterogene Gruppe zutreffen

\section{Empfehlung für weiterführende Forschung}
\label{ch:diskussion:empfehlungen}
- Wesentlich mehr Bewertungen bei Präferenzen --> Lag an Liste? --> Liste: Kein Coldstart
- Kein grundsätzlicher Cold Start zu erwarten
- Kein Gemischtwarenladen, sondern spezialisieren --> Spiegelt sich in UX-Projekt wieder
- Die Leute präferieren 50 Prozent der gesuchten Fähigkeiten in den Beispielprojektpositionen --> Ist andere Hälfte nicht präferiert oder egal?
- Im hinteren Teil würde man immer wieder die gleichen kriegen --> Neue kommen nur schwer rein, dabei wollen es ja sehr viele machen
- vllt noch schreiben, dass wenn die mit eingespielt werden, wahrscheinlich dann hinten landen und es deswegen wichtig ist, die zu pflegen
- Auf Kurven eingehen

Aus Ergebnissen:
- Wie viel von gesuchten Fähigkeiten aus Projekten aus Long Tail?
- Wenn ich einen Mitarbeiter für eine bestimmte Technologie suche, wird er diese in 50 Prozent aller Fälle nicht präferieren

- Wie viele Cloud, KI? --> Wie viele nur 1 Bewertung (Ausreißer)
- Kein Coldstart bei Präferenzen
- Es sind immer mind. 19 andere Personen auf einem Skill verbunden

Projekt A: unilateral/bilateral\\
Projekt B: bilateral/unilateral\\
Projekt C: bilateral/unilateral\\
Projekt D: unilateral/bilateral\\
Projekt E: bilateral/unilateral


\shorthandon{"}
