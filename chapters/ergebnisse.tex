\shorthandoff{"}
\chapter{Forschungsergebnisse}
\label{ch:ergebnisse}

\section{Fähigkeiten und Präferenzen der Mitarbeiter}
\label{ch:ergebnisse:analyse}

\subsection{Fähigkeitsbewertungen in Umfrage und Intranet}
\label{ch:ergebnisse:analyse:intranetUndUmfrage}
An der Umfrage unter den Mitarbeitern haben N=23 Personen aus dem Fachbereich \JES der EXXETA AG teilgenommen. Diese Angestellten haben im Rahmen der Befragung 1408 Präferenzbewertungen abgegeben, welche sich auf 370 einzelne Fähigkeiten verteilen. Das entspricht knapp über 61 abgegebenen Wünschen pro Mitarbeiter. Git ist mit 18 Beurteilungen die meist präferierte Fähigkeit.

Im Intranet des Unternehmens haben die 23 Angestellten 643 Bewertungen hinsichtlich ihrer bereits beherrschten Fähigkeiten abgegeben. Damit verfügt eine Person über etwa 28 Kompetenzen. In Summe beherrschen die Mitarbeiter des Fachbereichs \JES 212 der \anzFaehigkeiten unterschiedlichen, im Intranet gespeicherten Fähigkeiten. Java ist mit 16 Beurteilungen die meist beherrschte Kompetenz.

Abbildung \ref{fig:ergebnisse:analyse:abb1} zeigt, dass sowohl bei Darstellung der präferierten Fähigkeiten, als auch bei Betrachtung der beherrschten Kompetenzen, der in Kapitel \ref{ch:empfehlungssysteme:cf:speicherbasiert} vorgestellte lange (Ratten-)Schwanz gut erkennbar ist. Dieser ist in beiden Fällen jedoch weniger stark ausgeprägt, als in der Referenzdarstellung aus Abbildung \ref{fig:empfehlungssysteme:cf:speicherbasiert:abb1}.

\begin{figure}[h]
	\centering
	\includegraphics[width=1\textwidth]{gfx/long-tail-insgesamt.png}
	\caption{Langer (Ratten-)Schwanz bei beherrschten und präferierten Fähigkeiten der Mitarbeiter}
	\label{fig:ergebnisse:analyse:abb1}
\end{figure}

Bei der gemeinsamen Betrachtung von Kompetenzen und Wünschen ist auf Mitarbeiterebene festzustellen, dass ein durchschnittlicher Angestellter etwa 75 Fähigkeiten als vorhanden und/oder präferiert bewertet hat. Abbildung \ref{fig:ergebnisse:analyse:abb3} zeigt, zu welchen Anteilen die Kompetenzen als beherrscht und/oder gewünscht markiert wurden.

\begin{figure}[h]
	\centering
	\includegraphics[width=1\textwidth]{gfx/auswertung-anteil-an-faehigkeiten.png}
	\caption{Anteil beherrschter und präferierter Fähigkeiten bei einem durchschnittlichen Mitarbeiter}
	\label{fig:ergebnisse:analyse:abb3}
\end{figure}

In Abbildung \ref{fig:ergebnisse:analyse:abb3} ist zu erkennen, dass ein durchschnittlicher Angestellter ca. 37 Prozent seiner insgesamt beurteilten Kompetenzen gleichzeitig beherrscht (orange markiert und orange-blau schraffiert). Von diesen beherrschten Fähigkeiten werden nur knapp über die Hälfte präferiert (orange-blau schraffiert). Dem stehen etwa 63 Prozent an Fähigkeiten gegenüber, welche der Angestellte zwar präferiert, aber nicht beherrscht (blau markiert).

In den vorliegenden Daten des Intranets ist darüber hinaus zu beobachten, dass vier bzw. ca. 17 Prozent der Mitarbeiter keine einzige Fähigkeit bewertet haben. Diese Angestellten sind seit Einführung des Kompetenz-Bewertungssystems durchgehend in einem Projekt tätig und haben daher ihre Fähigkeiten noch nicht gepflegt. Bei der Umfrage gab es dagegen keinen Mitarbeiter, welcher keine einzige Fähigkeit als Präferenz ausgewählte.

\subsection{Bewertungen hinsichtlich der Projektpositionen}
\label{ch:ergebnisse:analyse:projektpositionen}
Im Rahmen der vorliegenden Master-Thesis wurden fünf beispielhafte Projektpositionen definiert und in Kapitel \ref{ch:methodik:evaluation} vorgestellt. Abbildung \ref{fig:ergebnisse:analyse:abb5} zeigt, wie viele der befragten Mitarbeiter die für die jeweiligen Stellen durchschnittlich gesuchte Fähigkeit beherrschen bzw. präferieren.

\begin{figure}[h]
	\centering
	\includegraphics[width=1\textwidth]{gfx/anteil-bewertungen-je-projektposition.png}
	\caption{Anteil an Mitarbeitern, welche die in den Beispielprojektpositionen gesuchten Fähigkeiten beherrschen und/oder präferieren}
	\label{fig:ergebnisse:analyse:abb5}
\end{figure}

In Abbildung \ref{fig:ergebnisse:analyse:abb5} ist zu erkennen, dass ca. ein Drittel aller Mitarbeiter die durchschnittlich gesuchte Fähigkeit jeder Projektposition beherrschen (orange markiert und orange-blau schraffiert). Der Anteil an Angestellten, welche eine Fähigkeit beherrschen und gleichzeitig präferieren (orange-blau schraffiert) ist bei den Projektpositionen A bis D jedoch wesentlich höher, als der Anteil an Mitarbeitern, welche die Kompetenz beherrschen und nicht präferieren (orange markiert). Abschließend ist zu Abbildung \ref{fig:ergebnisse:analyse:abb5} festzustellen, dass die orange-blau schraffierten und blau markierten Anteile an Mitarbeitern im Durchschnitt gleich groß sind.

\section{Ergebnisse der Fallstudie}
\label{ch:ergebnisse:fallstudie}

\subsection{Erwartete Zufriedenheit der Projektmitarbeiter}
\label{ch:ergebnisse:fallstudie:umfrageMitarbeiter}
In der Umfrage unter den Angestellten wurde erhoben, welche Zufriedenheit die Mitarbeiter der EXXETA AG mit Tätigkeiten auf den Projektpositionen aus Abbildung \ref{fig:methodik:evaluation:abb2} prognostizieren. Die Ergebnisse sind in Abbildung \ref{fig:ergebnisse:fallstudie:abb1} dargestellt.

\begin{figure}[h]
	\centering
	\includegraphics[width=1\textwidth]{gfx/mitarbeiter-zufriedenheit-umfrage.png}
	\caption{Anzahl an Mitarbeitern, welche zufrieden bzw. unzufrieden mit der Tätigkeit auf den jeweiligen Beispielprojektpositionen wären}
	\label{fig:ergebnisse:fallstudie:abb1}
\end{figure}

In Abbildung \ref{fig:ergebnisse:fallstudie:abb1} ist zu erkennen, dass die Mitarbeiter überwiegend eine hohe Zufriedenheit mit den Projektpositionen A und B prognostizieren. Mit einer Tätigkeit auf den Projektpositionen C und E zeigen sich die Angestellten eher unzufrieden. Projektposition D stehen die Mitarbeiter gespalten gegenüber, sodass etwa die Hälfte der Befragten zufrieden und die andere Hälfte unzufrieden mit dieser Tätigkeit wäre.

Abbildung \ref{fig:ergebnisse:analyse:abb7} zeigt, für wie viele der 23 befragten Mitarbeiter der bilaterale Empfehlungsansatz gegenüber dem unilateralen Vorgehen für eine höhere Zufriedenheit seitens der Angestellten sorgte. Wie in Kapitel \ref{ch:methodik:evaluation} beschrieben, wird die Entstehung einer höheren Zufriedenheit mit den Projekttätigkeiten angenommen, wenn das bilaterale System die Angestellten bei einer prognostizierten Zufriedenheit höher und bei einer erwarteten Unzufriedenheit niedriger positioniert als die unilaterale Anwendung.

\begin{figure}[h]
	\centering
	\includegraphics[width=0.7\textwidth]{gfx/zufriedenheit-projekte.png}	
	\caption{Ergebnisse des bilateralen Empfehlungsansatzes im Vergleich zum unilateralen Vorgehen hinsichtlich der Mitarbeiterzufriedenheit}
	\label{fig:ergebnisse:analyse:abb7}
\end{figure}

In Abbildung \ref{fig:ergebnisse:analyse:abb7} ist zu erkennen, dass der bilaterale Empfehlungsansatz einen Großteil der Angestellten für die Projektpositionen A und B zugunsten einer höheren Zufriedenheit positionierte. Bei den Projektpositionen D und E sind erreiche das bilaterale Vorschlagsverfahren für knapp über die Hälfte der Mitarbeiter eine höhere Zufriedenheit. Bei Projektposition C konnte dagegen der unilaterale Empfehlungsansatz für den Großteil der Angestellten eine höhere Zufriedenheit erzielen.

\subsection{Prognostizierte Arbeitsleistung der Projektmanager}
\label{ch:ergebnisse:fallstudie:arbeitsleistung}
An der Umfrage unter den Projektmanagern haben N=6 Personen teilgenommen. Fünf der Teilnehmer sind im Bereich \JES tätig. Eine Person stammt aus einer anderen Abteilung, welche jedoch ähnliche Technologien bei der Projekttätigkeit einsetzt. Abbildung \ref{fig:ergebnisse:fallstudie:arbeitsleistung:abb1} zeigt, von den Vorschlägen welches Empfehlungsansatzes die Projektmanager eine höhere Arbeitsleistung erwarten.

\begin{figure}[h]
	\centering
	\includegraphics[width=1\textwidth]{gfx/ergebnisse-projektmanager-arbeitsleistung.png}	
	\caption{Ergebnisse der Umfrage unter den Projektmanager hinsichtlich der erwarteten Arbeitsleistung der Mitarbeiter}
	\label{fig:ergebnisse:fallstudie:arbeitsleistung:abb1}
\end{figure}

An den Ergebnissen aus Abbildung \ref{fig:ergebnisse:fallstudie:arbeitsleistung:abb1} ist zu erkennen, dass die Projektmanager bei vier der fünf Projektpositionen von den vorgeschlagenen Mitarbeitern des bilateralen Empfehlungsansatzes eine höhere Arbeitsleistung erwarten. Für Projektposition A prognostiziert sogar kein einziger Projektmanager eine höhere Leistung von den unilateralen Empfehlungen. Auffällig ist außerdem, dass die Hälfte der Teilnehmer von den Vorschlägen des bilateralen Empfehlungssystems für Projektposition D eine viel höhere Arbeitsleistung erwartet. Einzig bei den Vorschläge für Projektposition C sind keine Unterschiede zwischen der erwarteten Arbeitsleistung des uni- und des bilateralen Empfehlungssystems erkennbar.

\subsection{Einschätzungen hinsichtlich möglicher Unterforderung}
\label{ch:ergebnisse:fallstudie:kurven}
Sowohl unter den Projektmanagern als auch den Mitarbeitern wurde in den jeweiligen Umfragen erhoben, wie sie mögliche Unterforderung bei der Projektarbeit bewerten. Die Ergebnisse sind in Abbildung \ref{fig:ergebnisse:fallstudie:kurven:abb1} dargestellt. In der Grafik sind die Antwortmöglichkeiten aus den Abbildungen \ref{fig:methodik:evaluation:abb3} bzw. \ref{fig:methodik:evaluation:manager:abb3} durch die entsprechenden Kurven aus Darstellung \ref{fig:methodik:versuchsaufbau:unilateral:abb2} ersetzt.

\begin{figure}[h]
	\centering
	
	\subfloat[Projektmanager]{\includegraphics[width = 0.5\textwidth]{gfx/ergebis-kurve-projektmanager.png}}
	\subfloat[Mitarbeiter]{\includegraphics[width = 0.5\textwidth]{gfx/ergebis-kurve-mitarbeiter.png}}
	
	\caption{Umgang mit Unterforderung bei Projektmanagern und Mitarbeitern}
	\label{fig:ergebnisse:fallstudie:kurven:abb1}
\end{figure}

In Abbildung \ref{fig:ergebnisse:fallstudie:kurven:abb1} ist zu erkennen, dass sowohl die befragten Projektmanager als auch die teilnehmenden Mitarbeiter mehrheitlich angaben, Unterforderung bei der Projektarbeit zu vermeiden.
\shorthandon{"}
