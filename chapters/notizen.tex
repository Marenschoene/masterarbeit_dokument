\shorthandoff{"}
\chapter{Notizen}
\label{ch:notizen}

\section{Relevanz und Motivation}
\label{ch:notizen:relevanzMotivation}
Warum ist das Thema relevant?
\begin{itemize}
    \item Personen und Jobs die nicht zusammenpassen bedeuten für Unternehmen Aufwand (Kosten, Zeit, Qualitätseinbuße, Unzufriedenheit).
    \item Nicht nur im Recruiting relevant, sondern auch innerhalb von Organisationen % (http://www.timothy-judge.com/Cable%20&%20Judge%20OBHDP%201996.pdf). Dort steckt viel Potenzial.)
    \item Daher: Berücksichtigen diverser Attribute von Personen und Umgebung bei der Auswahl von Personen für Berufe/Jobpositionen (d.h. weg vom alleinigen skill matching zum job matching).
    \item Dadurch soll die Performance der Mitarbeiter im Job, sowie deren Mitarbeiterzufriedenheit gesteigert werden.
\end{itemize}

\section{Wie funktioniert die Empfehlung von Personen für Jobs?}
\label{ch:notizen:maEmpfehlung}
Wie funktionieren Empfehlungen allgemein?
\begin{itemize}
    \item Erklärung Nutzer, Element, Nutzer-Element-Kombination.
    \item Dann Erklärung, wie Empfehlungen allgemein bestimmt werden (2 Teilschritte).
    \item Allgemein erfolgt die Empfehlung von Nutzern bzw. Elementen in Empfehlungssystemen in zwei Teilschritten: % Erklärung S. 405 unten: https://link.springer.com/content/pdf/10.1007/s11577-021-00753-z.pdf
    \begin{enumerate}
        \item Vorhersage unbekannter Nutzer-Element-Kombinationen
        \item Ranking und Ausgabe der Top-K-Nutzer
    \end{enumerate}
\end{itemize}
% 1. schritt geht davon aus, dass Daten zu Attributen unvollständig sind und daher vorhergesagt werden müssen
% 2. Schritt ist dann die Bestimmung der passenden Mitarbeiter und der Ausgabe der Empfehlung (damit wird sich in der Arbeit befasst)

% Erklärung, dass für die Lösung der Forschungsfrage der zweite Schritt relevant ist
Übergang: Wie funktioniert die Empfehlung von Mitarbeitern allgemein? (d.h. Zuweisung von Personen für Jobs/Jobpositionen)
\begin{itemize}
    \item Es sollen Personen empfohlen werden, die am besten auf einen Job / eine Stelle passen.
    \item Am besten passen = Kompatibilität zwischen Person und Job. % (file:///C:/Users/masc6/Downloads/1887_3147276-A%20Comprehensive%20Assessment%20of%20the%20Person-Environment%20Fit%20Dimensions%20and%20Their%20Relationships%20With%20Work-Related%20Outcomes.pdf, S. 568)
    \item Die Kompatibilität zwischen Individuum und Job ist in der Literatur als Person-Environment-Fit bekannt. % (file:///C:/Users/masc6/Downloads/1887_3147276-A%20Comprehensive%20Assessment%20of%20the%20Person-Environment%20Fit%20Dimensions%20and%20Their%20Relationships%20With%20Work-Related%20Outcomes.pdf)
    \item Erklärung Person-Environment-Fit als Maß dafür, wie gut Person und Umgebung zueinander passen bzw. sich gegenseitig ergänzen (Stichwort komplementär und ergänzender Fit). Erklärung, was unter Person, Umgebung und P-E-Fit verstanden wird.
    \item Person-Environment-Fit besagt, dass ein Übereinstimmen zwischen Person und Umgebung positive effekte erzielt (Zufriedenheit, Leistung), während ein Misfit negative Effekte verursachen kann. % (file:///C:/Users/masc6/Downloads/1887_3147276-A%20Comprehensive%20Assessment%20of%20the%20Person-Environment%20Fit%20Dimensions%20and%20Their%20Relationships%20With%20Work-Related%20Outcomes.pdf)
    \item Vier Komponenten des PE-Fit: PJ-Fit, PO-Fit, PG-Fit, PS-Fit. % (file:///C:/Users/masc6/Downloads/1887_3147276-A%20Comprehensive%20Assessment%20of%20the%20Person-Environment%20Fit%20Dimensions%20and%20Their%20Relationships%20With%20Work-Related%20Outcomes.pdf)
    \item Berufen auf PE-Fit als Funktion von Person-Organization-Fit, Need-Supply-Fit und Demand-Ability-Fit % (S. 3, https://link.springer.com/content/pdf/10.1007/s12144-022-03461-9.pdf)
    \item -> hier noch entscheiden, ob an den 4 dimensionen oder den 3 orientiert wird
\end{itemize}

Ablauf Berechnung PE-Fit:
\begin{enumerate}
    \item Bestimmen, welche Kriterien für die Empfehlung von Mitarbeitern herangezogen werden sollen (skills, job-attribute, ogranizational-attribute).
Nach Russell ist die Performance von Personen in Jobpositionen grundsätzlich von drei Komponenten abhängig:
\begin{itemize}
    \item dem Skillmatch (Anforderungs-Fähigkeiten-Fits)
    \item dem Jobmatch (Person-Job-Fit) und
    \item dem Organizational-Match (Person-Organization-Fit) \cite{russell:book}.
\end{itemize}
Demnach orientieren sich die Attribute von Personen und Jobs für die Empfehlung in der Praxis meist an einer der Komponenten beziehungsweise einer Kombination dieser.
 \item Bestimmen des Fits (Match) -> wie wird diese Übereinstimmung berechnet?
Wie bestimme ich den Fit? Direkt oder Indirekte Berechnung des Fits. % (https://onlinelibrary.wiley.com/doi/epdf/10.1111/j.1744-6570.1996.tb01790.x?saml_referrer)
\begin{itemize}
    \item direkt: PE-Fit ergibt sich aus dem direkten Bestimmen der Kompatibilität zwischen Person und Umgebung
    \item indirekt: PE-Fit ergibt sich indirekt aus dem Vergleich zwischen den zwei unabhängige voneinander bewerteten Variablen Person und Umgebung
\end{itemize}
Weiter: Subjective and objective indirect PE-Fit.
\begin{itemize}
    \item subjecktiv: Person und Umgebung werden unabhängig voneinander durch die Zielperson bestimmt
    \item objektiv: Person und Umgebung werden unabhängig voneinander bestimmt, unabhängig von der Wahrnehmung einer Person
\end{itemize} % (S. 291, French JRP Jr, Rogers W, Cobb S. (1974). Adjustment as person–environment fit. In CoelhoDAHGV, Adams JE (Ed.),Coping and adaptation)
Problem der Commensurability (Dimensionen bei Erhebung von Person und Environment -> person und umgebung getrennt voneinander oder auf einer Skala?) % http://psychology.iresearchnet.com/industrial-organizational-psychology/recruitment/person-environment-fit/

\item Ausgabe der Person(en) mit der größten Übereinstimmung
\end{enumerate}

Annahme: indirekte Berechnung für diesen Anwendungsfall am sinnvollsten, da Umgebung (angeforderte Fähigkeiten, angebotene Fähigkeiten) separat von der Person (zur Verfügung stehende Fähigkeiten, präferierte Fähigkeiten) erhoben werden.

Welche Möglichkeiten der indirekten Berechnung gibt es?
\begin{itemize}
    \item Ähnlichkeitsmaß
    \item Differenzmaß
    \item ...
\end{itemize}

Frage: wie können die verschiendenen Ausprägungen des Fits in der Berechnung vereint werden? Wie können diese gewichtet werden? simple linear combination, non-linear combination % (file:///C:/Users/masc6/Downloads/SekiguchiHuberOBHDP.pdf)

------------------

Wie werden Präferenzen von Nutzern in Empfehlungen miteinbezogen?

Vorab: Unterscheidung personalisierte und nicht-personalisierte Empfehlungssysteme % file://wsl%24/Ubuntu/home/masc6/Projects/masterarbeit/literatur/Die%20Ordnung%20von%20Empfehlungen.pdf
\begin{itemize}
    \item nicht-personalisierte / schwach-personalisierte Empfehlungssysteme -> eigentlich bewegen wir uns hier, wenn wir die besten Mitarbeiter für ein Projekt ermitteln wollen (Popularitätsmetriken, z.B. "Top 10 Songs in Deutschland", "Am häufigsten gekauft")
    \item personalisierte Empfehlungssysteme ("Songs für dich", "Das könnte dich auch interessieren")-> hier werden Präferenzen im eigentlichen Sinn miteinbezogen
\end{itemize}

Im Regelfall werden Präferenzen in Empfehlungssystemen dafür verwendet vorherzusagen, welche anderen Elemente im System einen Nutzer oder eine Gruppe an Nutzern interessieren könnten.
Im vorliegenden Anwendungsfall handelt es sich bei den angegebenen Präferenzen jedoch eigentlich eher um Bewertungen, und zwar Bewertungen der Fähigkeiten im System durch die Nutzer (Mitarbeiter).
Der Unterschied zu personalisierten Empfehlungssystemen im klassischen Sinn liegt also darin, dass in dem vorliegenden Anwendungsfall nichts "vorhergesagt" wird, sondern sich lediglich für einen Weg, die Bewertung von Fähigkeiten in die Ermittlung des PE-Fits miteinzubeziehen, entschieden werden muss.

Frage: Sollen die Präferenzen der Mitarbeiter (Like, Dislike) als separate Attribute betrachtet werden? -> das würde dann in die Richtung der verschiedenen Ausprägungen des PE-Fit gehen (in diesem Fall Anforderungen-Fähigkeiten-Fit und Angebot-Bedürfnisse-Fit). Das Ergebnis der Fits würde dann kombiniert werden und gemeinsam den PE-Fit ergeben. Der PE-Fit würde sich dann also aus einer Kombination der einzelnen Fits ergeben. Würde man den PE-Fit zukünftig noch genauer ermitteln wollen, liesen sich in dieser Darstellung recht verständlich neue Attribute hinzufügen (z.B. Kapazität).
Oder sollen die Präferenzen unmittelbar den Wert einer Fähigkeit beeinflussen (d.h. wie in Johannes Thesis bedeutet 1 Like zb addition eines wertes und ein dislike Abzug o.Ä.)?

------------------

Wie erhebe ich Präferenzen? % siehe: atomistic, molecular, molar (https://psycnet.apa.org/fulltext/2006-08435-006.pdf?auth_token=7270682a9555be377243d8180634ce7719154c9e), file:///C:/Users/masc6/Downloads/1887_3147276-A%20Comprehensive%20Assessment%20of%20the%20Person-Environment%20Fit%20Dimensions%20and%20Their%20Relationships%20With%20Work-Related%20Outcomes.pdf

------------------

Überprüfen des Outcomes:
Wie erhebe ich Zufriedenheit? % Bsp: S. 289 outcome, S. 300 (http://www.timothy-judge.com/Cable%20&%20Judge%20OBHDP%201996.pdf), file:///C:/Users/masc6/Downloads/1887_3147276-A%20Comprehensive%20Assessment%20of%20the%20Person-Environment%20Fit%20Dimensions%20and%20Their%20Relationships%20With%20Work-Related%20Outcomes.pdf, S. 572 file:///C:/Users/masc6/Downloads/1887_3147276-A%20Comprehensive%20Assessment%20of%20the%20Person-Environment%20Fit%20Dimensions%20and%20Their%20Relationships%20With%20Work-Related%20Outcomes.pdf, S. 6 https://www.ncbi.nlm.nih.gov/pmc/articles/PMC7437360/pdf/fpsyg-11-01740.pdf

Zukünftig:
Um Empfehlung weiter zu optimieren genügt es nicht die Anforderungs-Fähigkeiten-Fit um die Präferenzen der Personen zu ergänzen, sondern auch Dinge wie Persönlichkeit, Organizational-Fit, Person-Group-Fit, etc.
Hier: MA wurden bereits von Exxeta rekrutiert. Es kann davon ausgegangen werden, dass der Organizational-Fit in dem Rahmen bereits abgeklärt wurde.

\newpage

\section{Fragen}
\label{ch:notizen:fragen}

\shorthandon{"}