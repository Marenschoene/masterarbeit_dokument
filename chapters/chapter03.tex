\shorthandoff{"}
\chapter{Empfehlungssysteme}
\label{ch:empfehlungssysteme}
\section{Einführung}
\label{ch:empfehlungssysteme:einfuehrung}
Der Begriff des Empfehlungssystems ist im englischsprachigen Raum auch unter Bezeichnungen wie "Recommender System" \cite[S. 1]{lu:2015}, "Recommender Engine" \cite[S. 1]{panigrahi:2016} und "Recommendation System" \cite[S. 1]{ebesu:2018} verbreitet. Er wurde erstmals im Jahr 1997 von \textcite[S. 1]{resnick:1997} geprägt. Dass der Begriff gerade zu diesem Zeitpunkt entstand, ist auf die zur damaligen Zeit stark wachsende Internetnutzung und die damit verbundenen einfachen Möglichkeiten zur Sammlung und Auswertung großer Mengen an Nutzerdaten zurückzuführen \cite[S. xvii]{recommenderSystems:2016}.\\
Besonders bekannt für den Einsatz von Empfehlungssystemen sind große IT-Konzerne wie Amazon, Facebook, Google und Netflix \cite[S. 1]{zarzour:2018}. Diese Unternehmen nutzen Recommender Engines, um ihren Kunden personalisierte Vorschläge zu den Inhalten ihrer Plattformen anzuzeigen \cite[S. 2]{jeckmans:2013}. In vielen Fällen entfällt dabei für den Anwender vollständig die Notwendigkeit einer manuellen Suche \cite[S. 1]{comibingCareer:2013}.\\
Der Einsatz von Empfehlungssystemen wird in der Literatur kritisch diskutiert. Beispielsweise beobachteten \textcite[S. 17f.]{alfano:2020}, dass das Empfehlungssystem der Videostreaming-Plattform YouTube dazu tendiert, neuen Anwendern zur Verlängerung ihrer Nutzungszeit verschwörungstheoretische Inhalte auszuspielen. Eine Untersuchung von Forschern des sozialen Netzwerks Facebook kam zu dem Ergebnis, dass deren Recommender Engine Nutzern verstärkt Inhalte präsentiert, welche konform mit deren Ideologien sind \cite[S. 2]{bakshy:2015}. \textcite[S. 1ff.]{pariser:2012} prägte für diese Art der Personalisierung den Begriff der Filterblase.\\
Empfehlungssysteme haben aber auch einen bedeutenden Anteil am wirtschaftlichen Erfolg großer Internetplattformen. So führen beispielsweise \textcite[S. 6f.]{sharma:2015} etwa 30 Prozent des Internetverkehrs beim Online-Händler Amazon unmittelbar auf den Einsatz von Empfehlungssystemen zurück. \textcite[S. 5]{gomezuribe:2016} stellten bei einer Analyse der Streaming-Plattform Netflix fest, dass ca. 80 Prozent der Nutzungszeit auf Videos entfällt, welche Nutzern ohne vorherige Suche von einer Recommender Engine angezeigt wurden.\\
Um solche Vorschläge generieren zu können, suchen Empfehlungssysteme relevante Inhalte basierend auf den Präferenzen der Anwender aus \cite[S. 1]{das:2017}. Zu diesem Zweck müssen benötigte Nutzerdaten zunächst erhoben und in einer maschinell auswertbaren Struktur gespeichert werden.

\section{Zugrundeliegende Datenstruktur}
\label{ch:empfehlungssysteme:arbeitsweise}
Empfehlungssysteme können die Präferenzen ihrer Nutzer sowohl über explizite als auch implizite Rückmeldungen erfassen. Explizites Feedback erhalten Plattformen beispielsweise über abgegebene Produktbewertungen oder "Gefällt mir"-Angaben in sozialen Netzwerken. Um implizite Rückmeldungen auszuwerten, werden häufig Verhaltensweisen der Nutzer aufgezeichnet. Hierbei kann es sich beispielsweise um Suchverläufe oder die Wiedergabedauer von Videos handeln \cite[S. 3]{pu:2012}.\\
Das gesammelte Feedback überführen Analysten in die Struktur von Matrizen \cite[S. 11f.]{recommenderSystems:2016}. Ein Beispiel für eine Matrix mit Bewertungen der Fähigkeiten von Mitarbeitern ist in Tabelle \ref{tbl:empfehlungssysteme:arbeitsweise:tbl1} dargestellt.
\begin{table}[h]
	\centering
	\begin{tabular}{c|c|c|c|c|c|c|c}
	 & JavaScript & Java & MySQL & Db2 & Hadoop & Spark & ... \\
	\hline
	Doe, Jane & 3 & ? & 2 & ? & ? & ? & ... \\
	Doe, John & ? & 4 & 3 & 3 & 1 & ? & ... \\
	Musterfrau, Erika & ? & 5 & ? & ? & 5 & 3 & ... \\
	Mustermann, Max & 2 & 1 & 1 & ? & ? & ? & ... \\
	... & ... & ... & ... & ... & ... & ... & ... \\
	\end{tabular}
	\caption{Beispiel für die Matrixdarstellung von Fähigkeiten}
	\label{tbl:empfehlungssysteme:arbeitsweise:tbl1}
\end{table}\\
In Tabelle \ref{tbl:empfehlungssysteme:arbeitsweise:tbl1} sind in der ersten Spalte die Mitarbeiter eines Unternehmens gespeichert. Diese werden als Nutzer (User) bezeichnet. In der Kopfzeile der folgenden Spalten sind verschiedene Fähigkeiten eingetragen, welche Elemente (Items) genannt werden. In der Mitte der Tabelle befinden sich die Bewertungen (Ratings) der Fähigkeiten \cite[S. 1f.]{strub:2016}. Im Beispiel aus Tabelle \ref{tbl:empfehlungssysteme:arbeitsweise:tbl1} wurden die Beurteilungen auf einer Skala von eins bis fünf vergeben. Diese bewerteten Matrix-Einträge werden  als beobachtet (observed) oder spezifiziert (specified) bezeichnet. Unbewertete Elemente sind mit einem Fragezeichen gekennzeichnet und werden unbeobachtet (unobserved) oder fehlend (missing) genannt \cite[S. 8]{recommenderSystems:2016}.\\
Zahlreiche Wissenschaftler in der Literatur sind sich einig, dass für die Empfehlung geeigneter Kandidaten für eine Stelle bzw. Projektposition ein einfacher Abgleich zwischen gesuchten und vorhandenen Fähigkeiten in der Matrix eine unzureichende Lösung darstellt \cite[S. 1]{enhancingERecruitment:2012}\cite[S. 1]{faerber:2003}\cite[S. 2]{prospect:2010} und der Komplexität der Aufgabe nicht gerecht wird \cite[S. 1]{malinowski:2008}. So kritisieren beispielsweise \textcite[S. 1f.]{mitre:2014}, dass bei einem solchen Ansatz Synonyme und verwandte Fähigkeit nicht in die Suche einbezogen werden. Um diesem Problem zu begegnen, existieren in der Literatur zahlreiche unterschiedliche Ansätze, Recommender Enginges zu implementieren. Einer davon ist die Umsetzung eines wissensbasierten Empfehlungssystems \cite[S. 2f.]{dwivedi:2017}.

\section{Wissensbasierte Empfehlungssysteme}
\label{ch:empfehlungssysteme:wissensbasierteAnsaetze}
Bei einem wissensbasierten Empfehlungssystem werden die Schlüsselwörter der Matrix aus Tabelle \ref{tbl:empfehlungssysteme:arbeitsweise:tbl1} um weiteres Domänenwissen angereichert, welches in die Suche einbezogen wird \cite[S. 168f.]{recommenderSystems:2016}. Dieses Vorgehen wird beispielsweise auch von der Anwendung SAP R/3 Human Resources in der Praxis angewendet \cite[S. 2]{malinowski:2006}.\\
Unternehmen können beim Erstellen von Wissensdatenbanken auf bereits vorhandene Ontologien zurückgreifen. Beispielsweise stellt die Europäische Kommission mit \acs{ESCO} explizit zum Zweck der Stellenbesetzung eine mehrsprachige Ontologie mit vordefinierten Kompetenzen, Fähigkeiten und Qualifikationen bereit \cite[S. 1ff.]{leVrang:2014}. Ein vergleichbares Angebot existiert mit \acs{ONet} auch von der Regierung der Vereinigten Staaten von Amerika \cite[S. 2]{aCombinedRepresentation:2018}.\\
In solchen Wissensdatenbanken können Unternehmen zu Stellen passende Mitarbeiter über semantische Suchen abfragen. Hierbei kann das System über hinterlegte Regeln sowohl Synonyme als auch Beziehungen berücksichtigen \cite[S. 2f.]{singto:2013}. Jedoch werden Mitarbeiter in den Ergebnissen nur ausgegeben, wenn sie die Suchanfrage exakt erfüllen. Aus diesem Grund stellen \textcite[S. 3]{bianchini:2008} bei semantischen Suchen eine hohe Genauigkeit der Resultate fest, bemängeln jedoch die Flexibilität der Verfahren.\\
Auch ist es möglich, innerhalb der Ontologien über Graphenalgorithmen die Übereinstimmungen zwischen Fähigkeiten zu berechnen \cite[S. 1f.]{balachander:2018}. Bei solchen, auf Ähnlichkeitsberechnungen basierenden Verfahren, beobachten \textcite[S. 4]{bianchini:2008} eine hohe Flexibilität bei der Suche, kritisieren jedoch die mangelnde Genauigkeit der Verfahren.\\
Um die Nachteile beider Ansätze auszugleichen, implementierten die Forscher \textcite[S. 4ff.]{semanticMatchmaking:2009} ein eigenes wissensbasiertes Empfehlungssystem. Dieses sollte gleichzeitig hohe Genauigkeit und Flexibilität gewährleisten. Für dieses Vorhaben entwickelten die Wissenschaftler eine Ontologie, welche die Fähigkeiten der Mitarbeiter sehr feingranular erfasst. Einzelne Kompetenzen mussten dabei über mehrere Einträge spezifiziert werden. Zu Stellen passende Personen wurden anschließend über einen Algorithmus ermittelt, welcher semantische Schlussfolgerungen mit Ähnlichkeitsberechnungen kombinierte. Mit diesem Ansatz erreichten \textcite[S. 11f.]{semanticMatchmaking:2009} ihr Ziel, ein genaues und zugleich flexibles wissensbasiertes Empfehlungssystem zu implementieren. Jedoch muss kritisch angemerkt werden, dass die Pflege der Fähigkeiten in der Ontologie als sehr aufwändig erscheint. Somit muss in Frage gestellt werden, ob Mitarbeiter ein solches System zuverlässig im Unternehmensalltag pflegen würden. Auch \textcite[S. 2]{aCombinedRepresentation:2018} beobachten in anderen Job-Ontologien wie \acs{ONet}, dass Informationen über Fähigkeiten häufig nicht aktuell gehalten werden.\\
Sofern die Ergebnisse nicht vollständiger Präzision unterliegen müssen, ziehen viele Wissenschaftler daher die Entwicklung anderer flexiblerer Empfehlungssysteme den wissensbasierten Ansätzen vor. Meist entstehen dabei Implementierungen im Bereich des kollaborativen oder inhaltsbasierten Filterns. Diese Ansätze verfolgen das Ziel, unbeobachtete Bewertungen aus beobachteten Präferenzen abzuleiten und daraus Vorschläge zu generieren. Dazu verwenden sie die Daten zusätzlicher Mitarbeiter und Projekte zur Bestimmung von Vorschlägen \cite[S. 3ff.]{recommenderSystems:2016}.\\

\section{Kollaboratives Filtern}
\label{ch:empfehlungssysteme:cf}

\section{Inhaltsbasiertes Filtern}
\label{ch:empfehlungssysteme:cfundcb}
Implementierungen auf Basis des kollaborativen Filterns fokussieren die Daten anderer Mitarbeiter. Dabei werden dem Anwender Stellen empfohlen, für welche sich auch andere Nutzer interessieren, die ihm ähnlich sind \cite[S. 3]{jobMatcher:2020}. Diese Art von Empfehlungssystem ist für die vorliegende Problemstellung jedoch ungeeignet, da hier Mitarbeiter für Projekte empfohlen werden sollen und nicht umgekehrt. Dieses Problem identifizierten auch \textcite[S. 2]{mitre:2014}. Um dennoch ein System auf Basis des kollaborativen Filterns entwickeln zu können, entwarfen die Forscher einen fiktiven "Pseudo-Mitarbeiter". Diesem wiesen sie alle für das Projekt relevanten Fähigkeiten zu. Über Ähnlichkeitsmaße wurden die Angestellten ausgewählt, welche die höchste Übereinstimmung mit dem Pseudo-Mitarbeiter vorweisen konnten. Zu diesem Ansatz muss kritisch angemerkt werden, dass es sich entgegen der Angabe der Wissenschaftler nicht um kollaboratives, sondern um inhaltsbasiertes Filtern handelt.\\
Beim inhaltsbasierten Filtern werden Mitarbeiter für Projekte empfohlen, welche in ihren Fähigkeiten eine möglichst hohe Übereinstimmung mit den im Projekt benötigten Kompetenzen aufweisen \cite[S. 139]{recommenderSystems:2016}. So verwendeten \textcite[S. 2]{mitre:2014} zwar einen Pseudo-Mitarbeiter, dieser stellte jedoch lediglich eine Repräsentation der im Projekt geforderten Fähigkeiten da.\\
Einen ähnlichen Ansatz verfolgten auch \textcite[S. 6ff.]{buildingVectorRepresentations:2020}. Diese stellten die Fähigkeiten der Mitarbeiter und die im Projekt benötigten Kompetenzen in Form von Vektoren dar. Auch hier wurden über Ähnlichkeitsalgorithmen diejenigen Kandidaten ausgewählt, welche die höchste Übereinstimmung mit den gesuchten Fähigkeiten aufweisen konnten.\\
Solche, auf Ähnlichkeitsberechnungen basierende Implementierung, werden als "speicherbasierte Ansätze" bezeichnet. Darüber hinaus existieren auch "modellbasierte Ansätze", welche Verfahren aus dem Bereich des maschinellen Lernens und des Data Minings zur Generierung von Empfehlungen verwenden \cite[S. 9]{recommenderSystems:2016}. So entwickelten beispielsweise \textcite[S. 5ff.]{personJobFit:2018} ein Empfehlungssystem auf Basis eines neuronalen Netzes, welches die Eignung einer Person für eine Stelle aus vergangenen Bewerbungsdaten vorhersagt.\\
Ob eine speicher- oder modellbasierte Implementierung geeigneter ist, muss im Einzelfall entschieden werden. \textcite[S. 4]{peerToPeer:2008} stellen diesbezüglich fest, dass speicherbasierte Ansätze in der Praxis häufig wesentlich einfacher umzusetzen sind, da weniger Parameter aufeinander abgestimmt werden müssen. Modellbasierte Ansätze benötigen dafür eine kürzere Berechnungszeit, welche laut \textcite[S. 2]{weightedSimilarity:2015} insbesondere bei steigender Datenmenge vorteilhaft ist. Außerdem bieten modellbasierte Verfahren einen einfacheren Umgang mit dem Cold Start- und dem Sparse Data-Problem \cite[S. 4]{peerToPeer:2008}.

\section{Cold Start- und Sparse Data-Problem}
\label{ch:empfehlungssysteme:coldStartUndSparseData}
Algorithmen im Bereich des kollaborativen bzw. inhaltsbasierten Filterns setzen voraus, dass ausreichend Fähigkeits-Daten der Mitarbeiter vorhanden sind. Verfügt ein Angestellter ausschließlich über Fähigkeiten, welche kein anderer Mitarbeiter besitzt oder welche für kein Projekt exklusiv ausgeschrieben sind, wird dieser von den bisher vorgestellten Verfahren im Empfehlungsprozess nicht berücksichtigt. Dieses Phänomen wird als "Cold Start"-Problem bezeichnet \cite[S. 1]{coldStart:2002}. Um ihm entgegenzuwirken, ist in der Literatur die Implementierung hybrider Verfahren verbreitet. Diese kombinieren Ansätze des kollaborativen und inhaltsbasierten Filterns innerhalb eines Systems \cite[S. 8]{malinowski:2008}. So konnten beispielsweise \textcite[S. 8]{combiningCbAndCFCostSensitiveApproach:2017} nachweisen, dass sie durch die Implementierung eines hybriden Verfahrens die Empfehlungen bei Stellensuchen verbessern konnten. \textcite[S. 16]{hybridImmunizing:2017} zeigten, dass deren hybrides Job-Empfehlungssystem trotz der Kombination von kollaborativem und inhaltsbasiertem Filtern eine nach wie vor sehr hohe Performance aufweisen konnte.\\
Neben dem Cold Start sollten Empfehlungssysteme auch eine Lösung für das Sparse Data-Problem bieten. Dieses bezeichnet das Phänomen, dass in der Praxis für einen Großteil der Fähigkeiten nur sehr wenige Bewertungen vorliegen \cite[S. 8]{recommenderSystems:2016}. So stellten \textcite[S. 3]{mitre:2014} bei der Implementierung ihres Projekt-Empfehlungssystems fest, dass über die Hälfte der ca. 17.000 vergebenen Fähigkeiten von nur je einem Mitarbeiter angegeben wurden. Um auch bei einer solch geringen Datendichte Empfehlungen generieren zu können, schlagen einige Wissenschaftler auch hier die Implementierung hybrider Verfahren vor \cite[S. 3]{weightedSimilarity:2015}.\\
\textcite[S. 1]{malinowski:2008} kritisieren jedoch, dass auch ein hybrides Empfehlungssystem nicht ausreicht, um Mitarbeiter für Projekte zu empfehlen. Deren Einschätzung zu Folge ist die Implementierung eines bilateralen Empfehlungssystems notwendig.

\newpage
- \cite[S. xvii]{recommenderSystems:2016}: Thema der RS erhielt eine steigende Wichtigkeit in den 90ern, als das Web ein wichtiges Medium für Wirtschaft und Online-Handel wurde --> Es wurde früh erkannt, dass das Web unerreichte Möglichkeiten zur Personalisierung bereit hielt, welche in anderen Kanälen nicht zur Verfügung standen --> Web bot einfache Datensammlung und Nutzeroberfläche, um empfohlene Items in einer unaufdringlichen Weise anzuzeigen\\
- \cite[S. xvii]{recommenderSystems:2016}: Bekannteste Methoden: kollaboratives filtern, inhaltsbasiertes filtern und wissensbasierte Systeme\\
- \cite[S. 1]{recommenderSystems:2016}: Wichtigkeit des Webs war eine treibende Kraft für die Entwicklung der RS-Technologie --> Wichtiger Katalysator: Einfachheit, mit der Nutzer Feedback über Likes und Dislikes oder Bewertungen vergeben können / Es gibt auch implizites Feedback, z.B. indem der Nutzer sich ein Element ansieht oder einkauft --> Vergangene Interaktion mit dem Item --> Diese sind oft ein guter Indikator für zukünftige Wahl\\
- \cite[S. 1]{recommenderSystems:2016}: Grundidee des RS: Verschiedenen Datenquellen nutzbar machen und daraus Kundeninteressen schlussfolgern\\
- \cite[S. 1f.]{recommenderSystems:2016}: Ausnahme: Wissensbasierte RS --> Hier werden Empfehlungen eher auf Basis von Nutzer-spezifizierten Anforderungen als auf Vergangenheitsdaten generiert\\
- \cite[S. 2]{recommenderSystems:2016}: Kollaboratives Filtern bedeutet die Bewertung vieler Nutzer in einem kollaborativen Weg einzubeziehen, um fehlende Bewertungen schlusszufolgern\\
- \cite[S. 2]{recommenderSystems:2016}: Inhaltsbasiertes Filtern: Content spielt die primäre Rolle im Empfehlungsprozess --> Bewertungen der Nutzer und Attribut-Beschreibungen werden genutzt, um Empfehlungen abzugeben\\
- \cite[S. 2]{recommenderSystems:2016}: Knowledge-Based: Nutzer spezifizieren Interaktiv ihre Interessen --> Nutzerspezifikation wird mit Domänenwissen kombiniert, um Empfehlungen zu generieren\\
- \cite[S. 3]{recommenderSystems:2016}: Zwei primäre Empfehlungsprobleme: 1. Vorhersageproblem --> Hier gibt es spezifizierte bzw. observed items und fehlende bzw. unobserved Werte --> Auch genannt: Matrix completion problem --> Ziel: fehlende Werte durch lernenden Algorithmus vorherzusagen / 2. Ranking-Version des Problems: Bestimmung der top k-Items --> Auch genannt: top-k recommendation problem --> Hier sind absolute Werte nicht wichtig; 4. Erhöhung der Empfehlungs-Diversität (Wenn alle top-k Items sehr ähnlich sind, erhöht das die Gefahr, dass sich der Nutzer für keines der Items interessiert --> Sind die Empfehlungen diverser, ist die Wahrscheinlichkeit höher, dass der Nutzer zumindest ein Item mag)\\
- \cite[S. 4]{recommenderSystems:2016}: Ziele: 1. Relevanz (Relevant für Nutzer); 2. Neuheit (Nutzer hat das in der Vergangenheit noch nicht gesehen --> Quelle); 3. Serendipität (Nutzer bekommt etwas vorgeschlagen, das ihn positiv überrascht) / Beispiel: Indisches Restaurant eröffnet in der Nähe von jemandem der oft indisch isst --> Das Restaurant zu empfehlen wäre Neu, aber nicht Serendipität --> Serendipität wäre, wenn der Nutzer ein ethiopisches Restaurant empfohlen bekommt / Serendipität kann langfristig zu einem neuen Interesse des Nutzers und somit zu strategischen Vorteilen für den Verkäufer führen; Führt aber auch oft zur Empfehlung irrelevanter Items\\
- \cite[S. 5]{recommenderSystems:2016}: GroupLens war ein Pionier-RS --> Forschungs-Prototyp für die Empfehlung von Nachrichtenartikeln\\
- \cite[S. 5]{recommenderSystems:2016}: Amazon (Quelle) war auch ein Pionier, insbesondere für den Handel --> Haben sehr früh die Nützlichkeit dieser Technologie erkannt; Bewertungen werden durch ein 5-Sterne-System gespeichert; Auch werden Surf-Daten geloggt / Unterscheidung explizite und implizite Bewertung \\
- \cite[S. 5f.]{recommenderSystems:2016}: Streaming-Anbieter; Bietet explizit Beispiele für Empfehlungen basierend auf angesehenen Items --> Hilft dem Nutzer zu entscheiden, ob dieser den Film sehen möchte --> Hilft dem Nutzer zu verstehen, weshalb er einen bestimmten Film sehen möchte / Netlifx-Preis / Cinematch\\
- \cite[S. 6]{recommenderSystems:2016}: Google News (Quelle): Empfiehlt Nachrichten basiernd auf der Click-Historie / Klick auf Artikel wird als positives Feedback gewertet \\
- \cite[S. 7]{recommenderSystems:2016}: Facebook: Empfiehlt potentielle Freunde --> Erhöht Vernetzung und Wachstum des sozialen Netzwerks\\
- \cite[S. 8]{recommenderSystems:2016}: Zwei Arten von Daten: 1. Nutzer-Item Interaktionen (genutzt für Kollaboratives Filtern) und Attribut-Informationen (genutzt für Content-Based --> nutzen auch Bewertungen, aber von einem einzigen Nutzer und nicht von allen)\\
- \cite[S. 8]{recommenderSystems:2016}: Nutzer spezifiziert explizit Anforderungen; Historische Daten werden nicht genutzt, dafür externes Wissen und constraints \\
- \cite[S. 8]{recommenderSystems:2016}: Hybrides System kombiniert die Stärken mehrerer der vorgestellten Systeme --> Sind robuster\\
- \cite[S. 8]{recommenderSystems:2016}: Kollaboratives Filtern nutzt Daten mehrerer Nutzer, um Empfehlungen zu generieren --> Hauptherausforderung: Daten sind sparse --> Meiste Items sind nicht bewertet; Spezifiziert Bewertungen werden als "observed" bezeichnet, unspezifizierte Bewertungen werden als "unobserved" oder missing bezeichnet --> Jetzt schaut man was ähnlichen Nutzern gefällt / \cite[S. 9]{recommenderSystems:2016}: Speicherbasierte (Memory-Based): Auch genannt Neighborhood-based collaborative filtering systems --> Bewertungen von dem Zielnutzer ähnlichen anderen Nutzern werden zur Vorhersage genutzt (user-Based) oder item-based: Hier werden für ein Zielitem andere ähnliche Items berechnet\\
- \cite[S. 9]{recommenderSystems:2016}: Vorteil memory-based: einfach zu implementieren und resultierende Ergebnisse sind oft einfach zu erklären --> Aber: Funktioniert nicht gut mit sparse Data --> Bei wenig Daten ist die Vorhersage wenig robust --> Das ist aber oft kein Problem, wenn nur die top-k items benötigt werden\\
- \cite[S. 9]{recommenderSystems:2016}: Modellbasiert: Nutzt machine-learning und data mining, um Vorhersagen zu treffen; z.B. Entscheidungsbäume, regelbasierte Modelle; Bayessche Methoden oder latent Faktor-Modelle --> Viele dieser Methoden wie latent Faktor Modelle haben eine hohe Abdeckung sogar bei sparse Data\\
- \cite[S. 10]{recommenderSystems:2016}: Es wurde gezeigt, dass Kombinationen aus modell- und speicherbasierten Ansätzen sehr exakte Ergebnisse lieferten (Quelle)\\
- \cite[S. 15]{recommenderSystems:2016}: Nachteile Content-Based: 1. Bieten manchmal offensichtliche Bewertungen, wegen der Nutzung von Schlüsselwörtern oder Inhalt --> Wenn Nutzer niemals ein Produkt mit speziellen Keywords genutzt hat, wird dieses niemals empfohlen --> Reduziert Diversität / 2. Gut geeignet, um Vorhersagen für neue Items zu liefern, aber nicht für neue User --> Grund: Historische Bewertungen von Nutzern benötigt \\
- \cite[S. 15]{recommenderSystems:2016}: (Quelle) Es wird oft diskutiert, ob wissensbasierte Systeme sich von content-basierten unterscheiden --> Genutzt zB wenn Nutzer Keywords angeben können\\
- \cite[S. 15]{recommenderSystems:2016}: Knowledge based Systeme sind geeignet, wenn Items nicht sehr oft verkauft werden --> z.B. Wohnungen, Automobile, Luxusgüter --> Hier sind oft nicht ausreichende Bewertungen vorhanden / Problem ist auch bekannt als Cold-Start-Porblem: Es sind nicht ausreichende Bewertungen für einen Empfehlungsprozess vorhanden \\
- \cite[S. 16]{recommenderSystems:2016}: Übersichtstabelle zu Content-Based, Kollaborativ und Wissensbasiert \\
- \cite[S. 16f.]{recommenderSystems:2016}: Knowledge-Based Systeme sind einzigartig, da sie dem Nutzer explizit erlauben zu spezifizieren, was sie wollen / Es gibt unterschiedliche Arten: 1. Constraint-Basierte RS: Nutzer spezifizieren Anforderungen oder Beschränkungen (z.B. obere oder untere Limits) für Item-Attribute --> Regeln werden genutzt, um Anforderungen auf Item-Attribute zu matchen (Mit Quellen) / 2. Case-based: Nutzer spezifiziert Fälle als Ankerpunkte und Ähnlichkeitsmaße in den Attributen werden für Vorschläge genutzt --> Ergebnisse können auch als neue Ziele genutzt werden --> Interaktiver Prozess, um den Nutzer zu leiten (Mit Quellen) / Bei beiden: Nutzer hat die Möglichkeit seine Anforderungen anzupassen / Interaktivität: 1. Conversational Systems; 2. Search-Based Systems; 3. Navigation-based Recommendation\\
- \cite[S. 18]{recommenderSystems:2016}: Es ist bemerkenswert, dass sowohl knowledge als auch content based signifikant von den Attributen er Items abhängen --> Deshalb haben Knowledge-Based Systeme oft dieselben Nachteile wie Content-based --> z.B. Vorschläge können offensichtlich sein; (Quelle) Knowledgebased Systeme werden oft als "Cousins" von Inhaltsbasiert betrachtet --> Unterschied: Inhaltsbasiert lernt aus vergangenem Verhalten, KB nutzt nur aktuelle Nutzerspezifikationen \\
- \cite[S. 24]{recommenderSystems:2016}: Cold-Start Problem: Wenn initial nur sehr wenige Bewertungen verfügbar sind, ist es schwer traditionelle Collaborative Filter Modelle anzuwenden --> Hier sind content-based und knowledge-based robuster --> Jedoch könnte Knowledge oder Content nicht immer verfügbar sein\\
- \cite[S. 29]{recommenderSystems:2016}: Neighboorhood collaborative Filtering werden auch als speicherbasierte Algorithmen bezeichnet; 2 Typen: User-based: Bewertungen von ähnlichen Nutzern zu Nutzre A werden für Empfehlung genutzt --> Es wird der Mittelwert der "Peer Group"-Bewertungen für jedes Item bestimmt; 2. Item-basiert: Bewertung von Nutzer A für Artikel B soll vorhergesagt werden --> Artikel, die Artikel B sehr ähnlich sind werden bestimmt und in ein Set S geladen --> Durchschnittliche Bewertung aller Bewertungen von Nutzer A in S ist die Bewertung für B\\
- \cite[S. 32]{recommenderSystems:2016}: Long-Tail: Ein sehr kliener Anteil an Items hat sehr viele Bewertungen --> Das sind die beliebten Items; Mehrheit der Items wird sehr wenig bewertet / Die populären Items haben wichtige Implikationen für den Empfehlungsprozess: 1. Sind oft sehr konkurrenzfähige Produkte mit geringem Profit --> Weniger bewertete Produkte bringen höheren Prfit --> (Quelle) Amazon macht den meisten Profit durch den Verkauf von Produkten aus dem Long-Tail; 2. Wegen der geringeren Bewertungen ist es schwerer zuverlässige Vorhersagen im Long-Tail zu tätigen --> (Quelle) viele RS haben die Tendenz beliebte Items eher zu empfehlen --> Negativer Einfluss auf Diversität --> Nutzer werden gelangweilt durch immer die selben Empfehlungen; 3. Neigherhood based collaborative Filtering nutzen oft die hochfrequentierten Items --> sind oft nicht repräsentativ\\
- \cite[S. 33]{recommenderSystems:2016}: Beispiele zum kollaborativen Filtern: Kann über Nachbarschaften zwischen Nutzern (User-based Models) (Wenn Alice und Bob in der Vergangenheit ähnliche Filme bewertet haben, kann Alice Bewertung für Terminator genutzt werden, um Bobs nicht vergebene Bewertung für Terminator vorherzusagen) oder Nachbarschaft zwischen Items (Item-Based Models) --> Ähnliche Items werden vom Nutzer ähnlich bewertet --> Um Bobs Bewertung für Terminator vorherzusagen, können Bobs Bewertungen für Alien and Predator verwendet werden / Neighborhood Methoden können als Generalisierung von Nearest Neighbor Klassifikatoren aus der ML-Literatur betrachtet werden --> Unterschied: Beim Kollaborativen Filtern können Ähnlichkeiten sowohl in Spalten als auch Zeilen gefunden werden --> Bei ML nur in Rows\\
- \cite[S. 34f.]{recommenderSystems:2016}: User-Based Neighborhood Models: Berechne Ähnlichkeit zwischen allen Nutzern i und einem Zielnutzer u --> Dazu muss eine Ähnlichkeitsfunktion verwendet werden --> Nimm dazu für jedes Nutzer-Paar alle Items, die beide bewertet haben und berechne darüber die Ähnlichkeit --> z.B. mit Pearson Korrelations-Koeffizient --> Danach kann man ein Set mit k Nutzern auswählen, welche die höchste Ähnlichkeit aufweisen --> Durchschnitt derer Bewertungen kann zur Vorhersage genutzt werden / Achtung: Nutzer könnten unterschiedlich gut bewerten (grundsätzlich besser oder grundsätzlich schnlechter), deshalb: Bewertungen mean-centern, bevor die durchschnittliche Bewertung der Peer-Group bestimmt wird --> Mean-Centern: Durchschnittliche Bewertung eines Nutzers von der Rohbewertung des Items abziehen --> S. 39: Negative Bewertungen sollten gefiltert werden\\
- \cite[S. 40f.]{recommenderSystems:2016}: Item-Based Neighborhood Models: Ähnlichkeiten werden zwischen Items (Spalten) berechnet --> Erstmal mean centern --> Dann werden alle Nutzer geladen, die das Ziel-Item bewertet haben --> Ähnlichkeit berechnen z.B. Mit AdjustedCosinus (bietet hier bessere Ergebnisse als Pearson) --> Nimm top k-Items --> Gewichteten Durchschnitt berechnen (unter Einbeziehung der Bewertungen des Nutzers für Hebelwirkung)\\
- \cite[S. 42]{recommenderSystems:2016}: Item-based bieten oft relevantere Empfehlungen, da die eigenen Bewertungen des Nutzers mit einbezogen werden, dafür kann user-based für mehr Diverstität führen --> Wenn bei Item-Based der Nutzer das erste Item nicht mag, mag er wahrscheinlich gar keins; S. 43: Item-based bietet auch konkreten Grund für Empfehlung "Weil du Star Wars gesehen hast, gefällt dir vielleicht auch .."\\
- \cite[S. 44]{recommenderSystems:2016}: Vorteile Neighborhood: Einfacher und intuitiver Ansatz; Nachteil: Berechnung kann bei großen Datenmengen unpraktisch werden --> Kann langsam werden (vgl. Offline- Online Phase)\\
- \cite[S. 44]{recommenderSystems:2016}: Kombination von Item- und User-Based: Wichtig: Beide müssen denselben Algorithmus nutzen --> Zuvor mean-centern --> Für Zieleintrag ähnlichste Zeilen und Spalten bestimmen --> Für Zieleintrag die gewichtete Kombination aus den top-k ähnlichsten bestimmen\\
- \cite[S. 46]{recommenderSystems:2016}: Alternativ: Offline-Clustern\\
- \cite{recommenderSystems:2016}: Allgemeine Anmerkung: Laufzeiten O(..) wurden im Text genannt / Grundsätzlich: Hauptprobleme bei Memory-Based: Sparsity und hohe Berechnungszeit\\
- \cite[S. 60]{recommenderSystems:2016}: Sparsity ist ein großes Problem bei Nachbarschafts-Berechnungen  --> Graphen können helfen --> Können für Nutzer, Items oder beide Konzipiert werden --> Bestimmt auch Nachbarschaft --> Ist aber effektiver bei sparse Settings
- \cite[S. 61]{recommenderSystems:2016}: User-Item-Graph ist ungerichtet und bipartite --> Enthält Items, User und über Kanten die Beziehung zwischen beiden (wenn Nutzer ein Item bewertet hat) --> ungerichtete Beziehung --> Vorteil: Es muss nicht viele Daten geben, damit ein kürzester Weg zw. zwei Nutzern existiert --> Man kann die indirekte Beziehung bestimmen --> Graphenalgorithmen aucj bei Link Prediction in Social Media genutzt (bzw. Vanilla Recommendation Problem)\\
- \cite[S. 61]{recommenderSystems:2016}: Random-Walk: Häufig genutzt für Web-Ranking wie z.B. PageRank oder SimRank, um die k-ähnlichsten Items ausgehend von einem Startitem zu finden / Bei Pearsons müssen Nutzer direkt miteinander verbunden sein --> Hier reicht indirekt --> Deshalb besserer Umgang mit Sparse Data / Katz Measure: Gewichtete Nummer von Gängen zwishen Paaren von Knoten, um die Affinität zwischen beiden zu bestimmen --> Gewicht ist ein Faktor zwischen 0 und 1, welcher typischerweise mit zunehmender Länge abnimmt --> Katz Measure ist die gewichtete Anzahl von Walks zwischen paaren von Knoten --> Auch oft genutzt zur Link Prediction --> Intuition: Wenn zwei Nutzer die selbe Nachbarschaft haben, gibt es die Neigung einen Link zwischen beiden im User-Item Graph zu erstellen --> Die Neigung wird mit der Anzahl von Wegen zwischen beiden bestimmt / Sobald über Katz die Nachbarschaft bestimmt ist, kann man Vorhersagen berechnen wie zuvor bei der Nachbarschaft\\
- \cite[S. 61]{recommenderSystems:2016}: User-User Graph besser geeignet\\
----\\
- \cite[S. 71]{recommenderSystems:2016}: Modell-Basierte Ansätze: Anwendung von ML-Methoden --> Trennung in Trainins und Vorhersagephase --> z.B. Entscheidungsbäume, Regelbasiert, Bayes, Regressions, SVM, Neuronale Netze\\
- \cite[S. 72]{recommenderSystems:2016}: Bei kollaborativem Filtern keine Unterscheidung zw. Trainings- und Testdaten --> Bei Modell-basiert findet diese Trennung statt\\
- \cite[S. 72]{recommenderSystems:2016}: Vorteile Model-based gegenüber Neighborhood: 1. Modelle sind kleiner als Original-Matrix --> Weniger Platz benötigt; 2. Schneller; Achtung: Overfitting, Dimensionen;
- \cite[S. 74]{recommenderSystems:2016}: Beispiel: Entscheidungsbaum\\
- \cite[S. 76]{recommenderSystems:2016}: Problem wenn Entscheidungsbaum zum Kollaborativen Filtern genutzt wird: Keine klare Trennung zwischen Feature- und Klassenvariablen; Sparse Data --> Lösung: Mehrere Entscheidungsbäume für jedes Item / Alternativ: Dimensionsreduzierung --> Es werden davon Nachteile diskutiert\\
- \cite[S. 84]{recommenderSystems:2016}: Problem Overfitting: Tritt auf, wenn die Bewertungsmatrix sparse ist und die Anzahl an bewerteten Items klein --> Datengetriebene Ansätze können nicht robust sein --> Diskutiert warum\\
- \cite[S. 134]{recommenderSystems:2016}: Latent-Faktor sind state of the Art in Collaborativem Filtern\\
- \cite{recommenderSystems:2016}: Content-Based nutzt zusätzliche Infos aus Nutzerprofilen bzw. Beschreibungen von Items --> Funktioniert aber ansonsten analog\\
- \cite[S. 161]{recommenderSystems:2016}: Vorteile Contentbased gegenüber Kollaborativem Filtern: 1. Wenn ein neues Item hinzugefügt wird und noch keine Bewertung hat, wird kollaboratives Filtern nicht funktionieren --> Kollaborative Systeme haben cold-start Probleme bei neuen Nutzern und neuen Items; 2. Bietet Erklärungen; 3. ...\\
- \cite[S. 161]{recommenderSystems:2016}: Nachteile Content-based: 1. Tendiert dazu, items zu finden, die ähnlich zu denen sind, die der Nutzer schon kennt; 2. Löst Cold-Start-Problem nicht für neue Nutzer\\
- \cite[S. 199]{recommenderSystems:2016}: Knowledge Based kann besser mit Cold start umgehen, da keine Vergangenheitsdaten benötigt werden --> Dafür schlechter für Personalisierung geeignet --> Mit derselben Suchanfrage bekommen unterschiedliche Nutzer dieselben Ergebnisse\\
- \cite[S. 199f.]{recommenderSystems:2016}: 3 Möglichkeiten Hybride Systeme umzusetzen: 1. Ensemble Dsign: Ergebnisse mehrerer Algorithmen werden in einen Output verpackt; 2. Monolithisches Design: Integrierter Empfehlungsalgorithmus wird durch die Nutzung verschiedener Datentypen erstellt --> Klare Trennung zwischen den Teilen (z.B. kollaborativ und content) oft nicht vorhanden; 3. Mixed System: Es werden mehrere Algorithmen verwendet und alle Ergebnisse Seite an Seite ausgegeben --> Meist wird unter dem Ensemble das Hybride System verstanden; Auch möglich mehrfach dasselbe System zu kombinieren\\
- \cite[S. 328]{recommenderSystems:2016}: Alternative zu Katz: Jaccard oder PageRank\\
- \cite[S. 443]{recommenderSystems:2016}: Reripocale Systeme zeichnen sich dadurch aus, dass sowohl Nutzer als auch Items Präferenzen haben --> Asymmetrisches Problem



\newpage
Aufbau:
- Einleitung
- Wissensbasiert
- Kollaborativ und inhaltsbasiert (speicherbasiert)
- Modellbasiert
- Cold-Start und Sparse Data
- Hybrid
- Reciprocal
- (Evaluation)

%\section{Bilaterale Empfehlungssysteme}
%\label{ch:empfehlungssysteme:bilateraleVerfahren}
%Die Idee des bilateralen Empfehlungssystems basiert auf der Arbeit von Jeffrey R. Edwards, einem Forscher im Bereich des organisationalen Verhaltens \cite[S. 3]{malinowski:2006}. \textcite[S. 2ff.]{edwards:1991} untersuchte, unter welchen Bedingungen eine Person grundsätzlich für eine Stelle geeignet ist. Dabei stellte er fest, dass neben der Befähigung des Mitarbeiters für eine bestimmte Position auch dessen Wünsche bzw. Bedürfnisse zur vorgesehenen Stelle passen müssen. Diese zweite Ebene muss laut \textcite[S. 1]{malinowski:2006} ebenfalls von Empfehlungssystemen berücksichtigt werden. Betrachtet ein solches System neben den Anforderungen des Personalsachbearbeiters an die Fähigkeiten des Mitarbeiters auch die Wünsche des Angestellten, sprechen die Wissenschaftler von einem bilateralen Empfehlungssystem.\\
%Es ist festzustellen, dass der Wunsch des Mitarbeiters in der Literatur nicht einheitlich interpretiert wird. So entwickelten \textcite[S. 1ff.]{applyingDataMining:2014} ein bilaterales Empfehlungssystem auf Basis von Data Mining-Technologien. Die Wissenschaftler verstehen dabei unter dem Wunsch des Nutzers dessen Präferenz für ein bestimmtes Gehalt oder die Bekanntheit eines potenziellen Arbeitgebers. \textcite[S. 4ff.]{malinowski:2006} interpretieren den Wunsch des Nutzers dagegen als dessen Präferenz für bestimmte Stellenprofile.\\
%Allgemein ist festzustellen, dass zu bilateralen Empfehlungssystemen bislang nur sehr wenig Literatur existiert \cite[S. 2f.]{jobRecommenderSystemsASurvey:2012}.\\

% bemerken, dass zu diesem Forschungsgebiet bislang sehr wenig Literatur existiert.\\
%- \cite{applyingDataMining:2014}: Nutzerprofil --> Data Mining --> Klassifikation in Gruppe --> CB unter Einbezug der persönlichen Präferenzen (Präferenzen kommen aus Eingabeformular und werden aus Bewerbungshistorie gemint) / Präferenzen: z.B. lieber besser angesehenes Unternehmen oder mehr Geld / Mehr Fokus auf aktuelle Präferenzen, als Historie / Erstellen Entscheidungsbaum / Beziehen Domänenwissen mit ein, um Skills besser matchen zu können / Fazit: Präferenzen mit einzubeziehen erhöht wie akkurat das Ergebnis ist / Wenn Kandidat einen Karriereweg verfolgt, fokussiert das System auf die letzten Jobs\\
%- \cite{malinowski:2008}: "Existing systems only consider whether a person has the requiredtechnical skills and abilities for a job." / Quelle 2: Existing approaches usually consider only unary attri-butes that are tied directly to an individual (e.g.educational data) and–based on them–assess theaptitude of a candidate in relation to the job requirements\\
%- \cite{jobRecommenderSystemsASurvey:2012}: Quelle 15: Bilaterales Sytstem / Quelle: Piazzato baut Reciporal Reommender für Jobs / Zu Reciprocal wenig Literatur
%- \cite{hybridImmunizing:2017}: Entwickeln hybrides System unter Beachtung des bilateralen Systems \\
%- \cite{malinowski:2006}: Theorie zeigt, dass für ein gutes Match eine bilaterale Beziehung bestehen muss\\
%- \cite{malinowski:2008} orientiert sich an Edwards \cite{edwards:1991}, der sagt, dass man noch die Wunsch-Ebene mit einbeziehen muss / Traditioneller Ansatz in Literatur: Nur Skill-Abgleich --> eigentlich muss man beide Perspektiven mit einbeziehen / Setzt Fokus auf Person-Team Fit / Aktuelle Systeme fokussieren nur auf die Abilities-Ebene --> Auch Match zwischen Person und Team wird nicht berücksichtigt / Quelle 44 und 47 entwickelten RS, die Needs-Supplies beachten \\
%- \cite{comibingCareer:2013}: Vorteil Keyword-Abgleich: Nutzer hat viel Kontrolle --> Kann aber nicht bewerten, ob er auch zu den geeignetsten Kandidaten für den Job gehört / System soll entwickelt werden, bei welchem Nutzern nur Jobs empfohlen werden, welche für ihre Karriere förderlich sind --> Es werden Jobs aus der Historie anderer Nutzer ermittelt / Malinowski: Bilaterale Beziehung --> Benötigt: Reciporal RS --> Vergleich Online-Dating: Dort oft CF, hier aber ungeeignet wegen der Data Sparsity / Nutzen Daten von 2.410 LinkedIn-Nutzern, entfernten aktuellsten Job und versuchten diesen vorauszusagen / Es wird eine Flag eingeführt, welche einen Vertrauenswert in das Ergebnis mit angibt (kann stark oder schwach sein) / Apache OpenNLP überführt Text in Keyworte, wobei ähnliche Begriffe erkannt werden / Baseline: Kosinus-Ähnlichkeit / Entwickeln ein kaskadierenes System, wobei jedes System unabhängig voneinander Empfehlungen macht und diese dann unter Beachtung der Flag von einem Multiplexer kombiniert werden --> Kosinus wird nur verwendet, wenn zu wenig Ergebnisse vorliegen / Einzelne Systeme beziehen sich auf unterschiedliche Abschnitte in CV und Ausschreibung \\
%- \cite{malinowski:2006}: Person-Job Fit ist ein Teilgebiet des P-E Fit --> Konzentriert sich aber nur auf den P-J Fit / Edwards \cite{edwards:1991} sagt, dass man auch den Wunsch mit einbeziehen muss / Bilateral --> Deshab werden zwei Systeme parallel entwickelt (CV-RS und Job-RS) --> Bezieht aber Präferenzen der Kandidaten aus Vergangenheitsdaten mit ein --> Getestet mit Studis \\
%- \cite{edwards:1991}: P-J Fit betrachtet zwei Ebenen (rein psychologische Betrachtung) \\
%- \cite{exploringJobRecommentations:2019}: Versucht Kompetenz-Lücken zu finden (Ähnlich Quelle 6 und 9) --> Fokus auf Design nicht auf technischer Umsetzung / Auch Quelle 7 gut \\
%- \cite{dynamicUserProfile:2013}: Quelle 9 (Gauch et al) entwickelten ein dynamisches Nutzerprofil --> Dynamische Veränderung von Interssen und Präferenzen des Nutzers wurden bedacht \\
%- \cite{aCombinedRepresentation:2018}: Gutes Empfehlungssystem schlägt nicht Job vor, der zu Skills passt, sondern schlägt auch Skills vor, die Person erwerben kann, um neue Position zu erhalten / Quelle 8: Skill-GAP, da zu schnell neue Technologien kommen --> Skill-Gap muss eliminiert werden, daher muss erst der Gap bestimmt werden / Ähnlichkeiten zwischen Skills werden bestimmt, um vorzuschlagen, welcher Skill bei Kandidat noch hinzugefügt werden könnte / Skills werden aus Ausschreibungen und Bewerbungen extrahiert --> 2 Graphen: Skill und Skill-Occurence Graphen --> In einem wird nach Ähnlichkeiten gesucht, um Zukunft vorherzusagen / Ziel: Vorhersage des zukünftigen Jobs basierend auf aktuellem Job --> Dafür 20 Mio. Bewerbungen verwendet / Über O*Net wurden Jobs in verschiedene Gruppen kategorisiert / System empfiehlt Jobs und Skills / Nutzen hybriden Ansatz aus CB und CF
\shorthandon{"}