\shorthandoff{"}
\chapter{Ausblick}
\label{ch:ausblick}
% 1 Seite
Wie diese Veränderung in der Logistik aussehen könnte, zeigen schon heute Pilotprojekte einiger Unternehmen, darunter die Volkswagen AG. VW entwickelt zur Zeit ein Konzept zur Verkehrsoptimierung mittels Quantencomputer für seine Tochtergesellschaft MAN, die unter anderem Busse produziert. Das Ziel ist es dabei, dass die Busse nicht mehr vorgegebenen Routen durch die Stadt folgen, sondern nur noch die Haltestellen kennen und ein Quantencomputer in Echtzeit die aktuell optimale Route zwischen den Haltestellen berechnet. Praktisch eingesetzt wurde dieses System bereits 2019 im Rahmen einer Technik-Messe in Lissabon. Dort wurde das System für 4 Buslinien mit insgesamt 26 Haltestellen eingesetzt. Jetzt möchte Volkswagen das System zur Marktreife entwickeln, sodass es auch in anderen Städten eingesetzt werden kann.
Die Quanten-Revolution könnte also schneller beginnen, als wir uns das heute vorstellen können.
Damit bedanke ich mich für eure Aufmerksamkeit


im von einem Meilenstein in der Geschichte der Quantencomputer und gehen davon aus, dass erste univer

- (Problem hat kaum einen praktsichen Nutzen)
- Dennoch für viele Forscher ein Meilenstein
- Dauert noch bis zur Marktreife
- Wenn es soweit ist, liesen sich mit Quantencomputern aber komplexe Probleme effizient lösen, die heute nicht berechenbar sind. z.B. Optimierungsalgorithmen in der Logistik 
- Polymale Laufzeit statt exponentiel


\shorthandon{"}