\shorthandoff{"}
\chapter{Fazit}
\label{ch:fazit}
Die Besetzung offener Projektpositionen ist eine immer häufiger auftretende und bedeutsamer werdende Tätigkeit in der Wirtschaft. Unterstützend eingesetzte Empfehlungssysteme richten sich bislang zumeist einseitig entweder an Personalverantwortliche oder an Stellensuchende. Bilaterale Empfehlungssysteme basieren dagegen auf dem psychologischen Konzept des \aclp{PEFit} und betrachten gleichermaßen die Präferenzen beider Parteien. In der vorliegenden Master-Thesis wurde überprüft, ob der Einsatz eines bilateralen Empfehlungssystems bei der Besetzung offener Projektpositionen gleichzeitig zu einer höheren Zufriedenheit der Angestellten und zu einer gesteigerten Arbeitsleistung im Unternehmen führt. 

Zur Validierung der Fragestellung wurde im Rahmen dieser Arbeit ein graphenbasiertes Empfehlungssystem entwickelt. Dieses sortierte die Mitarbeiter eines Unternehmens für fünf vordefinierte Projektpostionen über einen uni- und einen bilateralen Ansatz. Die bilaterale Variante bezog sowohl die von den Angestellten beherrschten Fähigkeiten als auch deren präferierte Kompetenzen in die Empfehlungsbestimmung ein. Der unilaterale Ansatz betrachtete dagegen lediglich die beherrschten Fähigkeiten der Mitarbeiter.

Anschließend folgte die Durchführung einer Fallstudie unter Projektmanagern und Angestellten eines Unternehmens. Hierbei wurde auf Seiten der Mitarbeiter überprüft, ob der bilaterale Empfehlungsansatz die Angestellten im Vergleich zur unilateralen Variante bei den vordefinierten Projektpositionen höher positioniert, wenn diese eine hohe Zufriedenheit mit einer betrachteten Projektposition prognostizieren bzw. niedriger positioniert, wenn diese eine geringe Zufriedenheit erwarten. Die Projektmanager erhielten die ersten fünf vorgeschlagenen Mitarbeiter jedes Empfehlungsverfahrens für die vordefinierten Projektpositionen. Sie bewerteten, von den Mitarbeitern welcher Liste sie eine höhere Arbeitsleistung bei einer Tätigkeit auf der betrachteten Stelle erwarten. Die Ergebnisse der Fallstudie wurden hinsichtlich der folgenden Forschungsfrage analysiert: \forschungsfrage

Die Auswertung der Umfragen zeigt, dass die Forschungsfrage bestätigt werden kann, wenn die Mitarbeiter mehrheitlich eine Tätigkeit auf einer betrachteten Projektposition präferieren. In diesem Fall sorgt der bilaterale Empfehlungsansatz im Vergleich zur unilateralen Variante sowohl für eine höhere Zufriedenheit auf Seiten der Angestellten als auch aus Perspektive der Projektmanager für eine gesteigerte prognostizierte Arbeitsleistung der Mitarbeiter. Lehnen die Angestellten dagegen mehrheitlich die Tätigkeit auf einer betrachteten Projektposition ab, sorgt das bilaterale Vorgehen sowohl auf Seiten der Mitarbeiter für eine geringere Zufriedenheit als auch aus Perspektive der Projektverantwortlichen für eine niedrigere erwartete Arbeitsleistung von den Angestellten.

Als Ursache für diese Einschränkung wird die Erhebung der Mitarbeiter-Präferenzen betrachtet. Das bilaterale Empfehlungssystem nutzte die im Rahmen dieser Master-Thesis erhobenen booleschen Bewertungen, um die präferierten Fähigkeiten der Angestellten stärker zu gewichten. Bei nicht gewünschten Kompetenzen wurde jedoch nicht unterschieden, ob ein Angestellter dieser Fähigkeit neutral gegenübersteht oder ob er diese nicht bei der Projektarbeit anwenden möchte. Dementsprechend sortierte das bilaterale Empfehlungssystem zwar zufriedene Mitarbeiter in den Ergebnissen höher ein, unzufriedene Angestellte wurden jedoch nicht niedriger angeordnet.

Aus diesem Grund wird für folgende Arbeiten empfohlen, den im Rahmen dieser Arbeit implementierten Empfehlungsansatz zu erweitern. Hierbei sollten die Präferenzen nicht über boolesche Werte, sondern über Abstufungen der Form "möchte ich anwenden", "neutral", "möchte ich nicht anwenden" erhoben werden. Bei der Implementierung sollten die Mitarbeiter bei vorhandenem Wunsch weiterhin höher positioniert, bei negativer Antwortoption sollten sie jedoch zusätzlich niedriger einsortiert werden. Unter Betrachtung dieser Veränderungen sollte die Evaluation unter Mitarbeitern und Projektmanagern nochmals durchgeführt und die Forschungsfrage erneut untersucht werden.

Es ist davon auszugehen, dass eine folgende Arbeit unter Beachtung der empfohlenen Optimierungen uneingeschränkt für eine gesteigerte Zufriedenheit bei den Mitarbeitern und eine höhere erwartete Arbeitsleistung der Angestellten seitens der Projektmanager sorgen wird. In diesem Fall sollte die Vorgehensweise zur Implementierung von Empfehlungssystemen zur Auswahl von Personen zukünftig grundsätzlich überdacht werden. Beispielsweise könnten bilaterale Empfehlungssysteme über die Problemstellung dieser Master-Thesis hinaus auch bei der Einstellung neuer Mitarbeiter oder bei der Auswahl von Auszubildenden bzw. Studenten in der Lehre unterstützend zum Einsatz kommen. Es ist zu erwarten, dass die Einführung bilateraler Empfehlungssysteme auch in solchen Bereichen zu höherer Leistung und Zufriedenheit und damit einer geringeren Fluktuation bzw. Abbrecherquote führen wird. Die weitere Erforschung bilateraler Empfehlungssysteme sollte daher in Zukunft eine wichtige und lohnenswerte Investition für Unternehmen zur Optimierung ihrer personellen Ressourcen darstellen.
%Technische Möglichkeiten erleichtern und verbessern die Identifikation und Auswahl geeigneter Mitarbeiter in Projekten, tragen zur optimalen Nutzung personeller Ressourcen und zur Motivation und Förderung der Mitarbeiter bei. Ausbau und Verfeinerung der oben genannten Ansätze können daher lohnende Investitionen für Unternehmen sein.
\shorthandon{"}
