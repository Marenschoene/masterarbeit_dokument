\definecolor{exxetagray}{gray}{0.75}
\definecolor{itemcolor}{RGB}{179,217,255}
\definecolor{usercolor}{RGB}{255,204,179}

\shorthandoff{"}
\chapter{Fazit}
\label{ch:fazit}
Der Einsatz von Empfehlungssystemen zur Entscheidungsunterstützung hat in den vergangenen Jahren zunehmend an Bedeutung gewonnen.
Während Empfehlungssysteme traditionell für die Empfehlung von Gegenständen, Orten oder Dokumenten eingesetzt werden, hat die Empfehlung von Personen durch den Zuwachs an sozialen Netzwerken vermehrt an Relevanz erlangt.
Systeme, in denen Personen die Inhalte von Empfehlungen bilden und der Erfolg einer Empfehlung von der Bedürfniserfüllung beider beteiligten Parteien abhängt, werden als wechselseitige Empfehlungssysteme bezeichnet.

Ein Einsatzgebiet wechselseitiger Empfehlungssysteme stellt die Besetzung offener Projektpositionen durch passende Mitarbeiter in projektgetriebenen Unternehmen dar.
Nach aktuellem Stand der Forschung blieb bislang unklar, wie ein Empfehlungssystem gestaltet werden muss, um

- an sich ja, wir sehen im durchschnitt, dass zufriedenheit in vier von fünf projekten gesteigert werden konnte, auch arbeitsleistung konnte zuteils gesteigert werden
- dennoch ausreisser vorhanden
- diese in teilen auf datengrundlage zurückzuführen
- darüber hinaus wird davon ausgegangen, dass sowohl zufriedenheit als auch arbeitsleistung von unbekannten faktoren beeinflusst wird
- diese gilt es zu identifizieren und in kommenden Forschungen zu berücksichtigen

% Zurückgreifen auf einige Formulierungen aus der Diskussion!

Die Erkenntnisse, die Link im Rahmen seiner Arbeit gewinnen konnte, stützen die These, dass ein Berücksichtigen der Präferenzen von Mitarbeitern die Zufriedenheit und Arbeitsleistung in Projekten steigen kann.

\shorthandon{"}