\shorthandoff{"}
\chapter{Fazit (WIP)}
\label{ch:fazit}
\section{Fazit}
\label{ch:fazit:fazit}
- Ob man die Anforderungen der Stelle erfüllt, ist dem Mitarbeiter eigentlich egal. Ihn interessiert es nur, ob dadurch seine Werte verstärkt/erfüllt werden. Es wäre also interessant, herauszufinden, wieso ein Mitarbeiter z.B. mehr Python anwenden würde. Mehr Gehalt wegen Data Science? Neugier für neue Sprache? Kumpel arbeitet in dieser Abteilung? ... \\
- Eigentlich müsste der Prozess des Motivations-Herausfindens dem kompletten Einstellungsprozess vorgelagert sein \note{Kann sich die Motivationslage nicht ändern?}. Bzw. sogar dem Studium \\
- Grenze: \textcite{cable:1997} stellten fest, dass das Bauchgefühl des Interviewers besser vorhersagen kann, ob eine P zur O passt. Wenn also ein Unternehmen so klein ist, dass der "Staffer" alle Berater persönlich kennt, kann er wahrscheinlich besser zuordnen als die KI \\
- Anmerkung von mir: Während \textcite{parsons:1909} 1909 also noch davon ausging, dass alles möglichst objektiv und wissenschaftlich korrekt gemessen werden muss, gehen Psychologen heute davon aus, dass primär die subjektive Wahrnehmung eine Hauptrolle spielt \\
- Subjektive Wichtigkeiten bewerten wie bei Story-Points in Scrum

\section{Fragen}
\label{ch:fazit:fragen}
- Die historischen Bücher sprechen immer nur von "Männern" --> kann man daraus einfach "Menschen" machen?\\ \note{Zitattechnisch: Nein. Auch inhaltlich ist das eine spannende Frage. Für Motivationslagen spielen soziale, kulturelle und gesellschaftliche Aspekte ja eine Rolle.}
- Gründungsvater ist ein englisches Zitat --> Wie zitieren?\\ \note{Üblich ist das Orginalzitat mit zusätzlicher Übersetzung}
- Wie umgehen mit den englischen Begriffen? z.B. fit, Need, Desire, etc. \note{Ggf. als Eigenbegriffe nutzen}

\section{Anmerkungen}
\label{ch:fazit:anmerkungen}
- "Referent" klingt komisch (Christian)\\
- Zweite Seite und "I. Thesis" ist unnötig (Christian)\\
- JSON und REST im Abkürzungsverzeichnis muss nicht sein (Christian)\\
- Wort: "Fallstudie" (Christian)\\
- Forschungsfrage: Symmetrische sollte klarer formuliert werden --> Kommt etwas hinterher --> Eher an den Anfang (Andreas)\\
- Forschungsfrage ist sehr lang (Christian)\\
- Einleitung: Warum erhöht die dezentrale Kommunikation die Kreativität? (Christian)\\
- Zu Empfehlungssystemen: Könnten wir uns nicht auf einen Teilgraphen beschränken? (Jan)\\
- Verlinkungen z.B. PE-Fit unterstreichen? (Nina)\\
- Formel \ref{frml:verwandteArbeiten:formel1} raus? (Nina)\\
- Bilddiskussion in Kapitel \ref{ch:verwandteArbeiten:aufDemPEFitBasierendeBilateraleSysteme:pjUndPtFit} in Präsens? (Nina)\\
- \ref{ch:verwandteArbeiten:aufDemPEFitBasierendeBilateraleSysteme:bilateraleVertrauensbestimmung}: Was ist an der Formel noch nicht optimal? (Nina)\\
- Zeile (Z) bei wörtlichen Zitaten aufnehmen? (Christian)\\
- Historischer Teil bei \ac{PEFit} ist unnötig, um Thesis zu verstehen - nur nice-to-know (Christian)\\
- Viel "die Wissenschaftler" (Christian)\\
- Bilder im \ac{PEFit}-Kapitel auf deutsch und dann englische Begriffe im Fließtext weglassen / Auch Long Tail-Bild auf deutsch (Christian)\\
- Ist \ref{fig:personEnvironmentFit:auswirkungenErhoehterAngebote:formel4} notwendig? (Christian)\\
- Qualität von \ref{fig:personEnvironmentFit:wichtigkeiten:abb2} könnte besser sein (Christian)\\
- "e-lancer", P-E Misfit, Pearson Korrelation, Sparsity Problem und Recommender Engine kursiv? (Christian)\\
- Aufkommen der Recommender Engines, YouTube, Filterblase --> an der Grenze, ob es relevant ist (Christian)\\
- Ist es häufig so, dass man direkte Treffer hat oder dass man indirekte Treffer hat? --> Sucht man nach verbreiteten Skills oder nach exotischem? --> Einfach nur als Beobachtung (Andreas)\\
- S. 22: Nicht nur die Skalierung ist verändert --> Verhältnisse --> Relative Werte verändern sich / Statt Skalierung: Glättung - dazu sollten noch 1-2 Sätze rein und das erklären und noch ein Argument dazu rein, warum das kein Problem ist, sondern wieso das vielleicht sogar gut ist (Andreas)
\shorthandon{"}