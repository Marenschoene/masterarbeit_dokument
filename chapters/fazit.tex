\shorthandoff{"}
\chapter{Fazit (WIP)}
\label{ch:fazit}
\section{Fazit}
\label{ch:fazit:fazit}
- Ob man die Anforderungen der Stelle erfüllt, ist dem Mitarbeiter eigentlich egal. Ihn interessiert es nur, ob dadurch seine Werte verstärkt/erfüllt werden. Es wäre also interessant, herauszufinden, wieso ein Mitarbeiter z.B. mehr Python anwenden würde. Mehr Gehalt wegen Data Science? Neugier für neue Sprache? Kumpel arbeitet in dieser Abteilung? ... \\
- Eigentlich müsste der Prozess des Motivations-Herausfindens dem kompletten Einstellungsprozess vorgelagert sein \note{Kann sich die Motivationslage nicht ändern?}. Bzw. sogar dem Studium \\
- Grenze: \textcite{cable:1997} stellten fest, dass das Bauchgefühl des Interviewers besser vorhersagen kann, ob eine P zur O passt. Wenn also ein Unternehmen so klein ist, dass der "Staffer" alle Berater persönlich kennt, kann er wahrscheinlich besser zuordnen als die KI \\
- Anmerkung von mir: Während \textcite{parsons:1909} 1909 also noch davon ausging, dass alles möglichst objektiv und wissenschaftlich korrekt gemessen werden muss, gehen Psychologen heute davon aus, dass primär die subjektive Wahrnehmung eine Hauptrolle spielt \\
- Subjektive Wichtigkeiten bewerten wie bei Story-Points in Scrum

\section{Fragen}
\label{ch:fazit:fragen}
- Die historischen Bücher sprechen immer nur von "Männern" --> kann man daraus einfach "Menschen" machen?\\ \note{Zitattechnisch: Nein. Auch inhaltlich ist das eine spannende Frage. Für Motivationslagen spielen soziale, kulturelle und gesellschaftliche Aspekte ja eine Rolle.}
- Gründungsvater ist ein englisches Zitat --> Wie zitieren?\\ \note{Üblich ist das Orginalzitat mit zusätzlicher Übersetzung}
- Wie umgehen mit den englischen Begriffen? z.B. fit, Need, Desire, etc. \note{Ggf. als Eigenbegriffe nutzen}

\section{Anmerkungen Jan}
\label{ch:fazit:anmerkungenJan}
- Zu Empfehlungssystemen: Könnten wir uns nicht auf einen Teilgraphen beschränken?
\shorthandon{"}