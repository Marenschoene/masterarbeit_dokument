\shorthandoff{"}
\chapter{Fazit}
\label{ch:fazit}
Die vorliegende Arbeit zeigte auf, dass die Besetzung offener Projektpositionen eine immer häufiger auftretende Tätigkeit in der Wirtschaft sein wird. Bisher entwickelte Empfehlungssysteme im Bereich der Personalauswahl richten sich zumeist einseitig entweder an Personalverantwortliche oder an Stellensuchende. \textcite[S. 1ff.]{malinowski:2006} zu Folge müsste stattdessen ein bilaterales Empfehlungssystem eingesetzt werden. Diese Art der Anwendung basiert auf dem Konzept des \aclp{PEFit}. Gemäß dieser Theorie führt die gemeinsame Betrachtung der Präferenzen von Mitarbeitern und Personalverantwortlichen gleichzeitig zu einer höheren Zufriedenheit der Angestellten und zu einer gesteigerten Arbeitsleistung im Unternehmen. Diese Master-Thesis verfolgte das Ziel, die folgende Forschungsfrage zu überprüfen: \forschungsfrage.

Um diese Forschungsfrage zu überprüfen, wurde ein graphenbasiertes Empfehlungssystem entwickelt. Dieses sortierte die Mitarbeiter eines Unternehmens für fünf vordefinierte Projektpostionen über einen uni- und einen bilaterale Ansatz. Die bilaterale Variante bezog sowohl die von den Angestellten beherrschten Fähigkeiten, als auch deren präferierte Kompetenzen in die Empfehlungsbestimmung ein. Der unilaterale Ansatz betrachtete dagegen lediglich die beherrschten Fähigkeiten der Mitarbeiter.

Anschließend wurde eine Fallstudie unter Projektmanagern und Angestellten eines Unternehmens durchgeführt. Hierbei wurde auf Seiten der Mitarbeiter überprüft, ob der bilaterale Empfehlungsansatz die Angestellten im Vergleich zur unilateralen Variante bei den vordefinierten Projektpositionen höher positioniert, wenn diese eine hohe Zufriedenheit mit einer betrachteten Projektposition prognostizieren bzw. niedriger positioniert, wenn diese eine geringe Zufriedenheit erwarten. Die Projektmanager erhielten die ersten fünf vorgeschlagenen Mitarbeiter jedes Empfehlungsverfahrens für die vordefinierten Projektpositionen. Sie bewerteten, von den Mitarbeitern welcher Liste sie eine höhere Arbeitsleistung bei einer Tätigkeit auf der betrachteten Stelle erwarten.

Durch die Auswertung der Ergebnisse der Fallstudie wurde die folgende Forschungsfrage analysiert: \forschungsfrage\\
Die Resultate der Umfragen zeigten, dass die Forschungsfrage bestätigt werden kann, wenn die Mitarbeiter mehrheitlich mit einer Tätigkeit auf der betrachteten Projektposition zufrieden sind. In diesem Fall sorgt der bilaterale Empfehlungsansatz im Vergleich zur unilateralen Variante sowohl für eine höhere Zufriedenheit auf Seiten der Angestellten, als auch für eine gesteigerte prognostizierte Arbeitsleistung bei den Projektmanagern. Zeigen sich die Mitarbeiter dagegen mehrheitlich unzufrieden mit einer betrachteten Projektposition, sorgt das bilaterale Vorgehen sowohl auf Seiten der Mitarbeiter für eine geringere Zufriedenheit, als auch aus Perspektive der Projektverantwortlichen für eine niedrigere erwartete Arbeitsleistung.

Als Ursache für diese Einschränkung wird die Erhebung der Mitarbeiter-Präferenzen betrachtet. Das bilaterale Empfehlungssystem nutzte die im Rahmen dieser Master-Thesis erhobenen booleschen Werte, um die präferierten Fähigkeiten der Angestellten stärker zu gewichten. Bei nicht gewünschten Kompetenzen wurde jedoch nicht unterschieden, ob ein Angestellter dieser Fähigkeit neutral gegenübersteht oder ob er diese nicht bei der Projektarbeit anwenden möchte. Dementsprechend sortierte das bilaterale Empfehlungssystem zwar zufriedene Mitarbeiter in den Ergebnissen höher ein, unzufriedene Angestellte wurden jedoch nicht niedriger angeordnet.

Aus diesem Grund wird für folgende Arbeiten empfohlen, den im Rahmen dieser Arbeit implementierten Empfehlungsansatz zu erweitern. Hierbei sollten die Präferenzen nicht über boolesche Werte, sondern über Abstufungen der Form "möchte ich anwenden", "neutral", "möchte ich nicht anwenden" erhoben werden. Bei der Implementierung sollten die Mitarbeiter bei vorhandenem Wunsch weiterhin höher positioniert werden, bei negativer Antwortoption sollten sie jedoch zusätzlich niedriger einsortiert werden. Unter Betrachtung dieser Veränderungen sollte die Evaluation unter Mitarbeitern und Projektmanagern nochmals durchgeführt und die Forschungsfrage erneut untersucht werden.

Es ist davon auszugehen, dass eine folgende Arbeit unter Beachtung der empfohlenen Optimierungen die Hypothese von \textcite[S. 1ff.]{malinowski:2006} final bestätigen wird. Unter dieser Bedingung sollte die Vorgehensweise zur Implementierung von Empfehlungssystemen zur Auswahl von Personen grundsätzlich überdacht werden. Beispielsweise könnten bilaterale Empfehlungssysteme über diese Master-Thesis hinaus auch bei der Einstellung neuer Mitarbeiter oder bei der Auswahl von Auszubildenden bzw. Studenten in der Lehre unterstützend zum Einsatz kommen. Es ist zu erwarten, dass die Einführung bilateraler Empfehlungssysteme auch in derartigen Bereichen zu höherer Leistung und Zufriedenheit und einer geringeren Fluktuation bzw. Abbruchquote führen wird.
\shorthandon{"}