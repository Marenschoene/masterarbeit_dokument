\shorthandoff{"}
\chapter{Fazit}
\label{ch:fazit}
Die Besetzung offener Projektpositionen ist eine immer häufiger ausgeführte und bedeutsamer werdende Tätigkeit in der Wirtschaft. Unterstützend eingesetzte Empfehlungssysteme richten sich bislang zumeist einseitig entweder an Personalverantwortliche oder an Stellensuchende. Bilaterale Empfehlungssysteme basieren auf dem psychologischen Konzept des \aclp{PEFit} und beachten die Präferenzen beider Parteien gleichermaßen. In der vorliegenden Master-Thesis wurde überprüft, ob der Einsatz solcher Anwendungen bei der Besetzung offener Projektpositionen gleichzeitig zu einer gesteigerten Zufriedenheit der Mitarbeiter und zu einer höheren erwarteten Arbeitsleistung der Angestellten seitens der Projektmanager führt. 

Zur Validierung der Fragestellung wurde im Rahmen dieser Arbeit ein graphenbasiertes Empfehlungssystem entwickelt. Dieses sortierte die Mitarbeiter eines Beratungsunternehmens für fünf vordefinierte Projektpositionen über einen uni- und einen bilateralen Ansatz. Die bilaterale Variante bezog die Präferenzen von Projektmanagern und Angestellten in die Empfehlungsbestimmung ein. Der unilaterale Ansatz beachtete ausschließlich die Anforderungen der Projektverantwortlichen.

Anschließend folgte die Durchführung einer Fallstudie unter Projektmanagern und Angestellten. Hierbei wurde auf Seiten der Mitarbeiter überprüft, ob der bilaterale Empfehlungsansatz die Angestellten höher positioniert, wenn diese eine hohe Zufriedenheit mit einer betrachteten Projektposition prognostizieren bzw. niedriger einordnet, wenn diese eine geringe Akzeptanz zeigen. Die Projektmanager erhielten die ersten fünf vorgeschlagenen Mitarbeiter jedes Empfehlungsverfahrens für die vordefinierten Projektpositionen. Sie bewerteten, von den Angestellten welcher Liste sie eine höhere Arbeitsleistung bei einer Tätigkeit auf den betrachteten Stellen erwarten. Die Ergebnisse der Fallstudie wurden hinsichtlich der folgenden Forschungsfrage analysiert: \forschungsfrage

Die Auswertung der Fallstudie zeigt, dass die Forschungsfrage bestätigt werden kann, wenn die Mitarbeiter mehrheitlich eine Tätigkeit auf einer betrachteten Projektposition präferieren. In diesem Fall sorgt der bilaterale Empfehlungsansatz im Vergleich zur unilateralen Variante auf Seiten der Angestellten für eine höhere Zufriedenheit und aus Perspektive der Projektmanager für eine gesteigerte prognostizierte Arbeitsleistung der Mitarbeiter. Lehnen die Angestellten dagegen mehrheitlich die Tätigkeit auf einer Projektposition ab, sorgt das bilaterale Vorgehen auf Seiten der Mitarbeiter für eine geringere Zufriedenheit und aus Perspektive der Projektverantwortlichen für eine niedrigere erwartete Arbeitsleistung der Angestellten.

Als Ursache für diese Einschränkung wird die Erhebung der Mitarbeiter-Präferenzen betrachtet. Das bilaterale Empfehlungssystem nutzte boolesche Werte, um die präferierten Fähigkeiten der Angestellten zu erfassen und entsprechend stärker zu gewichten. Bei nicht gewünschten Kompetenzen wurde dabei nicht unterschieden, ob ein Angestellter dieser Fähigkeit neutral gegenübersteht oder ob er diese nicht bei der Projektarbeit anwenden möchte. Dementsprechend sortierte das bilaterale Empfehlungssystem zwar zufriedene Mitarbeiter in den Ergebnissen höher ein, die Positionierung unzufriedener Angestellter blieb dagegen unverändert.

Aus diesem Grund wird für folgende Arbeiten empfohlen, den im Rahmen dieser Arbeit implementierten Empfehlungsansatz zu erweitern. Hierbei sollten die Präferenzen nicht über boolesche Werte, sondern über Abstufungen der Form "möchte ich anwenden", "neutral", "möchte ich nicht anwenden" erhoben werden. Bei der Implementierung sollten die Mitarbeiter bei vorhandenem Wunsch weiterhin höher positioniert, bei negativer Antwortoption sollten sie jedoch zusätzlich niedriger einsortiert werden. Unter Beachtung dieser Veränderungen sollte die Evaluation unter Mitarbeitern und Projektmanagern nochmals durchgeführt und die Forschungsfrage erneut untersucht werden.

Es ist davon auszugehen, dass eine folgende Arbeit durch Ergänzung der empfohlenen Optimierungen die Forschungsfrage uneingeschränkt bestätigen wird. In diesem Fall sollte die Vorgehensweise zur Implementierung von Empfehlungssystemen zur Auswahl von Personen grundsätzlich überdacht werden. Der Fokus sollte hierbei von der einseitigen Betrachtung der Anforderungen von Individuum und Umgebung verstärkt auf die Bedürfnisse beider Parteien gerichtet werden. Beispielsweise könnten bilaterale Empfehlungssysteme auch bei der Einstellung neuer Mitarbeiter oder bei der Auswahl von Studenten in der Lehre unterstützend zum Einsatz kommen. Es ist zu erwarten, dass ein bilateraler Auswahlprozess auch in diesen Bereichen zu höherer Leistung und Zufriedenheit und damit einer geringeren Fluktuation bzw. Abbrecherquote führen wird. Außerdem können die erhobenen Präferenzen langfristig zur strategischen Weiterentwicklung von Person und Umgebung zur Anwendung kommen. So ist es beispielsweise möglich, Bildungsangebote oder Anreizsysteme dynamisch an sich ändernde Bedürfnisse anzupassen und so Personen nachhaltig an die Organisation zu binden. Diese Perspektiven zeigen, dass die weitere Erforschung bilateraler Empfehlungssysteme auch über diese Master-Thesis hinaus eine wichtige und lohnenswerte Investition für Institutionen zur Optimierung ihrer personellen Ressourcen darstellt.
%Technische Möglichkeiten erleichtern und verbessern die Identifikation und Auswahl geeigneter Mitarbeiter in Projekten, tragen zur optimalen Nutzung personeller Ressourcen und zur Motivation und Förderung der Mitarbeiter bei. Ausbau und Verfeinerung der oben genannten Ansätze können daher lohnende Investitionen für Unternehmen sein.
\shorthandon{"}
