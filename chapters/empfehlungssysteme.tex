\definecolor{exxetagray}{gray}{0.75}
\definecolor{itemcolor}{RGB}{179,217,255}
\definecolor{usercolor}{RGB}{255,204,179}

\shorthandoff{"}
\chapter{Empfehlungssysteme}
\label{ch:empfehlungssysteme}

\section{Einführung}
\label{ch:empfehlungssysteme:einführung}
Nach \textcite[S. 4]{ricci:inbook} können Empfehlungssysteme (engl.: \ac{RS}) allgemein als Techniken und Werkzeuge verstanden werden, die einem Nutzer oder einer Gruppe an Nutzern eines Systems, aus einer Menge möglicher Entitäten, potenziell nützliche Elemente vorschlagen.
% Allgemein können Empfehlungssysteme (engl.: \ac{RS}) als Techniken und Werkzeuge verstanden werden, die einem Nutzer oder einer Gruppe an Nutzern eines Systems, aus einer Menge möglicher Entitäten, potenziell nützliche Elemente vorschlagen \cite[S. 4]{ricci:inbook}\cite[S. 1]{klahold:book}.
% Im einfachsten Sinne können Empfehlungssysteme (engl.: \ac{RS}) als Techniken und Werkzeuge verstanden werden, die Nutzern eines Systems aus einer Menge an Elementen potenziell nützliche Elemente vorschlagen \cite[S. 10]{ricci:inbook}.
% Etwas feingranularer definiert \textcite[S. 1]{klahold:book} Empfehlungssysteme als Systeme, die einem Nutzer oder einer Gruppe an Nutzern in einem gegebenen Kontext aus einer Menge möglicher Entitäten eine Teilmenge "nützlicher" Elemente empfehlen.
Aus der Definition von \textcite[S. 4]{ricci:inbook} lassen sich grundlegende Komponenten, die für das Erstellen von Empfehlungen in Empfehlungssystemen von Bedeutung sind, abgrenzen.
Für ein fundiertes Verständnis sind die einzelnen Komponenten und deren Zusammenspiel in Abbildung \ref{fig:empfehlungssysteme:einführung:abb1} dargestellt.

\begin{figure}[H]
    \centering
	\includegraphics[width=0.8\textwidth]{gfx/komponenten-empfehlungssystem.png}
	\caption[Übersicht Empfehlungssystem]{Übersicht Empfehlungssystem\\
	(Eigene Darstellung in Anlehnung an \cite[S. 2]{klahold:book})}
	\label{fig:empfehlungssysteme:einführung:abb1}
\end{figure}

% Klärungsbedarf: Nutzer bewusst immer als nutzer bezeichenn, damit klar sit was gemeint ist, das ist sehr häufig in der literatur so
% Hier fragen, ob bildbeschreibung sinn macht. ich will allgemeine übersicht abbilden, nutzer und elemente können ja auch lediglich inputs darstellen und nicht komponenten DES empfehlungssystems
Wie aus Abbildung \ref{fig:empfehlungssysteme:einführung:abb1} hervorgeht, stehen sich bei Empfehlungssystemen grundsätzlich zwei Komponenten gegenüber.
Auf der einen Seite stehen Empfänger von Vorschlägen, welche in Empfehlungssystemen als Nutzer bezeichnet werden \cite[S. 8]{ricci:inbook}.
Dem entgegen stehen Entitäten, die den Anwendern eines Systems empfohlen werden können.
Entitäten werden in Empfehlungssystemen als Elemente bezeichnet und bilden den Inhalt von Vorschlägen.
Dabei kann es sich bei Elementen sowohl um Gegenstände wie CDs, Bücher und Hotels handeln, als auch beispielsweise Personen, Orte oder Prozesse \cite[S. 4]{klahold:book}.
% Unter dem Kontext einer Empfehlung versteht \textcite[S. 1]{klahold:book} das Zusammenspiel aus der Situation, in der eine Empfehlung gegeben wird, dem Profil eines Nutzers sowie der Menge möglicher Entitäten zum Zeitpunkt einer Empfehlung.
% Unter dem Kontext einer Empfehlung verstehen \textcite[S. 191]{adomavicius:3:inbook} zusätzliche Parameter wie Uhrzeit, Ort oder die Gegenwart anderer Personen zum Zeitpunkt einer Empfehlung.\footnote{\textcite[S. 1]{klahold:book} versteht unter dem Kontext das Zusammenspiel aus der Situation, in der eine Empfehlung gegeben wird, dem Profil eines Nutzers sowie der Menge möglicher Entitäten zum Zeitpunkt einer Empfehlung. Es wird angenommen, dass der Kontext nach dem Verständnis von \textcite{ricci:book} gleichzusetzen ist mit der Situation nach dem Verstängnis von \textcite{klahold:book} und die Berücksichtigung von Nutzerprofil und Elementen nach dem Verständnis von \textcite{ricci:book} implizit stattfindet.}

% Hier Johannes fragen, ob das verständlicher ist
Nutzer und Elemente können in Empfehlungssystemen miteinander in Beziehung stehen.
Grundlage dieser Beziehungen bilden dokumentierte Interaktionen zwischen Nutzern und Empfehlungssystem, wie der Kauf oder die Bewertung eines Produktes durch einen Anwender \cite[S. 8ff.]{ricci:inbook}.
Dokumentierte Interaktionen werden in Empfehlungssystemen als Nutzer-Element-Interaktion oder Transaktion bezeichnet \cite[S. 8]{recommenderSystems:2016}\cite[S. 9]{ricci:inbook}.

Die bekannteste Form der Nutzer-Element-Interaktion stellen Bewertungen dar.
Bewertungen von Elementen in Empfehlungssystemen können sowohl explizit als auch implizit erfolgen.
Unter expliziter Bewertung wird in Empfehlungssystemen die ausdrückliche Bewertung eines Elements durch einen Nutzer verstanden \cite[S. 9]{ricci:inbook}.
Ein Beispiel stellt die aktive Bewertung eines Produktes durch einen Nutzer in einem Online-Shop in Form einer Produktrezension (z.B. "Gefällt-mir"-Angabe) dar.
Im Vergleich dazu werden unter impliziten Bewertungen passiv erzeugte Verhaltensdaten, wie der Suchverlauf oder die Kaufhistorie eines Nutzers, verstanden, die indirekt mögliche Präferenzen eines Nutzers vermuten lassen \cite[S. 149]{jadidinejad:inproceedings}\cite[S. 403]{unternährer:article}.

Neben Transaktionen können Nutzer bzw. Elemente in Empfehlungssystemen zusätzliche Attribute besitzen \cite[S. 8]{recommenderSystems:2016}.
Beispiele für Attribute eines Nutzers sind das Geschlecht oder Alter eines Anwenders, während Attribute eines Produktes beispielsweise die Produktkategorie oder beschreibende Schlagwörter.
% Unterschied Kombination und Interaktion? Kombibnation ist nicht zwangsläufig das vorhandensein einer Interaktion

\section{Problemstellung}
\label{ch:empfehlungssysteme:problemstellung}
% Im vorangegangenen Kapitel wurden grundlegende Komponenten von Empfehlungssystemen und deren Zusammenspiel erläutert.
% Das grundlegende Problem in Empfehlungssystemen besteht in der Auswahl der Elemente aus einer Menge an Entitäten, von denen angenommen werden kann, dass diese für die Empfänger einer Empfehlung von Relevanz sind \cite[S. 734f.]{adomavicius:inproceedings}\cite[S. 76]{jannach:inproceedings}.
Das grundlegende Problem in Empfehlungssystemen besteht in der Auswahl der Elemente, von denen angenommen werden kann, dass diese für die Empfänger einer Empfehlung von Relevanz sind \cite[S. 734f.]{adomavicius:inproceedings}\cite[S. 76]{jannach:inproceedings}.
Relevanz bedeutet in dem Kontext von Empfehlungssystemen, dass ein Element für einen Nutzer bzw. eine Gruppe an Nutzern einen maximalen Nutzen bietet \cite[S. 49f.]{adomavicius:inproceedings:2}\cite[S. 219]{lakiotaki:inproceedings}.
Formal ausgedrückt besteht das Problem darin, einem Nutzer $c$ eine Menge an Elementen $s^{'}$ $\in$ $S$ zur Verfügung zu stellen, für die gilt:
\begin{equation}\label{eq1}
    \forall c\in C,  s'_c = arg\max_{s \in S} R(c,s)
\end{equation}
Hierbei gibt $C$ die Menge aller Nutzer eines Systems und $R(c,s)$ den Nutzen einer Kombination $(c,s)$ an Element $s$ und Nutzer $c$ (im Folgenden \ac{N-E-K} genannt \cite[S. 3]{recommenderSystems:2016}) an \cite[S. 734f.]{adomavicius:inproceedings}\cite[S. 219]{lakiotaki:inproceedings}.

% Der Kontext $K$ wird in \textcite[S. 1]{klahold:book} explizit für die Ermittlung des Nutzen einer Nutzer- Element- Kombination angeführt.
% Unter dem Kontext einer Empfehlung verstehen \textcite[S. 191]{adomavicius:3:inbook} zusätzliche Parameter wie Uhrzeit, Ort oder die Gegenwart anderer Personen zum Zeitpunkt einer Empfehlung.
% Grundsätzlich stellen kontextbasierte Empfehlungssysteme, das heißt Syteme, die den Kontext von Empfehlungen für die Ermittlung des Nutzen berücksichtigen, eine Erweiterung traditioneller Empfehlungssyteme dar \cite[S. 744ff.]{adomavicius:inproceedings}.
% In der Literatur wird der Nutzen einer Nutzer- Element- Kombination traditionell ohne Berücksichtigung des Kontextes betrachtet, weshalb der Kontext von Empfehlungen in dieser Arbeit vernachlässigt wird \cite[S. 734f.]{adomavicius:inproceedings}\cite[S. 219]{lakiotaki:inproceedings}\cite[S. 3]{jawaheer:article}.
% In Anlehnung an \textcite[S. 3]{jawaheer:article} wird der interessierte Leser für eine detaillierte Übersicht zu kontextbasierten Empfehlungssystemen auf \textcite[S. 191ff]{adomavicius:3:inbook} verwiesen.
% In den Formulierungen der Gleichung in \cite[S. 734f]{adomavicius:inproceedings} und \cite[S. 219]{lakiotaki:inproceedings} wird jedoch nicht explizit auf den Kontext als Bedingung zur Erfüllung von Gleichung \ref{eq1} hingewiesen.
% Es wird angenommen, dass der Kontext für die Ermittlung der Güte einer Empfehlung implizit berücksichtigt wird und daher auf eine explizite Nennung verzichtet wurde. In Anlehnung an \cite[S. 734f]{adomavicius:inproceedings} und \cite[S. 219]{lakiotaki:inproceedings} wurde daher auf die explizite Unterscheidung im weiteren Verlauf verzichtet.

Bei dem beschriebenen Problem in Gleichung \ref{eq1} handelt es sich um ein Maximierungsproblem.
Nach \textcite[S. 1]{book:kallrath} können Maximierungsprobleme der Kategorie der Optimierungsprobleme zugeordnet werden.
Die Problematik des Ermittelns einer nutzenmaximierenden Teilmenge aus einer Menge an Elementen gehört demnach zu der Kategorie der Optimierungsprobleme.
% Da es sich bei dem Problem des Ermittelns einer nutzenmaximierenden Teilmenge aus einer Menge an Elementen um ein Maximierungsproblem handelt, kann das Problem der Kategorie der Optimierungsprobleme, wie in \textcite[S. 1]{book:kallrath} definiert, zugeordnet werden.

% Einigung: Nützlichkeit bezieht sich nicht auf die Güte einer Empfehlung, sondern auf den Nutzen einer Kombination. Güte beschreibt Wert einer Empfehlung.
Der Begriff der "Nützlichkeit" wird in der Literatur vermehrt verwendet, um den Wert eines Elements für einen Nutzer zu beschreiben \cite[S. 10f.]{ricci:inbook}\cite[S. 1]{klahold:book}\cite[S. 735]{adomavicius:inproceedings}.\footnote{Unter der Nützlichkeit kann in dem Kontext von Empfehlungssystemen auch die Güte einer Empfehlung verstanden werden \cite[S. 37ff.]{klahold:book}.}
Dabei kann der Nutzen sowohl unmittelbar über Angaben eines Anwenders erschlossen, als auch über eine Nutzenfunktion des Empfehlungssystems berechnet werden \cite[S. 735]{adomavicius:inproceedings}.
Da die Ermittlung des Nutzens durchaus komplex sein kann, wird der Aspekt der "Nützlichkeit" nachfolgend ausführlich aufgegriffen.

\section{Nutzenfunktion} % Unterschied Nutzen und Güte?
\label{ch:empfehlungssysteme:nutzenfunktion}
% Frage: ist mit der Güte die Güte einer Empfehlung gemeint (gefällt einem nutzer die Empfehlung ja/nein), oder ist damit die Güte im Sinne von die Qualität von Empfehlungssystemen gemeint?
% Der erfolgreiche Einsatz von Empfehlungssystemen wird maßgeblich von der Qualität der ausgesprochenen Empfehlungen bestimmt.
Wie aus Gleichung \ref{eq1} hervorgeht, wird für den Nutzen einer \ac{N-E-K} in Empfehlungssystemen allgemein eine Funktion $R$ angenommen.
Diese Funktion wird Nutzenfunktion (engl.: utility function) genannt und ist traditionell definiert als \cite[S. 195]{adomavicius:3:inbook}\cite[S. 3]{jawaheer:article}:
\begin{equation}\label{eq2}% Dabei handelt es sich um eine lineare Abbildung, bei der über die Funktion u jeder Kombination aus der nxm matrix, genau ein Wert aus R, also der Rating matrix, zugeordnet werden kann
    R: Nutzer \times Element \rightarrow R_{0}
\end{equation}
Die Nutzenfunktion ordnet jeder Kombination $(c,s)$ an Nutzer $c \in C$ und Element $s \in S$ eines Empfehlungssystems einen Nutzen $R(c,s) \in R_{0}$ zu.
$R_{0}$ umfasst alle Werte, die der Nutzen von Kombinationen annehmen kann \cite[S. 49f]{adomavicius:inproceedings:2}.
% In Systemen, in denen der Wert einer Transaktion beispielsweise über eine explizite, binäre Bewertung (z.B. "Gut", "Schlecht") eines Nutzers für ein Element bestimmt wird, würde $R_{0}$ genau die zwei Ausprägungen umfassen, die eine Bewertung annehmen kann (hier: "Gut" und "Schlecht").

Grundsätzlich kann die Nutzenfunktion in Empfehlungssystemen jede beliebige Funktion darstellen \cite[S. 735]{adomavicius:inproceedings}.
Besteht das Ziel eines Empfehlungssystems bespielsweise darin, den Gesamtumsatz eines Unternehmens zu maximieren, so kann der Nutzen einer \ac{N-E-K} über eine Profit-Funktion ermittelt werden \cite[S. 735]{adomavicius:inproceedings}.
% Solche Systeme, in denen der Nutzen von Nutzer- Element- Kombinationen über eine eigenständige Nutzenfunktion ermittelt wird, zählen zu den nutzenbasierten Empfehlungssystemen und sind nicht Inhalt dieser Arbeit.
In den meisten Empfehlungssystemen wird für den Nutzen $R(c,s)$ unmittelbar die Bewertung (engl.: Rating) $r_{cs}$ eines Element $s$ durch einen Nutzer $c$ angenommen \cite[S. 735]{adomavicius:inproceedings}\cite[S. 9]{ricci:inbook}\cite[S. 11]{recommenderSystems:2016}.
Daher ist die Nutzenfunktion in der Literatur auch als Rating-Funktion oder Rating-Matrix bekannt \cite[S. 49]{adomavicius:inproceedings:2}\cite[S. 91]{ekstrand:article}\cite[S. 11]{recommenderSystems:2016}.
Aus Gründen der Einfachheit wird für den weiteren Verlauf der Arbeit ebenfalls der Wert einer Bewertung $r_{cs}$ für den Nutzen $R(c,s)$ einer \ac{N-E-K} $(c,s)$ angenommen (d.h. $R(c,s) = r_{cs}$).
% Diese Abbildung einer Funktion als Matrix folgt der mathematischen Definition von Matrizen als lineare Funktion.
% Es wird davon ausgegangen, dass die gängige Verwendung des Ratings als Maß des Nutzen eines Elements für einen Nutzer daher kommt, dass Empfehlungssysteme traditionell in dem Kontext item-to-people-recommendation eingesetzt wurden.
% In solchen systemen ist es meistens das ziel, die elemente zu empfehlen, die den nutzer am meisten ansprechen. -> Folglich Nutzen einer Nutzer-Element-Kombination bedeutet Nutzen eines Elements für einen Nutzer. (kann aber ja auch bedeuten nutzen eines elements für einen nutzer im sinne von profitmaximierung)
% Unter der Annahme ist damit erklärbar, dass mit einer gewissen Konfidenz davon ausgegangen werden kann, dass die Bewertung eines Nutzers für ein Element auch dessen gefühlten Nutzen des ELements für ihn abbildet.

%Bewertungen von Elementen in Empfehlungssystemen können sowohl explizit als auch implizit erfolgen.
%Unter expliziter Bewertung kann in Empfehlungssystemen die ausdrückliche Bewertung eines Elements durch einen Nutzer verstanden werden.
%Ein Beispiel stellt die aktive Bewertung eines Produktes durch einen Nutzer in einem Online-Shop in Form einer Produktrezension (z.B. "Gefällt-mir"-Angabe) dar.
%Im Vergleich dazu werden unter impliziten Bewertungen in Empfehlungssystemen passiv erzeugte Verhaltensdaten, wie der Suchverlauf oder die Kaufhistorie eines Nutzers, verstanden, die indirekt mögliche Präferenzen eines Nutzers vermuten lassen \cite[S. 149]{jadidinejad:inproceedings}\cite[S. 403]{unternährer:article}.
% IDEE: hier einfügen Rating Scales? Ordinal, numeric, etc.

Tabelle \ref{tab1} stellt beispielhaft den Ausschnitt einer Rating-Matrix $R$ dar, welche explizite Bewertungen von Mitarbeitern zu ihren Fähigkeiten abbildet \cite[S. 735]{adomavicius:inproceedings}\cite[S. 16]{link:booklet}.
Hierbei stellen Mitarbeiter die Nutzer und Fähigkeiten die Elemente des Systems dar.
Der Nutzen einer Kombination $R(c,s)$ wird unmittelbar über die Bewertung $r_{cs}$ eines Mitarbeiters $c$ für eine Fähigkeit $s$ bestimmt.
Eine Bewertung kann jeden Wert in $R_{0}=\{1;2;3;4;5\}$ annehmen.

\begin{table}[htbp]
    \begin{center}
    \begin{tabular}{c|c|c|c|c|c|c}
    {} & {\textbf{Java}} & {\textbf{Python}} & {\textbf{MySQL}} & {\textbf{MongoDB}} & {\textbf{HDFS}} & {\textbf{Spark}}\\
    \hline
    \textbf{D., Jane} & ? & 4 & 3 & 3 & ? & ?\\
    \hline
    \textbf{D., John} & 3 & ? & 2 & ? & 1 & ?\\
    \hline
    \textbf{M., Erika} & ? & ? & ? & 3 & 5 & 3\\
    \hline
    \textbf{M., Max} & 2 & 3 & 1 & ? & ? & ?\\
    \end{tabular}
    \end{center}
    \caption[Beispielhafte Darstellung einer Rating-Matrix ]{Beispielhafte Darstellung einer Rating-Matrix \\
	(Eigene Darstellung in Anlehnung an \cite[S. 16]{link:booklet})}
	\label{tab1}
\end{table}

Der Rating-Matrix $R$ können alle angegebenen Bewertungen von Mitarbeitern zu ihren Fähigkeiten entnommen werden.
So wird beispielsweise ersichtlich, dass Jane D. im Verhältnis zu den anderen Mitarbeitern ihre Fähigkeiten in Python am höchsten bewertet hat.
Darüber hinaus geht aus Tabelle \ref{tab1} hervor, dass einige Kombinationen an Nutzer und Element anstelle einer Bewertung $r_{cs}$ ein Fragezeichen beinhalten.
Dieses kennzeichnet die Kombinationen des Empfehlungssystems, für die keine Bewertungen vorhanden sind.
Das kann beispielsweise bei Nutzern der Fall sein, die noch keine Elemente bewertet haben, oder bei Elementen, die neu zu einer Anwendung hinzugefügt wurden und noch keine Interaktion mit Nutzern des Systems aufweisen.
Wie im nachfolgenden Kapitel deutlich werden wird, stellt die Vorhersage fehlender Nutzer-Element-Interaktionen eine der Kernaufgaben von Empfehlungssystemen dar.
% Nach \textcite[S. 37]{klahold:book} besteht eine der Herausforderungen in der Ermittlung des tatsächlichen Nutzen einer Empfehlung darin, dass für eine objektive Beurteilung die genaue Nutzenfunktion einer Empfehlung an einen Nutzer bekannt sein muss.
% Hier ergänzen nach gespräch mit Andreas. Quellen dazu: Klahod S. 37ff und hier: https://api-depositonce.tu-berlin.de/server/api/core/bitstreams/b490302e-8b6a-4300-946f-b5763c2b47d7/content

\section{Erstellen von Empfehlungen}
\label{ch:empfehlungssysteme:empfehlungserstellung}
Das Erstellen von Empfehlungen ist eine nicht-triviale Aufgabe, die bereits seit Jahrzehnten Inhalt zahlreicher Veröffentlichungen und Konferenzen ist.
Grundsätzlich kann das Erstellen von Empfehlungen in Empfehlungssystemen in zwei Phasen unterteilt werden \cite[S. 405]{unternährer:article}\cite[S. 854]{adomavicius:4:inbook}:
\begin{enumerate}
	\item Vorhersage-Phase
	\item Ranking-Phase
\end{enumerate}

In der Vorhersage-Phase werden für alle \ac{N-E-K}-en, für die keine Transaktionen vorliegen, deren Nutzer-Element-Interaktion vorhergesagt \cite[S. 3]{recommenderSystems:2016}.
Für die Ranking-Phase werden die (angenommenen) Nutzer-Element-Interakti\-onen aus Phase 1 herangezogen, in Abhängigkeit ihrer Nützlichkeit sortiert und das Element bzw. die Elemente mit dem größten Nutzen empfohlen \cite[S. 854]{adomavicius:4:inbook}.
Nachfolgend werden die einzelnen Phasen im Detail erläutert.

\subsection{Vorhersage}
\label{ch:empfehlungssysteme:empfehlungserstellung:prediction}
% Ratings normalerweise nur vorhanden bei nutzern, die diese produkte in der vergangenheit bewertet haben, S. 735, file://wsl%24/Ubuntu/home/masc6/Projects/masterarbeit/literatur/Toward_the_next_generation_of_recommender_systems_a_survey_of_the_state-of-the-art_and_possible_extensions.pdf
Wie aus Gleichung \ref{eq1} hervorgeht, ist das Vorhandensein einer Nutzer-Ele\-ment-Interaktion Voraussetzung, um die Nützlichkeit einer Kombination im Vergleich zu anderen \ac{N-E-K}-en zu bewerten.
Eine gängige Eigenschaft von Umgebungen, in denen Empfehlungssysteme ihren Einsatz finden, ist, dass die zugrundeliegenden Daten unvollständig sind \cite[S. 735]{adomavicius:inproceedings}.
Das bedeutet, dass nicht für jede Kombination an Nutzer und Element ein Wert für $r_{cs}$ vorliegt (wie in Tabelle \ref{tab1} durch Fragezeichen gekennzeichnet).
So ist es beispielsweise anhand der Daten aus Tabelle \ref{tab1} nicht möglich zu Beurteilen, welche der beiden Mitarbeiter Jane D. und Erika M. die Fähigkeit Java besser beherrscht.
% Würde beispielsweise für Erika M. aus Tabelle \ref{tab1} versucht zu Bestimmen, ob ihre Fähigkeiten in Java oder Python höher sind, könnte anhand von Gleichung \ref{eq1} aufgrund der fehlenden Nutzer-Element-Interaktion keine Aussage getroffen werden.

Aufgabe eines Empfehlungssystems ist es daher, die fehlenden Bewertungen für \ac{N-E-K}-en der Rating-Matrix $R$ vorherzusagen.
Sei $\hat{r}_{cs}$ der angenommene Wert einer Nutzer-Element-Interaktion, so wird $\hat{r}_{cs}$ in Empfehlungssystemen allgemein wie folgt vorhergesagt (in Anlehnung an \cite[S. 743]{adomavicius:inproceedings}):\footnote{Voraussetzung der Gültigkeit der Formel \ref{eq3} ist die Annahme, dass gilt $r_{cs} = R(c,s)$. Das bedeutet, dass für den Nutzen einer \ac{N-E-K} unmittelbar der Wert der Bewertung $r_{cs}$ eines Nutzers $c$ für das Element $s$ angenommen werden kann (Vgl. Kapitel \ref{ch:empfehlungssysteme:nutzenfunktion}).}
% Für die Vorhersage einer Nutzer- Element- Kombination $\hat{r}_{cs}$ der Matrix R kann allgemein folgende Formel angenommen werden (in Anlehnung an \cite[S. 743]{adomavicius:inproceedings}):\footnote{Voraussetzung der Gültigkeit der Formel \ref{eq3} ist die Annahme, dass gilt $r_{cs} = R(c,s)$. Das bedeutet, dass für den Nutzen einer \ac{N-E-K} unmittelbar der Wert der Bewertung $r_{cs}$ eines Nutzers $c$ für das Element $s$ angenommen werden kann (Vgl. Kapitel \ref{ch:empfehlungssysteme:nutzenfunktion}).}
\begin{equation}\label{eq3}
\hat{r}_{cs} =
    \begin{cases}
        r_{cs} & \textnormal{wenn } r_{cs} \neq \textnormal{ ?} \\
        R(c,s) & \textnormal{wenn }r_{cs} = \textnormal{ ?} \\
    \end{cases}
\end{equation}
Das heißt, grundsätzlich wird $\hat{r}_{cs}$ entweder unmittelbar über vorhandene Nutzer-Element-Interaktionen $r_{cs}$, oder bei fehlenden Transaktionen für $(c,s)$ unter Anwendung der Nutzenfunktion $R$ vorhergesagt werden.
%Demnach wird eine Bewertung $\hat{r}_{cs}$ entweder unmittelbar über vorhandene Nutzer-Element-Interaktionen $r_{cs}$, oder bei fehlenden Transaktionen für $(c,s)$ unter Anwendung der Nutzenfunktion $R$ vorhergesagt werden.
% Unterschied ist, dass hier rij sich von u unterscheidet und bei uns einfach erstmal R als ein wert angenommen wird
% Wichtig hier klar zu definieren, dass die Nützlichkeit oft so definiert ist, dass sie entweder explizit gegeben ist, oder über u berechnet wird für fehlende werte von r
% bei uns wird dann quasi u für alle werte berechnet, da wir ja nicht explizit einen wert annehmen, sondern Ro sich aus r1 und r2 usw zusammensetzt

In der Praxis existieren verschiedene Techniken, um fehlende Nutzer-Ele\-ment-Interaktionen vorherzusagen.
Empfehlungssysteme können anhand der eingesetzten Technik für die Vorhersage in verschiedene Kategorien unterteilt werden \cite[S. 8 ff.]{recommenderSystems:2016}.
Zu den bekanntesten Kategorien von Empfehlungssystemen zählen:
\begin{itemize}%Sind die mathematischen Bezeichnungen hier verwirrend oder sinnvoll?
	\item \textit{Systeme des Kollaborativen Filterns:} Vorhersage fehlender Nutzer-Element-Interaktionen $r_{cs}$ eines Zielnutzers $c$ anhand vorhandener Nutzer-Ele\-ment-Interaktionen $r_{c's}$ anderer Nutzer $c' \in C \backslash \{c\}$ des Systems, die dem Zielnutzer $c$ ähnlich sind.
	\item \textit{Inhaltsbasierte Systeme:} Vorhersage fehlender Nutzer-Element-Interakti\-onen $r_{cs}$ eines Zielnutzers $c$ anhand vorhandener Nutzer-Element-Inter\-aktionen $r_{cs'}$ des Zielnutzers zu Elementen $s' \in S \backslash \{s\}$, die dem Zielelement $s$ ähnlich sind.
	\item \textit{Wissensbasierte Systeme:} Vorhersage fehlender Nutzer-Element-Interakti\-onen eines Zielnutzers anhand der aktiven Angabe von Präferenzen durch den Zielnutzer.
	\item \textit{Hybride Systeme:} Kombination verschiedener Techniken für die Vorhersage fehlender Nutzer-Element-Interaktionen.
\end{itemize}

Der Fokus dieser Arbeit liegt auf der zweiten Phase des Empfehlungserstel\-lungs-Prozesses, weshalb auf eine eingehende Erläuterung der Techniken nachfolgend verzichtet wird.
Für eine detaillierte Übersicht der bekanntesten Werkzeuge wird auf \textcite[S. 8ff.]{recommenderSystems:2016} verwiesen.

\subsection{Ranking}
\label{ch:empfehlungssysteme:empfehlungserstellung:recommendation}
Die zweite Phase umfasst die eigentliche Erstellung der Empfehlungen, die an einen Nutzer zurückgegeben werden.
Ziel ist es, einem Nutzer $c$ in Abhängigkeit der vollständigen Rating-Matrix $R$ eine Empfehlung an Elementen $s'$ mit dem größtmöglichen Nutzen zu generieren \cite[S. 87]{ekstrand:article}.
% Wichtig zu unterscheiden, dass die Empfehlungen nicht zwangsäufig die präferenzen des nutzers empfehlen müssen, da diese auch von anderen kriterien abhängen könnne, siehe quelle unten
% HIER WEITERMACHEN mit S. 87, file://wsl%24/Ubuntu/home/masc6/Projects/masterarbeit/literatur/Collaborative%20Filtering%20Recommender%20Systems.pdf

In Abhängigkeit des Anwendungsfalls kann eine Empfehlung verschiedene Formen annehmen.
So kann eine Empfehlung eine Gruppe an Elementen darstellen.
Ein Beispiel stellen Empfehlungen auf Webseiten von Klamottenlabeln dar, die verschiedene Kleidungsstücke gemeinsam als vollständige Outfits empfehlen.
In den meisten Fällen geben Empfehlungssysteme eine sortierte Liste der nützlichsten Elemente zurück (Top-K-Elemente), in der für jedes Element der angenommene Wert für einen Nutzer angegeben ist \cite[S. 6]{ricci:inbook}\cite[S. 3]{recommenderSystems:2016}.
Dabei wird die Nützlichkeit einer Kombination, wie in Kapitel \ref{ch:empfehlungssysteme:empfehlungserstellung:prediction} erläutert, zumeist anhand eines Ratings durch einen Nutzer bestimmt, oder über eine Nutzenfunktion in Phase 1 vorhergesagt.
Diese Ratings erfolgen in gängigen Empfehlungssystemen zumeist anhand eines Kriteriums, beispielsweise in Form eines übergreifenden Ratings für ein Element.

In der Praxis kann es vorkommen, dass von einem vorhergesagten Wert einer \ac{N-E-K} nicht unmittelbar auf den tatsächlichen Rang einer Kombination im Ranking geschlossen werden kann, beispielsweise da noch andere Kriterien den Nutzen (und somit den Rang) einer Kombination beeinflussen \cite[S. 405]{unternährer:article}.
Wie in Kapitel \ref{ch:erweiterungen} deutlich werden wird, ist die Bestimmung des Nutzen eines Elements für einen Nutzer anhand eines einzelnen Kriteriums (e.g. dem Rating) nicht immer ausreichend, um dessen Nutzen zuverlässig zu bestimmen \cite[S. 847]{adomavicius:4:inbook}.
% ------- Hier (oder oben?) kurze Zusammenfassung der Begrifflichkeiten (siehe aggarwal, S. xviii, file:///C:/Users/masc6/OneDrive/Persoenliche_Unterlagen/Uni/Masterthesis/Aggarwal2016_Book_RecommenderSystems.pdf)

Abbildung \ref{fig:empfehlungssysteme:recommendation:abb1} illustriert den Prozess der Empfehlungserstellung am Beispiel eines Ausschnitts der Mitarbeiter-Fähigkeiten Matrix aus Tabelle \ref{tab1}.
Das Beispiel stellt den Empfehlungerstellungsprozess für die Mitarbeiterin Erika M. dar.
Dabei soll mithilfe des Empfehlungssystems die Fähigkeit vorschlagen werden, von der angenommen werden kann, dass Erika M. in dieser Fähigkeit die meisten Kenntnisse im Vergleich zu den anderen Fähigkeiten aufweist (e.g. $argmax(\hat{r}_{\textnormal{Erika M., s}})$).
% am die dieser am Gesucht: Element des Mitarbeiters, für das er/sie die höchsten Fähigkeiten aufweist, bspw. weil vorgeschalgen wird in dem Bereich eine Schulung zu geben, oder MA gesucht wird, der experte in einem bestimmten gebiet ist

\begin{figure}[H]
    \centering
	\includegraphics[width=0.9\textwidth]{gfx/phasen-empfehlungserstellung.png}
	\caption[Phasen der Empfehlungserstellung]{Phasen der Empfehlungserstellung\\}
	\label{fig:empfehlungssysteme:recommendation:abb1}
\end{figure}

Empfehlungen dieser Art können beispielsweise sinnvoll sein, wenn Mitarbeitern Fähigkeiten vorgeschlagen werden sollen, die sie verhältnismäßig schnell erwerben können.
% Empfehlungen dieser Art können beispielsweise sinnvoll sein, wenn Mitarbeitern Fähigkeiten vorgeschlagen werden sollen, die viele Ähnlichkeiten zu bereits existierenden Fähigkeiten eines Mitarbeiters aufweisen.
% Dies kann sinnvoll sein, wenn Mitarbeiter möglichst schnell eine Fähigkeit erlernen sollen.
% Wenn angenommen wird, dass Mitarbeiter, die bereits Fähigkeiten besitzen, welche Ähnlichkeiten zu der gefragten Fähigkeit aufweisen, schneller die gefragte Fähigkeit erlernen als Mitarbeiter, die diese Fähigkeiten nicht oder weniger gut beherrschen.
So kann es sinnvoll sein der Mitarbeiterin Erika M. die Fähigkeit MySQL für eine Schulung vorzuschlagen, da aufgrund ihrer Vorkenntnisse in MongoDB angenommen werden kann, dass sie sich MySQL im Verhältnis zu anderen unbekannten Fähigkeiten (z.B. Python) schnell aneignen kann.

\section{Präferenzen in Empfehlungssystemen}
\label{ch:empfehlungssysteme:preferences}
Im Zusammenhang mit Empfehlungssystemen fällt häufig der Begriff der Präferenzen (engl.: preferences) \cite[S. 2]{ricci:inbook}\cite[S. 11f.]{recommenderSystems:2016}.
% Brauch man hier überhaupt ein zitat, wenn ich die Annahme selber aufstelle?
In diesem Kapitel werden Präferenzen allgemein erläutert und in dem Kontext von Empfehlungssystemen inhaltlich abgegrenzt.

Nach dem Verständnis der Bundeszentrale für politische Bildung können Präferenzen grundsätzlich als Vorlieben und Verhaltensweisen von Personen verstanden werden, "[...], die bewirken, dass Güter unterscheidbar werden."\cite{pollert:book}
Ein einfaches Beispiel der Präferenzen von Personen stellt die Vorliebe von Schülern für bestimmte Fächer in der Schule dar.
Hierbei sind Schüler die Personen, die Präferenzen besitzen und Güter die verschiedenen Fächer, die Schüler besuchen müssen.
Basierend auf verschiedenen Gründen können Schüler bestimmte Fächer präferieren.
Das heißt, dass sie einige Fächer gegenüber Anderen bevorzugen können.

Wenn Präferenzen in dem Kontext von Empfehlungssystemen aufkommen, wird sich in der Regel auf die Vorlieben und Verhaltensweisen der Nutzer des Systems bezogen.
Beispielsweise kann unter der Präferenz eines Nutzers in einem Online-Shop die Bevorzugung eines Produktes gegenüber eines Anderen Verstanden werden.

Ein Verständnis der Präferenzen von Nutzern eines Systems kann für die erfolgreiche Ermittlung von Empfehlungen von großer Bedeutung sein. % \cite[S. 1]{jawaheer:article}
In der Literatur wird unter Präferenzen eines Nutzers zumeist dessen explizite und / oder implizite Bewertung von Elementen verstanden \cite[S. 37]{berkovsky:2:article}\cite[S. 743]{adomavicius:inproceedings}\cite[S. 11]{recommenderSystems:2016}\cite[S. 129]{ekstrand:article}.\footnote{Vgl. Kapitel \ref{ch:empfehlungssysteme:nutzenfunktion}.}
Die Präferenzen von Nutzern sind jedoch keinesfalls trivial.
Dies hängt unter anderem damit zusammen, dass einer expliziten Bewertung (z.B. einer Gefällt-mir-Angabe in Facebook) durch einen Nutzer aus der Sicht eines Nutzers ein unterschiedlicher Wert zukommen kann als aus Sicht des Systems.
Darüber hinaus kann auch die Bewertungsskala eines Elements Einfluss auf die Genauigkeit von Präferenzen haben.
So kann eine fehlende Gefällt-mir-Angabe für ein Element durch einen Nutzer in Facebook nicht zwangsläufig als Abneigung für das Element verstanden werden \cite[S. 11]{recommenderSystems:2016}.

Anhand von Präferenzen ist es möglich Empfehlungen für Nutzer eines Empfehlungssystems zu personalisieren.
In Abhängigkeit des Grads der Personalisierung können Empfehlungen personalisierten bzw. nicht-personalisierten Empfehlungen zugeordnet werden (in Anlehnung an \cite[S. 400]{unternährer:article}).

\subsection{Personalisierte und Nicht-Personalisierte Empfehlungen}
Unter nicht-personalisierten Empfehlungen versteht \textcite[S. 400]{unternährer:article} Empfehlungen, die allen Nutzern die gleichen Elemente anzeigen.
Das bedeutet, alle Nutzer eines Empfehlungssystems bekommen, unabhängig von deren Präferenzen oder Nutzerprofil, dieselben Elemente vorgeschlagen (z.B. "Top 10 Lieder aller Zeiten").

Für die Ermittlung nicht-personalisierter Empfehlungen können Popularitätsmetriken herangezogen werden.
Dies folgt der Annahme, dass Popularität einen guten Prädiktor der Präferenzen aller Nutzer eines Systems darstellt \cite[S. 406]{unternährer:article}.
Popularitätsmetriken stellen sortierte Listen dar, über die Elemente in ordinale Relationen gesetzt werden können (e.g. Element A ist "besser" als Element B, da es einen höheren Rang aufweist) \cite[S. 404ff.]{unternährer:article}.
Die Listen werden basierend auf "aggregierten Präferenzen" aller Nutzer eines Systems generiert.
Hierfür muss im Vorfeld bestimmt werden, welcher Indikator für die Bestimmung der Popularität in einem System geeignet ist (z.B. die Anzahl an Klicks oder Gefällt-mir-Angaben) \cite[S. 406]{unternährer:article}.

Nicht-personalisierte Empfehlungen müssen nicht allein auf absoluter Popularität beruhen.
Nach \textcite[S. 405]{unternährer:article} ermöglichen zeitlich, räumliche und soziale Popularität, absolute Popularität zu begrenzen.
So können Nachrichten-Plattformen beispielsweise gezielt die populärsten Artikel des Tages (zeitliche Popularität) auf ihrer Startseite anzeigen, anstelle der populärsten Artikel aller Zeiten (absolute Popularität).
Unter der Annahme, dass die Relevanz der Beiträge mit deren Aktualität zusammenhängt, kann durch Verwendung von zeitlicher Popularität für die Ermittlung von Empfehlungen die Relevanz von Beiträgen auf der Startseite erhöht werden \cite[S. 405]{unternährer:article}.
% Fragwürdig, ob das nicht auch schon eine art der personalisierung ist? Es ist anzunehmen, dass nicht-personalisiert einfach bedeutet, dass nicht die präfeenzen eines nutzers speziell verwendet werden, um gezielt empfehlungen zu schalten, sondern eben allgemein

Der maßgebende Unterschied zu personalisierten Empfehlungen besteht darin, dass das Erstellen von nicht-personalisierten Empfehlungen für einen Nutzer nicht auf dessen individuellen Präferenzen beruht.
Empfehlungen können somit völlig unabhängig davon erstellt werden, ob ein Nutzer eines Empfehlungssystems bereits eine Interaktion mit einem System aufweist oder nicht.
Dies ermöglicht Systemen auch Empfehlungen für Nutzer zu erstelllen, für die beispielsweise keine Präferenzen bekannt sind.\footnote{Das Problem ist in der Literatur als das Kaltstart-Problem bekannt \cite[S. 407]{unternährer:article}.}
Ausschlaggebend ist lediglich, dass Informationen vorliegen, die es ermöglichen die Popularität von Elementen zu ermitteln.
So kann beispielsweise kein populärster Artikel des Tages einer Nachrichten-Plattform basierend auf der Popularität ermittelt werden, wenn überhaupt keine Interaktionen mit Artikeln des Systems vorliegen.

Im Gegensatz zu nicht-personalisierten Empfehlungen stellen personalisierte Empfehlungen Vorschläge dar, die basierend auf individuellen Präferenzen von Nutzern erstellt werden (bspw. über explizite oder implizite Bewertungen).
Der Grad der Personalisierung ermöglicht eine weitere Unterteilung der personalisierten Empfehlungen in schwach-personalisierte Empfehlungen und stark-personalisierte Empfehlungen.
Unter schwacher Pesonalisierung versteht \textcite[S. 407]{unternährer:article} stereotypisierende Empfehlungen, die Nutzer, basierend auf deren Attributen (z.B. Wohnort, Alter, Geschlecht), bestimmten Kategorien zuordnen.
Prinzipiell ähneln schwach-personalisierte Empfehlungen den nicht-personalisierten Empfehlungen, da sie auch basierend auf Popularitätsmetriken erstellt werden.
Der Unterschied besteht darin, dass die Popularität von Elementen innerhalb einer Kategorie ermittelt wird.
Die Empfehlungen der populärsten Elemente für einen Nutzer erfolgen dann in Abhängigkeit der kategorialen Zugehörigkeit des Nutzers \cite[S. 407ff.]{unternährer:article}.

Stark-personalisierte Empfehlungen verzichten auf pauschale und kategoriale Vorentscheidungen und beziehen sich für das Ermitteln von Empfehlungen auf paarweise Relationen (sog. "Good Matches") zwischen Nutzern untereinander bzw. zwischen Nutzern und deren Elementen \cite[S. 415]{unternährer:article}.
Solche Sachverhalte sind vorwiegend in Systemen des Kollaborativen Filterns und Inhaltsbasierten Empfehlungssystemen anzutreffen.
Hierbei werden nicht einzelne Nutzer oder Elemente bewertet, etwa wie viele Fünf-Sterne-Rezensionen ein Produkt erhalten hat. 
Stattdessen wird der "Fit" einer Kombination an Nutzer und Element bewertet.
Das bedeutet, es wird eine Relation zwischen einem  Nutzer und einem Element erstellt, die dann mit anderen Relationen verglichen werden kann.
Beispielsweise kann eine Fünf-Sterne-Bewertung eines Produktes durch einen Nutzer als starker "Fit" des Produktes mit dem Nutzer gesehen werden.
Diese Beziehung zwischen Nutzer und Produkt kann dann mit anderen Beziehungen verglichen werden, um daraus Präferenzen von Nutzern abzuleiten \cite[S. 417]{unternährer:article}.
% Popularitätsmetriken bewerten, wie gut ein element ist und ob es besser oder schlechter als ein anderes ist, während personalisierte empfehlungen beziehungen zwischen nutzern und elementen bewerten

Eine Voraussetzung für das Erstellen personalisierter Empfehlungen ist die Abbildung der Präferenzen der Nutzer eines Systems \cite[S. 36]{berkovsky:2:article}.

\subsection{Zugrundeliegende Datenstruktur}% Frage ist, wie Präferenzen in RS aussehen, erwähnen
Präferenzen von Nutzern werden in Empfehlungssystemen über sogenannte Nutzermodelle dargestellt \cite[S. 9]{ricci:inbook}\cite[S. 2]{jawaheer:article}\cite[S. 246]{berkovsky:article}.
Nach \textcite[S. 249f.]{berkovsky:article} speichern Nutzermodelle die Präferenzen traditionell in einer zweidimensionalen Matrix $R'_{gen}$:
\begin{equation}\label{eq4}
    R'_{gen}: Nutzer_{Attribut} \times Element_{Attribut} \rightarrow rating
\end{equation}
Hierbei kennzeichnet die Dimension $Nutzer_{Attribut}$ Attribute der Nutzer (z.B. ID, Name, Geschlecht) und die Dimension $Element_{Attribut}$ Attribute der Elemente (z.B. ID, Produktkategorie, Genre).
Das $rating$ stellt die Bewertungen der Nutzer für Elemente dar \cite[S. 250]{berkovsky:article}.

In Abhängigkeit der eingesetzten Technik innerhalb eines Empfehlungssystems (Vgl. Kapitel \ref{ch:empfehlungssysteme:empfehlungserstellung:prediction}) muss die Matrix $R'_{gen}$ Nutzermodelle mit unterschiedlichen Daten abbilden können. % Zitat? \cite[S. 248]{berkovsky:article}
Hierfür sind in Abbildung \ref{fig:empfehlungssysteme:preferences:abb1} drei Beispiele unterschiedlicher Nutzermodelle grafisch dargestellt, welche nachfolgend erläutert werden.

In Systemen des kollaborativen Filterns werden Präferenzen von Nutzern für Elemente in der Rating-Matrix $R$ gespeichert \cite[S. 246ff.]{berkovsky:article}.
Nutzer und Elemente werden dabei jeweils über ein eindeutiges Attribut (z.B. ID, Name) beschrieben.
Daraus ergibt sich nach \textcite[S. 250]{berkovsky:article} für die Abbildung der Nutzermodelle in Matrix $R_{CF}$ folgende Form:
\begin{equation}\label{eq5}
    R_{CF}: Nutzer_{ID} \times Element_{ID} \rightarrow rating
\end{equation}
Die Dimensionen $Nutzer_{ID}$ und $Element_{ID}$ stellen die eindeutig beschreibenden Attribute der Nutzer und Elemente dar.
Anhand des Beispiels aus Kapitel \ref{ch:empfehlungssysteme:nutzenfunktion} ist in Abbildung \ref{fig:empfehlungssysteme:preferences:abb1:1} ein Beispiel einer Nutzer$_{ID}$$\times$Element$_{ID}$-Matrix abgebildet.\footnote{Für eine bessere Übersicht wurde das Beispiel auf die drei Fähigkeiten Java, Python und MySQL reduziert und die Bewertungen lediglich für einen Beispiel-Mitarbeiter (hier: John D.) angegeben.}
Mitarbeiter und Fähigkeiten werden jeweils eindeutig über deren Namen gekennzeichnet (d.h. Attribut ID = Name).
% Der Sachverhalt wird nachfolgend anhand des Beispiels aus Kapitel \ref{ch:empfehlungssysteme:nutzenfunktion} erläutert.
% Hierfür sind in Abbildung \ref{fig:empfehlungssysteme:preferences:abb1:1} die Nutzermodelle einiger Mitarbeiter des Beispiels in einer Matrix dargestellt.\footnote{Für eine bessere Übersicht wurden das Beispiel auf die drei Fähigkeiten Java, Python und MySQL reduziert und die Bewertungen lediglich für einen Beispiel-Mitarbeiter (hier: John D.) angegeben.}

\begin{figure}[H]
    \centering
    \subfloat[Nutzer$_{ID}\times$Element$_{ID}$ - Matrix]{\includegraphics[height=1.5in]{gfx/abbildung-präferenzen-A.png}\label{fig:empfehlungssysteme:preferences:abb1:1}}\\
    \subfloat[Nutzer$_{ID}\times$Element$_{Attribut}$ - Matrix]{\includegraphics[height=1.48in]{gfx/abbildung-präferenzen-B.png}\label{fig:empfehlungssysteme:preferences:abb1:2}}\\
    \subfloat[Nutzer$_{Attribut}\times$Element$_{Attribut}\times$Kontext$_{Attribut}$ - Matrix]{\includegraphics[height=1.85in]{gfx/abbildung-präferenzen-C.png}\label{fig:empfehlungssysteme:preferences:abb1:3}}
  \caption[Abbildung von Präferenzen]{Abbildung von Präferenzen\\
	(Eigene Darstellung in Anlehnung an \cite[S. 198]{adomavicius:3:inbook})}\label{fig:empfehlungssysteme:preferences:abb1}
\end{figure}
% Werden Nutzermodelle in Systemen folglich über eine Nutzer$\times$Element-Matrix abgebildet, können die Präferenzen eines Nutzers unmittelbar aus der Rating-Matrix abgelesen werden.

Ein Nutzermodell $UM_{c}$ entspricht nun genau einer Sammlung aller Bewertungen eines Nutzers $c$ dieser Matrix.
Nach \textcite[S. 41]{berkovsky:2:article} kann diese Sammlung beispielsweise als Liste paarweiser Einträge dargestellt werden:
\begin{equation}\label{eq6}
    UM_{c} = \{s_{1}:r_{1},\textnormal{ }s_{2}:r_{2},\textnormal{ }...,\textnormal{ }s_{k}:r_{k}\}
\end{equation}
Dabei stellt jedes Paar $s_{j}:r_{j}$ die Bewertung $r_{j}$ eines Elements $s_{j}$ durch den Nutzer $c$ dar \cite[S. 41]{berkovsky:2:article}.
Daraus ergibt sich beispielsweise für den Nutzer John D. das Nutzermodell $UM_{John\textnormal{ }D.} = \{Java: 3, Python:\textnormal{ }, MySQL: 2\}$ (Vgl. Abbildung \ref{fig:empfehlungssysteme:preferences:abb1:1}).

Ein Vergleich der Matrix $R_{CF}$ aus Abbildung \ref{fig:empfehlungssysteme:preferences:abb1:1} mit der Rating-Matrix aus Tabelle \ref{tab1} zeigt, dass Abbildung \ref{fig:empfehlungssysteme:preferences:abb1:1} lediglich einen vereinfachten Ausschnitt der Rating-Matrix darstellt.
In Systemen des kollaborativen Filterns sind die Nutzer$_{ID}$$\times$Element$_{ID}$-Matrix zur Abbildung der Nutzermodelle und die Rating-Matrix folglich identisch.

In inhaltsbasierten Empfehlungssystemen stellen nicht die Bewertungen der Nutzer für Elemente als Ganzes deren Präferenzen dar, sondern die Bedeutung einzelner Elementattribute für einen Nutzer.
Nutzermodelle müssen folglich nicht die Bewertungen von Nutzern für Elemente, sondern die Wichtigkeit einzelner Elementattribute für einen Nutzer darstellen \cite[S. 42]{berkovsky:2:article}.
Nach \textcite[S. 42]{berkovsky:2:article} entspricht ein Nutzermodell demnach einer Sammlung der Wichtigkeiten (meist in Form von Gewichten) einzelner Elementattribute für einen Nutzer.
Ein solches Nutzermodell lässt sich ebenfalls als eine Liste paarweiser Einträge abbilden:
\begin{equation}\label{eq7}
    UM_{c} = \{a_{1}:w_{a(1)}, a_{2}:w_{a(2)}, ..., a_{k}:w_{a(k)}\}
\end{equation}
Jedes Paar $a_{j}:w_{a(j)}$ stellt das Gewicht $w$ eines Elementattributs $a$ für den Nutzer $c$ dar \cite[S. 42]{berkovsky:2:article}.
Für die Darstellung der Nutzermodelle eines inhaltsbasierten Systems ergibt sich nach \textcite[S. 251]{berkovsky:article} folgende Matrix:
\begin{equation}\label{eq8}
    R_{CB}: Nutzer_{ID} \times Element_{Attribut} \rightarrow rating
\end{equation}
Hierbei umfasst Element$_{Attribut}$ das Attribut, durch das die Elemente eines Systems beschrieben werden können \cite[S. 251]{berkovsky:article}.

Abbildung \ref{fig:empfehlungssysteme:preferences:abb1:2} illustriert eine solche Nutzer$_{ID}$$\times$Element$_{Attribut}$-Matrix anhand des Beispiels aus Kapitel \ref{ch:empfehlungssysteme:nutzenfunktion}.
Als Elementattribut wurde für dieses Beispiel die Typisierung einer Programmiersprache mit den Ausprägungen "Statisch" und "Dynamisch" gewählt.
Ein Nutzermodell enthält für alle Ausprägungen des Attributs Gewichte $w$, die den Wert eines Elementattributs für einen Nutzer abbilden \cite[S. 251]{berkovsky:article}.
Daraus ergibt sich beispielsweise für den Nutzer John D. das Nutzermodell $UM_{John\textnormal{ }D.} = \{Statisch:\textnormal{ }, Dynamisch: 1.0\}$ (Vgl. Abbildung \ref{fig:empfehlungssysteme:preferences:abb1:2}).

Um die Gewichte einer Nutzer$_{ID}$$\times$Element$_{Attribut}$-Matrix zu ermitteln, existieren in der Literatur verschiedene Verfahren \cite[S. 42]{berkovsky:2:article}.
Grundsätzlich wird die Nutzer$_{ID}$$\times$Element$_{Attribut}$-Matrix auf Basis der Rating-Matrix generiert.
Bei den Gewichten in Abbildung \ref{fig:empfehlungssysteme:preferences:abb1:2} handelt es sich lediglich um ein willkürliches Beispiel, da der Fokus dieses Kapitels auf der Darstellung der Präferenzen und nicht auf den Verfahren für die Ermittlung der Attributgewichte liegt.
% Wie gewichtung hier beschreiben, ohne ins detail zu gehen?

Neben der traditionellen Darstellung der Präferenzen von Nutzern in einem zweidimensionalen Raum, können Präferenzen auch in dreidimensionaler Form abgebildet werden.
Diese Darstellung beruht auf der Annahme, dass Präferenzen von Nutzern für Elemente zusätzlich durch kontextuelle Information (z.B. Zeit, Ort, Gesellschaft) beeinflusst werden \cite[S. 195]{adomavicius:3:inbook}.
Eine zweidimensionale Darstellung der Präferenzen wie in Gleichung \ref{eq4} vernachlässigt demnach eine entscheidende Dimension: den Kontext von Präferenzen \cite[S. 253]{berkovsky:article}.
Für die Darstellung der Nutzermodelle wird daher in kontextbasierten Empfehlungssystemen von einer Matrix $R_{CA}$ ausgegangen \cite[S. 254]{berkovsky:article}:
\begin{equation}\label{eq9}
    R_{CA}: Nutzer_{Attribut} \times Element_{Attribut} \times Kontext_{Attribut} \rightarrow rating
\end{equation}
Matrix $R_{CA}$ erweitert die zweidimensionale Darstellung aus Gleichung \ref{eq4} um die Dimension $Kontext_{Attribut}$, welche den Kontext einer Bewertung angibt \cite[S. 254]{berkovsky:article}.\footnote{\textcite[S. 249ff.]{berkovsky:article} merken an, dass es sich bei den Dimensionen $Nutzer_{Attribut}$ und $Element_{Attribut}$ auch um Sammlungen von Attributen handeln kann. Dies ist der Fall, wenn Nutzer oder Elemente für das Abbilden von Präferenzen über mehrere Attribute beschrieben werden. Angenommen die Präferenzen von Nutzern können über $n$ Attribute des Nutzers und $m$ Attribute der Elemente ($n \cup m > 1$) beschrieben werden, so ergibt sich für $R'_{gen}$ eine multidimenisonale $n+m$-Matrix. Dies gilt analog für die Dimension $Context_{Attribut}$ \cite[S. 254f.]{berkovsky:article}.}
% MIT AUFNEHMEN JA/ NEIN?
% In Anlehnung an \textcite[S. 250ff]{berkovsky:article} und \textcite[S. 198]{adomavicius:3:inbook} kann für die Abbildung der zugehörigen Nutzermodelle Liste an Tripeln angenommen werden:
% \begin{equation}\label{eq10}
%     UM_{c} = \{(s_{1},q_{1}):r_{1},\textnormal{ }(s_{2},q_{2}),\textnormal{ }...,\textnormal{ }(s_{k},q_{k})\}
% \end{equation}
% Jeder Eintrag $(s_{k},q_{k}):r_{k}$ in $UM_{c}$ stellt die Bewertung eines Nutzers $c$ für ein Element

Eine dreidimensionale Darstellung von Nutzermodellen ist beispielhaft in Abbildung \ref{fig:empfehlungssysteme:preferences:abb1:3} dargestellt.
Die abgebildete Nutzer$_{ID}$$\times$Element$_{ID}$$\times$-Kontext$_{ID}$-Matrix stellt eine Erweiterung der Matrix aus Abbildung \ref{fig:empfehlungssysteme:preferences:abb1:1} um die kontextuelle Information "Teamzugehörigkeit" dar.
Dem liegt die Annahme zugrunde, dass die Einschätzung der eigenen Fähigkeiten eines Mitarbeiters von den Teamkollegen abhängt, mit denen ein Mitarbeiter zusammenarbeitet.
Folglich kann ein Mitarbeiter für eine Fähigkeit mehrere Bewertungen aufweisen.

Ein Eintrag $R_{CA}(c_{i},s_{j},q_{t}) = r_{ijt}$ der Matrix stellt die Bewertung $r_{ijt}$ einer Fähigkeit $s_{j}$ durch einen Mitarbeiter $c_{i}$ innerhalb eines Teams $q_{t}$ dar \cite[S. 198]{adomavicius:3:inbook}.
Ein Nutzermodell repräsentiert eine Sammlung aller Bewertungen von Fähigkeiten durch einen Mitarbeiter in Abhängigkeit der Teamzugehörigkeit.
Das Nutzermodell $UM$ des Mitarbeiters John D. ist in Abbildung \ref{fig:empfehlungssysteme:preferences:abb1:3} beispielhaft dargestellt.
% Die angegebenen Bewertungen sind willkürlich gewählt.
% Hier was anderes als Teamzugehörigkeit finden! das macht in dem kontext keinen sinn

% Bei popularität: nur bewertung von element ist ausschlaggeben, ggf. noch kategorie, aber sonst reicht bewertung, um ordinale Relation zu erstellen -> d.h. über alle nutzer verteilt die durchschnittlich höchste bewertung -> es existieren nicht wirklich individuelle präferenzen, eher ableitung aus aggregierten präferenzen
% Bei paarweiser Relation: Bewertung einer Beziehung, d.h. betrachten von beziehungen zwischen Nutzern und Elementen, bzw. sogar zwischen Nutzer und Nutzern -> präferenz bedeutet hier die ordinale ordnung von elementen für einen Nutzer (hier habe ich leider keine referenz zu literatur, lediglich eine annahme, muss also weiterschauen)
% das ist der wichtige unterschied zu uns, da wir präferenzen als ein attribut des mitarbeiters bzw als ein attribut des nutzers verstehen, und nicht als interpretation einer relation
\section{Wechselseitige Empfehlungssysteme}
\label{ch:empfehlungssysteme:rrs}
% Problem: ist reciprocal recommender nur, wenn das empfehlende element auch selber entscheidet, wen es annimmt?
Nach \textcite[S. 2429]{palomares:inproceedings} werden Empfehlungssysteme größtenteils in Online-Diensten eingesetzt, um Nutzern basierend auf ihren (expliziten oder impliziten) Präferenzen potenziell interessante Elemente vorzuschlagen.
Zu klassischen Inhalten von Empfehlungen zählt \textcite[S. 2429]{palomares:inproceedings} Objekte wie Musik, Filme, Konsumgüter und Restaurants. 
 
% Als wechselseitig werden Empfehlungssysteme bezeichnet, in denen Personen den Inhalt von Empfehlungen bilden und für die Empfehlungserstellung sowohl die Präferenzen der Empfänger, als auch der empfohlenen Personen berücksichtigt werden (bspw. in Online-Dating-Plattformen oder Recruiting-Systemen) \cite[S. 35]{li:inproceedings}\cite[S. 2199]{akehurst:inproceedings}\cite[S. 207]{pizzato:2010}.
Empfehlungssysteme, in denen Personen die Inhalte von Empfehlungen bilden und diese Empfehlungen reziprok sind, werden in der Literatur als wechselseitige Empfehlungssysteme (engl.: \ac{RRS}) bezeichnet \cite[S. 35]{li:inproceedings}\cite[S. 2199]{akehurst:inproceedings}\cite[S. 207]{pizzato:2010}.
Reziprok bedeutet, dass für eine erfolgreiche Empfehlung sowohl die Präferenzen des Empfängers einer Empfehlung, als auch die Präferenzen der empfohlenen Person berücksichtigt werden \cite[S. 447]{pizzato:2013}\cite[S. 22]{kleinerman:inproceedings}.
Bekannte Domänen für den Einsatz wechselseitiger Empfehlungssysteme sind Online-Dating und (semi-)automatisiertes Recruiting.
Nachfolgend wird das Konzept der RRS und dessen zugrundeliegende Problemstellung erläutert.

\subsection{People-to-People-Empfehlung}% oder umbenennen in Allgemeines Konzept
\label{ch:empfehlungssysteme:rrs:people_to_people}
Ein grundlegender Unterschied zwischen klassischen Empfehlungssystemen und RRS liegt in der Art der Empfehlung.
Diese können in \ac{I2P}- und \ac{P2P}-Empfehlungen unterschieden werden \cite[S. 62f.]{kim:inproceedings}.

In der Literatur wird unter traditioneller I2P-Empfehlung die Empfehlung klassischer Inhalte (z.B. Bücher, Filme, Hotels) verstanden \cite[S. 2199]{akehurst:inproceedings}\cite[S. 2429]{palomares:inproceedings}.
In I2P-Empfehlungen ist der Erfolg einer Empfehlung durch die Akzeptanz des Empfehlungsempfängers (d.h. des Nutzers) gekennzeichnet \cite[S. 131]{kleinerman:2:inproceedings}\cite[S. 546]{koprinska:inbook}.
Für personalisierte I2P-Empfehlungen verwenden Empfehlungssysteme Wissen über die Präferenzen der Nutzer, um ihnen potenziell interessante Elemente zu empfehlen \cite[S. 403]{terveen:article}.

% Ein grundlegendes Problem, welches dieser Art der Empfehlung löst, ist das Problem der Informationsüberlastung \cite[S. 403]{terveen:article}.
% Indem Empfehlungssysteme Nutzern aus einem Überangebot an (unbekannten) Elementen potenziell nützliche Elemente vorschlagen, können sie Nutzer bei der Entscheidungsfindung unterstützen.

Im Vergleich zu traditionellen Empfehlungssystemen basieren RRS auf P2P-Empfehlungen.
Unter P2P-Empfehlung (auch: social matching \cite[S. 208]{pizzato:2010}) wird die Empfehlung von Personen an Nutzer eines Empfehlungssystems verstanden.
Im Gegensatz zu I2P-Empfehlungen, handelt es sich bei den empfohlenen Elementen nicht um leblose Objekte, sondern um Menschen, die anderen Menschen empfohlen werden \cite[S. 2]{kazienko:inbook}.
Das bedeutet, dass die empfohlenen Elemente Präferenzen besitzen und auch positiv oder negativ reagieren können \cite[S. 2]{kazienko:inbook}.
Dies ist ein besonderer Aspekt, den es zu beachten gilt, da dadurch die Qualität der Empfehlungen beeinflusst werden kann \cite[S. 208]{pizzato:2010}\cite[S. 2199]{akehurst:inproceedings}.

\subsection{Abgrenzung traditioneller Empfehlungssysteme}
\label{ch:empfehlungssysteme:rrs:traditional_vs_rrs}
Bei Empfehlungen in RRS handelt es sich um reziproke P2P-Empfehlungen \cite[S. 207]{pizzato:2010}.
Das heißt, der Erfolg einer Empfehlung in RRS hängt von der Berücksichtigung der bilateralen Präferenzen ab, nicht, wie in traditionellen Systemen, lediglich von den Präferenzen der Nutzer \cite[S. 1468]{yildirim:article}.
So reicht es beispielsweise in einem Recruiting-System nicht aus, lediglich die Präferenzen eines Akteurs (bspw. dem Recruiter) zu berücksichtigen.
Damit dem Recruiter auch Personen vorgeschlagen werden, die ein potenzielles Angebot annehmen würden, müssen die Präferenzen der Stellensuchenden ebenfalls in Betracht gezogen werden.
Andernfalls besteht die Gefahr, dass einem Recruiter Stellensuchende vorgeschlagen werden, die aufgrund ihrer Präferenzen ein Angebot des Recruiters ablehnen.

Ein Großteil der P2P-Empfehlungen beruhen auf dem Aspekt der Reziprozität (engl.: reciprocity) \cite[S. 545]{koprinska:inbook}.
Jedoch ist nicht jede P2P-Empfehlung unmittelbar reziprok.
Empfehlungen können auch P2P-Empfehlungen sein und lediglich die Präferenzen des Nutzers berücksichtigen \cite[S. 2429]{palomares:inproceedings}.
Ein Beispiel stellen P2P-Empfehlungen des sozialen Netzwerks Twitter dar \cite[S. 2429]{palomares:inproceedings}.
Der Vorschlag einer anderen Person zu folgen beruht dabei lediglich auf den Präferenzen des Empfängers des Vorschlags.

Neben dem Aspekt der Reziprozität besitzen RRS zusätzliche Eigenschaften, welche diese im Vergleich zu traditionellen Empfehlungssystemen inhärent komplex machen\cite[S. 2429]{palomares:inproceedings}\cite[S. 35]{li:inproceedings}\cite[S. 207]{pizzato:2010}.
Die wichtigsten Unterschiede zwischen RRS und traditionellen RS sind in Tabelle \ref{tab2} zusammengefasst \cite[S. 546]{koprinska:inbook}.

In RRS ist es von Bedeutung die Passivität von Personen zu berücksichtigen \cite[S. 35]{li:inproceedings}.
Dazu gehört hauptsächlich, reaktive Personen in Empfehlungen einzubeziehen, die nicht aktiv den Kontakt zu anderen Personen suchen \cite[S. 459]{pizzato:2013}.
Im Vergleich dazu können traditionelle Empfehlungssysteme durchaus über inaktive Elemente verfügen, das heißt Elemente, die in keinen Empfehlungen vorkommen \cite[S. 208]{pizzato:2010}.

Nutzer traditioneller Empfehlungssysteme haben in der Regel wenig Anreiz ihre Nutzerprofile zu pflegen \cite[S. 546]{koprinska:inbook}.
Explizite Angaben sind daher meist rar.
Währenddessen verwenden Nutzer traditioneller Systeme diese meist über einen längeren Zeitraum, weshalb der Anteil impliziter Präferenzen verhältnismäßig groß ist \cite[S. 208]{pizzato:2010}.
In RRS ist das Pflegen eines Nutzerprofils mit Präferenzen und persönlichen Daten meist fester Bestandteil.
Daher sind Nutzer in RRS eher gewillt Angaben zu machen \cite[S. 208]{pizzato:2010}.
Es gilt jedoch zu beachten, dass die angegebenen Präferenzen nicht zwangsläufig der Realität entsprechen \cite[S. 457]{pizzato:2013}.
Dies kann unter anderem daran liegen, dass Nutzer diese selbst falsch einschätzen \cite[S. 457]{pizzato:2013}.

In wechselseitigen Empfehlungssystemen sind Empfehlungen für Nutzer endlich.
Das bedeutet, dass ein Element nicht unbegrenzt Nutzern eines Systems vorgeschlagen werden kann.
Beispielsweise kann ein Nutzer einer Online-Dating-Plattform nicht zeitgleich fünfzig verschiedene Personen daten \cite[S. 35]{li:inproceedings}.
Im Vergleich dazu sind Elemente in traditionellen Empfehlungssystemen für Empfehlungen an unterschiedliche Nutzer nicht begrenzt \cite[S. 1468]{yildirim:article}.
Das bedeutet, ein Element kann zeitgleich jedem Nutzer eines Systems vorgeschlagen werden.
Diese Eigenschaft wird als Endlichkeit bezeichnet \cite[S. 35]{li:inproceedings}.
% Weiterer unterschied: RRS unterstützen keine n-to-n connections % S. 423, https://link.springer.com/content/pdf/10.1007/978-1-0716-2197-4.pdf

\begin{table}[htbp]
    \begin{center}
        \begin{tabular}{p{0.2\textwidth}|p{0.35\textwidth}|p{0.35\textwidth}}
             & {\textbf{Traditionelle RS}} & {\textbf{Reciprocal RS}} \\
            \hline
            Reziprozität & Erfolg einer Empfehlung ist durch den Empfänger bestimmt. &  Erfolg einer Empfehlung ist sowohl durch den Empfänger als auch der empfohlenen Person bestimmt. \\
            \hline
            Passivität & Es wird akzeptiert, dass einige Elemente nicht Teil irgendwelcher Empfehlungen sind. &  Auch reaktive Nutzer müssen Teil von Empfehlungen sein. \\
            \hline
            Rolle der Nutzerprofile & Kein Anreiz für Nutzer zum Pflegen des Nutzerprofils. &  Das Pflegen von Nutzerprofilen wird erwartet. \\
            \hline
            Endlichkeit & Ein Element kann allen Nutzern vorgeschlagen werden. & Beliebte Personen sollten nicht allen Nutzern vorgeschlagen werden. \\
    \end{tabular}
    \end{center}
    \caption[Grundlegende Unterschiede von RS und RRS]{Grundlegende Unterschiede von RS und RRS \\
	(Eigene Darstellung in Anlehnung an \cite[S. 546]{koprinska:inbook} und \cite[S. 35f.]{li:inproceedings})}
	\label{tab2}
\end{table}

\subsection{Problemstellung}
Nach \textcite[S. 35]{li:inproceedings} stellt das Zufriedenstellen der Bedürfnisse beider Akteure, das heißt von sowohl Empfänger einer Empfehlung als auch empfohlener Person, die größte Herausforderung wechselseitiger RS dar.
Das Problem ist nachfolgend in Abbildung \ref{fig:empfehlungssysteme:rrs:abb1} veranschaulicht.\footnote{Es gilt zu beachten, dass in RRS die Elemente von Empfehlungen (empfohlene Personen) gleichzeitig auch Nutzer desselbigen Systems darstellen können.}
Hierfür wurde die vereinfachte Darstellung des Empfehlungssystems aus Abbildung \ref{fig:empfehlungssysteme:einführung:abb1} aufgegriffen.

Auf der linken Seite der Abbildung sind die Komponenten eines traditionellen Empfehlungssystems dargestellt.
Elemente, die die Bedürfnisse des Nutzers $X$ erfüllen, sind durch einen empfangenden roten Pfeil gekennzeichnet.
Fehlende Verbindungen zwischen Elementen und dem Nutzer $X$ bedeuten, dass ein Element die Bedürfnisse des Nutzers $X$ nicht erfüllt.
Soll dem Nutzer $X$ das Element empfohlen werden, welches seine Bedürfnisse am stärksten erfüllt, würde nach Abbildung \ref{fig:empfehlungssysteme:rrs:abb1} eindeutig Element $C$ ausgewählt werden.

Auf der rechten Seite von Abbildung \ref{fig:empfehlungssysteme:rrs:abb1} sind die Komponenten eines wechselseitigen Empfehlungssystems vereinfacht dargestellt.
Zusätzlich zu den präferierten Elementen des Nutzers $X$ sind in der Darstellung auch die Präferenzen der empfohlenen Personen abgebildet.
Daraus wird ersichtlich, dass Person $C$ die Bedürfnisse des Nutzers $X$ erfüllt, aber eine Empfehlung von Person $C$ für Nutzer $X$ nicht die Bedürfnisse von Person $C$ erfüllen würde.
Eine Empfehlung von Person $A$ würde die Bedürfnisse von Person $A$ erfüllen, dann jedoch die Bedürfnisse des Nutzers $X$ unerfüllt lassen.
Das Problem besteht folglich darin, aus einer Menge an Personen eine passende Person für Nutzer $X$ zu ermitteln, sodass sowohl die Bedürfnisse des Nutzers als auch der empfohlenen Person möglichst optimal erfüllt werden.

\begin{figure}[H]
    \centering
	\includegraphics[width=1.0\textwidth]{gfx/traditional-vs-rrs.png}
	\caption[Traditionelle vs. wechselseitige Empfehlungssysteme]{Traditionelle vs. wechselseitige Empfehlungssysteme\\}
	\label{fig:empfehlungssysteme:rrs:abb1}
\end{figure}

Nach \textcite[S. 36]{li:inproceedings} kann diese Problemstellung der reziproken Empfehlung mathematisch als das "Stable Marriage"-Problem modelliert werden.
Das "Stable-Marriage"-Problem beschreibt das Problem der Bestimmung einer optimalen Paarung (engl.: Match) zweier Mengen an Elementen anhand ihrer Präferenzen, sodass kein Element durch Paarung mit einem anderen Partner besser gestellt wäre \cite[S. 36]{li:inproceedings}\cite[S. 67]{diaz:inproceedings}.

Um in dem Szenario einer reziproken Empfehlung zu ermitteln, ob ein Paar optimal ist, müssen nach \textcite[S. 36]{li:inproceedings} für jeden Nutzer $c \in C$ eine Selbstbeschreibung $F_{c}^{d}$ und Präferenzen $F_{c}^{p}$ bekannt sein.
Analog müssen für jede Person (Element) $s \in S$ des Systems eine Selbstbeschreibung $F_{s}^{d}$ und Präferenzen $F_{s}^{p}$ bekannt sein.
Eine optimale Paarung definieren \textcite[S. 36]{li:inproceedings} wie folgt: 

\begin{definition}[Optimale Paarung]\label{def:1}
    Sei ein Nutzer $c \in C$ und eine Person (Element) $s \in S$ gegeben, dann ist das Paar $(c,s)$ optimal, wenn $F_{c}^{d}$ maximal $F_{s}^{p}$ und $F_{s}^{d}$ maximal $F_{c}^{p}$ erfüllt.
\end{definition}

Für reziproke Empfehlungen wird meist kein hartes Match zwischen Nutzer und Element gesucht, sondern eine sortierte Liste an potenziellen Partnern für einen Nutzer \cite[S. 67]{diaz:inproceedings}.
Das heißt, das Ziel der Empfehlungserstellung an einen Nutzer $c$ besteht darin dem Nutzer eine Liste $L_{s} \subset S$ an Personen $s$ zu empfehlen, sodass die Präferenzen $F_{s}^{p}$ und $F_{c}^{p}$ erfüllt sind \cite[S. 36]{li:inproceedings}.

Nach \textcite[S. 36]{li:inproceedings} ist das Bestimmen einer optimalen Paarung in RRS erschwert, da neben der Reziprozität auch Eigenschaften wie die Verfügbarkeit von Elementen (Endlichkeit) berücksichtigt werden müssen.
So kann es sein, dass eine Person anhand ihrer Präferenzen optimal zu einem Nutzer passt, die empfohlene Person aber in einem gegebenen Zeitraum nicht verfügbar ist.
Um dies bei der Bestimmung von Paaren zu berücksichtigen, unterscheiden \textcite[S. 37]{li:inproceedings} daher zusätzlich zwischen optimalem und erfolgreichem Match von Nutzern und Elementen.
Eine erfolgreiche Paarung definieren \textcite[S. 37]{li:inproceedings} wie folgt:

\begin{definition}[Erfolgreiche Paarung]\label{def:2}
    Sei ein Nutzer $c \in C$ und eine Person (Element) $s \in S$ gegeben, dann ist ein Paar $(c,s)$ erfolgreich, wenn $F_{c}^{d}$ suboptimal $F_{s}^{p}$ und $F_{s}^{d}$ suboptimal $F_{c}^{p}$ erfüllt, unter Berücksichtigung der Verfügbarkeit von $c$ und $s$.
\end{definition}

Demnach kann eine Paarung auch dann erfolgreich sein, wenn Beschreibungen von Nutzern bzw. Elementen die Präferenzen des Partners nur teilweise erfüllen \cite[S. 37]{li:inproceedings}.
Dafür wird die Verfügbarkeit von Personen berücksichtigt.
% nochmal hier schauen, eher genau so formulieren, wie es dort steht: file://wsl%24/Ubuntu/home/masc6/Projects/masterarbeit/literatur/MEET%20-%20A%20Generalized%20Framework.pdf

\shorthandon{"}