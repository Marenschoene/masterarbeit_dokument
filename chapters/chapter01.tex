\shorthandoff{"}
\chapter{Einführung}
\label{ch:intro}

\section[Problemstellung]{1.1 Problemstellung}
\label{sec:intro:motivation}
In Unternehmen aller Branchen beobachten Wissenschaftler eine Abkehr von großgewachsenen, zentralen und hierarchischen Organisationsstrukturen. Stattdessen setzen Betriebe zunehmend auf kleine, dezentrale und flexible Teams, welche in Projektarbeiten neue Produkte entwickeln und Dienstleistungen erbringen \cite[S. 15]{elanceEconomy:1999}. Besonders stark nimmt dieser Trend seit Mitte der 1990er Jahre zu \cite[S. 8]{whittington:1999}. Ein Hauptgrund für den Wandel ist die fortschreitende Digitalisierung. Diese ermöglicht kurze Kommunikations- und Entscheidungswege und macht so zentrale und hierarchische Unternehmensstrukturen zunehmend überflüssig. Gleichzeitig ermöglicht sie höhere Kreativität und Flexibilität innerhalb der Organisation \cite[S. 17]{elanceEconomy:1999}.\\
Die Deutsche Gesellschaft für Projektmanagement e.V. \cite[S. 16]{gpm:2015} stellte fest, dass ein durchschnittlicher Angestellter in Deutschland im Jahr 2013 mehr als ein Drittel seiner Arbeitszeit mit Projekttätigkeiten verbrachte. Die Organisation erwartet, dass die Projektarbeit in Zukunft einen noch größeren Anteil an der Gesamtarbeitszeit in Anspruch nehmen wird. Mitarbeiter sind folglich nicht mehr über mehrere Jahre hinweg in einer Abteilung des Unternehmens tätig, sondern finden sich immer wieder zu neuen, kurzlebigen Projekten zusammen.\\
Mit Blick auf diese Entwicklung stellen \textcite[S. 14]{elanceEconomy:1999} sogar in Frage, ob Unternehmen in ihrer heutigen Form zukünftig überhaupt noch existieren werden. Sie prognostizieren die Entstehung einer "E-Lance Economy". Darunter verstehen sie eine Gesellschaft von elektronisch verbundenen Freiberuflern, welche sie als "e-lancer" bezeichnen. Alle Marktteilnehmer sind in dieser Wirtschaftsform rechtlich selbstständig und finden sich immer wieder zu neuen, temporären Netzwerken zusammen, um gemeinsam an Projekten zu arbeiten.\\
Auch wenn dieses Szenario noch in ferner Zukunft liegt, zeichnet sich ab, dass die Zusammenstellung von Teams für einzelne Projekte ein immer häufiger stattfindender Prozess in der Wirtschaft sein wird \cite[S. 2]{malinowski:2008}. Gleichzeitig zeigen repräsentative Studien, dass die Personalabteilungen deutscher Unternehmen schon heute bei der Besetzung von Stellen überlastet sind \cite[S. 244]{personalbeschaffung:2017}. So fällt es insbesondere Personalsachbearbeitern bekannter Unternehmen schwer, die hohe Anzahl an Kandidaten zu überblicken. Die HR-Abteilungen weniger bekannter Organisationen verfügen dagegen häufig über zu wenig Angestellte für eine ausreichende Prüfung der Unterlagen \cite[S. 8]{hays:2013}. Diese Überlastung der Personalabteilungen wird durch fehlende Automatisierungsangebote in Bezug auf die Zusammenführung von Kandidaten und Stellen zusätzlich verschärft \cite[S. 15]{hays:2013}.

\section[Zielsetzung]{1.2 Zielsetzung}
\label{sec:intro:zielsetzung}
Durch den fortschreitenden und branchenübergreifenden Wandel von Linien- zu Projektorganisationen besteht die Notwendigkeit, ein IT-System für Personalsachbearbeiter und Projektmanager zu entwickeln. Dieses soll das Ziel verfolgen, passende Mitarbeiter des Unternehmens für offene Projektpositionen zu empfehlen. Die vorliegende Arbeit basiert auf einer Literaturrecherche und beantwortet die folgende Forschungsfrage: "Welche Ansätze existieren nach aktuellem Stand der Forschung zur Implementierung eines IT-Systems für die Empfehlung geeigneter Mitarbeiter für offene Projektpositionen in Unternehmen?"\\
Es wird angemerkt, dass sich vergleichsweise wenige Veröffentlichungen explizit mit der Entwicklung von Systemen zur Empfehlung von Mitarbeitern für Projektpositionen beschäftigen. Ein wesentlich größerer Forschungsbereich ist das Vorschlagen von Personen für Stellen im Allgemeinen. Beide Arten von Systemen sind jedoch technologisch gleich konzipiert. Aus diesem Grund bezieht die vorliegende Literaturrecherche auch Werke mit ein, welche sich mit der Empfehlung von Personen für Stellen im Allgemeinen befassen. Der Fokus der vorliegenden Ausarbeitung liegt ausschließlich auf der Empfehlung von Mitarbeitern, welche aufgrund ihrer Fähigkeiten zu denen in einem Projekt gesuchten Kompetenzen kompatibel sind. Weitere Forschungsschwerpunkte wie die Transferierung unstrukturierter Bewerbungsdaten in semantische Fähigkeiten oder die automatisierte Aktualisierung von Kompetenz-Datenbanken werden in dieser Arbeit nicht thematisiert.
%Schon heute sind die Personabteilungen von Unternehmen jeder Größe bei der Besetzung von Stellen überlastet. Gleichzeitig lässt der Trend zu projektbasierten Organisationen erwarten, dass Personaler zusätzlich zu ihrer bisherigen Tätigkeit bereits eingestellte Mitarbeiter häufig in Projekte vermitteln müssen. Fehlende Automatisierung könnte die Überlastung der Personalabteilungen weiter verschärfen. Aus diesem Grund
\shorthandon{"}